\documentclass[12pt]{article}
\usepackage[a4paper, margin=2.54cm]{geometry}

% español
\usepackage[spanish]{babel}

% imágenes
\usepackage{graphicx}
%\graphicspath{{img}}

% fuentes de conjuntos numéricos
\usepackage{amsfonts}

% símbolos
\usepackage{amsmath, amssymb}

% gráficos
%\usepackage{tikz}

% plots
%\usepackage{pgfplots}
%\pgfplotsset{width=10cm, compat=1.9}

% averiguar
\setlength{\jot}{8pt}
\setlength{\parindent}{0cm}

% espacio entre párrafos
\usepackage[skip=10pt plus1pt]{parskip}

% cancelar términos
\usepackage{cancel}

% links
%\usepackage[colorlinks=true, 
%    urlcolor=blue]{hyperref}

% shapes
%\usetikzlibrary{shapes.geometric}

% incluir pdfs
%\usepackage{pdfpages}

\title{Física I\\Cheatsheet 2\(^{\circ}\) parcial}
\author{Daniel Ise}
\date{Octubre, 2024}

\begin{document}

\maketitle

\tableofcontents

\pagebreak

\section{Cantidad de movimiento}

\section{Cuerpo rígido}

\subsection{Cinemática rotacional}

\textbf{Velocidad angular.} Velocidad angular inicial
más aceleración angular por tiempo.

\begin{align}
    \omega = \omega_{0} + \alpha t
\end{align}

\section{Movimiento periódico}

\section{Ondas}

\section{Sonido}

\section{Mecánica de fluidos}

\end{document}

