\documentclass[12pt]{article}
\usepackage[a4paper, margin=2.54cm]{geometry}
\usepackage[spanish]{babel}

% imágenes
%\usepackage{graphicx}
%\graphicspath{{img}}

% fuentes de conjuntos numéricos
\usepackage{amsfonts}

% math
\usepackage{amsmath, amssymb}

% gráficos y plots
\usepackage{tikz}
%\usepackage{pgfplots}
%\pgfplotsset{width=10cm, compat=1.9}
\usetikzlibrary{babel}

\setlength{\jot}{8pt}
\setlength{\parindent}{0cm}

% espacio entre párrafos
\usepackage[indent=12pt]{parskip}

% cancelar términos
\usepackage{cancel}

% links
%\usepackage[colorlinks=true,
%    urlcolor=blue]{hyperref}

% shapes
%\usetikzlibrary{shapes.geometric}

% incluir pdfs
%\usepackage{pdfpages}

\begin{document}

\thispagestyle{empty}

\begin{center}
	\vspace*{.5cm}
	\includegraphics[scale=.6]{~/Pictures/udemm-logo.png}\\
	\vspace{.2cm}
	\Large
	\textbf{Facultad de Ingeniería}\\
	\textbf{Ingeniería en Sistemas}\\
	\vspace{2cm}

	\Huge
	Física I\\
	Cheatsheet Parcial N\(^\circ\) 2\\
	\vfill

	\raggedright
	\Large
	Docentes:
	\begin{itemize}
		\item[] Lic. Francisco González \\
	\end{itemize}
	Alumno:
	\begin{itemize}
		\item[] Daniel Ise
	\end{itemize}
	Legajo:
	\begin{itemize}
		\item[] 28547
	\end{itemize}
	Fecha:
	\begin{itemize}
		\item[] Noviembre, 2024
	\end{itemize}
\end{center}

\pagebreak

\tableofcontents



\pagebreak

\section{Cantidad de movimiento}

\subsection{Momento}

El momento -o cantidad de movimiento-
es una medida que refiere a la velocidad con que se mueve un cuerpo,
esto es,
un objeto con masa,
y se puede expresar como:

\begin{equation}
	P = mv
\end{equation}

Siendo \(m\) la masa del cuerpo en cuestión y \(v\) la velocidad del mismo.
El momento se mide en \(kg\cdot\frac{m}{s}\). Es una cantidad vectorial.

El momento se puede relacionar con la fuerza, ya que si recordamos,
fuerza se puede expresar como:

\begin{align*}
	F = ma
\end{align*}

Si reescribimos:

\begin{align*}
	F = m\cdot\frac{\Delta v}{\Delta t} \\
	F = \frac{m \Delta v}{\Delta t}     \\
	F = \frac{\Delta P}{\Delta t}
\end{align*}

La fuerza se puede expresar como el cambio del Momento lineal en el
tiempo.
De ello deducimos que, en ausencia de fuerzas externas,
el momento lineal se mantiene constante.

\subsection{Impulso}


El impulso es el cambio en la cantidad de movimiento de cuerpo,
esto es,
una partícula con masa:

\begin{equation}
	J = \Delta P = mP - mP_{0}
\end{equation}

El impulso, naturalmente, se mide también en \(kg\cdot\frac{m}{s}\).

Sabemos que la fuerza es \(F = \Delta P / \Delta t\).
Si despejamos el cambio en el momento lineal,
obtenemos \(\Delta P = F \Delta t\).
Por lo tanto:

\begin{equation}
	J = F \Delta t
\end{equation}

Desde este punto de vista,
el impulso se puede definir como la aplicación de una fuerza durante un intervalo de tiempo.

\subsection{Conservación del momento}

En física,
se habla de conservación cuando una cantidad se mantiene en dos momentos de un sistema.
La conservación del momento implica que el sistema considerado se encuentra \textit{aislado},
esto es,
no existen fuerzas externas al mismo actuando sobre él.
En estas condiciones, podemos afirmar que:

\begin{align*}
	m_{1}v_{1} + \ddots + m_{n}v_{n} = m_{1f}v_{1f} + \ddots + m_{nf}v_{nf}
\end{align*}

Esto quiere decir que la sumatoria del estado inicial del sistema es igual a la 
sumatoria del estado final del sistema.

\pagebreak

\section{Cuerpo rígido}

\pagebreak

\section{Movimiento periódico}

\pagebreak

\section{Ondas}

\pagebreak

\section{Sonido}

\end{document}
