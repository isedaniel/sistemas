\documentclass[12pt]{article}
\usepackage[a4paper, margin=2.54cm]{geometry}
\usepackage[spanish]{babel}

% imágenes
%\usepackage{graphicx}
%\graphicspath{{img}}

% fuentes de conjuntos numéricos
\usepackage{amsfonts}

% math
\usepackage{amsmath, amssymb}

% gráficos y plots
\usepackage{tikz}
%\usepackage{pgfplots}
%\pgfplotsset{width=10cm, compat=1.9}
\usetikzlibrary{babel}

\setlength{\jot}{8pt}
\setlength{\parindent}{0cm}

% espacio entre párrafos
\usepackage[indent=12pt]{parskip}

% cancelar términos
\usepackage{cancel}

% links
%\usepackage[colorlinks=true,
%    urlcolor=blue]{hyperref}

% shapes
%\usetikzlibrary{shapes.geometric}

% incluir pdfs
%\usepackage{pdfpages}

\begin{document}

\thispagestyle{empty}

\begin{center}
	\vspace*{.5cm}
	\includegraphics[scale=.6]{~/Pictures/udemm-logo.png}\\
	\vspace{.2cm}
	\Large
	\textbf{Facultad de Ingeniería}\\
	\textbf{Ingeniería en Sistemas}\\
	\vspace{2cm}

	\Huge
	Física I\\
	Cheatsheet Parcial N\(^\circ\) 2\\
	\vfill

	\raggedright
	\Large
	Docentes:
	\begin{itemize}
		\item[] Lic. Francisco González \\
	\end{itemize}
	Alumno:
	\begin{itemize}
		\item[] Daniel Ise
	\end{itemize}
	Legajo:
	\begin{itemize}
		\item[] 28547
	\end{itemize}
	Fecha:
	\begin{itemize}
		\item[] Noviembre, 2024
	\end{itemize}
\end{center}

\pagebreak

\tableofcontents



\pagebreak

\section{Cantidad de movimiento}

\subsection{Momento}

El momento -o cantidad de movimiento-
es una medida que refiere a la velocidad con que se mueve un cuerpo
-esto es,
un objeto con masa-
y se puede expresar como:

\begin{equation}
	P = mv
\end{equation}

Siendo \(m\) la masa del cuerpo en cuestión y \(v\) la velocidad del mismo.
El momento se mide en \(kg\cdot\frac{m}{s}\). Es una magnitud vectorial.

El momento se puede relacionar con la fuerza, ya que si recordamos,
fuerza se puede expresar como:

\begin{align*}
	F = ma
\end{align*}

La aceleración es el cambio en la velocidad durante un intervalo de tiempo (\(\Delta v / \Delta t\)), reescribimos:

\begin{align*}
	F = m\cdot\frac{\Delta v}{\Delta t} \\
	F = \frac{m \Delta v}{\Delta t}     \\
	F = \frac{\Delta P}{\Delta t}
\end{align*}

La fuerza se puede expresar como el cambio del Momento lineal en el
tiempo.
De ello deducimos que, en ausencia de fuerzas externas,
el momento lineal se mantiene constante.

\subsection{Impulso}


El impulso es el cambio en la cantidad de movimiento de un objeto:

\begin{equation}
	J = \Delta P = mP - mP_{0}
\end{equation}

El impulso, naturalmente, se mide también en \(kg\cdot\frac{m}{s}\).

Sabemos que la fuerza es \(F = \Delta P / \Delta t\).
Si despejamos el cambio en el momento lineal,
obtenemos \(\Delta P = F \Delta t\).
Por lo tanto:

\begin{equation}
	J = F \Delta t
\end{equation}

Desde este punto de vista,
el impulso se puede definir como la aplicación de una fuerza durante un intervalo de tiempo.

\subsection{Conservación del momento}

En física,
se habla de conservación cuando una magnitud se mantiene constante durante
el período de tiempo en que se considera al sistema.
La conservación del momento implica que el sistema considerado
se encuentra \textit{aislado},
es decir,
no existen fuerzas externas al mismo actuando sobre él.
O, lo que en términos prácticos es lo mismo,
la fuerza neta total que actúa sobre él es igual a 0.
En estas condiciones, podemos afirmar que:

\begin{align*}
	m_{1i}v_{1i} + \dots + m_{ni}v_{ni} = m_{1f}v_{1f} + \dots + m_{nf}v_{nf}
\end{align*}

Esto quiere decir que la sumatoria del momento inicial del sistema es igual a la
sumatoria del momento final del sistema.

\subsection{Choques}

En mecánica se distinguen dos tipos de choques:
\begin{itemize}
	\item Elásticos
	\item Inelásticos
\end{itemize}

Un choque elástico es aquel en el cual la energía cinética total se
conserva, por lo cual la cantidad de energía perdida en la colisión es nula
-o prácticamente nula y, por lo tanto, despreciable.

\begin{align*}
	K_{A0} + K_{B0} = K_{A} + K_{B} \\
\end{align*}

Un choque inelástico es aquel en el que los objetos que colisionan pierden parte
\textit{significativa} de su energía en el choque,
que se disipa en deformación de los objetos,
calor y ondas de sonido.

\begin{align*}
	K_{A0} + K_{B0} \neq K_{A} + K_{B}
\end{align*}

Recordar que energía cinética se define como:

\begin{align*}
	K = \frac{1}{2}mv^{2}
\end{align*}

Es decir, mitad de la masa por la velocidad al cuadrado.

Hay que resaltar que,
con independencia de lo que suceda con la energía en el sistema,
el momento lineal se puede conservar si se dan las condiciones para ello.

La ventaja de las colisiones elásticas es que permiten emplear la igualdad de
energía cinética al comienzo y final del período en que se considera el sistema.
Empleando este igualdad, se puede arribar a una expresior muy simple y elegante
para resolver problemas de choques elásticos.

Dados dos objetos, \(A\) y \(B\), 
la igualdad de energía cinética a principio y final implica que:

\begin{equation}
	v_{Ai} + v_{Af} = v_{Bi} + v_{Bf}
\end{equation}

La ventaja de las colisiones inelásticas es que permiten unificar las masas y,
por lo tanto, tener una sola variable de velocidad del otro lado de la igualdad,
que aplicaría a las dos masas consideradas.

\subsection{Centro de masa}

El centro de masa es un punto que puede ser considerado como equiparable al 
conjunto de la masa de un objeto,
esto es,
si aplico una fuerza sobre el centro de masa de un objeto,
equivaldría a empujar al objeto completo.

Si,
por el contrario,
aplico una fuerza fuera del centro de masa de un objeto,
el resultado sería la rotación de dicho cuerpo.

El centro de masa de un cuerpo coincide con su centro geométrico,
siempre y cuando la distribución de su masa sea homogénea.

Ahora,
dado un sistema arbitrario,
¿cómo podemos conocer su centro de masa?

En ese caso debemos guiarnos por la expresión:

\begin{align*}
	X = \frac{\sum_{i=1}^{n}m_{i}x_{i}}{\sum_{i=1}^{n}m_{i}}
\end{align*}

Es decir,
la posición del centro de masa (X)
es igual a la sumatoria de cada masa multiplicada por su posición,
dividido por la sumatoria de todas las masas.
Lo importante para que esta expresión se cumpla es la \textit{consistencia},
es decir,
usar el mismo punto de referencia \(x=0\) de principio a fin.

\pagebreak

\section{Cuerpo rígido}

\pagebreak

\section{Movimiento periódico}

\pagebreak

\section{Ondas}

\pagebreak

\section{Sonido}

\end{document}
