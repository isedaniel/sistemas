\documentclass[12pt]{article}
\usepackage[a4paper, margin=2.54cm]{geometry}
\usepackage[spanish]{babel}

% imágenes
%\usepackage{graphicx}
%\graphicspath{{img}}

% fuentes de conjuntos numéricos
\usepackage{amsfonts}

% math
\usepackage{amsmath, amssymb}

% gráficos y plots
\usepackage{tikz}
%\usepackage{pgfplots}
%\pgfplotsset{width=10cm, compat=1.9}
\usetikzlibrary{babel}

\setlength{\jot}{8pt}
\setlength{\parindent}{0cm}

% espacio entre párrafos
\usepackage[indent=12pt]{parskip}

% cancelar términos
\usepackage{cancel}

% links
%\usepackage[colorlinks=true,
%    urlcolor=blue]{hyperref}

% shapes
%\usetikzlibrary{shapes.geometric}

% incluir pdfs
%\usepackage{pdfpages}

\begin{document}

\thispagestyle{empty}

\begin{center}
	\vspace*{.5cm}
	%\includegraphics[scale=.6]{~/Pictures/udemm-logo.png}\\
	\vspace{.2cm}
	\Large
	\textbf{Facultad de Ingeniería}\\
	\textbf{Ingeniería en Sistemas}\\
	\vspace{2cm}

	\Huge
	Física I\\
	Cheatsheet Parcial N\(^\circ\) 2\\
	\vfill

	\raggedright
	\Large
	Docentes:
	\begin{itemize}
		\item[] Lic. Germán González \\
	\end{itemize}
	Alumno:
	\begin{itemize}
		\item[] Daniel Ise
	\end{itemize}
	Legajo:
	\begin{itemize}
		\item[] 28547
	\end{itemize}
	Fecha:
	\begin{itemize}
		\item[] Noviembre, 2024
	\end{itemize}
\end{center}

\pagebreak

\tableofcontents



\pagebreak

\section{Cantidad de movimiento}

\subsection{Momento}

El momento -o cantidad de movimiento-
es una medida que refiere a la velocidad con que se mueve un cuerpo
-esto es,
un objeto con masa-
y se puede expresar como:

\begin{equation}
	P = mv
\end{equation}

Siendo \(m\) la masa del cuerpo en cuestión y \(v\) la velocidad del mismo.
El momento se mide en \(kg\cdot\frac{m}{s}\). Es una magnitud vectorial.

El momento se puede relacionar con la fuerza, ya que si recordamos,
fuerza se puede expresar como:

\begin{align*}
	F = ma
\end{align*}

La aceleración es el cambio en la velocidad durante un intervalo de tiempo (\(\Delta v / \Delta t\)), reescribimos:

\begin{align*}
	F = m\cdot\frac{\Delta v}{\Delta t} \\
	F = \frac{m \Delta v}{\Delta t}     \\
	F = \frac{\Delta P}{\Delta t}
\end{align*}

La fuerza se puede expresar como el cambio del Momento lineal en el
tiempo.
De ello deducimos que, en ausencia de fuerzas externas,
el momento lineal se mantiene constante.

\subsection{Impulso}


El impulso es el cambio en la cantidad de movimiento de un objeto:

\begin{equation}
	J = \Delta P = mP - mP_{0}
\end{equation}

El impulso, naturalmente, se mide también en \(kg\cdot\frac{m}{s}\).

Sabemos que la fuerza es \(F = \Delta P / \Delta t\).
Si despejamos el cambio en el momento lineal,
obtenemos \(\Delta P = F \Delta t\).
Por lo tanto:

\begin{equation}
	J = F \Delta t
\end{equation}

Desde este punto de vista,
el impulso se puede definir como la aplicación de una fuerza durante un intervalo de tiempo.

\subsection{Conservación del momento}

En física,
se habla de conservación cuando una magnitud se mantiene constante durante
el período de tiempo en que se considera al sistema.
La conservación del momento implica que el sistema considerado
se encuentra \textit{aislado},
es decir,
no existen fuerzas externas al mismo actuando sobre él.
O, lo que en términos prácticos es lo mismo,
la fuerza neta total que actúa sobre él es igual a 0.
En estas condiciones, podemos afirmar que:

\begin{align*}
	m_{1i}v_{1i} + \dots + m_{ni}v_{ni} = m_{1f}v_{1f} + \dots + m_{nf}v_{nf}
\end{align*}

Esto quiere decir que la sumatoria del momento inicial del sistema es igual a la
sumatoria del momento final del sistema.

\subsection{Choques}

En mecánica se distinguen dos tipos de choques:
\begin{itemize}
	\item Elásticos
	\item Inelásticos
\end{itemize}

Un choque elástico es aquel en el cual la energía cinética total se
conserva, por lo cual la cantidad de energía perdida en la colisión es nula
-o prácticamente nula y, por lo tanto, despreciable.

\begin{align*}
	K_{A0} + K_{B0} = K_{A} + K_{B} \\
\end{align*}

Un choque inelástico es aquel en el que los objetos que colisionan pierden parte
\textit{significativa} de su energía en el choque,
que se disipa en deformación de los objetos,
calor y ondas de sonido.

\begin{align*}
	K_{A0} + K_{B0} \neq K_{A} + K_{B}
\end{align*}

Recordar que energía cinética se define como:

\begin{align*}
	K = \frac{1}{2}mv^{2}
\end{align*}

Es decir, mitad de la masa por la velocidad al cuadrado.

Hay que resaltar que,
con independencia de lo que suceda con la energía en el sistema,
el momento lineal se puede conservar si se dan las condiciones para ello.

La ventaja de las colisiones elásticas es que permiten emplear la igualdad de
energía cinética al comienzo y final del período en que se considera el sistema.
Empleando este igualdad, se puede arribar a una expresior muy simple y elegante
para resolver problemas de choques elásticos.

Dados dos objetos, \(A\) y \(B\),
la igualdad de energía cinética a principio y final implica que:

\begin{equation}
	v_{Ai} + v_{Af} = v_{Bi} + v_{Bf}
\end{equation}

La ventaja de las colisiones inelásticas es que permiten unificar las masas y,
por lo tanto, tener una sola variable de velocidad del otro lado de la igualdad,
que aplicaría a las dos masas consideradas.

\subsection{Centro de masa}

El centro de masa es un punto,
perteneciente a un objeto o sistema de objetos,
para el cual,
si sumamos la posición de todos sus elementos ponderados por su masa,
obtenemos 0.

En un cuerpo rígido y de distribución uniforme,
el centro de masa se ubica en su centroide.
Puede ubicarse también fuera de la masa del sistema,
por ejemplo,
en el caso de un anillo.

Para concer el centro de masa de un sistema arbitrario recurrimos a la expresión:

\begin{align*}
	C_{x} = \frac{\sum_{i=1}^{n}m_{i}x_{i}}{M}
\end{align*}

Es decir,
la posición del centro de masa (C)
es igual a la sumatoria de la masa de cada elemento,
multiplicada por su posición,
dividida por la sumatoria de la masa de todos los elementos (M).
Es importante cuidar la \textit{consistencia},
es decir,
usar el mismo punto de referencia \(x=0\) para todas las posiciones.

Cuando tenemos problemas de dos dimensiones se recurre a la misma expresión,
considerando las componentes horizontal y vertical por separado,
siguiendo la misma recomendación:
cuidar la consistencia \(x=0 \land y=0\) para todos los elementos.

El centro de masa es útil en tres tipos de problemas.

En primer lugar,
cuando una fuerza se aplica de manera uniforme sobre un objeto,
por complejo que este sea,
se puede considerar que la fuerza se aplica sobre el centro de masa,
facilitando así el trabajo con el mismo.

En segundo lugar,
y como consecuencia directa del enunciado anterior,
en mecánica clásica generalmente se considera al campo gravitatorio como una fuerza uniforme,
por lo que esta se aplicaría directamente sobre el centro de masa,
facilitando así problemas que de otra forma podrían ser muy complejos.

Y en tercer y último lugar,
considerar que la fuerza de gravedad se aplica sobre el centro de masa tiene una aplicación interesante en los problemas de volteo.
Si tomamos el caso de un vehículo mal cargado,
por ejemplo,
una camioneta con la mayoría de su carga ubicada sobre su lado izquierdo,
cuando el vector peso
-que podemos trazar desde el centro de masa y hacia la tierra-
sobresale de los puntos de contacto de la camioneta con el suelo,
podemos estar seguros de que existirá un volteo.
El grado máximo de inclinación que soporta un objeto se conoce como \textit{límite de volteo}.

\pagebreak

\section{Cuerpo rígido}

\subsection{Principales variables del movimiento angular}

Equiparable al desplazamiento cuando hablamos de movimiento lineal,
tenemos el \textit{desplazamiento angular} (\(\theta\)),
que refiere al ángulo que forma el objeto con su rotación,
medido en radianes.

\begin{equation}
	\Delta\theta = \theta_{f} - \theta_{i}
\end{equation}

Para dar cuenta de la distancia recorrida,
denominada \textit{longitud de arco} (\(S\)),
es necesario considerar el radio (\(r\)) de la circunferencia,
multiplicado por el ángulo medido en radianes
\footnote{La ventaja de medir ángulos en radianes es que esta y otras conversiones se pueden hacer de manera directa.}:

\begin{equation}
	S = \theta\cdot r
\end{equation}

La \textit{velocidad angular} (\(\omega\)) es otra variable importante.
Refiere a la tasa de cambio del desplazamiento angular en relación al tiempo.

\begin{equation}
	\Delta\omega = \frac{\Delta\theta}{\Delta t}
\end{equation}

Vinculada a la velocidad angular encontramos el concepto de \textit{rapidez} (\(v\)),
que refiere ya no al ángulo que describe el objeto en movimiento sino a la trayectoria,
siguiendo la circunferencia,
en relación al tiempo.
Como todas las variables que refieren a la circunferencia,
se encuentra multiplicando a la velocidad angular por el radio.

\begin{equation}
	v = \frac{S}{\Delta t} = \omega r
\end{equation}

En cuanto a la aceleración, distinguimos tres tipos.

En primer lugar,
tenemos la \textit{aceleración angular} (\(\alpha\)),
que refiere al cambio en la velocidad angular en relación al tiempo.

\begin{equation}
	\alpha = \frac{\Delta\omega}{\Delta t}
\end{equation}

En todo movimiento circular existe un tipo de aceleración cuya función es cambiar la dirección del vector velocidad.
Esta aceleración se denomina \textit{aceleración centrípeta} (\(a_{c}\)).

\begin{equation}
	a_{c} = \frac{v^{2}}{r}
\end{equation}

Por último,
tenemos la \textit{aceleración tangencial},
que resulta de la multiplicación de la aceleración angular por el radio:

\begin{equation}
	a_{t} = \alpha r
\end{equation}

\subsection{Cinemática rotacional}

Como pasa con sus contrapartes,
dentro de la cinemática en una sola dimensión,
las variables del movimiento angular se relacionan siguiendo algunas fórmulas.

Dado el caso que la aceleración angular sea constante, 
la velocidad angular en función del tiempo se puede expresar como:

\begin{equation}
	\omega = \omega_{i} + \alpha t
\end{equation}

Por su parte,
el desplazamiento angular sigue la expresión:

\begin{equation}
	\theta = \omega_{i}t + \frac{1}{2}\alpha t^{2}
\end{equation}

\subsection{Torque}

Dado un eje,
-que podemos situar convenientemente en \(x=0\)-,
llamamos \textbf{torca} (\(\tau\)) al producto de una fuerza aplicada por la distancia al mismo.
Se mide en Newton-metro aunque,
a diferencia del trabajo,
la torca no es fuerza aplicada en la misma dirección que el movimiento del objeto.

\begin{equation}
	\tau = Fd
\end{equation}

El torque cuando se aplica con un ángulo con respecto al vector \(d\) es:

\begin{equation}
	\tau = Fd\sen(\theta)
\end{equation}

\subsection{Dinámica rotacional}

La aceleración angular es igual al torque sobre la inercia rotacional:

\begin{equation}
	\alpha = \frac{\tau}{I}
\end{equation}

Llamamos inercia rotacional
\footnote{También se la denomina momento rotacional} 
a la masa multiplicada por el radio al cuadrado:

\begin{equation}
	I = mr^{2}
\end{equation}

Recordar:
si son sistemas, siempre sumatorias ponderadas,
como en todo sistema considerado en mecánica clásica.


\pagebreak

\section{Movimiento periódico}

\pagebreak

\section{Ondas}

\pagebreak

\section{Sonido}

\end{document}
