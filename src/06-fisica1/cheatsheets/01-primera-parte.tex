\documentclass[12pt]{article}
\usepackage[a4paper, margin=2.54cm]{geometry}

% español
\usepackage[spanish]{babel}

% imágenes
%\usepackage{graphicx}
%\graphicspath{{img}}

% fuentes de conjuntos numéricos
\usepackage{amsfonts}

% símbolos
\usepackage{amsmath, amssymb}

% gráficos
%\usepackage{tikz}

% plots
%\usepackage{pgfplots}
%\pgfplotsset{width=10cm, compat=1.9}

% averiguar
\setlength{\jot}{8pt}
\setlength{\parindent}{0cm}

% espacio entre párrafos
\usepackage[skip=10pt plus1pt]{parskip}

% cancelar términos
\usepackage{cancel}

% links
%\usepackage[colorlinks=true, 
%    urlcolor=blue]{hyperref}

% shapes
%\usetikzlibrary{shapes.geometric}

% incluir pdfs
%\usepackage{pdfpages}

\title{Física I\\Cheatsheet Primer Parcial}
\author{Daniel Ise}
\date{Septiembre, 2024}

\begin{document}

\maketitle

\tableofcontents

\pagebreak

\section{Cinemática}

\subsection{Desplazamiento}

Es la distancia recorrida independientemente de la trayectoria.

Trayectoria \(\implies\) escalar.
Desplazamiento \(\implies\) vectorial.
\begin{align*}
    \Delta x = x_{f} - x_{i}
\end{align*}

\subsection{Rapidez}

Distancia sobre tiempo. Distancia es escalar \(\implies\) rapidez es escalar.
\begin{align*}
    \overline{s} = \frac{\Delta d}{\Delta t}
\end{align*}

\subsection{Velocidad}

Desplazamiento en función del tiempo. Desplazamiento es vectorial \(\implies\) velocidad es vectorial.
\begin{align*}
    v_{x} = \frac{\Delta x}{\Delta t}
\end{align*}

Velocidad instantánea.
\begin{align*}
    v_{x} = \lim_{\Delta t \to 0} \frac{\Delta x}{\Delta t}
\end{align*}

\subsection{Aceleración}

Cambio en la velocidad en el tiempo. Velocidad vectorial \(\implies\) acelereación vectorial.
También tiene promedio y derivada.
\begin{align*}
    a = \frac{\Delta v}{\Delta t}
\end{align*}

\subsection{Movimiento rectilíneo uniforme}

Velocidad constante sobre línea recta \(\implies\) aceleración nula
\begin{align*}
    v = v_{i} &  & x = x_{i} + v_{i} \cdot t &  & \overline{v} = \vec{v}
\end{align*}

\subsection{Movimiento rectilíneo uniformemente variado}

Aceleración constante sobre línea recta.
\begin{align*}
    a = a_{i} &  & v = at + v_{i} &  & x = \frac{1}{2}at^{2} + v_{i} + x_{i}
\end{align*}

\subsection{Coordenadas intrínsecas de la aceleración}

El vector velocidad puede cambiar en módulo o dirección.
Para ello, se utiliza un sistema de referencia intrínseco para la aceleración.
Se divide en un componente tangencial y otro normal.
El tangencial es tangente a la trayectoria y responsable del módulo de \(\vec{v}\).
El normal es centrípeto y responsable de la dirección de \(\vec{v}\).
\begin{align*}
    |a_{t}| = \frac{d|v|}{dt} &  & |a_{n}| = \frac{|v|^{2}}{r}
\end{align*}

\subsection{Tiro oblicuo}

Objeto lanzado por el aire,
con velocidad inicial y ángulo respecto del suelo,
suponiendo efecto constante de la gravedad,
y despreciando efecto del rozamiento del aire.

MRU en dirección horizontal,
MRUV en dirección vertical (caída libre).

\textbf{Ecuaciones en dirección horizontal.}
\begin{align*}
    v_{x} = v_{i}\cdot\cos \alpha &  & a_{x} = 0 &  & x = v_{x}\cdot t
\end{align*}
\begin{align*}
    A = \frac{2v_{i}^{2}\cdot\sin\alpha\cdot\cos\alpha}{g}
\end{align*}

\textbf{Ecuaciones en dirección vertical.}
\begin{align*}
    v_{y} = v_{i}\cdot \sen \alpha - g\cdot t &  & a_{y} = -g &  & y = \frac{1}{2} \cdot (-g) \cdot t^{2} + v_{y} \cdot t
\end{align*}
\begin{align*}
    H = \frac{v_{i}^{2} \cdot \sen^{2}\alpha}{2g} &  & t_{v} = \frac{2v_{y}}{g}
\end{align*}

\subsection{Velocidad relativa en dos dimensiones}

Velocidad relativa de A visto de B, dado un observador fijo T que ve a los dos,
es:
\begin{align*}
    \vec{v}_{AB} = \vec{v}_{AT} - \vec{v}_{BT}
\end{align*}

\section{Dinámica}

\subsection{Fuerza}

Se denomina fuerza a la \textit{interacción} que puede cambiar el estado de
movimiento de un objeto
\footnote{Aunque una fuerza no equilibrada también podría \textit{deformar}
    un objeto.}.

Una fuerza produce aceleración. La aceleración es una magnitud vectorial.
\(\implies\) La fuerza es una magnitud vectorial.

Cuando varias fuerzas actúan sobre un mismo objeto,
nos interesa el vector de \textbf{fuerza neta},
que es igual a la suma vectorial de las fuerzas actuantes.
\begin{align*}
    \vec{R} = \vec{F}_{1} + \vec{F}_{2} + \cdots + \vec{F}_{n}
\end{align*}

\subsection{Leyes de Newton}

Son descripciones a partir de la observación.
Son fundamentales porque no puede deducirse ni demostrarse.
Son la base de la mecánica clásica.
En realidad serían más bien principios que leyes,
similares a los axiomas en matemática.

\subsubsection{Primera ley}

"Todo cuerpo permanece en su estado de reposo o de movimiento rectilíneo
y uniforme a menos que se le obligue a varias dicho estado mediante 
fuerzas que actúen sobre él."

Es decir,
cuando la fuerza neta actúante sobre un objeto es 0,
su aceleración es 0.

\end{document}
