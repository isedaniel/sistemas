\documentclass[12pt]{article}
\usepackage[a4paper, margin=2.54cm]{geometry}

% español
\usepackage[spanish]{babel}

% imágenes
%\usepackage{graphicx}
%\graphicspath{{img}}

% fuentes de conjuntos numéricos
\usepackage{amsfonts}

% símbolos
\usepackage{amsmath, amssymb}

% gráficos
%\usepackage{tikz}

% plots
%\usepackage{pgfplots}
%\pgfplotsset{width=10cm, compat=1.9}

% averiguar
\setlength{\jot}{8pt}
\setlength{\parindent}{0cm}

% espacio entre párrafos
\usepackage[skip=10pt plus1pt]{parskip}

% cancelar términos
\usepackage{cancel}

% links
%\usepackage[colorlinks=true, 
%    urlcolor=blue]{hyperref}

% shapes
%\usetikzlibrary{shapes.geometric}

% incluir pdfs
%\usepackage{pdfpages}

\title{Física I\\Cheatsheet Primer Parcial}
\author{Daniel Ise}
\date{Septiembre, 2024}

\begin{document}

\maketitle

\tableofcontents

\pagebreak

\section{Cinemática}

\subsection{Desplazamiento}

Es la distancia recorrida independientemente de la trayectoria.

Trayectoria \(\implies\) escalar.
Desplazamiento \(\implies\) vectorial.
\begin{align*}
  \Delta x = x_{f} - x_{i}
\end{align*}

\subsection{Rapidez}

Distancia sobre tiempo. Distancia es escalar \(\implies\) rapidez es escalar.
\begin{align*}
  \overline{s} = \frac{\Delta d}{\Delta t}
\end{align*}

\subsection{Velocidad}

Desplazamiento en función del tiempo. Desplazamiento es vectorial \(\implies\) velocidad es vectorial.
\begin{align*}
  v_{x} = \frac{\Delta x}{\Delta t}
\end{align*}

Velocidad instantánea.
\begin{align*}
  v_{x} = \lim_{\Delta t \to 0} \frac{\Delta x}{\Delta t}
\end{align*}

\subsection{Aceleración}

Cambio en la velocidad en el tiempo. Velocidad vectorial \(\implies\) acelereación vectorial.
También tiene promedio y derivada.
\begin{align*}
  a = \frac{\Delta v}{\Delta t}
\end{align*}

\subsection{Movimiento rectilíneo uniforme}

Velocidad constante sobre línea recta \(\implies\) aceleración nula
\begin{align*}
  v = v_{i} &  & x = x_{i} + v_{i} \cdot t &  & \overline{v} = \vec{v}
\end{align*}

\subsection{Movimiento rectilíneo uniformemente variado}

Aceleración constante sobre línea recta.
\begin{align*}
  a = a_{i} &  & v = at + v_{i} &  & x = \frac{1}{2}at^{2} + v_{i} + x_{i}
\end{align*}

\subsection{Coordenadas intrínsecas de la aceleración}

El vector velocidad puede cambiar en módulo o dirección.
Para ello, se utiliza un sistema de referencia intrínseco para la aceleración.
Se divide en un componente tangencial y otro normal.
El tangencial es tangente a la trayectoria y responsable del módulo de \(\vec{v}\).
El normal es centrípeto y responsable de la dirección de \(\vec{v}\).
\begin{align*}
  |a_{t}| = \frac{d|v|}{dt} &  & |a_{n}| = \frac{|v|^{2}}{r}
\end{align*}

\subsection{Tiro oblicuo}

Objeto lanzado por el aire,
con velocidad inicial y ángulo respecto del suelo,
suponiendo efecto constante de la gravedad,
y despreciando efecto del rozamiento del aire.

MRU en dirección horizontal,
MRUV en dirección vertical (caída libre).

\textbf{Ecuaciones en dirección horizontal.}
\begin{align*}
  v_{x} = v_{i}\cdot\cos \alpha &  & a_{x} = 0 &  & x = v_{x}\cdot t
\end{align*}
\begin{align*}
  A = \frac{2v_{i}^{2}\cdot\sin\alpha\cdot\cos\alpha}{g}
\end{align*}

\textbf{Ecuaciones en dirección vertical.}
\begin{align*}
  v_{y} = v_{i}\cdot \sen \alpha - g\cdot t &  & a_{y} = -g &  & y = \frac{1}{2} \cdot (-g) \cdot t^{2} + v_{y} \cdot t
\end{align*}
\begin{align*}
  H = \frac{v_{i}^{2} \cdot \sen^{2}\alpha}{2g} &  & t_{v} = \frac{2v_{y}}{g}
\end{align*}

\subsection{Velocidad relativa en dos dimensiones}

Velocidad relativa de A visto de B, dado un observador fijo T que ve a los dos,
es:
\begin{align*}
  \vec{v}_{AB} = \vec{v}_{AT} - \vec{v}_{BT}
\end{align*}

\section{Dinámica}

\subsection{Fuerza}

Se denomina fuerza a la \textit{interacción} que puede cambiar el estado de
movimiento de un objeto
\footnote{Aunque una fuerza no equilibrada también podría \textit{deformar}
  un objeto.}.

Una fuerza produce aceleración. La aceleración es una magnitud vectorial.
\(\implies\) La fuerza es una magnitud vectorial.

Cuando varias fuerzas actúan sobre un mismo objeto,
nos interesa el vector de \textbf{fuerza neta},
que es igual a la suma vectorial de las fuerzas actuantes.
\begin{align*}
  \vec{R} = \vec{F}_{1} + \vec{F}_{2} + \cdots + \vec{F}_{n}
\end{align*}

\subsection{Leyes de Newton}

Son descripciones a partir de la observación.
Son fundamentales porque no puede deducirse ni demostrarse.
Son la base de la mecánica clásica.
En realidad serían más bien principios que leyes,
similares a los axiomas en matemática.

\subsubsection{Primera ley}

\textbf{Ley de inercia.}
``Todo cuerpo permanece en su estado de reposo o de movimiento rectilíneo
y uniforme, a menos que se le obligue a varias dicho estado mediante
fuerzas que actúen sobre él.''

Es decir,
cuando la fuerza neta sobre un objeto es 0,
su aceleración es 0,
manteniendo su estado,
sea \textit{reposo} o \textit{movimiento rectilíneo uniforme}.
\begin{align*}
  \sum\vec{F} = 0 \implies a=0
\end{align*}

\subsubsection{Segunda ley}

``La aceleración de un objeto es proporcional a la fuerza neta que actúa sobre
él,
e inversamente proporcional a su masa.
La dirección de la aceleración es igual a la de la fuerza neta aplicada.''
\begin{align*}
  \sum\vec{F}=m\cdot a
\end{align*}

\subsubsection{Tercera ley}

\textbf{Ley de acción y reacción.}
``Si A ejerce una fuerza sobre B, B ejerce una fuerza sobre A igual en
módulo y dirección, pero de sentido contrario.''

Una fuerza no \textit{reacciona} contra la otra. Se dan de forma simultánea,
actuando sobre cuerpos diferentes, formando un \textbf{par acción-reacción}.
El par nunca puede darse sobre un único cuerpo.
\begin{align*}
  \vec{F}_{AB} = -\vec{F}_{BA}
\end{align*}

\subsection{Definición de newton}

\begin{align*}
  1N=1kg\cdot \frac{m}{s^{2}}
\end{align*}

\subsection{Peso y masa}

El peso es quivalente a la fuerza gravitacional ejercida sobre un objeto.
Varía con la posición. La masa no.
\begin{align*}
  w = m\cdot g
\end{align*}

\subsection{Fricción}

Resistencia al movimiento cuando dos materiales entran en contacto.

\textbf{Fricción entre sólidos.}
Generalmente se distingues tres tipos:

\textbf{Fricción estática.}
Cuando la fuerza de fricción es suficiente como para impedir un movimiento
relativo entre superficies.
Su magnitud máxima se ejerce justo antes de que el objeto empiece a deslizarse.
Su dirección es paralela a la superficie de contacto,
con sentido opuesto al movimiento inminente.
\begin{align*}
  f_{s} = \mu_{s}\cdot N
\end{align*}

\textbf{Fricción cinética.}
Cuando hay movimiento relativo entre las superficies en contacto.
Cuando el objeto empieza a deslizarse, la fricción pasa de estática a cinética.
Su dirección también es paralela a la superficie de contacto,
con sentido opuesto al movimiento relativo.
\begin{align*}
  f_{k} = \mu_{k}\cdot N
\end{align*}

\textbf{Fricción de rodamiento.}
Cuando una supericie gira sobre otra.

Generalmente, \(\mu_{s} > \mu_{k}\).
El intervalo de valores típicos es (0.03, 1).

\subsection{Movimiento circular uniforme}

Dado que el MCU requiere aceleración hacie el centro,
la fuerza neta debe estar dirigida también hacia el centro.
La magnitud de la aceleración, por su parte, permanece constante.
\begin{align*}
  \sum F=m\cdot a_{c} = m \cdot \frac{v^{2}}{r}
\end{align*}

\subsubsection{Peralte}

Una curva no require fricción cuando:
\begin{align*}
  F_{n}\cdot \sen\theta = m \cdot \frac{v^{2}}{r}
\end{align*}

\subsection{4 fuerzas de la naturaleza}

En física clásica solo interesan la fuerza gravitacional y electromagnética.

\subsubsection{Fuerza gravitacional}

Responsable de los grandes movimientos del universo.
Siempre atractiva y de alcance infinito.
La fuerza resultante atrae hacia el centro de gravedad de un objeto.
En el interior de los átomos no juega un papel importante.

\subsubsection{Fuerza electromagnética}

Responsable de la interacción entre partículas con carga eléctrica,
por extensión,
de todas las reacciones químicas,
por extensión,
de todos los fenómenos biológicos.
Actúan sobre partículas cargadas eléctricamente.
Naturaleza atractiva y repulsiva.
Radio de interacción infinito.

Son manifestaciones de intercambio de fotones.

Mantiene unidos a los átomos y moléculas.
Es dominante en la estructura atómica y molecular.

\subsubsection{Fuerza nuclear fuerte}

Alcance menor que la interacción eléctrica, pero magnitud mayor.

Papel fundamental en las reacciones termonucleares que ocurren dentro del Sol,
generando luz y calor.

Mantiene unidos a los protones en el núcleo.

\subsubsection{Fuerza nuclear débil}

Actúa entre partículas elementales.

Responsable de un tipo de desintegración radiactiva.

Es débil, pero permite que las estrellas produzcan luz y energía.

\section{Trabajo y energía}

\textbf{Trabajo.} 
Decimos que una fuerza \textit{realiza} un trabajo
cuando hay un desplazamiento del centro de masa del cuerpo
sobre el que ésta se aplica,
siendo en la misma dirección de dicha fuerza.
El trabajo realizado por la fuerza es quivalente a la energía necesaria 
para el desplazamiento.
\begin{align*}
  W = F \cdot \Delta x
\end{align*}

\textbf{Energía.} 
Capacidad para realizar un trabajo. 
No se crea ni se destruye: se convierte de una forma a otra.

\textbf{Energía cinética.}
Energía que un objeto tiene debido a su movimiento.
\begin{align*}
  K = \frac{1}{2}m\cdot v^{2}
\end{align*}

\textbf{Energía potencial.}
Potencial que un objeto tiene debido a su posición particular en un campo gravitacional.
\begin{align*}
  U = m\cdot g\cdot h
\end{align*}

\textbf{Energía mecánica.}
Sumatoria de la energía cinética y potencial
\begin{align*}
  E_{M} = E_{P} + E_{K}
\end{align*}

\textbf{Energía térmica.}
Es una forma particular de la energía cinética.
Dada fricción, se puede calcular la energía térmica.
\begin{align*}
  \Delta E_{T} = \mu_{k}\cdot F_{n}\cdot d
\end{align*}

\subsubsection{Conservación de la energía}

En física tres cantidades se conservan: 
energía, momento y momento angular.

\subsubsection{Potencia}

Medida de la tasa a la que se realiza un trabajo o se transfiere energía.
Se mide mediante los watts, simbolizados \(W\),
que representan trabajo -o energía- por segundo, es decir,
\(\frac{J}{s}\), por lo cual se puede entender a la potencia como:
\begin{align*}
  P = \frac{\Delta E}{\Delta t}
\end{align*}

\section{Gravitación}

La fuerza de atracción de dos masas disminuye como el inverso de la distancia
de separación \(r\) al cuadrado. Siendo \(G\) la constante de gravitación 
universal.
\begin{align*}
  F = \frac{G\cdot m_{1}\cdot m_{2}}{r^{2}}
\end{align*}

\end{document}
