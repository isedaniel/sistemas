\documentclass{article}
\usepackage[margin=2.54cm]{geometry}
%\usepackage{graphicx}               % imágenes
%\graphicspath{{img}}
\usepackage{amsfonts}               % fuentes de conjuntos numéricos
\usepackage{amsmath, amssymb}       % símbolos
%\usepackage{tikz}                   % gráficos
%\usepackage{pgfplots}               % plots
%\pgfplotsset{width=10cm, compat=1.9}
\setlength{\jot}{8pt}
\setlength{\parindent}{0cm}
\usepackage{parskip}                % espacio entre párrafos
\usepackage{cancel}                 % cancelar términos
%\usepackage[colorlinks=true, 
%    urlcolor=blue]{hyperref}        % links
%\usetikzlibrary{shapes.geometric}   % shapes
%\usepackage{pdfpages}               % incluir pdfs

\title{Física I - Segunda clase}
\author{Daniel Ise}
\date{8 de agosto de 2024}

\begin{document}

\maketitle

\section*{Cinemática en una dimensión}

\textbf{Mecánica.} Estudio del movimiento. Se dividie en cinemática y dinámica.

\textbf{Cinemática.} Estudio del movimiento sin tener en cuenta las causas.
Tiene 3 variables principales: posición, velocidad, aceleración.

\textbf{Movimiento.} Cambio de posición en un sistema de referencia, a través
del tiempo. Si la posición no cambia, decimos que el cuerpo está en reposo.

\textbf{Cuerpo.} Un punto en un sistema de referencia.

\end{document}