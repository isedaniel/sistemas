\documentclass{article}
\usepackage[a4paper, margin=2.54cm]{geometry}
%\usepackage{graphicx}               % imágenes
%\graphicspath{{img}}
\usepackage{amsfonts}               % fuentes de conjuntos numéricos
\usepackage{amsmath, amssymb}       % símbolos
%\usepackage{tikz}                   % gráficos
%\usepackage{pgfplots}               % plots
%\pgfplotsset{width=10cm, compat=1.9}
\setlength{\jot}{8pt}
\setlength{\parindent}{0cm}
\usepackage{parskip}                % espacio entre párrafos
\usepackage{cancel}                 % cancelar términos
%\usepackage[colorlinks=true, 
%    urlcolor=blue]{hyperref}        % links
%\usetikzlibrary{shapes.geometric}   % shapes
%\usepackage{pdfpages}               % incluir pdfs

\title{Ejercicios Unidad 1\\Magnitudes y mediciones}
\author{Daniel Ise}
\date{29 de agosto de 2024}

\begin{document}

\maketitle

\textbf{1.1. Basandose en que cualquier magnitud derivada puede ser expresada en 
función de las magnitudes fundamentales: longitud, masa y tiempo, establecer las
dimensiones de las siguientes magnitudes en el Sistema Internacional.}

\textbf{a. Área}

\[\text{Longitud al cuadrado: }m^{2}\]

\textbf{b. Volumen}

\[\text{Longitud al cubo: }m^{3}\]

\textbf{c. Velocidad}

\[\text{Longitud / Tiempo: }\frac{m}{s}\]

\textbf{d. Aceleración}

\[\text{Velocidad / Tiempo:} \frac{m}{s^{2}}\]

\textbf{e. Fuerza}

\[\text{Masa x Aceleración:} 1 N = 1 kg \cdot \frac{m}{s^{2}}\]

\textbf{f. Densidad}

\[\text{Masa / volumen: }\frac{kg}{m^{3}}\]

\textbf{g. Presión}

\[\text{Fuerza / Área: }1 Pa = 1 \frac{N}{m^{2}} = 1 \frac{kg}{s^{2} \cdot m}\]

\textbf{h. Trabajo}

\[\text{Fuerza x distancia: }1 J = 1 N \cdot m = 1 kg \cdot \frac{m^{2}}{s^{2}}\]

\textbf{i. Potencia}

\[\text{Trabajo / tiempo: }1 W = 1 \frac{J}{s} = 1 \frac{kg \cdot m^{2}}{s^{3}}\]

\hfill

\textbf{1.2. Conversiones de unidades.}

\textbf{a. 25 km a m}

\[25 km = 25 \cancel{km} \cdot 1000 \frac{m}{\cancel{km}} = 25 000 m\]

\textbf{b. 4 ns a s}

\[4 ns = 4 \cancel{ns} \cdot 10^{-9} \frac{s}{\cancel{ns}} = 4 \cdot 10^{-9} s\]

\textbf{c. 40 \(\mu\)W a W}

\[40 \cancel{\mu W} \cdot 10^{-6} \frac{W}{\cancel{\mu W}} = 4 \cdot 10^{-5} W\]

\textbf{d. 3 MW a W}

\[3 \cancel{MW} \cdot 10^{6} \frac{W}{\cancel{MW}} = 3 \cdot 10^{6}\]

\end{document}
