\documentclass{article}
\usepackage[margin=2.54cm]{geometry}
%\usepackage{graphicx}               % imágenes
%\graphicspath{{img}}
\usepackage{amsfonts}               % fuentes de conjuntos numéricos
\usepackage{amsmath, amssymb}       % símbolos
%\usepackage{tikz}                   % gráficos
%\usepackage{pgfplots}               % plots
%\pgfplotsset{width=10cm, compat=1.9}
\setlength{\jot}{8pt}
\setlength{\parindent}{0cm}
\usepackage{parskip}                % espacio entre párrafos
\usepackage{cancel}                 % cancelar términos
%\usepackage[colorlinks=true, 
%    urlcolor=blue]{hyperref}        % links
%\usetikzlibrary{shapes.geometric}   % shapes
%\usepackage{pdfpages}               % incluir pdfs

\title{Clase de Física I}
\author{Daniel Ise}
\date{15 de agosto de 2024}

\begin{document}

\maketitle

Hacemos algunos problemas.

Dada una expresión sobre la \textbf{posición} de una partícula, si tomo su
derivada obtengo una expresión sobre su \textbf{velocidad}. Si tomo su $2^a$
derivada obtengo una expresión de su \textbf{aceleración}.

Cuando aceleración y velocidad tienen el mismo signo, decimos que un movimiento
es \textbf{acelerado}. No importa en sí el signo, sino cuando ambos comparten
signo.

\textbf{Movimiento en 2 y 3 dimensiones.} Cuando el movimiento es en 2
dimensiones

\textbf{Posición.} Se tiene en cuenta el \textbf{origen}, que se designa $O$.
Definido el origen, la posición es un vector $\vec{P}$, que se puede notar
como un punto $(x,y)$.

\textbf{Desplazamiento.} El cambio de posición se denota con el vector
desplazamiento. El vector diferencia es el vector desplazamiento.
$$\vec{r} = \vec{r}_f - \vec{r}_i$$.

\textbf{Velocidad media.} El módulo es igual al módulo del vector desplazamiento
dividido por tiempo transcurrido. Dirección y sentido son iguales a vector
desplazamiento.

\textbf{Aceleración media.} La aceleración media es el cambio en el vector 
velocidad considerado en un período de tiempo. 

\textbf{Composición de movimiento.} Cuando un cuerpo está sometido a una 
composición de movimientos, su movimiento resultante es la suma de los dos 
vectores.

\textbf{Tiro oblicuo.} Objeto lanzado en el aire con velocidad inicial $v_0$ y 
cierto ángulo respecto al suelo, se mueve describiendo trayectoria 
curva en un plano. Caída libre es lo mismo pero tirado de una altura y sin 
$v_0$. Si suponemos gravedad constante y desprecia rozamiento del aire, estamos
ante tiro oblicuo. Es una composición de movimiento MRU en el horizontal y MRUV
en el vertical. Resulta en una parábola.

\end{document}
