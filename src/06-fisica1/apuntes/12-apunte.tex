\documentclass[12pt]{article}
\usepackage[a4paper, margin=2.54cm]{geometry}

% español
\usepackage[spanish]{babel}

% imágenes
%\usepackage{graphicx}
%\graphicspath{{img}}

% fuentes de conjuntos numéricos
\usepackage{amsfonts}

% símbolos
\usepackage{amsmath, amssymb}

% gráficos
%\usepackage{tikz}

% plots
%\usepackage{pgfplots}
%\pgfplotsset{width=10cm, compat=1.9}

% averiguar
\setlength{\jot}{8pt}
\setlength{\parindent}{0cm}

% espacio entre párrafos
\usepackage[skip=10pt plus1pt]{parskip}

% cancelar términos
\usepackage{cancel}

% links
%\usepackage[colorlinks=true, 
%    urlcolor=blue]{hyperref}

% shapes
%\usetikzlibrary{shapes.geometric}

% incluir pdfs
%\usepackage{pdfpages}

\title{Física I\\Apunte de clase}
\author{Daniel Ise}
\date{10 de octubre de 2024}

\begin{document}

\maketitle

\tableofcontents

\section{Mecánica de la rotación}

Hasta ahora hemos visto el movimiento de traslación,
centrado en el desplazamiento y la aceleración.

En esta sección vamos a aplicar ideas similares a objetos que pueden rotar.

\textbf{Cuerpo rígido.}
Cuerpo que no se deforma con la aplicación de fuerzas,
la distancia entre las partículas que lo forman permanece constante.

\section{Cinemática del movimiento rotacional}

Está vinculado al movimiento circular uniforme,
de las partículas que forman el cuerpo rígido.

\subsection{Posición angular}

Describe la rotación de una partícula,
mediante sus coordenadas \(x\) e \(y\).

Para trabajar con movimiento angular usamos radianes.
\begin{equation}
    180^{\circ} = \pi rad
\end{equation}

\subsection{Velocidad angular}

Cambio de la posición angular en relación al tiempo.
\begin{equation}
    \omega = \frac{\Delta\theta}{\Delta t}
\end{equation}

Su unidad en el SI es:
\begin{equation}
    \omega = \frac{rad}{s}
\end{equation}

Si el pulgar queda para arriba en el giro es positiva.
Si el pulgar queda hacia abajo es negativa.

\subsection{Velocidad tangencial}

Dos puntos en un disco que gira tienen la misma velocidad angular.
Sin embargo,
el punto cuya distancia al centro sea mayor va a tener una mayor
\textbf{velocidad tangencial}.

\subsection{Aceleración centrípeta}

Todo movimiento circular es acelerado,
por lo menos,
para mantener la trayectoria circular.
Esta aceleración se denomina aceleración centrípeta,
y se define como:
\begin{equation}
    a_{cen}=\frac{v^{2}}{r} = \omega^{2}\cdot r
\end{equation}

\subsection{Aceleración angular}

Es el cambio en la velocidad angular en función del tiempo.
\begin{equation}
    \alpha_{m} = \frac{\Delta\omega}{\Delta t}
\end{equation}

En el SI se expresa como:
\begin{equation}
    \frac{rad}{s^{2}}
\end{equation}

\textbf{Ejemplo.}
\begin{align*}
    \alpha = \frac{40\pi rad/s - 8\pi rad/s}{2 s}\\
    \alpha = \frac{16\pi rad}{s^{2}}\\
\end{align*}

\subsection{Período y frecuencia}

Período es la cantidad de vueltas que da por segundo.
Frecuencia es el tiempo que le lleva dar una vuelta.

La frecuencia en el SI se expresa en hertz(Hz).

Por ejemplo, una rueda que gira \(6\pi rad/s\).

\section{Dinámica de la rotación}

Un cuerpo rígio en rotación es masa en movimiento.
Por tanto, 
tiene energía cinética.

Sabemos que \(v = \omega \cdot r\),
entonces la expresión de la energía cinética se puede reescribir como:
\begin{equation}
    K_{t} = \frac{1}{2}\sum m\cdot r^{2} \cdot \omega^{2}
\end{equation}

A la expresión \(\sum m\cdot r^{2}\) se la define como el 
\textbf{momento de inercia}, que se denota como \(I\),
por lo tanto, reescribimos:
\begin{equation}
    K_{t} = \frac{1}{2} I \cdot \omega^{2}
\end{equation}

\end{document}
