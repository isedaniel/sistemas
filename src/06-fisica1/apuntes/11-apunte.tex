\documentclass[12pt]{article}
\usepackage[a4paper, margin=2.54cm]{geometry}

% español
\usepackage[spanish]{babel}

% imágenes
%\usepackage{graphicx}
%\graphicspath{{img}}

% fuentes de conjuntos numéricos
\usepackage{amsfonts}

% símbolos
\usepackage{amsmath, amssymb}

% gráficos
%\usepackage{tikz}

% plots
%\usepackage{pgfplots}
%\pgfplotsset{width=10cm, compat=1.9}

% averiguar
\setlength{\jot}{8pt}
\setlength{\parindent}{0cm}

% espacio entre párrafos
\usepackage[skip=10pt plus1pt]{parskip}

% cancelar términos
\usepackage{cancel}

% links
%\usepackage[colorlinks=true, 
%    urlcolor=blue]{hyperref}

% shapes
%\usetikzlibrary{shapes.geometric}

% incluir pdfs
%\usepackage{pdfpages}

\title{Física I\\Apunte de clase}
\author{Daniel Ise}
\date{3 de septiembre de 2024}

\begin{document}

\maketitle

\tableofcontents

\section{Momento lineal y energía cinética}

Las dos se desprenden de masa por velocidad.
\begin{align*}
    E_{k} = \frac{1}{2}mv^{2}\\
    p = mv
\end{align*}

El momento lineal es el impulso de aceleró desde el reposo a rapidez actual.

Energía es trabajo necesario para acelerar la partícula.

Momento es fuerza en el tiempo, energía es fuerza en la distancia.

\section{Conservación del momento lineal}

También se conserva,
con mayor generalidad que la energía mecánica.

Lo bueno de las cosas que se conservan es que alcanza con saber 
el estado inicial y el final.

Así se simplifican problemas.

Para eso definimos sistemas, 
con partículas que forman parte y partículas que no.
Las fuerzas que el sistema ejerce entre sí son fuerzas internas.
Las fuerzas externas son ejercidas por partículas exteriores al sistema.

Siempre que las fuerzas sean internas, 
no hay variación en la cantidad de movimiento del sistema.

\textbf{Principio de conservación del momento lineal.}
Si la suma vectorial de las fuerzas externas
sobre un sistema es cero,
el momento lineal total del sistema es constante.

\textbf{Colisión.}
Choque de cuerpos,
interacción,
con cambios muy bruscos en el sistema,
en un momento muy corto de tiempo,
lo que hace que se desprecien resto de fuerzas,
y nos centramos en los cambios en la cantidad de movimiento.

Un choque puede ser:
\begin{itemize}
    \item elástico
    \item inelástico
    \item perfectamente inelástico, que tiene la máxima pérdida de energía cinética
\end{itemize}

En la realidad,
no hay choque completamente elástico.
Solo son aproximadamente elásticas.
Debería ser silenciosa para ser perfectamente elástica.

Las colisiones entre partículas subatómicas son perfectamente elásticas.

En un choque elástico se supone conservación del momento y energía cinética 
en su totalidad.

En un choque inelástico parte de la energía cinética se pierde.

En un choque totalmente inelástico los cuerpos terminan con la misma velocidad.

\section{Centro de masa}

Punto en el que se puede considerar concentrada toda la masa de un sistema 
o de un objeto.
Cuando se considera fuerza externa, se considera que todo actúa sobre el 
centro de masa del sistema.

\begin{equation}
    x = \frac{\sum m_{i}x_{i}}{\sum m_{i}}\\
\end{equation}

\begin{equation}
    y = \frac{\sum m_{i}y_{i}}{\sum m_{i}}\\
\end{equation}

El concepto de centro de masa se usa para estudiar el movimiento.

\end{document}

