\documentclass[12pt]{article}
\usepackage[a4paper, margin=2.54cm]{geometry}

% español
\usepackage[spanish]{babel}

% imágenes
%\usepackage{graphicx}
%\graphicspath{{img}}

% fuentes de conjuntos numéricos
\usepackage{amsfonts}

% símbolos
\usepackage{amsmath, amssymb}

% gráficos
%\usepackage{tikz}

% plots
%\usepackage{pgfplots}
%\pgfplotsset{width=10cm, compat=1.9}

% averiguar
\setlength{\jot}{8pt}
\setlength{\parindent}{0cm}

% espacio entre párrafos
\usepackage[skip=10pt plus1pt]{parskip}

% cancelar términos
\usepackage{cancel}

% links
%\usepackage[colorlinks=true, 
%    urlcolor=blue]{hyperref}

% shapes
%\usetikzlibrary{shapes.geometric}

% incluir pdfs
%\usepackage{pdfpages}

\title{Física I\\Profesor: Germán González\\Apunte de clase}
\author{Daniel Ise}
\date{17 de octubre de 2024}

\begin{document}

\maketitle

\tableofcontents

\section{Equilibrio de un cuerpo rígido}

Según primera ley de Newto, cuando suma de fuerzas actuando sobre un cuerpo es 0,
permanece en reposo o movimiento constante.
En ambos casos, hablamos de equilibro traslacional, que requiere fuerza neta 0.

Sin embargo, si las fuerzas actúan con momento harán que el cuerpo gire,
por lo tanto: equilibrio de un cuerpo rígido tiene dos condiciones
\begin{itemize}
    \item Fuerza neta igual a 0, equilibrio traslacional
    \item Torque neto igual a 0, equilibrio rotacional
\end{itemize}

Si no se satisfacen las dos, no hay equilibrio.



\end{document}

