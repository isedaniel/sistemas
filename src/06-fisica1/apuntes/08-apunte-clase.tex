\documentclass[12pt]{article}
\usepackage[a4paper, margin=2.54cm]{geometry}

% español
\usepackage[spanish]{babel}

% imágenes
%\usepackage{graphicx}               
%\graphicspath{{img}}

% fuentes de conjuntos numéricos
\usepackage{amsfonts}               

% símbolos
\usepackage{amsmath, amssymb}       

% gráficos
%\usepackage{tikz}                   

% plots
%\usepackage{pgfplots}               
%\pgfplotsset{width=10cm, compat=1.9}

% averiguar
\setlength{\jot}{8pt}
\setlength{\parindent}{0cm}

% espacio entre párrafos
\usepackage[skip=8pt plus1pt]{parskip}                

% cancelar términos
\usepackage{cancel}                 

% links
%\usepackage[colorlinks=true, 
%    urlcolor=blue]{hyperref}        

% shapes
%\usetikzlibrary{shapes.geometric}   

% incluir pdfs
%\usepackage{pdfpages}               

\title{Física I\\Apunte de clase}
\author{Daniel Ise}
\date{5 de septiembre de 2024}

\begin{document}

\maketitle
\tableofcontents

\section{Conservación de la energía}

\textbf{Principio de conservación de la energía.}
La idea de que se conserva es importante.
La energía no se crea ni se destruye.
Se transfiere de un sistema a otro,
se convierte de un tipo a otro,
pero no se crea ni se destruye.
Se aplica a otras ciencias,
como la biología.

\textbf{Modelo sistema.}
En temas de energía no interesa más el modelo sistema que el modelo partícula.
La energía pasa del sistema al entorno y del entorno al sistema.

\textbf{Ejemplo de transformación de energía.}
Potencial en cinética.

\textbf{Ejemplos de conservación de la energía.}
Un motor es un sistema que convierte energía química,
almacenada en el combustible,
en energía cinética en forma de movimiento y de calor.

\textbf{Qué es la energía.}
En el universo hay,
básicamente,
\textbf{materia} y \textbf{energía}.

\textbf{Materia.}
Le pasan cosas:
se calienta,
se mueve,
se deforma,
cambia de estado.
Para todo eso,
hace falta \textbf{energía}.

\textbf{Energía.}
Capacidad de un cuerpo o sistema para realizar trabajo.
Existen otras formas de transferencia de energía,
en resumen:
energía hace que \textit{funcionen} las cosas.

\textbf{Unidades de energía.}
\textit{Joule} en el \textbf{sistema internacional}.
Energía necesaria para elevar cuerpo de \(1 N\) hasta \(1 m\).

\textit{Caloría o kilocaloría.}
Cantidad de energía para aumentar la temperatura \(1^oC\).

\textit{Kilowatt-hora.}
Derivada de la potencia.

\section{Tipos de energía}

\subsection{Energía mecánica}
Suma de la \textbf{energía cinética} y \textbf{energía potencial}.

\subsection{Energía electromagnética}

\subsection{Energía química}
De los seres vivos y baterías.

\subsection{Energía nuclear}

\subsection{Otras denominaciones}
Energía térmica por ejemplo,
la consideran como separada,
otras dentro de la mecánica.
Lo mismo pasa con mareomotriz o eólica.

\section{Transferencia de energía}

\textbf{Mecanismos de transferencia de energía.}
La energía se transfiere entre cuerpos mediante el \textbf{trabajo}.

La energía sale de la radio mediante ondas mecánicas producidas por la bocina.

\textbf{Trabajo.} Es una forma de transferencia de energía.

\section{Potencia}

Trabajo en relación al tiempo.

\textbf{Potencia media.}
Diferencia de trabajo sobre diferencia de tiempo.

\textbf{Potencia instantánea.}
Derivada del trabajo en función del tiempo.

\textbf{Unidad de medidad.}
El \textit{watt} (W) en honor a James Watt.
Equivale a \(1 J/s\).

\end{document}