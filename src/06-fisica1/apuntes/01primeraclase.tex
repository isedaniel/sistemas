\documentclass{article}
\usepackage[margin=2.54cm]{geometry}
%\usepackage{graphicx}               % imágenes
%\graphicspath{{img}}
\usepackage{amsfonts}               % fuentes de conjuntos numéricos
\usepackage{amsmath, amssymb}       % símbolos
%\usepackage{tikz}                   % gráficos
%\usepackage{pgfplots}               % plots
%\pgfplotsset{width=10cm, compat=1.9}
\setlength{\jot}{8pt}
\setlength{\parindent}{0cm}
\usepackage{parskip}                % espacio entre párrafos
\usepackage{cancel}                 % cancelar términos
%\usepackage[colorlinks=true, 
%    urlcolor=blue]{hyperref}        % links
%\usetikzlibrary{shapes.geometric}   % shapes
%\usepackage{pdfpages}               % incluir pdfs

\title{Física I - Primera clase}
\author{Daniel Ise}
\date{8 de agosto de 2024}

\begin{document}

\maketitle

\section*{Cursada}
\begin{itemize}
    \item Lunes 8 a 9.30
    \item Jueves de 18 hasta que de.
\end{itemize}

El profe \textit{puede} que tome lista. Es importante
asistir. Hay que llevarla al día, practicar, igual que álgebra.

\section*{Física: ciencia experimental}

La física es una ciencia que observa y describe el comportamiento de fenómenos naturales.

Se basa en observación experimental y medición cuantitativa. 

Sus leyes se expresan en lenguaje matematico. Entender el tratamiento teórico
implica formación en matemática.

Un \textbf{fenómeno físico} se puede observar y medir. 

El primer científico fue Galileo (1564-1642). Ha empezado a hacer observaciones,
anotar, constrastar y buscar una formulación matemática. Es el padre del método
científico. 

Se puede entender mejor otras ciencias si primero entendemos la física. 

Se puede dividir en \textbf{clásica} y \textbf{moderna}. Vemos mecánica. No 
vemos termodinámica ni electromagnetismo (física 2). El 90\% es mecánica, que es
el estudio del movimiento. 

Vamos a utilizar el sistema internacional de unidades. Hay que repasar y ejercitarse.

\end{document}
