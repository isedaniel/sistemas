\documentclass{article}
\usepackage[a4paper, margin=2.54cm]{geometry}

% español
\usepackage[spanish]{babel}

% imágenes
%\usepackage{graphicx}               
%\graphicspath{{img}}

% fuentes de conjuntos numéricos
\usepackage{amsfonts}               

% símbolos
\usepackage{amsmath, amssymb}       

% gráficos
%\usepackage{tikz}                   

% plots
%\usepackage{pgfplots}               
%\pgfplotsset{width=10cm, compat=1.9}

% averiguar
\setlength{\jot}{8pt}
\setlength{\parindent}{0cm}

% espacio entre párrafos
\usepackage[skip=8pt plus1pt]{parskip}                

% cancelar términos
\usepackage{cancel}                 

% links
%\usepackage[colorlinks=true, 
%    urlcolor=blue]{hyperref}        

% shapes
%\usetikzlibrary{shapes.geometric}   

% incluir pdfs
%\usepackage{pdfpages}               

\title{Física I\\Apunte de clase}
\author{Daniel Ise}
\date{5 de septiembre de 2024}

\begin{document}

\maketitle
\tableofcontents

\section{Energía}

Capacidad de un sistema de realizar \textbf{trabajo}.

Siempre se asocia a un tipo de \textbf{fuerza}.

Calcular rapidez de una flecha disparada es \textit{difícil} solo con \textbf{leyes de Newton}.
Debido a que la cuerda del arco ejerce \textbf{fuerza variable}.

Hasta aquí hemos trabajado con modelo partícula.
Desde aquí en adelante, trabajamos con \textbf{modelo sistema}.

\subsection{Sistema.}

\textit{Parte} del universo sobre la que se enfoca el estudio.
Primero \textit{definir} sistema, después analizar.
Puede ser: dos planetas, una colección de objetos, etc.

\subsection{Trabajo.}

En física tiene un significado muy específico.
Trabajo implica \textbf{fuerza} y \textbf{desplazamiento}.
Describe \textit{cuantitativamente} lo que se \textit{logra} cuando
una \textbf{fuerza} \textit{mueve} un objeto cierta \textbf{distancia}.

El caso más sencillo es una \textbf{fuerza constante}.

\begin{align*}
    w = \vec{F} \cdot \vec{s}
\end{align*}

En general, lo únco que efectúa trabajo es la fuerza o \textbf{componente fuerza}
paralela al desplazamiento del objeto. Es decir, si empujo un coche,
descompongo en componente en x e y la fuerza ejercida. Solo efectúa trabajo la
fuerza ejercida en x en este caso: es la que hace que el cuerpo se mueva.

\begin{align*}
    W = F \cdot s \cdot \cos \theta
\end{align*}

\textbf{Ejemplos.}

Cuerpo de 15kg cae 10 metros, calcula trabajo.
\begin{align*}
    P = 15 \cdot 9.8 = 147 N      \\
    W = 147 N \cdot 10 m = 1470 J \\
\end{align*}

Sobre cuerpo de 10 kg, fuerza de 100N, \(\alpha 30\) y se desplaza 5m.
\begin{align*}
    W = 100N \cdot 5m \cdot \cos 30^o \\
    W = 433 J                         \\
\end{align*}

\subsection{Energía cinética.}

\(K = \frac{1}{2}mv^2\).

\textbf{Ejemplo.}

Coche de 650kg,
va a 90 km/h,
frena y reduce a 50 km/h.

Calcular: cinética inicial, final y trabajo de los frenos.

\begin{align*}
     & K = \frac{1}{2} \cdot 650kg \cdot (50000 \frac{m}{h} \cdot \frac{h}{3600 s})^2 \\
     & K = \frac{1}{2} \cdot 650kg \cdot (13.889 \frac{m}{s})^2                       \\
     & \boxed{K = 62693.90 J}
\end{align*}

Cuerpo de 20kg de masa,
se mueve a \(2\frac{m}{s}\),
aceleración de \(2\frac{m}{s^2}\) durante \(5s\).
Calcular trabajo.

\begin{align*}
     & \Delta W = \frac{1}{2} 20kg \cdot \left[2\frac{m}{s} + \frac{1}{2} 2\frac{m}{s^2} \cdot (5s)^2\right] \\
\end{align*}

\subsection{Energía potencial}

Se relaciona con la \textbf{fuerza gravitatoria}.
A mayor altura, mayor distancia con respecto al centro de la tierra.
A mayor distancia, mayor energía potencial.
Se indica con \(U\).

Cuando el cuerpo baja \(U \rightarrow K\).
Notar que \(U + K\) permanece constante si no hay rozamiento.

\textbf{Energía potencial gravitatoria.}

Debida a las diferencias de altura.
\[U_g = \frac{1}{2} g \cdot h\]

\textbf{Energía potencial elástica.}

Debida a un cuerpo deformable, como un resorte o banda elástica.
\[U_e = \frac{1}{2} k \cdot x^2\]

\subsection{Fuerzas conservativas}

Fuerzas que son capaces de guardar energía y despues devolverla, como el peso.

Una fuerza es conservativa cuando el trabajo realizado no depende de la trayectoria.

Si en un sistema solo actúan FC se conserva la energía mecánica.

Hay correspondencia entre fuerza conservativa y energía potencial.

\subsection{Fuerzas no conservativas}

No guardan energía para devolverla después.

El trabajo realizado por este tipo de fuerza depende de la trayectoria.

Un automóvil pierde la energía cinética por el rozamiento,
esta energía no hay manera de recuperarla.

Una pelota que se lanza hacia el aire pierde energía por el rozamiento de subida y de bajada.
No hay forma de recuperar esta energía.

\textbf{Ejemplo.}

Reacomodando muebles.
Se mueve un sillón de 40kg por 2.5m.
La distancia directa es 2.5m.
Sin embargo, para esquivar una mesa, hay que llevarlo por 3.5m.
Calcular diferencia de trabajo si \(\mu = 0.2\).

\begin{align*}
     & F_r = 0.2 \cdot 40kg \cdot 9.8\frac{m}{s^2} \\
     & F_r = 50N                                   \\
     & W_1 = 50 \cdot 2.5m = 125J                  \\
     & W_2 = 50 \cdot 3.5m = 175J                  \\
\end{align*}

Se observa que en fuerzas no conservativas el trabajo depende de la distancia.
\end{document}