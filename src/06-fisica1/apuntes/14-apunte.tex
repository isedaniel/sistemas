\documentclass[12pt]{article}
\usepackage[a4paper, margin=2.54cm]{geometry}

% español
\usepackage[spanish]{babel}

% imágenes
%\usepackage{graphicx}
%\graphicspath{{img}}

% fuentes de conjuntos numéricos
\usepackage{amsfonts}

% símbolos
\usepackage{amsmath, amssymb}

% gráficos
%\usepackage{tikz}

% plots
%\usepackage{pgfplots}
%\pgfplotsset{width=10cm, compat=1.9}

% averiguar
\setlength{\jot}{8pt}
\setlength{\parindent}{0cm}

% espacio entre párrafos
\usepackage[skip=10pt plus1pt]{parskip}

% cancelar términos
\usepackage{cancel}

% links
%\usepackage[colorlinks=true, 
%    urlcolor=blue]{hyperref}

% shapes
%\usetikzlibrary{shapes.geometric}

% incluir pdfs
%\usepackage{pdfpages}

\title{Física I\\Prof. Germán González\\Apunte de clase}
\author{Daniel Ise}
\date{24 de octubre de 2024}

\begin{document}

\maketitle

\tableofcontents

\section{Movimiento circular y movimiento periódico}

El movimiento armónico simple es la proyección del movimiento circular uniforme
sobre un diámetro.

Si Q, partícula en el extremo del círculo,
completa una revolución en T,
P, que describe el movimiento periódico,
completa un ciclo de ocilación en el mismo tiempo T.

\section{Péndulo simple}

Modelo idealizado,
que consiste en una masa puntual,
suspendida de un cordón sin masa y no estirable.

Si la masa se mueve a un lado de su posición de equilibrio,
oscila al rededor de dicha posición .

\textbf{Condición para movimiento oscilatorio armónico simple.}
La fuerza restauradora tiene que ser proporcional al desplazamiento.

Un péndulo con ángulo chico se puede considerar como movimiento 
armónico simple.

\section{Péndulo físico}

Cualquier péndulo que usa un cuerpo de tamaño finito,
en contraste con el simple.

\end{document}

