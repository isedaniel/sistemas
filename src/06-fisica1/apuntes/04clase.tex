\documentclass{article}
\usepackage[margin=2.54cm]{geometry}
%\usepackage{graphicx}               % imágenes
%\graphicspath{{img}}
\usepackage{amsfonts}               % fuentes de conjuntos numéricos
\usepackage{amsmath, amssymb}       % símbolos
%\usepackage{tikz}                   % gráficos
%\usepackage{pgfplots}               % plots
%\pgfplotsset{width=10cm, compat=1.9}
\setlength{\jot}{8pt}
\setlength{\parindent}{0cm}
\usepackage{parskip}                % espacio entre párrafos
\usepackage{cancel}                 % cancelar términos
%\usepackage[colorlinks=true, 
%    urlcolor=blue]{hyperref}        % links
%\usetikzlibrary{shapes.geometric}   % shapes
%\usepackage{pdfpages}               % incluir pdfs

\title{Clase de Física I}
\author{Daniel Ise}
\date{22 de agosto de 2024}

\begin{document}

\maketitle

\textbf{Dinámica.} Hasta ahora hemos visto \textbf{cinemática}. Las dos son 
ramas de la \textbf{mecánica}. Se centra en las \textbf{causas} del movimiento.
Incorpora los conceptos de \textbf{fuerza} y \textbf{masa}.

\textbf{Mecánica.} Es la rama de la física que estudia el movimiento. Se divide
en \textbf{cinemática} y \textbf{dinámica}. La mecánica estudia 
\textbf{mecanismos}.

\textbf{Fuerza.} Es una \textbf{interacción} entre dos cuerpos. Siempre refiere
a la fuerza que un cuerpo ejerce sobre otro. El cambio en el estado de 
\textbf{movimiento} es la evidencia de la fuerza. Si algo se mueve, 
hay algo haciendo fuerza. Una fuerza \textbf{acelera} un cuerpo. Y si no lo 
mueve lo deforma. 

\textbf{Características de la fuerza.} Es una magnitud \textbf{vectorial}. 
Como es un vector, aplicar las reglas de suma vectorial para considerar sus
efectos sobre un objeto.

\textbf{Clasificación de la fuerza.} De \textbf{contacto}. \textbf{A distancia}.

\textbf{Superposición de fuerzas.} El efecto de cualquier cantidad de fuerzas
aplicadas sobre un cuerpo es igual a la suma vectorial de fuerzas.

\textbf{Leyes de Newton.} Otros hablan de \textit{principios} de Newton. Son 
principios porque no se demuestran, sino que se postulan de manera similar a un
axioma. Son una descripción que resulta de la observación de cuerpos en 
movimiento. Son fundamentales porque no se demuestran ni se deducen. Son la base
de la \textbf{mecánica clásica}. No son universales: requiere modificaciones a 
velocidades muy altas (cercanas a la velocidad de la luz) o tamaños muy pequeños
(como en física cuántica).

\textbf{Primera ley. } Todo cuerpo permanece en su estado de reposo o movimiento
rectilíneo y uniforme, a menos que se le obligue a variar dicho estado 
mediante fuerzas que actúen sobre él. \textbf{Consecuencias.} Los objetos no 
pueden cambiar por sí mismos su estado. El cambio requiere \textbf{interacción}.
\textit{Sin} acción de \textbf{fuerzas} \textit{no hay} \textbf{aceleración}. 
\textbf{Reposo} y \textbf{movimiento rectilíneo uniforme} son dos estados 
equivalentes. 

\textbf{Inercia.} Tendencia natural de un objeto a mantener su estado de 
movimiento, sea \textbf{reposo} o \textbf{movimiento rectilíneo uniforme}. Una 
propiedad se relaciona con esta tendencia natural y es la \textbf{masa}. Cuanto
mayor es la \textbf{masa} de un cuerpo, mayor es su \textbf{inercia.}

\textbf{Masa.} Propiedad de un objeto que especifica su resistencia a cambiar su
velocidad. En el \textbf{sistema internacional} se usa como unidad el 
\textbf{kilogramo}. A mayor \textbf{masa}, mayor \textbf{fuerza} se necesita 
para el \textbf{movimiento}. La masa es una propiedad \textbf{extensiva}, porque
depende de la cantidad de \textbf{materia}, a diferencia de una propiedad 
\textbf{intensiva}, como el punto de fusión, que \textit{no} depende de la 
cantidad de \textbf{materia}.

\textbf{Segunda ley.} La \textbf{aceleración} de un objeto es 
\textit{proporcional} a la \textbf{fuerza} neta que actúa sobre él, e 
inversamente proporcional a su \textbf{masa}. La \textbf{dirección} de la 
la \textbf{aceleración} es igual a la de la fuerza neta. 
\(\vec{F}_{neta} = m\vec{a}\). Al ser de caracter \textbf{vectorial}, 
normalmente la usaremos en forma de componentes, con una ecuación para la fuerza
en \(y\) y otra para la fuerza en \(x\).

\textbf{Newton.} Unidad de fuerza. Se define como la fuerza necesaria para dar a
una masa de 1kg una aceleración de 1 metro por segundo. 
\(1N = 1kg \cdot \frac{1m}{s^2}\)

\textbf{Peso.} Magnitud relacionada a la de \textbf{masa}, pero no es lo mismo.
Peso es la magnitud de la \textbf{fuerza} gravitacional que ejerce la Tierra 
sobre un objeto. La fuerza de \textbf{gravedad} es una fuerza de 
\textbf{acción a distancia}. El peso varía de acuerdo a la \textbf{distancia} 
con el \textbf{centro} de la tierra.

\textbf{Aceleración en caída libre.} \(w = g\)

\textbf{Kilogramo fuerza.} Kilogramo solo refiere a kilogramo masa. Hay una 
unidad de fuerza que se llama \textbf{kilogramo fuerza} \(kgf\). Se usa en el
\textbf{sistema técnico}. Equivale al peso de una masa de \(1kg\) situada al 
nivel del mar. Aproximadamente \(10N\). Entonces: \(1kgf \approx 10N\).

\textbf{Partícula en equilibrio.} \(\sum\vec{F}_x = \sum\vec{F}_y = 0\).

\textbf{Diagrama de cuerpo libre.} Es necesario para resolver un problema de 
\textbf{dinámica}. \textit{Se evalúa saber hacerlo}. 



\end{document}
