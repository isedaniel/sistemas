\documentclass[12pt]{article}
\usepackage[a4paper, margin=2.54cm]{geometry}

% español
\usepackage[spanish]{babel}

% imágenes
%\usepackage{graphicx}
%\graphicspath{{img}}

% fuentes de conjuntos numéricos
\usepackage{amsfonts}

% símbolos
\usepackage{amsmath, amssymb}

% gráficos
%\usepackage{tikz}

% plots
%\usepackage{pgfplots}
%\pgfplotsset{width=10cm, compat=1.9}

% averiguar
\setlength{\jot}{8pt}
\setlength{\parindent}{0cm}

% espacio entre párrafos
\usepackage[skip=10pt plus1pt]{parskip}

% cancelar términos
\usepackage{cancel}

% links
%\usepackage[colorlinks=true, 
%    urlcolor=blue]{hyperref}

% shapes
%\usetikzlibrary{shapes.geometric}

% incluir pdfs
%\usepackage{pdfpages}

\title{Física I\\Apunte de clase}
\author{Daniel Ise}
\date{30 de septiembre de 2024}

\begin{document}

\maketitle

\tableofcontents

\section{Momento o cantidad de movimiento}

Vectorial, misma dirección y sentido que la velocidad.
\begin{align*}
    p = mv
\end{align*}

\section{Impulso}

Es la variación del momento.
\begin{align*}
    J = \Delta p \\
    J = F \Delta t \\
    J = \int_{t1}^{t2} \sum F dt
\end{align*}
\end{document}

