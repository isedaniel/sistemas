\documentclass{article}
\usepackage[margin=2.54cm]{geometry}
%\usepackage{graphicx}               % imágenes
%\graphicspath{{img}}
\usepackage{amsfonts}               % fuentes de conjuntos numéricos
\usepackage{amsmath, amssymb}       % símbolos
%\usepackage{tikz}                   % gráficos
%\usepackage{pgfplots}               % plots
%\pgfplotsset{width=10cm, compat=1.9}
\setlength{\jot}{8pt}
\setlength{\parindent}{0cm}
\usepackage{parskip}                % espacio entre párrafos
\usepackage{cancel}                 % cancelar términos
%\usepackage[colorlinks=true, 
%    urlcolor=blue]{hyperref}        % links
%\usetikzlibrary{shapes.geometric}   % shapes
%\usepackage{pdfpages}               % incluir pdfs

\title{Clase de Física I}
\author{Daniel Ise}
\date{29 de agosto de 2024}

\begin{document}

\maketitle

El ave vuela a rapidez \textbf{constante}. Qué tiene mayor \textbf{magnitud}, la
\textbf{gravedad} o la \textbf{fuerza ascendente del aire}. Al ser constante, la
\textbf{resultante de fuerzas} es 0. Si no sería 0, habría \textbf{aceleración}.

Un objeto se puede mover en ausencia de fuerza. \textbf{Verdadero}, se puede
mover a velocidad constante.

Una fuerza simple actúa, el objeto acelera. \textbf{Verdadero}, es una sola
fuerza.

\section*{Dinámica. Segunda parte.}

\begin{quote}
    Existen diferentes tipos de aplicaciones de fuerza y debes usar todos ellos.

    \raggedleft - Bruce Lee
\end{quote}

\textbf{Fuerzas de fricción.} Resistencia al movimiento por dos materiales en
contacto. Existe en todos los medios: \textbf{sólidos}, \textbf{líquidos} y
\textbf{gaseosos}. Tiene su \textbf{reacción}.

\textbf{3 tipos.} \textbf{Estática}. Cuando no hay movimiento relativo.
\textbf{Cinética}. Cuando hay movimiento. \textbf{Rodamiento.} Cuando una
superficie gira sobre otra. Ejemplo: empujo algo pesado y no se mueve:
rozamiento estático. Se empieza a mover y me cuesta: rozamiento dinámico o
deslizante.

\textbf{Magnitud de fuerza de fricción.} Hay proporcionalidad con el peso.
La \textbf{estática} se expresa como \(f_{s} \leq \mu_s N\). La estática es
menor o igual al \textbf{coeficiente de rozamiento} \((\mu)\) por \textbf{peso}.
Una vez que arranca a moverse, la fricción cambia a \textbf{cinética}:
\(f_k = \mu_k N\).

\textbf{Coeficiente de rozamiento.} Dependen de la naturaleza de las
superficies. En general, \(\mu_s > \mu_k\). No habría \(\mu > 1\).

\textbf{Ejemplo.} Disco de Hockey sobre estanque congelado. Rapidez inicial
\(20 \frac{m}{s}\). Se desliza 115 m y llega al resposo. Determine \(\mu\).

\begin{align*}
    f_k & = \mu \cdot mg                                        \\
    \mu & = \frac{f_k}{mg}                                      \\
    a   & = \frac{0 - 20}{115} \rightarrow a = \frac{-4}{23}    \\
    \mu & = \frac{\bcancel{m} \cdot -4}{23 \cdot \bcancel{m} g} \\
    & \boxed{\mu \approxeq 0.17}
\end{align*}

\textbf{Dinámica de la rotación.} Una partícula que se mueve a rapidez constante
\(v\) en trayectoria circular de radio \(r\), hay \textbf{aceleración normal} o
\textbf{centrípeta} de \(a_c = \frac{v^{2}}{r}\). Se explica con la 
\textbf{fuerza centrípeta}. Ejemplo: en una boleadora, la tensión de la cuerda 
obliga al cuerpo a tomar la curva.

\textbf{Fuerza centrífuga.} \textit{No existe} como tal. La \textbf{centrípeta} 
obliga al cuerpo a tomar la curva. Pero el cuerpo tiene tendencia natural 
(\textbf{inercia}) a moverse en línea recta. 

\textbf{Ejemplo.} Auto \(1500 kg\), se traslada sobre curva. Si el radio es 
\(35 m\), \(\mu\) es \(0.523\), encuentre rapidez máxima. 

\textbf{Aceleración centrípeta.} Se da cuando un móvil toma una curva.

\textbf{Peralte.} Ayuda a tomar la curva porque le da ángulo a la 
\textbf{normal}, que te tira hacia el centro.

\textbf{Ejemplo.} Auto \(1000 kg\), se traslada sobre curva a \(15 m/s\), radio 
\(50 m\), \(\mu\) es \(0.6\), ¿derrapa?.

\textbf{Ejemplo.} Misma curva, peraltada \(\theta\). Dada rapidez \(v\), 
determine peralte. 

\section*{Fuerzas fundamentales}

\textbf{Fuerza gravitacional.} Responsable de la caída de cuerpos. De largo 
alcance, sin relevancia a nivel atómico. Relevante con masas importantes.

\textbf{Fuerza electromagnética.} Responsable de interacción entre partículas 
con carga eléctrica. Por extensión, todas las reacciones químicas, por 
consiguiente fenómenos biológicos. Actúan sobre todas las partículas cargadas
eléctricamente. De naturaleza atractiva o repulsiva. 

\textbf{Fuerza nuclear fuerte.} Más fuerte que la electromagnética. Actúa dentro
del núcleo. Hace que no se desintegre. 

\end{document}
