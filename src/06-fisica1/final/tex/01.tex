\section{Magnitudes y Mediciones}

\subsection{La Física como ciencia experimental}

La \textbf{ciencia} es una \textit{actividad humana},
cuyo resultado es un \textit{cuerpo de conocimientos}.
Esta acumulación de conocimientos se acelera desde el siglo XVI,
cuando se empieza a \textit{utilizar la matemática para describir la naturaleza},
\textit{reduciendo la ambigüedad}
y facilitando la verificación (o refutación) de hipótesis
mediante la \textit{experimentación}.

Una clasificación de la ciencia puede hacerse distinguiendo
entre ciencias de la vida y ciencias físicas,
que estudiaría a los objetos sin vida.
Las ciencias de la vida se dividen en áreas tales como la zoología o la botánica,
y las ciencias físicas en física, geología, astronomía o química.
Dentro este último conjunto ubicamos a la física.

La física es una ciencia experimental,
que se aboca al estudio de fenómenos naturales que ocurren en el universo,
quizá la más fundamental de las ciencias.
Trabaja temas como el movimiento,
la fuerza, la energía, la materia, el calor, la luz.

Decimos que es una ciencia expermiental y se basa en el método científico,
puesto que parte de la formulación de hipótesis
que contrasta con pruebas objetivas para su verificación o refutación.

El \textbf{método científico}, en líneas generales,
parte de la observación de la naturaleza,
propone preguntas,
formula hipótesis,
deriva de ellas consecuencias observables,
y desarrolla experimentos -controlados y sistemáticos-
que pongan a prueba las hipótesis.
La compilación de hipótesis verificadas permite la elaboración de teorías,
que pueden a su vez realimentar el proceso:
promover la formulación de nuevas hipótesis,
que se contrasten en nuevos experimentos controlados.
Cuando una teoría se cumple de forma sistemática,
para un conjunto delimitado de fenómenos de la naturaleza,
se formula una ley,
que permitirá describir -y en algunos casos predecir-
el comportamiento del fenómeno en cuestión.

La física se ocupa de los principios fundamentales del universo,
puesto que es el cimiento de otras ciencias,
tales como la astronomía o la geología.
Es a su vez uno de los fundamentos de muchas disciplinas,
como la arquitectura, la medicina o las ingenierías.

La belleza de la física radica en su \textbf{simplicidad}:
asumiendo una cantidad pequeña de principios, conceptos y modelos,
puede ofrecer respuestas a fenómenos muy variados.

Se subdivide en 6 areas fundamentales:
\begin{enumerate}
    \item Mecánica clásica,
          que trabaja con objetos de gran magnitud en comparación con los átomos
          y que se desplazan a velocidades considerablemente inferiores a la de la luz.
    \item Termodinámica,
          que se centra en conceptos como el calor, el trabajo y el comportamiento
          estadístico, y suele trabajar con sistemas con gran cantidad de partículas.
    \item Electromagnetismo,
          que se centra en los fenómenos vinculados a la electricidad
          y los campos electromagnéticos.
    \item Óptica,
          que estudia la luz y sus interacciones con distintos materiales.
    \item Relatividad,
          objetos con cualquier rapidez,
          incluyendo aquellos que se acercan a la velocidad de la luz.
    \item Mecánica cuántica,
          vincula el comportamiento de a nivel macroscópico y con el submicroscópico.
\end{enumerate}

Se puede distinguir también la física clásica de la física moderna.
Mientras la física clásica incluye teorías, conceptos y leyes provenientes
de la mecánica clásica, la termodinámica,
la óptica y el electromagnetismo clásico,
la física moderna
-cuyo inicio se puede situar a fines del siglo XIX-
se aboca a fenómenos físicos que no pueden explicarse en el marco de la física clásica,
como los objetos que se aproximan a la velocidad de la luz
o el comportamiento submicroscópico.

En este curso de Física I estudiamos en particular la mecánica clásica.

\subsection{Mediciones y sistemas de unidades}

Para describir fenómenos naturales necesitamos hacer \textbf{mediciones}.
Una medición implica obtener una \textit{cantidad física},
como por ejemplo la longitud de un objeto.
Las mediciones cumplen un papel fundamental en la física:
sus \textbf{leyes físicas} se expresan como \textit{relaciones matemáticas}
entre \textit{cantidades físicas}.

Por dicho motivo,
es importante contar con un \textbf{estándar},
que permite que las unidades de medición sean reproducibles,
persistentes en el tiempo,
así como conocer la confiabilidad
y las limitaciones de las mediciones que hacemos.

En 1960 un comité internacional establece un conjunto de \textbf{estándares}
para las cantidades fundamentales de la ciencia.
Este estándar se conoce como \textbf{Sistema Internacional}
y establece las siguientes unidades fundamentales:

\vspace{.5cm}
\begin{table}[H]
    \centering
    \caption{Cantidades Fundamentales y Unidades}
    \vspace{.5cm}
    \begin{tabular}{ll}
        \hline
        Longitud              & Metro (m)      \\
        Masa                  & Kilogramo (Kg) \\
        Tiempo                & segundo (s)    \\
        Temperatura           & Kelvin (K)     \\
        Corriente             & Ampere (A)     \\
        Cantidad de sustancia & Mol            \\
        Intesidad lumínica    & Candela        \\
        \hline
    \end{tabular}
\end{table}
\vspace{.5cm}

En la materia vamos a usar el sistema internacional adoptado por el
Sistema Métrico Legal Argentino (SiMeLA).
Notar que,
por la temática del curso,
se usarán únicamente las unidades fundamentales de tiempo, masa y longitud.
El resto de unidades que veremos 
se expresan matemáticamente como una combinación de estas tres.

La \textbf{longitud} refiere a la distancia que existe entre dos puntos en el
espacio.
Su unidad de medida es el \textbf{metro},
cuya última redefinición en 1983 lo establece como
"la distancia recorrida por la luz en el vacío durante un tiempo de
\(1/299\,792\,458\) segundos".

La \textbf{masa} se relaciona con la cantidad de materia presente en un objeto,
o con la resistencia que este presenta frente a los cambios en su movimiento.
Es un propiedad \textit{inherente} al objeto e independiente del entorno.
La unidad fundamental de masa en el SI es el kilogramo
y refiere a "la masa de un cilindro de aleación de platino-iridio mantenido
en la Oficina Internacional de Pesos y Medidas de Sèvres, Francia".
Notar que, aunque se utilice la equivalencia \(1\,kg\approx2.2\,lb\),
el kilogramo es una unidad de masa y la libra una \textit{unidad de fuerza}.

El \textbf{tiempo} refiere a la \textit{duración y secuencia de los eventos}.
Su unidad de medida es el \textit{segundo} que se define como
"9\,162\,631\,770 veces el período de vibración de la radiación del átomo
cesio 133".

Además de las unidades,
utilizamos \textbf{prefijos},
como en \textit{nanosegundos} o \textit{milimetros}.
Estos funcionan como \textit{multiplicadores},
siendo los más comunes:

\vspace{.5cm}
\begin{table}[H]
    \centering
    \caption{Prefijos, abreviaturas y multiplicadores}
    \vspace{.5cm}
    \begin{tabular}{llr}
        \hline
        nano  & n       & \(\times10^{-9}\) \\
        micro & \(\mu\) & \(\times10^{-6}\) \\
        mili  & m       & \(\times10^{-3}\) \\
        centi & c       & \(\times10^{-2}\) \\
        deci  & d       & \(\times10^{-1}\) \\
        deca  & da      & \(\times10^{1}\)  \\
        hecto & h       & \(\times10^{2}\)  \\
        kilo  & k       & \(\times10^{3}\)  \\
        mega  & M       & \(\times10^{6}\)  \\
        giga  & G       & \(\times10^{9}\)  \\
        tera  & T       & \(\times10^{12}\) \\
        peta  & P       & \(\times10^{15}\) \\
        exa   & E       & \(\times10^{18}\) \\
        \hline
    \end{tabular}
\end{table}
\vspace{.5cm}

\subsection{Modelos y representaciones}

Un método de resolución de problemas en física consiste en plantear \textbf{modelos},
simplificaciones de la realidad,
que permiten resolver de forma relativamente simple problemas que podrían ser, 
de otra forma, demasiado complejos.
Son necesarios porque la operación concreta del universo 
es extremandamente compleja.
Su utilidad radica en su capacidad de ofrecer predicciones que,
de observarse en la realidad,
permiten concluir que es una aproximación \textit{válida} a la misma.

Se construyen aislando un fenómeno de la naturaleza,
considerando solo aquellos aspectos que tienen un efecto \textit{mesurable}
sobre el proceso que estamos estudiando.

Algunos de los principales modelos que usamos en Física I son: 
\begin{itemize}
    \item el modelo de \textbf{partícula}
    \item el modelo de \textbf{sistema}
    \item el modelo de \textbf{cuerpo rígido}
    \item y el modelo de \textbf{ondas}
\end{itemize}

Cuando veamos los problemas y aparezcan veremos las características de cada uno.

\subsection{Análisis dimensional}

Una \textbf{dimensión} denota la naturaleza física de una cantidad,
por ejemplo, 
no alcanza con decir que la distancia entre dos puntos es 3,
puesto que no es lo mismo que sean 3 metros o 3 pies.
Debemos indicar la dimensión.

El \textbf{análisis dimensional} señala que las unidades en las ecuaciones deben
de tratarse algebraicamente. 
Una implicancia de esto es que,
en una ecuación \textit{resuelta},
las unidades a ambos lados de la igualdad deben ser las mismas.
Que las unidades no sean coincidentes o no sean las esperadas es la indicación 
de un error,
de operación o de planteamiento.

Es por ello que,
por regla general,
\textbf{siempre} que operemos ecuaciones físicas debemos incluir las unidades.

\subsection{Conversión de unidades}

En relación al punto anterior,
a veces es necesario convertir unidades.
Por ejemplo,
si el problema se presenta en millas,
debemos pasarlo a la unidad del SI para medir longitudes,
esto es,
metros.

Entonces, si \(1 milla = 1\,609m\), 15 millas será:

\begin{equation*}
    d = 15 mi \cdot \frac{1\,609m}{1 mi} = 24\,135
\end{equation*}

Es decir,
para convertir de una unidad a otra simplemente multiplico por la 
\textbf{fracción de equivalencia}.

