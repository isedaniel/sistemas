\section{Cinemática de la partícula}

La rama de la física que estudia el movimiento,
sus causas y sus interacciones,
se conoce como \textbf{mecánica}.
La mecánica suele subdividirse,
a su vez,
en dos áreas:
\begin{enumerate}
    \item la cinemática 
    \item y la dinámica 
\end{enumerate}

La cinemática estudia el \textbf{movimiento} sin tener en cuenta sus causas ni sus
sus interacciones con el entorno.

La cinemática solo trabaja con longitudes.
No requiere considerar masas, fuerzas ni energías:
ese es trabajo de la dinámica.

En esta área de la mecánica usamos el \textbf{modelo partícula},
que considera al objeto como un \textit{punto} en un sistema de coordenadas,
puesto que otros aspectos del objeto,
-como sus dimensiones-
no son relevantes en este contexto.
Este modelo proporciona un punto de partida simple para el desarrollo teórico de 
la cinemática.

\subsection{Desplazamiento, velocidad y aceleración}

Para definir al concepto de movimiento,
necesitamos entender lo que es la \textbf{posición}.
Esta se define como la \textit{ubicación} de la \textit{partícula},
respecto a un \textbf{punto de referencia}
que funciona como el \textit{origen del sistema de coordenadas}.
Es una \textit{magnitud vectorial},
cuyo \textit{signo} cambia de acuerdo a si se encuentra de un lado u otro 
del sistema de referencia elegido.

Cuando hablamos de \textbf{distancia recorrida},
nos referimos a la longitud total de un trayecto por el cual se desplaza un 
objeto.
La \textit{distancia}, a diferencia de la posición, es una magnitud escalar,
nos indica solo una cantidad, pero \textit{no una dirección}.
Y con \textbf{trayectoria} nos referimos al conjunto de puntos que atraviesa 
un cuerpo, en un sistema de referencia considerado,
a lo largo de su trayecto.

Ahora,
cuando hablamos de \textbf{desplazamiento}
nos referimos al \textit{cambio en la posición} del objeto,
con \textit{independencia} de la trayectoria seguida.
Es una magnitud vectorial,
que puede ser negativa o positiva de acuerdo al sistema de referencia,
que se mide en \textit{metros} \((m)\),
como señala el SI,
y que viene dado por la expresión:

\begin{equation*}
    \Delta x = x_f - x_i
\end{equation*}

Notar que, si el \textit{sentido} del movimiento no cambia,
la distancia recorrida y el desplazamiento son iguales.

El \textbf{movimiento} es el \textit{cambio en la posición con el tiempo}.

Si relacionamos la \textit{distancia recorrida} y el \textit{tiempo},
nos referimos a la \textbf{rapidez},
que es un magnitud escalar,
y se expresa en \textit{metros por segundo} \((m/s)\).

\begin{equation*}
    s = \frac{d}{\Delta t}
\end{equation*}

Cuando hablamos de \textbf{rapidez media} nos referimos a la rapidez promedio
de un cuerpo atravesando una distancia determinado a lo largo de un intervalo de 
tiempo.

En cambio, la \textbf{rapidez instantánea} refiere al movimiento en un instante 
dado, y es la rapidez que muestra el velocímetro de un auto.

El concepto de \textbf{velocidad} refiere a un \textit{desplazamiento}
durante un período de tiempo determinado,
por lo cual es una \textit{magnitud vectorial}.
La unidad en que se mide es \textit{metros por segundo} \((m/s)\).

Velocidad media refiere al desplazamiento dividido por el tiempo:

\begin{equation*}
    \bar{v} = \frac{\Delta x}{\Delta t} = \frac{x_f - x_i}{t_f - t_i}
\end{equation*}

Como ya se ha señalado al hablar del desplazamiento,
\textit{no depende de la trayectoria},
sino que refiere al desplazamiento neto entre dos puntos.

El concepto de \textbf{velocidad instantánea}
refiere a la derivada de la velocidad evaluada en un punto,
es decir,
a la pendiente de la recta tangente en ese punto:

\begin{equation*}
    v_x = \lim_{\Delta t \to 0} \frac{\Delta x}{\Delta t}
\end{equation*}

De aquí en más,
cada vez que digamos \textit{media} o \textit{instantánea},
nos referimos a magnitudes vinculadas a un intervalo temporal.
Cuando decimos magnitud media,
nos referimos a su intensidad para un intervalo de tiempo determinado 
\(\Delta t\),
y cuando decimos magnitud instantánea,
nos referimos a su intensidad con ese intervalo tendiendo a 0 \(\Delta t \to 0\).

Así como la velocidad indica el cambio en la posición con respecto al tiempo,
la \textbf{aceleración} refiere a un cambio en la velocidad con respecto al tiempo.
Su unidad de medida es \textit{metros por segundo, por segundo},
es decir, \(v/s = m/s^{2}\).
Como se relaciona con la velocidad,
es naturalmente también una magnitud \textit{vectorial}.

La expresión de la aceleración media sería:

\begin{equation*}
    \bar{a} = \frac{\Delta v}{\Delta t}
\end{equation*}

Y la aceleración instantánea es:

\begin{equation*}
    a_x = \lim_{\Delta t \to 0}\frac{\Delta a}{\Delta t} = \frac{d\vec{v}}{dt}
\end{equation*}

Notar que,
como velocidad y aceleración son vectores,
pueden ser positivos o negativos,
de acuerdo al sentido que tengan en relación al sistema de referencia.

Si velocidad y aceleración tienen el mismo sentido,
el cuerpo tenderá a desplazarse cada vez más rápido.
Por el contrario,
si velocidad y aceleración tienen sentidos contrapuestos,
el cuerpo tenderá a bajar su velocidad.

De aquí en adelante vamos a ver algunos casos particulares de movimiento 
en una sola dimensión.

\subsection{Movimiento Rectilíneo Uniforme}

El movimiento rectilíneo uniforme es aquel que posee \textbf{aceleración nula},
lo que \textit{implica} que el cuerpo se mueve con \textbf{velocidad constante} 
y en \textbf{línea recta}.

\begin{align*}
    a = 0 \implies v = v_0 \quad\land\quad x = x_0 + v_0 \cdot t
\end{align*}

Como la velocidad no cambia:

\begin{align*}
    \bar{v} = v_x
\end{align*}

\subsection{Movimiento Rectilíneo Uniformemente Variado}

En este caso hablamos de un movimiento con trayectoria recta y aceleración 
constante.

\begin{align*}
    a = a_0 \implies v = a_0 \cdot t + v_0 \quad x = \frac{1}{2}a_{0}t^{2} + v_{0}t + x_0
\end{align*}

\subsection{Caída libre}

Un ejemplo de movimiento rectilíneo con aceleración -prácticamente- constante es 
la caída de un cuerpo en dirección a la Tierra por su atracción gravitacional.
Si omitimos el efecto del aire,
todo cuerpo en un lugar específico cae con la misma aceleración,
con independencia de su peso.
Si además su distancia de caída es pequeña en comparación al radio terrestre,
e ignoramos los efectos debidos a la rotación de la tierra,
la aceleración es constante.
El modelo que surge de estos supuestos se denomina \textbf{caída libre}.

La aceleración debida al efecto de la gravedad se simboliza con \(g\).
El valor aproximado cerca de la superficie terrestre es \(g = 9.8 m/s^{2}\).

\subsection{Cinemática en dos dimensiones}

Hasta aquí hemos trabajado con el movimiento en una sola dimensión,
en línea recta.
Para considerar el movimiento en dos o más dimensiones hay que tener en cuenta 
el caracter vectorial de la posición.
Esto implica describir el movimiento siempre en cada uno de sus componentes,
como si fueran movimientos independientes.

El vector posición definido en dos dimensiones será entonces la sumatoria de la 
componente horizontal y la componente vertical:

\begin{equation*}
    \vec{r} = x\hat{i} + y\hat{j}
\end{equation*}

La magnitud del vector desplazamiento viene dada por la expresión:

\begin{equation*}
    |\vec{r}| = \sqrt{x^{2} + y^{2}}
\end{equation*}

El vector desplazamiento tiene origen en la posición inicial y extremo en la 
posición final.

La \textbf{velocidad media} es el cociente entre el vector desplazamiento y el 
tiempo:

\begin{equation*}
    \vec{v} = \frac{\Delta\vec{r}}{\Delta t}
\end{equation*}

Su módulo es el módulo del desplazamiento dividido por el tiempo,
y su dirección y sentido son iguales a los del vector desplazamiento.

La \textbf{aceleración media} que es el cambio en el vector velocidad
en un intervalo de tiempo.

En general,
el vector velocidad puede cambiar en \textbf{módulo} y \textbf{dirección}.
En cada punto de la trayectoria de un cuerpo podemos distinguir un componente 
\textit{tangencial} a la trayectoria y un componente \textit{perpendicular} a la
misma.

El componente tangencial de la aceleración se denomina 
\textbf{aceleración tangencial},
y es reponsable por el cambio en el \textit{módulo de la velocidad}.
El componente perpendicular se denomina \textbf{aceleración normal},
o \textit{centrípeta},
apunta hacia el centro de la curva y es la responsable por los cambios en la 
dirección de la velocidad.
El módulo del vector aceleración sigue la expresión:

\begin{equation*}
    |\vec{a}| = \sqrt{a_t^{2} + a_n^{2}}
\end{equation*}

Además, el módulo de la aceleración tangencial es:

\begin{equation*}
    |a_t| = \frac{d|\vec{v}|}{dt}
\end{equation*}

Y el módulo de la aceleración normal es:

\begin{equation*}
    |a_n| = \frac{|\vec{v}|^{2}}{R}
\end{equation*}

Notar que,
cuando la aceleración tangencial es igual a 0,
podemos concluir que el módulo de la velocidad se mantiene constante,
esto es,
el cuerpo se encuentra en movimiento \textit{uniforme}.
Por otra parte,
si la aceleración normal es igual a 0,
la dirección del vector velocidad permanece constante,
por lo que el cuerpo se mueve de forma \textit{rectilínea}.

La aceleración en dos dimensiones funciona como 
un eje de coordenadas intrinseco.

\subsection{Composición de movimientos}

Llamamos composición de movimientos a la combinación en un mismo objeto de 
dos movimientos simultáneos e independientes.
Por ejemplo, un nadador atravesando un río:
tiene un movimiento en la dirección en que esta nadando,
y además otro movimiento en la dirección en que el río lo empuja.

\subsection{Tiro oblicuo}

El tiro oblicuo es, esencialmente, un movimiento compuesto.
Refiere al lanzamiento de un objeto,
con una velocidad inicial \(v_0\),
a un ángulo \(\theta\) con respecto al suelo.

Las suposiciones sobre las que se para el tiro oblicuo son dos:
la aceleración debida a la gravedad permanece constante para todo el trayecto;
y despreciamos el efecto del rozamiento debido al aire.

Con ambas suposiciones,
entendemos que el tiro oblicuo es un MRU sobre el eje horizontal \(x\),
y un MRUV respecto del eje vertical \(y\).
La composición de ambos movimientos resulta en una parábola.

\vspace{.5cm}
\begin{center}
    \begin{tikzpicture}
        \begin{axis}[
                xlabel=x,
                ylabel=y,
                axis lines=left,
            ]
            \addplot[domain=0:9, samples=100] {34/21*x - 8/45*x*x};
        \end{axis}
    \end{tikzpicture}
\end{center}
\vspace{.5cm}

Las expresiones para la componente en \(x\): 

\begin{align*}
    a_x = 0 \qquad v_x = v_0 \cdot \cos\theta \qquad x = v_x \cdot t 
\end{align*}

Por su parte, la componente en \(y\):

\begin{align*}
    a_y = -g \qquad v_y = v_0 \cdot \sen\theta - g \cdot t \qquad y = v_y \cdot t - \frac{1}{2}g \cdot t^{2}
\end{align*}

Para determinar el \textbf{momento de altura máxima} buscamos el \(t\) 
donde \(v_y\) se hace 0:

\begin{align*}
    v_0 \cdot \sen\theta - g \cdot t = 0 \\
    t = \frac{v_0 \cdot \sen\theta}{g}
\end{align*}

Si sustituimos en la expresión para \(y\) llegamos a una expresión para 
determinar rapidamente la \textbf{altura máxima}: 

\begin{align*}
    y_{max} = \frac{v_y^{2}}{2g}
\end{align*}

Por otra parte,
si queremos saber el \textbf{alcance} en \(x\),
suponiendo que aterriza a una altura igual a la altura de partida,
podemos estimar que el tiempo total de vuelo es el \textit{tiempo de altura máxima por 2},
sustituyendo en \(x\):

\begin{align*}
    x_{max} = \frac{2V_0^{2}\cos\theta\sen\theta}{g}
\end{align*}

\subsection{Velocidad relativa}

El concepto de velocidad relativa vincula la medición de velocidad desde dos 
observadores que se mueven uno respecto del otro.

Dados dos observadores que miden su distancia respecto a un punto \(P\),
podemos deducir que la posición del 1 es la sumatoria de la distancia con respecto 
al 2 y al punto \(P\):

\begin{align*}
    x_{1P} = x_2 + x_{2P}
\end{align*}

Si esas son las distancias, entonces las velocidades serán:

\begin{align*}
    v_{1P} = v_{2} + v_{2P}
\end{align*}

Si lo consideramos ahora en dos dimensiones,
llegamos a la \textbf{transformación galileana}:

Sea \(T\) un observador fijo respecto los móviles \(A\) y \(B\),
tenemos:
\begin{enumerate}
    \item \(\vec{r}_{AT}\): la posición del móvil A,
    visto desde T;
    \item \(\vec{r}_{BT}\): la posición de B visto desde T;
    \item \(\vec{r}_{AB}\) la posición de A visto desde B.
\end{enumerate}

La posición de A visto desde B es la ressta de vectores \(\vec{r}_{AB} = \vec{r}_{AT} - \vec{r}_{BT}\),
si derivamos llegamos a la expresión de las velocidades que es 
\(\vec{v}_{AB} = \vec{v}_{AT} - \vec{v}_{BT}\).

