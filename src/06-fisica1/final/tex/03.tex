\section{Dinámica de la partícula}

Hasta ahora hemos visto cómo podemos describir el movimiento,
desde el punto de vista de  la cinemática,
empleando conceptos como la posición, la velocidad y la aceleración.

En este capítulo vamos a preguntarnos por las \textit{causas del  movimiento}.
Y para dar respuesta a esas preguntas vamos a recurrir a nuevos conceptos,
como son \textbf{masa} y \textbf{fuerza},
y vamos a relacionarlos siguiendo las \textbf{leyes de Newton}.

\subsection{Fuerza}

La fuerza se puede definir como una interacción entre dos cuerpos,
cuya evidencia observable es un \textit{cambio en el movimiento}\footnote{Aunque una fuerza no equilibrada podría deformar un objeto.}.
Es decir, es capaz de producir una \textit{aceleración} y,
como esta, es una \textit{magnitud vectorial}.

Varias fuerzas pueden actuar sobre un mismo cuerpo.
Cuando esto sucede, debemos concentrarnos en su \textit{efecto combinado},
la \textbf{fuerza neta}.
Esto se conoce como \textbf{principio de superposición de las fuerzas}.
Notar que, al tratarse de una magnitud vectorial,
debemos seguir las reglas propias de las sumas de vectores.

La unidad del SI para medir la fuerza es el \textit{newton} \((N)\),
que equivale a la fuerza necesaria para acelerar 1 \(kg\) de masa 
1 \(m/s^{2}\).

La \textit{libra} (\(lb\)) es otra unidad usada para medir fuerzas,
equivale a \(1N = 4.448lb\).
El sistema técnico usa el \textit{kilogramo-fuerza} (\(kgf\)),
que equivale al peso de un kilogramo situado al nivel del mar,
es decir, \(1kg \cdot 9.81m/s^{2} = 1kgf = 9.81N\).

Las fuerzas pueden clasificarse en:
\begin{enumerate}
    \item fuerzas de \textbf{contacto}, que son aquellas que 
    surgen del contacto físico entre los objetos; y
    \item fuerzas de \textbf{acción a distancia}, 
    como la gravedad y el electromagnetismo,
    que no requieren del contacto. 
\end{enumerate}

\subsection{Leyes de Newton}

Las leyes de Newton son descripciones matemáticas que se desprenden de la 
observación de cuerpos en movimiento.
Son \textit{fundamentales}, en la medida en que no pueden deducirse ni 
demostrarse.
Antes bien,
funcionan como una \textit{axioma} en matemática,
un principio que no se demuestra sino que se da por válido y permite entender 
otros fenómenos de la naturaleza.

Las leyes de Newton son la base de la mecánica clásica,
aunque tienen ciertas limitaciones:
require de modificaciones para velocidades cercanas a la de la luz,
o para tamaños muy pequeños como el atómico.

\subsection{Primera ley de Newton: ley de la inercia}

La primera ley de Newton se puede enunciar como:

\blockquote{
    Todo cuerpo permanece en su estado 
    -que puede ser de reposo o movimiento rectilíneo uniforme- 
    a menos que una fuerza neta lo obligue a cambiar.}

Es decir, cuando la fuerza neta sobre un objeto es 0,
su aceleración es 0.

\begin{equation*}
    \sum \vec{F} = 0 \implies \vec{a} = 0
\end{equation*}

Algunos corolarios que se desprenden de esta ley:
\begin{enumerate}
    \item un objeto no puede cambiar por si solo su estado 
    \item un cambio de estado requiere de la interacción de dos objetos 
    \item el reposo \((v = 0)\) y el mru \(v = k\) son estados equivalentes 
\end{enumerate}

El puntapie inicial para la primera ley de Newton,
también conocida como \textbf{ley de la inercia},
lo da Galileo, en el siglo XVII,
quien postula que la naturaleza de un cuerpo no es el reposo,
sino \textit{resistir el cambio en su movimiento}.

Llamamos \textbf{inercia} a la tendencia natural que tiene un cuerpo a 
mantener su estado y a resistir al cambio.
La inercia se relaciona directamente con el concepto de \textbf{masa},
que vemos a continuación.

\subsection{Masa}

La masa se puede definir como la cantidad de resistencia que muestra un objeto a 
un cambio en su velocidad.
Es una propiedad inherente a un objeto, independiente del entorno.
Es una magnitud \textit{escalar}, que el SI se mide en kilogramos \((kg)\).

La masa es una \textit{medida cuantitativa} de la inercia:
a mayor masa, 
mayor la resistencia frente al cambio en el movimiento que presenta un cuerpo.

La forma más simple de comparar masas es medir pesos en el mismo lugar:
mismo peso en mismo lugar es misma masa.

\subsection{Segunda ley de Newton: }

La segunda ley relaciona la fuerza, la aceleración y la masa:

\blockquote{
    La aceleración de un objeto es proporcional a la fuerza neta,
    e inversamente proporcional a la masa.
}

Es decir:

\begin{equation*}
    \vec{F} = m\vec{a}
\end{equation*}

Hay que tener en cuenta algunas consideraciones:
\begin{enumerate}
    \item la ley implica la interacción entre dos cuerpos:
    sino yo podría acelerarme a mi mismo estirándome de la ropa
    \item la relación es válida para masas constantes:
    no es válida, por ejemplo,
    para un cohete que se impulsa quemando combustible y perdiendo masa.
    \item Además,
    requiere de un sistema de referencia inercial, 
    no es válida para vehículos con aceleración.
    \item Si la fuerza es variable, la ley se cumple para fuerza y acelaración 
    instantánea, haciendo obligatorio el uso de cálculo.
\end{enumerate}

En muchos problemas tocará considerar sumatoria de fuerzas y aceleraciones 
como componentes:

\begin{align*}
    \sum F_x = m \cdot a_x \qquad \sum F_y = m \cdot a_y
\end{align*}

Naturalmente, si la aceleración es 0 para cualquier componente,
la fuerza neta para dicho componente es 0.

\subsection{Diagrama de cuerpo libre}

Un diagrama de cuerpo libre es una representación esquemática de la acción 
de distintas fuerzas, considerando sus ángulos.

Por lo general es buena práctica escoger como eje la dirección de una de las 
fuerzas, de manera tal que la descomposición en componentes en \(xy\) sea más 
sencilla.

La metodología es simple:
\begin{enumerate}
    \item Se hace un diagrama identificando las fuerzas
    \item Se aísla el cuerpo para el cual construimos el diagrama
    \item Dibujamos vectores desde el origen, incluyendo ángulos
    \item Descomponemos en \(x\) y \(y\) las componentes necesarias y operamos 
\end{enumerate}

\subsection{Tercera ley de Newton: ley de acción y reacción}

La tercera ley establece que:

\blockquote{
    Si un objeto \(A\) ejerce fuerza sobre \(B\),
    \(B\) ejercerá una fuerza igual pero en sentido contrario sobre \(A\).
}

Las fuerzas se dan en pares, de manera simultánea,
con cada elemento ejerciendo fuerza sobre un \textit{cuerpo distinto}.
Un conjunto de fuerzas sobre el mismo cuerpo \textit{nunca} puede ser un 
par de acción-reacción.

\subsection{Fuerzas de fricción}

Refiere a la \textit{resistencia al movimiento}
que se da en el contacto entre dos materiales.
Esta resistencia existe para todo tipo de medios: sólidos, líquidos, gaseosos.

La fricción entre sólidos se clasifica generalmente en tres tipos:
\begin{enumerate}
    \item Fricción estática (\(f_s\))
    \item Fricción cinética (\(f_k\))
    \item Fricción rodante
\end{enumerate}

La fricción estática abarca aquellos casos en que la fuerza de fricción 
\textit{impide} el movimiento,
la fricción cinética incluye los casos en que la fuerza de fricción 
\textit{dificulta} un movimiento que se está realizando;
y la rodante es la fuerza de fricción de una superficie que gira sobre otra.
En esta materia nos vamos a concentrar en las dos primeras.

Ambas fuerzas de fricción dependen tanto de las dos superficies en contacto,
como de la \textit{carga} -que, como sabemos, es igual a la normal,
puesto que son un par de acción y reacción.

La fricción estática se maximiza justo antes de que un objeto comience a 
deslizarse, siendo su expresión:

\begin{equation*}
    f_s = \mu_s \cdot N = \mu_s \cdot mg
\end{equation*}

Una vez que el cuerpo comienza a moverse sobre la superficie,
la fricción se convierte en dinámica, que sigue una expresión parecida,
aunque con otro coeficiente de fricción:

\begin{equation*}
    f_k = \mu_k \cdot N = \mu_k \cdot mg
\end{equation*}

Los valores de los coeficientes dependen de la naturaleza de las superficies,
aunque suelen variar entre \(0.03\) y \(1\).
La dirección de una fuerza de fricción es paralela a la superficie de contacto,
en sentido opuesto al movimiento, sea este real o inminente.

\subsection{Dinámica de la rotación}

Cuando veíamos Movimiento Circular Uniforme,
veiamos que un móvil a rapidez constante \(v\),
con trayectoria circular de radio \(r\),
tenía una aceleración centrípeta que seguía la expresión \(a_c = v^{2}/r\).

Ahora que conocemos las leyes de Newton,
estamos en condiciones de afirmar que,
si existe una aceleración centrípeta,
es porque existe una \textbf{fuerza centrípeta}.

\begin{equation*}
    f_c = a_c \cdot m = m \cdot \frac{v^{2}}{r}
\end{equation*}

Ambas comparten dirección y sentido: hacia el centro de la circunferencia.
Como se trata de un MCU,
sabemos que la aceleración es constante,
por lo tanto la fuerza neta sobre el móvil también es constante.

Ojo con la "fuerza centrífuga".
Cuando vamos arriba de un automóvil,
lo que sentimos y erróneamente denominamos fuerza centrífuga es en 
realidad la inercia de nuestro propio cuerpo, 
que tiende a seguir en línea recta.
La presión del asiento y la carrocería hacen que nuestro cuerpo no siga en 
línea recta,
sino que acompañe el giro del auto.
Hay una sola fuerza, y es la centrípeta,
que corrige nuestra dirección para que hagamos la trayectoria circular.

Hay algunos problemas interesantes que aparecen en la fricción y el 
movimiento circular,
que se vinculan con el peralte de la ruta.

Por ejemplo, si un automovil circula por un camino plano,
la aceleración centrípeta máxima que puede ejercer para que no derrape 
es igual a la fuerza de fricción entre el auto y la ruta:

\begin{equation*}
    f_s = a_c \cdot m
\end{equation*}

Si hay peralte, la normal inclinada colabora con la fricción,
otorgando seguridad a mayores velocidades y en condiciones de menor fricción
como una ruta congelada.

\subsection{Movimiento circular no uniforme}

Si una partícula se mueve en trayectoria circular,
con rapidez variable,
existen dos componentes en la aceleración:
\begin{itemize}
    \item aceleración radial
    \item aceleración tangencial
\end{itemize}

En consecuencia la fuerza también tendrá un componente radial y uno tangencial.

\subsection{Las 4 fuerzas de la naturaleza}

Todas las fuerzas del universo se corresponden con una de estas 4:
\begin{enumerate}
    \item Fuerza gravitacional
    \item Fuerza electromagnética
    \item Fuerza nuclear fuerte
    \item Fuerza nuclear débil
\end{enumerate}

En este curso de física solo interesarán las dos primeras.

\subsection{Fuerza gravitacional}

Es la responsable de la caída de los cuerpos hacia la Tierra y de los movimientos 
a gran escala en el Universo.

Es siempre atractiva y su alcance es infinito,
aunque es inversamente proporcional al cuadrado de la distancia 
-al igual que la fuerza electromagnética.
Siempre produce atracción, con independencia de la composición de los cuerpos,
atrayendo el centro de gravedad de un objeto hacia el centro de otro.

Aunque es universal,
su fuerza a nivel atómico es muy baja,
por la pequeñisima magnitud de la masa en este nivel.

\subsection{Fuerza electromagnética}

Es responsable de la interacción entre partículas cargadas,
y por extensión, de todas las reacciones químicas.
Las fuerzas de contacto, que observamos y percibimos a nuestra escala,
son en esencia interacciones electromagnéticas entre las nubes de electrones
que conforman los materiales.

Las fuerzas eléctricas y magnéticas pueden ser de atracción y repulsión,
en constraste con la gravedad.
Pero disminuye su intensidad en la inversa del cuadrado de la distancia,
al igual que la gravedad.

También es la encargada de las uniones atómicas y moleculares,
por lo que es fundamental para la química.
Es tan dominante en este nivel que las otras tres fuerzas pueden ser 
consideradas insignificantes en la determinación de la estructura atómica y 
molecular.

\subsection{Fuerza nuclear fuerte}

Tiene un alcance menor al de las interacciones eléctricas,
pero una intensidad mucho mayor.
Juega un papel fundamental en las reacciones termonucleares en el núcleo del sol.

Como las cargas eléctricas iguales se repelen,
el nucleo atómico no podría existir.
La fuerza nuclear fuerte es la que mantiene unidos a los protones y neutrones,
siendo un centenar de veces más fuerte que la fuerza electromagnética.
Aunque los neutrones no poseen carga,
están sometidos a la fuerza nuclear fuerte.

\subsection{Fuerza nuclear débil}

Actúa entre partículas elementales,
siendo responsable de la desintegración beta.
Es una fuerza de corto alcance, 
del orden de \(10^{-13} mm\),
y además de baja intensidad:
si la fuerza nuclear débil tiene una intensidad de 1,
la fuerza electromagnética tendría un valor de \(1\,000\)
y la fuerte \(100\,000\,000\,000\).

Es importante, puesto que permite por ejemplo, la producción de luz solar.