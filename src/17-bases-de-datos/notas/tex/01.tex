\section{Primera clase, 16 de agosto}

\subsection{Sistema de Ficheros}
\begin{itemize}
    \item Surgen con la informatización
    \item Sigue un modelo descentralizado: cada area maneja su información
    \item Estos programas son independientes, enfocado a mantener ficheros
          y generar informes propios
    \item La falta de sincronización genera problemas:
          \begin{itemize}
              \item Se duplica información
              \item No se sincroniza
              \item Actualizar todos los ficheros lleva un costo
              \item Puede fallar (con mayor facilidad)
              \item Incompatibilidad entre formatos
              \item Rigidez en consultas, no flexible
          \end{itemize}
\end{itemize}

\subsection{Bases de datos}
\begin{itemize}
    \item Es un conjunto de datos
    \item Que modeliza hechos y objetos de una parte de la realidad
    \item Sirve de soporte a aplicaciones informáticas
    \item Debe estar almacenada físicamente en un \textbf{soporte informático}
    \item Deben relacionarse entre sí siguiendo una estructura lógica
    \item Nacieron para solucionar las limitaciones de los ficheros
          \begin{itemize}
              \item Consiguen independencia de datos: 
              no importa si consulta contaduría o compras:
              el proveedor es el proveedor y contamos con información 
              sincronizada y suficiente para resolver el problema concreto
          \end{itemize}
    \item Contiene datos de toda la organización 
    \item Los datos son coherentes lógicamente: 
    una amalgama de datos no es una base de datos
    \item También almacena \textbf{metadatos} 
    registrada en un diccionario de datos o \textbf{catálogo},
    permitiendo independencia entre lógica y física de datos
\end{itemize}

\subsection{Sistemas de gestión de bases de datos}
\begin{itemize}
    \item Ejemplos:
    \item SQLite
    \item MySQL
    \item Postgre
\end{itemize}

\subsection{Sistema de gestión de bases de datos}
\begin{itemize}
    \item Proporciona sistema de seguridad para usuarios
    \item Cuida la integridad de los datos: evitar que se corrompan
    \item Permitir la concurrencia: acceso simultáneo compartido
    \item Contar con sistema de recuperación que reestablece la BD frente a un problema de infraestructura
    \item Contar con diccionario o catálogo, que contiene descripción de los datos
\end{itemize}

\subsection{SQL}
\begin{itemize}
    \item Es un lenguaje estructurado para consultas 
    \item Se usa para gestionar y mantener bases de datos 
    \item Hay un estándar ANSI SQL que todo gestor debería cumplir 
    \item Pero no es limitante: los servicios implementan características 
    adicionales
    \item Si tengo una BD que cumple el estándar se reducen los costos de migración
\end{itemize}

\subsection{Ventajas de bases de datos}
\begin{itemize}
    \item Controla la redundancia: sin duplicidad
    \item Controla la consistencia: sin incoherencia
    \item Comparte los datos entre diversas apliaciones cliente
    \item Mantiene estándares 
    \item Mejora la integridad 
    \item Mejora la recolección y mantenimiento de datos 
    \item Todo esto redunda en mejora de productividad de la organización 
    \item Aumenta la posibilidad de concurrencia 
    \item Sirve para hacer copias de seguridad y recuperación ante fallos 
    \item Mejora la accesibilidad de los datos 
    \item Facilita el mantenimiento 
    \item Mejora la seguridad de los datos, permitiendo administración de permisos
\end{itemize}

\subsection{Inconvenientes de las bases de datos}
\begin{itemize}
    \item Costo de inversión: al principio puede ser dificil o costoso
    \item Mayor complejidad: requiere esfuerzo de capacitación en la organización,
    aunque existen implementaciones de complejidad variable
    \item Gran tamaño: puede llegar a ser exigente en infraestructura,
    aunque existen gestores más ligeros
\end{itemize}

\subsection{Historia de las bases de datos}
\begin{itemize}
    \item Arranca con tarjetas perforadas
    \item Primera generación: se almacenaba en cintas magnéticas,
    luego surgen discos duros
    \item Durante 50 al 70 predominan modelo jerárquico y modelo de red
    \item El disco rígido guarda archivos fragmentados, 
    por lo que era más rápido que la cinta
    \item \textbf{Modelo jerárquico}: se basa en el modelo de árbol,
    tiene un nodo raíz, que puede tener varios nodos hijos
    \item Útiles para aplicaciones que manejaban grandes volúmenes de información 
    \item Limitación: la redundancia de datos
    \item \textbf{Modelo de red:} usa grafos para representar información 
    \item Permite relaciones complejas entre datos 
    \item Supera las limitaciones del modelo jerárquico 
    \item La ventaja es que solucionaba la redundancia de datos 
    \item La desventaja es que incrementaba la complejidad de administrar datos 
    \item \textbf{Segunda generación:} Edgar Codd en 1970 sienta la base teórica
    del modelo de bases de datos.
    \item En 1976 Peten Chen, basado sobre el trabajo de Codd, crea el 
    Modelo de Entidad-Relación (MER).
    \item Sobre la base de Codd, otro grupo de IBM crea SEQUEL, que pasaría a 
    llamarse \textbf{SQL}
    \item Elliso, también basado en Codd, desarrolla para la CIA un RDBMS llamado
    Oracle, padre de los SQL 
    \item \textbf{Tercera generación:} evolución del modelo relacional,
    agregando solidez en BD orientadas a objetos, extensión de características,
    procesamiento distribuido
    \item Permite bases de datos distribuidas en distintos nodos,
    facilitando concurrencias
    \item BD activas, para aplicaciones críticas que responden automáticamente a 
    circunstancias
    \item BD deductivas, mezcla de BD con IA, permite deducir información 
    a partir de la introducida por el usuario 
    \item Web puede pensarse como una interfaz de acceso a BD, 
    proporcionan acceso a datos a través de \textbf{XML} o \textbf{JSON}
    \item En sentido \textit{NO ESTRICTO} se puede describir a la web como una 
    gran base de datos
    \item \textbf{Data warehouse}: sistema centralizado, 
    que soporta online analytical process (OLAP), 
    permitiendo hacer consultas en línea para toma de decisiones,
    antecesor del Big Data.
    \item \textbf{Data Mining}: procesos para descubrir patrones, tendencias o 
    relaciones significativas en grandes conjuntos de datos,
    para convertirlos en información útil para la toma de decisiones 
    \item ¿Está surgiendo una cuarta generación?
    \item Aparición de BD NoSQL: (Not Only SQL) Soporta SQL, pero tiene otras cosas 
    \item Big Data: procesamiento de volumen, velocidad, variedad y valor
\end{itemize}