\section{Segunda clase, 23 de agosto}

\subsection{Introducción a las BD}
\begin{itemize}
    \item Ficheros
    \item Biblioteca
    \item Guía telefónica
    \item Los tres son ejemplos de BD: 
    guardan información que vamos a poder consultar
\end{itemize}

\subsection{Sistemas de Gestión de Bases de Datos}
\begin{itemize}
    \item Jerárquico, con estructura tipo árbol,
    con nodo padre, nodos hijos
    \item Modelo rígido, que generaba duplicidad de datos 
    \item Modelo de red, implementa grafos, reduce el volumen de información,
    pero lo vuelve difícil de mantener, podía convertirse en una maraña de datos
    \item Sobre la base de eso Codd hace el \textbf{modelo de bases de datos 
    relacionales}, donde establece:
    \begin{itemize}
        \item Entidades, una por tabla 
        \item Cada tabla tiene: clave primaria, una identificación única para un 
        registro; clave foránea, que apunta a otra entidad en otra tabla
        \item Toda entidad tiene una \textbf{primary key} (PK)
        \item la \textbf{Foreign Key} (FK) es un campo que referencia a otra 
        entidad, en otra tabla
    \end{itemize}
\end{itemize}

\subsection{Conceptos importantes}
\begin{itemize}
    \item Entidad: algo de interés para el usuario de la BD
    \item Columna: es una pieza de dato almacenada 
    \item Registro: un conjunto de columnas que describen a una entidad o una 
    acción de una entidad 
    \item Tabla: conjunto de registros, conservada en memoria (no persistente)
    o almacenamiento (persistente)
    \item Result set: table no persistente, generalmente resulta de una consulta 
    SQL 
    \item Primary Key: columna que sirve de identificador único 
    \item Foreign Key: columna que sirve para identificar un registro en otra tabla
\end{itemize}

\subsection{SQL}
\begin{itemize}
    \item Codd propuso, DSL/Alpha se arma en IBM, SEQUEL
    \item ANSI hace el estándar en 1986 (ANSI: instituto estadounidense de estándares)
    \item En décadas posteriores se agregan otras características
    \item Vamos a ver MySQL, que tiene doble licencia, gratis para aprender
    \item Tenemos el motor de la base (servidor), y la interfaz (cliente)
\end{itemize}

\subsection{Modelado de Datos}
\begin{itemize}
    \item Forma de estructurar y organizar datos 
    \item Objetivos:
    \item Definir la estructura, la organización de los datos 
    \item Definir sus relaciones, cómo se conectan entre sí 
    \item Ayuda la comunicación, con XML y JSON
    \item Ayuda a documentar datos en procesos ETL, de actualización de BD,
    por ejemplo, traer ventas de otra surcursal, puede ser masivo,
    se genera en otro sistema y se trae en un formato, por ejemplo, JSON 
\end{itemize}

\subsection{Pasos del modelado}
\begin{enumerate}
    \item Mundo Real 
    \item Modelo de Entidad - Relación 
    \item Modelo Relacional, la  implementación lógica del MER 
\end{enumerate}

\subsection{Relación}
\begin{itemize}
    \item Asociación entre diferentes entidades 
    \item Puede ser reflexiva, bianaria, n-aria
    \item Reflexiva: se relaciona consigo misma 
    \item Ejemplo: una persona trabaja para otra. 
    \item Las dos entidades son iguales, aunque sean distintos registros 
    \item Binaria: vincula 2 entidades con una relación única de los 2 lados
    \item Ejemplo: yo tengo un auto, una relación 1 a 1, a través de relación
    poseer 
    \item N-aria: relación donde intervienen más de 2 entidades
    \item Ejemplo: periodista, artículo, y periódico
    \item Un periodista escribe muchos artículos, 
    publicados en distintos periódicos
\end{itemize}

Conceptualmente, tres tipos de relaciones:
\begin{itemize}
    \item Uno a uno: un registro de entidad A con un registro de entidad B 
    \item Uno a muchos: un registro de entidad A se relaciona con 0 o más registros 
    de entidad B, ejemplo: persona auto, tiene 0 o más autos 
    \item Muchos a muchos: un registro de A relaciona con 0 o más de B, a su vez,
    el registro de B se relaciona con uno o más de B 
\end{itemize}

\subsection{Modelo de Entidad Relación}
\begin{itemize}
    \item Es un modelo gráfico, un diagrama
    \item Representa a una entidad del mundo real 
    \item Tiene un nombre y \textbf{atributos} que corresponden a la entidad 
    \item Ejemplo: Entidad Persona, atributos nombre, apellido, género 
    \item El atributo principal permite identificar de maner unívoca a la entidad 
    \item Relaciones permiten identificar \textit{vínculos} con otras entidades,
    normalmente la relación es un verbo 
\end{itemize}

\subsection{Cardinalidad}
\begin{itemize}
    \item Es el tipo de relación por cantidad:
    \item Uno a uno, uno a muchos, muchos a muchos 
\end{itemize}

\subsection{Diagrama de Entidad Relación}
\begin{itemize}
    \item Se convierte el modelo entidad relación en tabla 
    \item Establecemos las relaciones entre tablas con PK/FK 
    \item Una relación muchos a muchos \textit{requiere} crear una nueva tabla,
    los registros de la tabla intermedia representan cada una de las relaciones,
    cada registro incluye las FK de las dos entidades relacionadas
\end{itemize}

\subsection{Tarea}
\begin{itemize}
    \item Queda en la diapo dos tareas, probar con Mermaid
\end{itemize}