\subsection*{Actividad 1.B}

Un número triangular se define como la cantidad de elementos que se 
pueden disponer en un triángulo equilátero,
de manera tal que el n-ésimo número triangular es el número de puntos,
en disposición triangular,
con base de n puntos.

\begin{figure}[h]
    \centering
    \includegraphics[width=.5\textwidth]{six-triangular.png}
\end{figure}

Los primeros ocho términos de la sucesión serían:

\begin{align*}
    \left\{1,3,6,10,15,21,28,36 \dots\right\}
\end{align*}

Un producto notable es una expresión algebraica que aparece con frecuencia,
que tiene un patrón específico y que, por lo tanto,
puede reescribirse con facilidad, agilizando cálculos.
Un ejemplo de un producto notable es el \textit{cuadrado del binomio},
\((a+b)^2\) que,
como vimos en el primer apartado del presente porfolio,
se puede reescribir como \(a^2 + 2ab + b^2\).

Otro ejemplo que suele aparecer con frecuencia es la \textit{diferencia de cuadrados},
esto es,
\(a^2 - b^2\),
que puede reformularse como \((a-b)(a+b)\).

La sucesión \(\left\{1,1,2,3,5,8,13,21,34,55 \dots\right\}\) es la 
sucesión de Fibonacci. La sucesión se obtiene,
como mencionamos en el apartado anterior,
partiendo de \(a_{1} = 0\) y \(a_{2} = 1\).
Desde allí en adelante,
operamos siguiendo la expresión \(a_{n} = a_{n-1} + a_{n-2}\),
por lo cual obtenemos,
por ejemplo,
\(a_{3} = 1 + 0 = 1\),
\(a_{4} = 1 + 1 = 2\),
\(a_{5} = 2 + 1 = 3\) y así sucesivamente.

A medida que \(n \to \infty\), 
el cociente de los consecutivos tiende al número irracional \(\phi\),
conocido como el \textit{número áureo},
cuyo valor aproximado es \(\phi = 1.61803398\dots\),
y su expresión algebraica \(\phi = \frac{1+\sqrt{5}}{2}\).

La sucesión de Fibonacci aparece en muchos fenómenos de la naturaleza,
como la disposición de las hojas en una planta,
la estructura de los caparazones de los caracoles
o la reproducción de las abejas.
