\subsection*{Actividad 2}

El concepto de integral tiene infinidad de usos en computación.
Esencialmente,
la computadora es un dispositivo que trabaja con datos discretos.
Sin embargo,
en la realidad,
los fenómenos de la naturaleza suelen tener comportamiento que se puede describir mejor apelando a funciones contínuas.

Ello en sí mismo es un desafío para la ingeniería de gran importancia,
que debe resolverse atendiendo a la especificidad de cada caso.

Para el presente trabajo, 
quería resaltar un uso en el procesamiento de señales,
entre ellas,
del sonido.
Empleando la transformada de Fourier,
las ondas de sonido,
que en la naturaleza tienen un comportamiento contínuo,
pueden convertirse,
apelando a algoritmos como la \textit{Fast Fourier Transformation},
que convierten el sonido desde sus ondas contínuas en datos discretos,
comprensibles por la máquina y por ende pasibles de ser guardados,
editados, compartidos y reproducidos.
