\subsection*{Actividad 1.B}

\textbf{Consigna. }
A partir de la deducción anterior, interpretar en forma gráfica la expresión: $x^2 + 4x + 4$

Dado que $(a + b)^2 = a^2 + 2ab + b^2$,
deducimos que:

\begin{align*}
	x^2 + 4x + 4 = x^2 + 2\cdot(2)\cdot(x) + 2^{2} = (x + 2)^{2}
\end{align*} 

Es decir,
gráficamente,
la expresión $x^2 + 4x + 4$ se puede representar como un cuadrado de lado $x + 2$.

\begin{align*}
	\begin{tikzpicture}
		% Cuadrado a
		\draw[cyan] (0,0) -- (2,0);
		\draw[cyan] (0,0) -- (0,2);
		\draw[cyan,dashed] (2,0) -- (2,2);
		\draw[cyan,dashed] (0,2) -- (2,2);
		\draw (1, -0.4) node{$x$};
		\draw (-0.4, 1) node{$x$};
		% Cuadrado b
		\draw[orange,dashed] (2,2) -- (3.5,2);
		\draw[orange,dashed] (2,2) -- (2,3.5);
		\draw[orange] (3.5,2) -- (3.5,3.5);
		\draw[orange] (2,3.5) -- (3.5,3.5);
		\draw (3.9, 2.75) node{$2$};
		\draw (2.75, 3.9) node{$2$};
		% Resto
		\draw[orange] (2,0) -- (3.5,0);
		\draw (2.75, -0.4) node{$2$};
		\draw[cyan] (3.5, 0) -- (3.5, 2);
		\draw (3.9, 1) node{$x$};
		\draw[cyan] (0, 3.5) -- (2, 3.5);
		\draw (1, 3.9) node{$x$};
		\draw[orange] (0, 2) -- (0, 3.5);
		\draw (-0.4, 2.75) node{$2$};
	\end{tikzpicture}
\end{align*}

De esta forma, 
sumando áreas de los dos cuadrados (\(x^{2}\) y \(2^{2}\)),
más los dos rectángulos (siendo cada uno de área \(2x\)),
llegamos a la expresión \(x^2 + 4x + 2^2\).

Esto es,
$(x+2)^2 = x^2 + 4x +4$.
