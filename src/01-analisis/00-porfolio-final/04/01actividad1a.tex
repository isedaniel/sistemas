\section*{Unidad 4: Derivadas y aplicaciones}
\subsection*{Actividad 1}

Modelo:

\begin{align*}
	h(t) &= 40t - 5t^2
\end{align*}

\subsection*{1.A}

La velocidad media se puede obtener siguiendo la expresión:

\begin{align*}
	V_m &= \frac{h(t_2) - h(t_1)}{t_2 - t_1}
\end{align*}

Evaluando esta expresión para los intervalos solicitados obtenemos:

\begin{center}
\begin{tabular}{ c c }
	Intervalo	&	Velocidad Media \\
	\hline \\
	$[0, 2]$	&	30\\	
	$[1, 4]$	&	15\\
	$[4, 7]$	&	-15\\
	$[1, 7]$	&	0
\end{tabular}
\end{center}

\textbf{Conclusiones}

Como cabría esperar de un modelo de lanzamiento vertical, la velocidad media 
es inicialmente más alta, decrece con el paso el tiempo y 
pasa a ser negativa en el tercer intervalo considerado. 
El cuarto intervalo la piedra se ubica a la misma altura en los segundos 1 y 7,
dando con ello una velocidad media de 0, aunque la piedra se encuentra en 
movimiento en ambos instantes.
