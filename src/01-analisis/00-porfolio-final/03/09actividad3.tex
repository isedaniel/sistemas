\subsection*{Actividad 3}

\textbf{Consigna.}
Proponer alguna situación problemática relacionada con su carrera
de grado, que plantee un enunciado de su modelización y la
aplicación del concepto de límite para el estudio de la misma.

El concepto de límite ocupa un lugar central en el análisis de algoritmos.
En particular, para el estudio de la eficiencia algorítmica se recurre a la cota superior asintótica,
cuya determinación depende del límite cuando la variable tiende a infinito.

Veamos un ejemplo. Siendo $n$ la cantidad de datos de entrada del algoritmo 1, la cantidad de operaciones que requiere
para arribar a un resultado sigue el modelo $3n^2 + 2^n$. Por otra parte, tenemos el algoritmo 2, cuya cantidad de operaciones
para arribar al mismo resultado crece según la función $2 \log n + 2n^4$. ¿Cómo determinamos cuál algoritmo es más eficiente?

Para ello, debemos determinar la cota superior asintótica de cada uno de los algoritmos, cuya expresión será aquel término que, diviendo
al modelo original, haga que el límite de la variable tendiendo a infinito sea igual a una constante. Entonces:

\begin{align*}
    \lim_{x \to \infty} \frac{3n^2 + 2^n}{3n^2} & = 1 + \infty \\
    \lim_{x \to \infty} \frac{3n^2 + 2^n}{3n^2} & = \infty
\end{align*}

Como vemos, el límite del cociente entre el modelo 1 y su primer término no da un valor constante. Probamos ahora con el segudo término:

\begin{align*}
    \lim_{x \to \infty} \frac{3n^2 + 2^n}{2^n} & = 0 + 1 \\
    \lim_{x \to \infty} \frac{3n^2 + 2^n}{2^n} & = 1
\end{align*}

Si operamos en cambio el cociente del modelo 1 por su segundo término, arribamos a un valor constante, en este caso de 1. Por ende,
la cota superior asintótica sigue a la expresión $2^n$. Veamos ahora el caso del segundo modelo.

\begin{align*}
    \lim_{x \to \infty} \frac{2 \log n + 2n^4}{2 \log n }                                   & = 1 + \frac{\infty}{\infty} \\
    1 + \lim_{x \to \infty} \frac{2 \cdot 4n^3}{2 \cdot \frac{1}{\ln 10} \cdot \frac{1}{x}} & = \infty
\end{align*}

En este caso, nuevamente el primer término no podría ser su cota superior asintótica, siéndolo por ende $2n^4$.
Teniendo estas expresiones para las cotas superiores, vemos cuál es su ubicación en la tabla de órdenes de complejidad:

\clearpage

\begin{table}[hbt!]
    \centering
    \begin{tabular}{ll}
        Notación                     & Nombre                              \\
        \hline
        $O(1)$                       & Orden constante                     \\
        $O(log log n)$               & Orden sublogarítmica                \\
        $O(log n)$                   & Orden logarítmica                   \\
        $O( n \sqrt{n})$             & Orden sublineal                     \\
        $O(n)$                       & Orden lineal o de primer orden      \\
        $O(n \log n)$                & Orden lineal logarítmica            \\
        $O(n^2)$                     & Orden cuadrática o de segundo orden \\
        $O(n^3)$                     & Orden cúbica o de tercer orden      \\
        $O(n^c)$                     & Orden potencial fija                \\
        $O(c^n) \text{, con } n > 1$ & Orden exponencial                   \\
        \hline
    \end{tabular}
\end{table}

Como puede observarse en la tabla, el algoritmo 1 podría ser clasificado como un algoritmo de orden exponencial, mientras el algoritmo 2 sería un algoritmo
de orden potencial fija. Como el orden exponencial está ubicado una fila por debajo del orden potencial, podemos concluir que el algoritmo 2 es más eficiente
que el 1.