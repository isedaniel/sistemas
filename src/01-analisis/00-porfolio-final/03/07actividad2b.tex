\subsection*{Actividad 2.B}

Para este ejercicio necesitamos, en primer lugar, una función que en $f(-5) = 1$.
Una función que cumple con este criterio es $f(x) = x + 6$.

Seguidamente, necesitamos que $\lim_{x \to 0^-}f(x) = \lim_{x \to 0^+}f(x) \neq f(0)$.
Para ello, $f(x) = 7, \text{ si } x = 0$
y $f(x) = \frac{x - 12}{x - 2} \text{ si } 0 < x < 5$. 
De esta manera cumplimos con la condición solicitada y además con $\nexists \lim_{x \to 2}$.

Por último, se pide que la función tenga una discontinuidad de segunda especie en $x = 5$, con $\lim_{x \to 5^-} \neq f(5)$, por lo cual la función por partes
puede definirse como sigue:

\begin{align*}
    f(x) =
    \begin{cases}
        x + 6,            & \text{si } -5 \leq x < 0 \\
        7,                & \text{si } x = 0         \\
        \frac{x-12}{x-2}, & \text{si } 0 < x < 5     \\
        7,                & \text{si } x = 5
    \end{cases}
\end{align*}

Gráficamente:

\begin{center}
    \begin{tikzpicture}
        \begin{axis}[
                axis lines = center,
                clip = false,
                legend pos = outer north east
            ]
            % x < 0
            \addplot [
                domain=-5:0,
                samples=50,
                color=cyan,
            ]
            {x+6};

            % x = 0
            \node[circle,fill,inner sep=2pt, color=cyan] at (axis cs:0,7) {};

            % x > 0
            \addplot [
                domain=0:1.5,
                samples=50,
                color=cyan,
            ]
            {(x-12)/(x-2)};

            \addplot [
                domain=2.5:5,
                samples=50,
                color=cyan,
            ]
            {(x-12)/(x-2)};

            % x = 5
            \node (n5) [circle, fill, inner sep=2pt, color=cyan] at (axis cs:5,7) {};

        \end{axis}
    \end{tikzpicture}
\end{center}
