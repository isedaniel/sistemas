\section*{Unidad 1: Números reales}
\subsection*{Actividad 1}

\subsection*{1.A}
Dado un cuadrado de lado $a$, con $a \in \mathbb{R^+}$, podemos definir su área como $a^2$.

\begin{center}
\begin{tikzpicture}
\draw[blue] (0,0) rectangle (2,2);
\draw (1, -0.4) node{$a$};
\end{tikzpicture}
\end{center}

De igual manera, si tomamos otro cuadrado, esta vez de lado $b$, con $b \in \mathbb{R^+}$, podemos establecer el área de este como $b^2$. 

\begin{align*}
\begin{tikzpicture}
\draw[red] (0,0) rectangle (1.5,1.5);
\draw (.75, -0.4) node{$b$};
\end{tikzpicture}
\end{align*}

Pasando a un tercer cuadrado, esta vez de lado $a+b$, podemos definir su área como $(a+b)^2$.

\begin{align*}
\begin{tikzpicture}
\draw[blue] (0,0) rectangle (2,2);
\draw (1, -0.4) node{$a$};
\draw[red] (2,2) rectangle (3.5,3.5);
\draw (3.9, 2.75) node{$b$};
\draw[red] (2,0) -- (3.5,0);
\draw (2.75, -0.4) node{$b$};
\draw[blue] (3.5, 0) -- (3.5, 2);
\draw (3.9, 1) node{$a$};
\draw[blue] (0, 3.5) -- (2, 3.5);
\draw (1, 3.9) node{$a$};
\draw[red] (0, 2) -- (0, 3.5);
\draw (-0.4, 2.75) node{$b$};
\draw (2.75, 3.9) node{$b$};
\draw (-0.4, 1) node{$a$};
\end{tikzpicture}
\end{align*}

De esta forma, deducimos que el área de un cuadrado de lado $a+b$ es igual a $a^2 + ab + ab + b^2$, es decir, $(a + b)^2 = a^2 + 2ab + b^2$.
