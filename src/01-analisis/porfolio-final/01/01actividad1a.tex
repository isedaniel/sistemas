\section*{Unidad 1: Números reales}

\subsection*{Actividad 1.A}
\textbf{Deducir y demostrar geométricamente, a partir de dos cuadrados
	de lado “a” y “b”, respectivamente, y otros dos rectángulos de largo
	“a” y ancho “b” que $(a + b)^2 = a^2 + 2ab + b^2$.}

Dado un cuadrado de lado $a$, con $a \in \mathbb{R^+}$, podemos definir su área como $a^2$.

\begin{center}
	\begin{tikzpicture}
		\draw[blue] (0,0) rectangle (2,2);
		\draw (1, -0.4) node{$a$};
	\end{tikzpicture}
\end{center}

De igual manera, podemos definir un cuadrado de lado $b$, con $b \in \mathbb{R^+}$, siendo su área $b^2$.

\begin{align*}
	\begin{tikzpicture}
		\draw[red] (0,0) rectangle (1.5,1.5);
		\draw (.75, -0.4) node{$b$};
	\end{tikzpicture}
\end{align*}

Entonces, si definimos un tercer cuadrado, de lado $a+b$, su área vendría dada por la expresión $(a+b)^2$. Gráficamente:

\begin{align*}
	\begin{tikzpicture}
		% Cuadrado a
		\draw[blue] (0,0) -- (2,0);
		\draw[blue] (0,0) -- (0,2);
		\draw[blue,dashed] (2,0) -- (2,2);
		\draw[blue,dashed] (0,2) -- (2,2);
		\draw (1, -0.4) node{$a$};
		% Cuadrado b
		\draw[red,dashed] (2,2) -- (3.5,2);
		\draw[red,dashed] (2,2) -- (2,3.5);
		\draw[red] (3.5,2) -- (3.5,3.5);
		\draw[red] (2,3.5) -- (3.5,3.5);
		\draw (3.9, 2.75) node{$b$};
		% Resto
		\draw[red] (2,0) -- (3.5,0);
		\draw (2.75, -0.4) node{$b$};
		\draw[blue] (3.5, 0) -- (3.5, 2);
		\draw (3.9, 1) node{$a$};
		\draw[blue] (0, 3.5) -- (2, 3.5);
		\draw (1, 3.9) node{$a$};
		\draw[red] (0, 2) -- (0, 3.5);
		\draw (-0.4, 2.75) node{$b$};
		\draw (2.75, 3.9) node{$b$};
		\draw (-0.4, 1) node{$a$};
	\end{tikzpicture}
\end{align*}

Como vemos, el cuadrado resultante es la sumatoria de las áreas de los cuadrados 1 y 2, más dos rectángulos de área $a \cdot b$. Por lo cual el área de un cuadrado de lado $a+b$ es igual a $a^2 + ab + ab + b^2$, es decir, $(a + b)^2 = a^2 + 2ab + b^2$.
