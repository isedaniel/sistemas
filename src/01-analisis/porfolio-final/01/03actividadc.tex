\subsection*{Actividad 1.C}
\textbf{Proponer una deducción de la fórmula Bhaskara o “resolvente”
	utilizada para la resolución de ecuaciones cuadráticas.}

Dados $a, b, c \in \mathbb{R}$, con $a \neq 0$, una expresión cuadrática puede definirse como el polinomio de estructura $ax^2 + bx + c = 0$, siendo $a$ el coeficiente cuadrático, $b$ el coeficiente lineal y $c$ el término independiente.

A partir de esta expresión es posible deducir la fórmula de Bhaskara o resolvente, despejando $x$ mediante el procedimiento de completar el cuadrado. Para ello, en primer lugar, dividimos todos los términos por $a$, de manera tal que el coeficiente cuadrático sea igual a 1. Despejamos además el término independiente hacia el otro lado de la igualdad.

\begin{align*}
	\frac{ax^2}{a} + \frac{bx}{a} + \frac{c}{a} & = \frac{0}{a}  \\
	x^2 + \frac{bx}{a}                          & = \frac{-c}{a} \\
\end{align*}

Nuestro objetivo ahora es buscar la estructura $x^2 + 2nx + n^2$, para comprimirla en $(x+n)^2$ y permitir así el despeje de $x$. Para ello, multiplicamos al término lineal por $\frac{2}{2}$, buscando que su estructura sea equivalente a $2nx$.

\begin{align*}
	x^2 + \frac{2}{2} \cdot \frac{bx}{a} & = - \frac{c}{a} \\
	x^2 + 2 \cdot \frac{b}{2a} \cdot x   & = - \frac{c}{a} \\
\end{align*}

De esta forma, $2 \cdot \frac{b}{2a} \cdot x$ sería equivalente a $2nx$, siendo $n = \frac{b}{2a}$. Para completar el cuadrado, sumamos ahora $n^2$ de ambos lados de la expresión.

\begin{align*}
	x^2 + 2 \cdot \frac{b}{2a} \cdot x + \left(\frac{b}{2a}\right)^2 & = \left(\frac{b}{2a}\right)^2 - \frac{c}{a} \\
\end{align*}

Con esta estructura, es posible expresar el polinomio de la izquierda como un binomio al cuadrado. Del otro lado de la igualdad operamos la resta de fracciones.

\begin{align*}
	\left(x + \frac{b}{2a}\right)^2 & = \frac{b^2 - 4ac}{4a^2} \\
\end{align*}

Operamos raíz cuadrada a ambos lados de la expresión, recordando añadir $\pm$ para no perder los dos resultados posibles para x. Luego despejamos $x$.

\begin{align*}
	x + \frac{b}{2a} & = \pm \frac{\sqrt{b^2 - 4ac}}{\sqrt{4a^2}}      \\
	x                & = \frac{-b}{2a} \pm \frac{\sqrt{b^2 - 4ac}}{2a} \\
\end{align*}

Finalmente, expresamos el cociente de la derecha como una sola fracción, resultando en:

\begin{align*}
	x & = \frac{-b \pm \sqrt{b^2 - 4ac}}{2a} \\
\end{align*}
