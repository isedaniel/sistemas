\subsection*{1.C}

Dados $a, b, c, x \in \mathbb{R}$, con $a \neq 0$, definimos como expresión cuadrática al polinomio de estructura $ax^2 + bx + c = 0$, siendo $a$ el coeficiente cuadrático, $b$ el coeficiente lineal y $c$ el término independiente.

Desde esta expresión es posible deducir la fórmula de Bhaskara o resolvente, recurriendo al método de completar el cuadrado. Para ello, en primer lugar dividimos todos los términos, a ambos lados de la igualdad, por $a$, de manera tal que el coeficiente cuadrático quede igual a 1. Pasamos además el término independiente hacia el otro lado de la igualdad.

\begin{align*}
	\frac{ax^2}{a} + \frac{bx}{a} + \frac{c}{a} &= \frac{0}{a}\\
	x^2 + \frac{bx}{a} &= \frac{-c}{a}\\
\end{align*}

Seguidamente, buscamos la estructura $x^2 + 2nx + n^2$, para poder comprimirla en $(x+n)^2$ y facilitar así el despeje de la $x$. Para ello, empezamos por multiplicar al término lineal por $\frac{2}{2}$, buscando que la estructura del término sea equivalente a $2nx$.

\begin{align*}
	x^2 + \frac{2}{2} \cdot \frac{bx}{a} &= - \frac{c}{a}\\
\end{align*}

De esta forma $2 \cdot \frac{b}{2a} \cdot x$ sería equivalente a $2nx$. Ahora sumamos $n^2$. Para mantener la igualdad, sumamos de ambos lados.

\begin{align*}
	x^2 + 2 \cdot \frac{b}{2a} \cdot x + \left(\frac{b}{2a}\right)^2 &= \left(\frac{b}{2a}\right)^2 - \frac{c}{a}\\
\end{align*}

Expresamos el término izquierdo como un binomio al cuadrado. Del otro lado de la igualdad operamos la resta de fracciones.

\begin{align*}
	\left(x + \frac{b}{2a}\right)^2 &= \frac{b^2 - 4ac}{4a^2}\\
\end{align*}

Operamos raíz cuadrada a ambos lados de la expresión, recordando añadir $\pm$ para no perder los dos resultados posibles para x. Luego despejamos $x$.

\begin{align*}
	x + \frac{b}{2a} &= \pm \frac{\sqrt{b^2 - 4ac}}{\sqrt{4a^2}}\\
	x &= \frac{-b}{2a} \pm \frac{\sqrt{b^2 - 4ac}}{2a}\\
\end{align*}

Finalmente, expresamos el cociente de la derecha como una sola fracción, resultando en:

\begin{align*}
	x &= \frac{-b \pm \sqrt{b^2 - 4ac}}{2a}\\
\end{align*}
