\subsection*{Actividad 1.D}
\textbf{Una inecuación es una desigualdad entre dos expresiones
	algebraicas con una o varias incógnitas.
	Los coeficientes de la x también pueden pasar al otro lado como en
	las ecuaciones, pero tenemos que cambiar el signo de
	desigualdad si el número es negativo. ¿Por qué?}

Dados tres números $a, b, x \in \mathbb{R}$, una inecuación puede expresarse como:

\begin{align*}
	ax & \leq b \\
\end{align*}

Dado un número $n \in \mathbb{R}$, sumar o restar $n$ a ambos miembros de la expresión mantiene la desigualdad:

\begin{align*}
	-2     & < 1     \\
	-2 - 1 & < 1 - 1 \\
	-3     & < -2
\end{align*}

La situación puede cambiar cuando operamos productos a ambos lados de la desigualdad.

\begin{align*}
	-2         & < 1         \\
	-2 \cdot 2 & < 1 \cdot 2 \\
	-4         & < 2
\end{align*}

Como vemos, la inecuación se sigue verificando para este producto, siendo ese el caso para todo $n \in \mathbb{R^+}$. Sin embargo, la situación cambia al multiplicar números negativos.

\begin{align*}
	-2 \cdot -2 & \not < 1 \cdot -2 \\
	4           & \not < -2         \\
	4           & > -2
\end{align*}

Al multiplicar ambos miembros de una inecuación por un número $n \in \mathbb{R^-}$, es necesario cambiar la orientación de la desigualdad, de manera tal que esta se siga verificando.
