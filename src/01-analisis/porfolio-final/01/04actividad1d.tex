\subsection*{1.D}

Dados $a, b, x \in \mathbb{R}$, una inecuación puede expresarse como:

\begin{align*}
	ax &\leq b\\
\end{align*}

Una inecuación funciona de manera similar para la suma y resta de número reales a ambos términos. Es decir, dado un número $x \in \mathbb{R}$, restando $x$ a ambos miembros de la expresión la inecuación se mantiene.

\begin{align*}
	-2 &< 1\\
	-2 - 1 &< 1 - 1\\
	-3 &< -2
\end{align*}

En el caso de operar multiplicaciones se debe tener un especial cuidado.

\begin{align*}
	-2 &< 1\\
	-2 \cdot 2 &< 1 \cdot 2\\
	-4 &< 2
\end{align*}

La inecuación se sigue verificando en esta multiplicación, siendo así $\forall x \in \mathbb{R^+}$. Sin embargo, la situación cambia al multiplicar números negativos.

\begin{align*}
	-2 \cdot -2 &\not < 1 \cdot -2\\
	4 &\not < -2\\
	4 &> -2
\end{align*}

Al multiplicar ambos miembros de una inecuación por un número $x \in \mathbb{R^-}$, es necesario cambiar la orientación de la desigualdad, de manera tal que esta se siga verificando.
