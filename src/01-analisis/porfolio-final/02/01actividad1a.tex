\section*{Unidad 2: Funciones Reales}

\subsection*{Actividad 1}
\subsection*{Modelización 1}

Jugando con un perro se arroja una pelota, cuya altura en función del tiempo sigue el modelo $h = \frac{4}{5} + 9t - 9.8t^2$. En este modelo la altura es la variable dependiente y el tiempo transcurrido, desde el lanzamiento hasta el aterrizaje, es la variable independiente. 

Gráficamente:

\begin{center}
\begin{tikzpicture}
\begin{axis}[
    axis lines = left,
    xlabel = \(t\),
    ylabel = {\(h\)},
    clip = false,
    legend pos = outer north east,
]
% costo
\addplot [
    domain=0:1, 
    samples=200, 
    color=cyan,
]
{4/5 + 9*x - x*x*49/5};
\addlegendentry{\(h = \frac{4}{5} + 9t - 9.8t^2\)}

\node[label={270:{(0.46,2.87)}},circle,fill,inner sep=2pt] at (axis cs:0.459,2.866) {};
\end{axis}
\end{tikzpicture}
\end{center}

El dominio de la función es $D = [0,1]$.

Para determinar su imagen buscamos el punto más alto, es decir, el vértice de la función cuadrática, siguiendo la expresión $\frac{-b}{2a}$. 

\begin{align*}
	\frac{-9}{-9.8 \cdot 2} &\approx 0.46
\end{align*}

Si evaluamos la función en $0.46$ obtenemos:

\begin{align*}
	\frac{4}{5} + 9 \cdot 0.459 - 9.8 \cdot 0.459^2 &\approx 2.87
\end{align*}

Entonces, $I = [0, 2.87]$
