\subsection*{Actividad 2.C}

Para el tramo menor a $x = -3$ necesitamos una asíntota horizontal en 0. Por ello, en este tramo la función sería $f(x) = \frac{-1}{x}$.
En $x = -3$ la función requiere de una discontinuidad de salto finito. Para ello, se propone que a partir de este punto $f(x) = \frac{-x}{x-2}$.
Manteniendo este función hasta $x = 2$ se cumple también con la necesidad de una asíntota vertical en $x = 2$.
Para el intervalo $[2, 4]$ se propone la función $x+1$, que mantiene el dominio de los números reales y a la vez $\lim_{x \to 2^+} = 3$.
Por último, para las $x$ mayores a 4, la función seguiría con $\frac{-x}{x-4}$, para, de este forma, tener una asíntota horizontal en $-1$.

\begin{align*}
    f(x) =
    \begin{cases}
        \frac{-1}{x},   & \text{si } x < -3          \\
        \frac{-x}{x-2}, & \text{si } -3 \leq x < 2   \\
        x+1,            & \text{si } 2 \leq x \leq 4 \\
        \frac{-x}{x-4}, & \text{si } 4 < x
    \end{cases}
\end{align*}

Gráficamente:

\begin{center}
    \begin{tikzpicture}
        \begin{axis}[
                axis lines = center,
                clip = false,
                legend pos = outer north east
            ]
            % x < -3
            \addplot [
                domain=-5:-3,
                samples=50,
                color=orange,
            ]
            {-1/x};

            % x < 2
            \addplot [
                domain=-3:1.8,
                samples=50,
                color=orange,
            ]
            {(-x)/(x-2)};

            % x < 4
            \addplot [
                domain=2:4,
                samples=50,
                color=orange,
            ]
            {x+1};

            % 4 < x
            \addplot [
                domain=4.5:7,
                samples=50,
                color=orange,
            ]
            {(-x)/(x-4)};

        \end{axis}
    \end{tikzpicture}
\end{center}
