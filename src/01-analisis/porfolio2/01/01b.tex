\subsection*{1.B}

La derivada de la función que describe la altura de la piedra es $40-10t$. Si igualamos a 0 obtenemos:

\begin{align*}
    40-10t &= 0 \\
    t &= \frac{-40}{-10}\\
    t &= 4
\end{align*}

Para comprobar que esta sea la altura máxima, evaluamos $h'$ en $t = 3$ y $t = 5$.

\begin{center}
\begin{tabular}{ c c }
	t	&	$h'_{(t)}$ \\
	\hline \\
	$3$	&	10\\	
	$4$	&	0\\
	$5$	&	-10
\end{tabular}
\end{center}

Como se desprende de la tabla, la velocidad en $t=3$ es todavía positiva, por lo cual la piedra continúa subiendo. En $t=5$, por el contrario, la velocidad es negativa, por lo que podemos concluir que la piedra ya se encuentra descendiendo. Por lo tanto, el punto $t=4$ constituye el máximo de la función.

Con estos datos en mente, podemos representar el modelo que describe el movimiento de la piedra, resaltando su punto máximo.

\begin{center}
\begin{tikzpicture}
\begin{axis}[
    axis lines = left,
    xlabel = \(t\),
    ylabel = {\(h(t)\)},
    clip = false,
]
% Gráfico
\addplot [
    domain=0:8, 
    samples=200, 
    color=blue,
]
{40*x - 5 * x^2};
\addlegendentry{\(40t - 5t^2\)}
\node[label={270:{(4,80)}},circle,fill,inner sep=2pt] at (axis cs:4,80) {};
\end{axis}
\end{tikzpicture}
\end{center}