\subsection*{Actividad 1.D}

Dados $a, b \in \mathbb{R}$, una inecuación puede expresarse como:

\begin{align*}
	a &\leq b\\
\end{align*}

Una inecuación funciona de manera similar para la suma y resta de número reales a ambos términos. Es decir, dado $m \in \mathbb{R}$, si resto $m$ a ambos miembros de la expresión la inecuación se mantiene. Por ejemplo:

\begin{align*}
	-2 &< 1\\
	-2 - 1 &< 1 - 1\\
	-3 &< -2
\end{align*}

Veamos un ejemplo multiplicando por $m$ con $m \in mathbb{R^+}$:


\begin{align*}
	-2 &< 1\\
	-2 \cdot 2 &< 1 \cdot 2\\
	-4 &< 2
\end{align*}

La inecuación se sigue verificando en la multiplicación, y es así $\forall m \in \mathbb{R^+}$.

Sin embargo, esta situación cambia al multiplicar por número negativos, es decir, con $m < 0$

\begin{align*}
	-2 \cdot -2 &\not < 1 \cdot -2\\
	4 &\not < -2\\
	4 &> -2
\end{align*}

Entonces, cuando multiplicamos ambos miembros de una inecuación por un número $m \in \mathbb{R^-}$, es necesario cambiar la orientación de la desigualdad para que esta se siga verificando.
