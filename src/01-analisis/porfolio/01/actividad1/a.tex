\subsection*{Actividad 1.A}

Dado un cuadrado de lado $a$, siendo $a \in \mathbb{R}$, podemos definir su área como $a^2$.

\begin{align*}
\begin{tikzpicture}
\draw[blue] (0,0) rectangle (2,2);
\draw (1, -0.4) node{$a$};
\end{tikzpicture}
\end{align*}

De igual manera, si tomamos otro cuadrado de lado $b$, con $b \in \mathbb{R}$, podemos establecer el área de este segundo cuadrado en $b^2$. 

\begin{align*}
\begin{tikzpicture}
\draw[red] (0,0) rectangle (1.5,1.5);
\draw (.75, -0.4) node{$b$};
\end{tikzpicture}
\end{align*}

Podemos definir un cuadrado de lado $a+b$, cuya área será igual a la expresión $(a+b)^2$. Representando gráficamente obtenemos:

\begin{align*}
\begin{tikzpicture}
\draw[blue] (0,0) rectangle (2,2);
\draw (1, -0.4) node{$a$};
\draw[red] (2,2) rectangle (3.5,3.5);
\draw (3.9, 2.75) node{$b$};
\draw[red] (2,0) -- (3.5,0);
\draw (2.75, -0.4) node{$b$};
\draw[blue] (3.5, 0) -- (3.5, 2);
\draw (3.9, 1) node{$a$};
\draw[blue] (0, 3.5) -- (2, 3.5);
\draw (1, 3.9) node{$a$};
\draw[red] (0, 2) -- (0, 3.5);
\draw (-0.4, 2.75) node{$b$};
\draw (2.75, 3.9) node{$b$};
\draw (-0.4, 1) node{$a$};
\end{tikzpicture}
\end{align*}

De esta forma, obtenemos un cuadrado de lado $a+b$, cuya área es la sumatoria del área de los polígonos que lo componen. El área del cuadrado cuyo lado es $(a+b)$ es igual a $a^2 + ab + ab + b^2$. Es decir, $(a + b)^2 = a^2 + 2ab + b^2$.
