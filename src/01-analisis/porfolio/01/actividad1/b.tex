\subsection*{Actividad 1.B}

Dado un número $x$, con $x \in \mathbb{R}$, podemos definir un cuadrado de lado $(x+2)$, cuya área se puede representar gráficamente de la siguiente forma:

\begin{align*}
\begin{tikzpicture}
\draw[blue] (0,0) rectangle (2,2);
\draw (1, -0.4) node{$x$};
\draw (2,2) rectangle (3.5,3.5);
\draw (3.9, 2.75) node{$2$};
\draw (2,0) -- (3.5,0);
\draw (2.75, -0.4) node{$2$};
\draw[blue] (3.5, 0) -- (3.5, 2);
\draw (3.9, 1) node{$x$};
\draw[blue] (0, 3.5) -- (2, 3.5);
\draw (1, 3.9) node{$x$};
\draw (0, 2) -- (0, 3.5);
\draw (-0.4, 2.75) node{$2$};
\draw (2.75, 3.9) node{$2$};
\draw (-0.4, 1) node{$x$};
\end{tikzpicture}
\end{align*}

Sumando las áreas de los cuadrados y rectángulos que componen al cuadrado de lado $(x+2)$ obtenemos que $(x+2)^2 = x^2 + 2x + 2x + 2^2$, que puede expresarse como $x^2 + 4x +4$.
