
Estas notas corresponden a las clases de Steven Skiena 
sobre Algoritmos y Estructuras de datos, 
\href{https://www.youtube.com/watch?v=22hwcnXIGgk&list=PLOtl7M3yp-DX6ic0HGT0PUX_wiNmkWkXx}{disponibles en Youtube}.
Este y otros recursos adicionales a la cursada vienen de la lista de lectura
 \href{https://teachyourselfcs.com/}{Teach Yourself Computer Science}.

\section{Lecture 1: Introduction to Algorithms}

Un algoritmo es una idea detrás de \textit{cómo} resuelve algo un programa.
Dado un input y un output,
un algoritmo es la forma en que el input se convierte en el output.

Un problema especifica ese input y el output deseado.

Lo que buscamos resolviendo un problema es que un algoritmo sea:
\begin{itemize}
    \item eficiente
    \item correcto
\end{itemize}

Eficiente tiene que ver con la notación O.
Con suficiente input, 
el procesamiento tiende a O.

Correcto quiere decir que tome todos los inputs del problema y 
devuelva correctamente los outputs. 

La corrección se prueba.

Necesitamos un lenguaje para describir algoritmos.
Hay distintas maneras.
Se lo puede describir en inglés (lenguaje natural).

A veces es bueno, a veces hace falta más precisión.

Un segundo nivel es \texttt{pseudocode}.

El tercero es implementar en un lenguaje de programación.

Generalmente,
describimos la idea general en lenguaje natural,
si hace falta precisión, pseudocode.
El lenguaje natural ayuda a entender el pseudo.

Con un ejemplo puedo demostrar la falta de corrección.