\section{Primera clase}

\subsection{Modalidad de cursada}

\begin{itemize}
    \item Consultas sobre los prácticos al principio de la clase 
    \item Desarrollo de teoría durante la clase 
    \item Consulta nuevamente al final 
\end{itemize}

\subsection{Forma de evaluación}

\begin{itemize}
    \item Parciales, promocionable con nota mínima de 6, promedio mayor a 7.
\end{itemize}

\subsection{Sugerencias}

\begin{itemize}
    \item Hacer ejercicios todas las semanas: matemática se aprende haciendo 
    ejercicios 
\end{itemize}

\subsection{Bibliografía}

\begin{itemize}
    \item Stewart, la mayoría de la práctica sale de ahí 
    \item Revisar teoría provista 
    \item Realizar prácticos provistos 
\end{itemize}

\subsection{Modalidad rendir final}

\begin{itemize}
    \item Examen escrito y oral
\end{itemize}

\subsection{Requerimientos}

\begin{itemize}
    \item Repasar identidades trigonométricas
    \item Repasar cónicas 
    \item Usar geogebra 
\end{itemize}

\subsection{Conceptos previos}

\begin{itemize}
    \item Espacio métrico
    \begin{itemize}
        \item Conjunto de puntos 
        \item que tienen una función asociada llamada \textbf{distancia}
        \item Y puede ser una recta, un plano, el espacio
    \end{itemize}
    \item Distancia 
    \begin{itemize}
        \item Módulo de la diferencia entre dos puntos 
        \item \(|P-Q| =\) distancia entre \(P\) y \(Q\)
        \item \(|P-Q| = 0 \implies P = Q\)
        \item \(|P-Q| + |P-R| \geq |Q-R|\), 
        (condición para que sean iguales: 
        que esten en la misma recta,
        dejando de ser por tanto un triángulo)
    \end{itemize}
    \item Espacio euclídeo n-dimensional
    \begin{itemize}
        \item Se denota \(R^{n}\)
        \item Sus elementos se denominan: duplas, ternas, n-uplas,
        de acuerdo a \(n\)
        \item \(R^{2}\): plano
        \item \(R^{3}\): espacio
        \item \(R^{n}\) con \(n>3\): hiperplano
    \end{itemize}
    \item Distancia euclídea en \(R^{n}\):
    \begin{itemize}
        \item \(|P-Q| = \sqrt{(q_1-p_1)^{2} + (q_2-p_2)^{2} + \cdots + (q_n-p_n)^{2}}\)
        \item Cada dimensión incorpora un \textit{término} al cálculo de la 
        distancia 
        \item Como hablamos de distancia: siempre positivo 
    \end{itemize}
\end{itemize}

\subsection{Conceptos de topología}

\textbf{Entorno.}
\begin{itemize}
    \item Siendo \(A\) punto en \(R^{n}\), un entorno con centro en \(A\),
    de radio \(\delta\), es el \textit{conjunto de puntos} a una distancia 
    respecto de \(A\) menor a \(\delta\)
    \item Es decir, \(E(A,\delta) = \{x \in \mathbb{R}: |x-A| < \delta\}\)
    \item En el plano, 
    \(E(A,\delta) = \{x \in \mathbb{R}^{2}: |x-a_1| < \delta \land |y-a_2| < \delta\}\)
    \item En otras palabras, en el plano genera un disco 
    \item En el espacio \(\mathbb{R}^{3}\), genera una esfera
    \item Entorno reducido: no incluye al centro \(A\)
\end{itemize}

\textbf{Punto.}
\begin{itemize}
    \item Decimos que es: interior: si al menos un entorno del punto está 
    totalmente incluido en el conjunto C 
    \item Exterior: si al menos un entorno está totalmente excluido de C 
    \item Frontera: no es ni interior ni exterior, siempre tiene parte 
    incluída y parte excluída.
    \item Punto de acumulación: cuando cualquier \textit{entorno reducido},
    es decir, que no incluye a \(A\), tiene intersección \textit{no vacía}
    con \(C\) (puede estar en frontera, naturalmente)
\end{itemize}

\textbf{Conjunto.}
\begin{itemize}
    \item Decimos que es: abierto: si todo punto incluído en \(C\) es interior,
    es decir, ninguno es frontera.
    \item cerrado: su complemento es abierto, es decir, tiene frontera 
    \item Acotado: lo puedo incluir \textit{en su totalidad} en una 
    circunferencia con centro en el origen, con independencia del radio 
    \item conexo: puedo unir 2 puntos cualesquiera, y la recta entre ellos 
    \textit{no se sale} del conjunto
\end{itemize}

\textbf{Ejemplo.}
\(f(x,y) = z = \sqrt{y-x^{2}}\)

\begin{itemize}
    \item Su \textit{dominio} son todos los \(y \geq x^{2}\)
    \item Su \textit{frontera} es \(y = x^{2}\), que está incluída
    \item Sus \textit{puntos interiores} son aquellos \(x\) que satisfacen 
    \(y - x^{2} > 0\)
    \item Es un \textit{dominio cerrado}, puesto que incluye a la frontera,
    y por lo tanto su complemento no
    \item \textit{No es acotado}, puesto que no se puede incluir en un círculo
    centrado en el origen
\end{itemize}

\subsection{Función de una variable}

\begin{itemize}
    \item A cada valor del \textit{dominio} de la función,
    le corresponde un \textit{y solo un} valor del codominio.
\end{itemize}

\subsection{Función de varias variables}
\begin{itemize}
    \item La idea es la misma 
    \item Si es una función que toma un par de escalares y devuelve un escalar,
    es un \textbf{campo escalar}
    \item \((x,y)\) es un punto en el conjunto \textit{dominio}
    \item como toda función tiene un único valor de salida, en este caso, \(z\)
    \item Sería función vectorial si: devuelve un vector
\end{itemize}

\subsection{Campo escalar}
\begin{itemize}
    \item Función que a cada par \((x,y)\) le asigna un \textit{único} valor \(z\)
    \item Dominio en \(\mathbb{R}^{2}\)
    \item Como en una función de una variable, dos pares pueden dar mismo \(z\)
    pero un solo par no puede devolver más de un \(z\)
\end{itemize}

\subsection{Como buscar dominio de campo escalar}
\begin{itemize}
    \item Todo par \((x,y)\) que deje a la función bien definida,
    es decir, misma condición que en función de una variable.
\end{itemize}

\subsection{Gráfica de función de dos variables}
\begin{itemize}
    \item Se realiza en el espacio
    \item El eje \(x\) debe estar a \(135^{\circ}\) del eje \(y\),
    siguiendo escala .7
    \item Graficamos por trazas: 
    hacemos que una o dos variables sean 0 y vemos que sale.
\end{itemize}

\textbf{Ejemplo.}
\begin{align*}
    z = 4 - 5x + 2y \\
\end{align*}

Si \(x = 0 \land y = 0 \implies z = 4\)
Si \(z = 0 \land y = 0 \implies x = 4/5\)
Si \(z = 0 \land x = 0 \implies z = -2\)

Con estos puntos podemos graficar el plano

\textbf{Ejemplo 2.}
\begin{equation*}
    z = 4x^{2} + y^{2}
\end{equation*}

En este caso, si hacemos dos variables igual a 0 no obtenemos información,
vamos de una en una:

\begin{align*}
    z = 0 \implies y = 
\end{align*}

Consultar de nuevo que pasa aquí.

\subsection{Interpretación geométrica}

Es una superficie de \(z\) para cada par \((x,y)\)

\subsection{Gráficas conocidas: cuádricas y cónicas}

\begin{itemize}
    \item en \(R^{2}\) \(x^{2} + y^{2} = 1\) es una circunferencia,
    en el espacio \(R^{3}\) es un cilindro con \(z\) libre.
    \item aclaración: llamamos cilindro a toda superficie de líneas paralelas 
    entre sí, a la que llamamos \textit{generatrices}, que van recorriendo una 
    curva, por lo que el cilindro puede (o no) ser la figura tradicional que 
    conocemos
\end{itemize}

\subsection{Elipsoide}

\begin{align*}
    \frac{x^{2}}{a^{2}} + \frac{y^{2}}{b^{2}} + \frac{z^{2}}{c^{2}} = 1
\end{align*}

\subsection{Paraboloide}

La palabra \textit{no elevada} al cuadrado es la que el paraboloide abraza:

\textbf{Consultar.}

\subsection{Cono:}

\begin{align*}
    \frac{x^{2}}{a^{2}} + \frac{y^{2}}{b^{2}} = \frac{z^{2}}{c^{2}}
\end{align*}

Notar que una es negativa y el cono es igual a 0.

\subsection{Hiperboloide}

Abraza la negativa al cuadrado 

\subsection{Hiperboloide 2 hojas}

Tiene dos variables negativas, y todas cuadradas.


\subsection{Curvas de nivel}

Curva que resulta de la intersección de un plano \(z=k\),
con \(k\) constante.
Si la proyecto sobre el piso, veo como evoluciona \(z\),
como en un mapa físico.

Algunas curvas conocidas son: isobaras (curvas de presión)
isotermas (areas de temperatura)

\subsection{Superficie de nivel}

Va de \(\mathbb{R}^{3} \to \mathbb{R}\). No es representable,
pero igualando a \(k\) obtengo \textit{superficie de nivel}

\subsection{Mapa de contorno}

Da una idea sobre una superficie.