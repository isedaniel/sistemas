\section{Tercera clase, 26 de agosto}

\subsection{Derivada en una función de una sola variable}
\begin{itemize}
    \item Aproximo tanto dos puntos que obtengo una recta tangente a la curva
\end{itemize}

\subsection{Derivada en función de dos variables}
\begin{itemize}
    \item Idea muy similar, pero no puedo derivar dos a la vez
    \item Derivo \(x\) o derivo \(y\)
    \item La nomenclatura es:
\end{itemize}

\begin{equation*}
    f_x = \frac{\partial f}{\partial x}
\end{equation*}

\begin{itemize}
    \item Entonces, derivar respecto de x es encontrar la pendiente de la
          curva que se genera por intersecar la superficie por un plano \(x = k\)
\end{itemize}

\subsection{Derivadas sucesivas}
\begin{itemize}
    \item Derivo una tomando la otra constante y luego derivo de nuevo,
          por la misma o la anterior \(f_{xx} \lor f_{xy}\)
    \item Teorema Claireaut-Schwarz: si la función es continua en el punto \(P\),
          la derivada segunda \(f_{xy} = f_{yx}\)
\end{itemize}

\subsection{Regla de la cadena}
\begin{itemize}
    \item Sale de funciones compuestas
    \item Derivo una y la otra que está adentro
    \item Tengo w(x,y), x(t) e y(t):
          \begin{enumerate}
              \item Puedo: sustituir y derivar como función de una variable
              \item Aplicar regla de la cadena:
          \end{enumerate}
\end{itemize}
\begin{equation*}
    \frac{dw}{dt} = \frac{\partial w}{\partial x}\frac{dx}{dt} + \frac{\partial w}{\partial y}\frac{dy}{dt}
\end{equation*}
\begin{itemize}
    \item Si tengo \(w(x,y)\), \(x(t,s)\) e \(y(t,s)\), es lo mismo,
          pero hay que hacer una para cada una de las variables \(s\) y \(t\):
\end{itemize}
\begin{equation*}
    \frac{\partial w}{\partial s} = \frac{\partial w}{\partial x}\frac{\partial x}{\partial s} + \frac{\partial w}{\partial y}\frac{\partial y}{\partial s}
\end{equation*}
\begin{equation*}
    \frac{\partial w}{\partial t} = \frac{\partial w}{\partial x}\frac{\partial x}{\partial t} + \frac{\partial w}{\partial y}\frac{\partial y}{\partial t}
\end{equation*}
\begin{itemize}
    \item Sugerencia: hacer los dos caminos
\end{itemize}

\subsection{Plano tangente a la superficie}
\begin{equation*}
    \vec{n}= \vec{u}\times\vec{v} = \begin{vmatrix}
        i & j & k \\ 1 & 0 & f_x(a,b) \\ 0 & 1 & f_y(a,b) \\
    \end{vmatrix}
\end{equation*}

\begin{itemize}
    \item Con esto tengo vector normal a la superficie en \((a,b)\)
    \item Con ese vector y el punto \((a,b)\) hacemos un plano tangente a la 
    superficie
    \item \( \vec{n} = (f_x, f_y, -1)\)
    \item Siempre que piden linealización o aproximación lineal están hablando de
    plano tangente
\end{itemize}

\subsection{Función diferenciable}
\begin{itemize}
    \item Si es diferenciable, es continua 
    \item No siempre es diferenciable si es continua
    \item diferenciable: si las derivadas parciales son continuas en el punto
\end{itemize}