\section{Segunda clase}

25 de marzo, 2025

\subsection{Límites de una función}

Trabajando en una variable.
El límite es L si y solo si a es punto de acumulación de f y,
para todo epsilon mayor a 0.

La definición de límite involucra una distancia.
Mientras más chica hago la distancia,
aparece el punto de acumulación al que es igual L.

Ahhora, cuando hablo de la distancia de un punto a otro, hay que tener en cuenta la definición pitagórica de la distancia.

\subsection{Límites de campos escalares}

Si tomo (x,y) que se acercan a (a,b),
la función se acerca al límite.

Este límite es el \textbf{límite doble} o \textbf{simultáneo}.

Fijando delta,
tomo un punto dentro del disco.

\subsection{Propiedades de los límites}

Uno de los operadores más nobles, distributivo con todo.

\subsection{Existencia de límite}

El límite existe cuando, acercándome a \(a\),
el límite es el mismo por los dos lados.

Ahora, encontramos indeterminación,
L'Hopital no aplica,
y el límite no alcanza con izquierda y derecha.

En el plano,
me puedo acercar por todos lados al punto.

Es muy dificil establecer que el límite existe,
salvo que lo hagamos por definición.

Más fácil, verificar que no existe.

Si me acerco por diferentes métodos y da distinto, establezco que no existe.

\subsection{Simultáneos}

Reemplazamos los dos.
Si da indeterminación, seguimos.

\subsection{Límites iterativos}

Hago límite de una. Al resultado, le hago el otro límite.
Si me da distinto, concluyo que no existe el límite.

Si tengo 0 arriba, no hace falta hacerlo: da 0.
Pruebo radiales.

\subsection{Límites radiales}

Me acerco por recta: \(y = m(x-x_0) + y_0\).
Entonces límite cuando \(y \to mx\).

\subsection{Límite parabólico}

\(y \to x^2\)

En general llegamos al parabólico. A veces,
por la forma, nos damos cuénta en cuál falla.
(Por la diferencia de grado numerador denominador es).

Cualquiera que me de un número, ya me afirma que no existe límite.
Me puedo saltear el que quiera también.

\subsection{Coordenadas polares}

Si trabajo con polares,
puedo asegurar que el límite existe.

Solo para funcion con denominador \(y^2+x^2\).

\subsection{Continuidad}

Igual a función 1 variable: 1. límite existe 2. función en el punto existe.