\section{Clase 13 de mayo}

Integrales múltiples:
la idea es igual a la integral de una función de 1 sola variable:
la idea de integral, acotada para un rectángulo,
bajo la superficie que devuelve un campo escalar,
es igual al volumen,
que resulta de dicha base rectangular,
por la distancia para cada punto hasta la superficie.

Es decir,
la idea es la misma.

Las propiedades también son las mismas:
la integral es distributiva para la suma y la resta de funciones;
el producto de una función por una constante es igual a la constante por el resultado de la integral
(es decir, puedo sacar la constante de la integral);
la integral de una región es igual a la suma de las integrales de 2 subregiones;
por último,
si una funcíon es superior a otra en toda la superficie, 
la integral es mayor a la otra en toda la superficie.

Si es a evaluar, es adimensional.
El \textbf{volumen} resulta del \textbf{contexto} del problema.
La \textbf{unidad} resulta del \textbf{contexto} de problema.

\subsection{Teorema de Fubini}

Teorema de Fubini
(que aplica también para derivadas parciales, Teorema de Schwarz allá)
implica que integrar en una  y después en otra para la misma región,
y viceversa,
es igual,
para los dos casos.

\subsection{Integrales dobles en recintos generales}

Refiere a recintos delimitados por funciones.
En este caso, si tengo funciones que dependen de y integro por x primero.
Si tengo funciones que dependen de x, integro por y primero.
Es decir, que sea constante.
Es para sacar la otra primero,
sino va a quedar dando vueltas
(probar).

Hay algunos que implica subdivisión si o si:
no hay manera de resolverlo sin hacer la subdivisión.
Aplciamos la propiedad de las integrales considerando subregiones.

\subsection{Cambio de variables en integrales dobles}

Cuando los recintos son parelelogramos o trapecios rotaos,
conviene meter transformación lineal,
cambiano variables para ir a imágenes rectas.

La transformación que se hace es biyectiva: x(u,v) y (u,v).
Condiciones: f contínua en región R,
x e y derivadas parciales contínuas en R,
\(\partial(x,v)/\partial(u,v)\) no nulo en R.

Calculamos el jacobiano de transformación.

Buscamos expresar x e y en u y v.
Porque \textit{literalmente} pasamos de paralelogramo a cuadrado.
Se convierte en una integral recontra fácil.

Hacemos un sistema de ecuaciones en funcíon de u y v.
De ahí el jacobiano y determinante.
Con el determinante hacemos la integral.
Y estamos.

\subsection{Clase que viene}

Trabajamos con cambio de variables a coordenadas polares.
El cambio de variable depende del contexto del problema de integración.

\subsection{Correcciones}

Grupo 1
\begin{enumerate}
    \item [d.] 116/3
    \item [f.] 46/3
    \item [g.] 4/15 (31-...)
    \item [i.] 21/2 ln2
\end{enumerate}

Grupo 2
\begin{enumerate}
    \item [a.] 75/2
    \item [c.] 16
    \item [e.] 36
\end{enumerate}

Grupo 3
\begin{enumerate}
    \item [b.] 9/8
    \item [f.] 1/3
    \item [p.] -12/5
    \item [r.] -4
\end{enumerate}

Grupo 4
\begin{enumerate}
    \item [c.] 31/8
    \item [e.] 9/2
\end{enumerate}
