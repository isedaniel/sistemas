\section{Clase 13 de mayo}

\subsection{Integrales múltiples}


Consideramos integral definida,
acotada para un rectángulo,
bajo la superficie que de un campo escalar.
Es decir, 
un volumen,
que resulta de multiplicar la base rectangular
por la distancia hasta la superficie.

Es decir,
similar a la integral definida con una variable.

Las propiedades también son las mismas:
la integral es distributiva para la suma y la resta de funciones;
si tenemos producto de función por constante podemos sacar la constante;
la integrar una región es igual a integrar dos subregiones;
por último,
si una funcíon es superior a otra en toda la superficie, 
la integral es superior a la otra en toda la superficie.

En cuanto a unidades:
Si la consigna dice solo evaluar: es adimensional.
El \textbf{volumen} resulta del \textbf{contexto} del problema.
La \textbf{unidad} resulta del \textbf{contexto} de problema.

\subsection{Teorema de Fubini}

(Similar a Teorema de Schwarz en derivadas parciales)

Implica que integrar en una variable y después en la otra,
de forma sucesiva,
es igual que hacerlo sucesivamente pero en sentido inverso.

\subsection{Integrales dobles en recintos generales}

Refiere a recintos que,
en lugar de ser rectángulos,
están delimitados por funciones.
En este caso, si tengo funciones que dependen de \(y\) integro por \(x\) primero.
Si tengo funciones que dependen de \(x\), integro por \(y\) primero.
Es decir, que las funciones sean constantes primero.
Ayuda a operar más fácil.

Hay algunos que implica subdivisión si o si:
aplicar propiedad de suma de subregiones.
No hay manera de operar sin subdivisión.

\subsection{Cambio de variables en integrales dobles}

Cuando los recintos son parelelogramos o trapecios rotados,
conviene operar primero transformación lineal,
cambiando variables para ir a recinto rectangular.

La transformación que se hace es biyectiva: x(u,v) y (u,v).
Condiciones necesarias: f contínua en región R,
x e y derivadas parciales contínuas en R,
\(\partial(x,v)/\partial(u,v)\) no nulo en R.

Calculamos el jacobiano de transformación.

Buscamos expresar x e y en u y v.
\textit{Literalmente} pasamos de paralelogramo a cuadrado.
Se convierte en una integral fácil.

Hacemos un sistema de ecuaciones en funcíon de u y v.
De ahí el jacobiano y determinante.
Con el determinante hacemos la integral.
Y estamos.

\subsection{Correcciones para ejercicios}

Grupo 1
\begin{enumerate}
    \item [d.] 116/3
    \item [f.] 46/3
    \item [g.] 4/15 (31-...)
    \item [i.] 21/2 ln2
\end{enumerate}

Grupo 2
\begin{enumerate}
    \item [a.] 75/2
    \item [c.] 16
    \item [e.] 36
\end{enumerate}

Grupo 3
\begin{enumerate}
    \item [b.] 9/8
    \item [f.] 1/3
    \item [p.] -12/5
    \item [r.] -4
\end{enumerate}

Grupo 4
\begin{enumerate}
    \item [c.] 31/8
    \item [e.] 9/2
\end{enumerate}
