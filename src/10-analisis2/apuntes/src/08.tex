\section{Clase 13 de mayo}

Integrales múltiples:
la idea es igual a la integral de una función de 1 sola variable:
la idea de integral, acotada para un rectángulo,
bajo la superficie que devuelve un campo escalar,
es igual al volumen,
que resulta de dicha base rectangular,
por la distancia para cada punto hasta la superficie.

Es decir,
la idea es la misma.

Las propiedades también son las mismas:
la integral es distributiva para la suma y la resta de funciones;
el producto de una función por una constante es igual a la constante por el resultado de la integral
(es decir, puedo sacar la constante de la integral);
la integral de una región es igual a la suma de las integrales de 2 subregiones;
por último,
si una funcíon es superior a otra en toda la superficie, 
la integral es mayor a la otra en toda la superficie.