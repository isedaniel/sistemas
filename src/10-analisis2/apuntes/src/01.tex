\section{Primera clase}

18 de marzo de 2025.

\subsection{Conceptos previos}

\textbf{Espacio métrico.}
Todo espacio que pueda calcular una distancia hablo de espacio métrico.

\textbf{Distancia.}
La distancia entre dos puntos P y Q es un real \textit{no negativo},
denotado \(|P-Q|\),
que cumple con las propiedades:

\begin{align*}
    |P-Q| = 0 \implies P = Q \\
    |P-Q| + |P-R| \geq |Q-R| \\
    |P-Q| = |Q-P|  
\end{align*}

\textbf{Espacio Euclídeo n-dimensional.}
Se denota \(R^{n}\),
y es el conjunto de todos los puntos de un espacio n dimensional,
cuyos elementos se denominan n-uplas.
Ejemplo, \(R^{2}\) indica espacio \textit{bidimensional},
plano,
formado por todos los pares ordenados de números reales.
Ejemplo 2, \(R^{3}\) es el espacio tridimensional,
formado por \textit{ternas} de números reales.
(Funciones de 2 puedo graficar.
De más no.)

Definimos \textbf{distancia euclídea} en \(R^{n}\) 
entre \(P = (x_{1},x_2 \dots x_n)\) y \(Q = (y_{1},y_2 \dots y_n)\) al número:

\begin{align*}
    |P-Q| = \sqrt{(y_1 - x_1)^{2} + (y_2 - x_2)^{2} + \dots + (y_n + x_n)^{2}}
\end{align*}

\subsection{Definiciones de topología}

Si \(A \in R^{n}\), 
se llama \textbf{entorno} de centro \(A\),
radio \(\delta\),
al conjunto de puntos que se encuentran a \textit{distancia} de \(A\) menor a \(\delta\).

El \textbf{entorno reducido},
centro A,
radio \(\delta\),
es el conjunto de puntos \textit{distintos} de \(A\).

Dado el conjunto \(C \subset R^{n}\):

Decimos que \(A\) es \textbf{punto interior} si existe por lo menos un entorno de \(A\)
completamente incluído en \(C\).

Decimos que es \textbf{punto exterior} si existe por lo menos un entorno de \(A\)
que no contiene \textit{ningún} elemento de \(C\).

\(A\) es \textbf{punto frontera} si y solo si no es interior ni exterior.

Es \textbf{punto de acumulación} si no pertenece a \(C\),
pero su entorno tiene intersección no vacía con \(C\).

Decimos que un conjunto es \textbf{abierto} si todos sus puntos son interiores.

Un conjunto es \textbf{cerrado} si su complemento es abierto.

Es \textbf{acotado} si su centro está en el origen.

Es \textbf{conexo} si dos puntos se puede unir
mediante poligonal
de número finito de lados,
cuyos puntos pertenecen \textit{todos}
al conjunto considerado.
(\textit{Criollo:} no tiene agujeros).

\subsection{Funcion de una variable}

Lo que trabajamos en análisis 1.
Regla, 
que asigna a cada \(x \in A\) un único \(y = f(x)\).

Aquí el \textbf{dominio} son puntos de la recta x,
puesto de \(x \in R\).

\subsection{Funciones de varias variables}

El dominio son puntos y la imagen es un número real.
Funciones que asignan a un par ordenado,
pertenciente al plano \((x,y)\) un valor de \(z\),
siendo \(z \in R\), \textit{no un punto}.

A lo sumo, trabajamos \(R^{2}\) o \(R^{3}\).
\(R^{2}\) se puede graficar.
\(R^{3}\) solo el dominio.

Este tipo de función se llama \(campo escalar\).

¿Qué se estudia de una función?

Una cosa es el \textbf{dominio},
es decir, 
el rango de valores de \((x,y)\) para los cuales la función está definida,
devuelve como respuesta un número real.

La \textbf{gráfica} de una función de 2 variables
da una superficie,
puesto que le doy un valor (x,y) y devuelve un real z.

El eje x se representa a \(-135^{\circ}\) del eje y.
(que son 90 + 45).

