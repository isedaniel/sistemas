\section{Primera clase}

18 de marzo, 2025.

\subsection{Conceptos previos}

Denominamos \textbf{espacio métrico}
a aquel espacio para el cual puedo calcular una distancia.

Dados dos puntos P y Q,
decimos que la \textbf{distancia} es un número real \textit{no negativo},
denotado \(|P-Q|\),
que cumple con las propiedades:

\begin{enumerate}
    \item \(|P-Q| = 0 \implies P = Q\)
    \item \(|P-Q| + |P-R| \geq |Q-R|\)
    \item \(|P-Q| = |Q-P|\)
\end{enumerate}

Llamamos \textbf{espacio euclídeo n-dimensional}
al conjunto de todos los puntos de un espacio n-dimensional,
cuyos elementos se denominan n-uplas.
Se denota \(R^{n}\).

Ejemplo, \(R^{2}\) indica espacio \textit{bidimensional},
plano,
formado por todos los pares ordenados de números reales.

Ejemplo 2, \(R^{3}\) es el espacio tridimensional,
formado por \textit{ternas} de números reales.

Definimos \textbf{distancia euclídea} en \(R^{n}\) 
entre \(P = (p_{1},p_2 \dots p_n)\) y \(Q = (q_{1},q_2 \dots q_n)\) al número:

\begin{align*}
    |P-Q| = \sqrt{(q_1 - p_1)^{2} + (q_2 - p_2)^{2} + \dots + (q_n - p_n)^{2}}
\end{align*}

\subsection{Definiciones de topología}

Si \(A \in R^{n}\), 
se llama \textbf{entorno} de centro \(A\),
radio \(\delta\),
al conjunto de puntos que se encuentran a \textit{distancia} de \(A\) menor a \(\delta\).

El \textbf{entorno reducido},
con centro en A,
de radio \(\delta\),
es el conjunto de puntos \textit{distintos} de \(A\),
(sin incluirlo).

Dado el conjunto \(C \subset R^{n}\):

Decimos que \(A\) es \textbf{punto interior} si existe por lo menos un entorno de \(A\)
completamente incluído en \(C\).

Decimos que \(A\) es \textbf{punto exterior} si existe por lo menos un entorno de \(A\)
que no contiene \textit{ningún} elemento de \(C\).

\(A\) es \textbf{punto frontera} si y solo si no es interior ni exterior a \(C\).

Es \textbf{punto de acumulación} si no pertenece a \(C\),
pero su entorno tiene intersección no vacía con \(C\).

Decimos que un conjunto es \textbf{abierto} si todos sus puntos son interiores
(es decir, su frontera no forma parte del conjunto).

Un conjunto es \textbf{cerrado} si su complemento es abierto,
(su frontera forma parte del conjunto).

Un conjunto es \textbf{acotado} si su centro está en el Origen \(O\).

Es \textbf{conexo} si, dados dos puntos cualesquiera,
pertenecientes a \(C\),
es posible unirlos mediante poligonal de número finito de lados,
cuyos puntos pertenecen \textit{todos}
al conjunto considerado.
(no tiene agujeros).

\subsection{Funcion de una variable}

Lo que trabajamos en análisis 1.
Una \textbf{función} es una regla, 
que asigna a cada \(x \in A\) un único \(y = f(x)\).

Aquí el \textbf{dominio} son los reales pertenecientes a la recta x,
puesto de \(x \in R\).

La \textbf{imagen} es el conjunto de los números reales que resultan de \(f(x)\).

\subsection{Funciones de varias variables}

También es una \textit{regla},
cuyo dominio son puntos y su imagen es un número real.

Asignan a un par ordenado,
pertenciente al plano \((x,y)\), 
un valor de \(z\),
siendo \((x,y)\) un punto y \(z\) un número real.

A lo sumo, trabajamos \(R^{2}\) o \(R^{3}\).

\(R^{2}\) se puede graficar.
Para \(R^{3}\) solo podemos graficar el dominio.

Llamamos a este tipo de función \textbf{campo escalar}.

Podemos estudiar el \textbf{dominio},
es decir, 
el rango de valores de \((x,y)\) para los cuales la función está definida,
que devuelven un número real \(z\).

La \textbf{gráfica} de una función de 2 variables
es una superficie.

El eje x se representa a \(-135^{\circ}\) del eje y
(que son 90 + 45).

\subsection{Curvas de nivel}

(O superificies de nivel en \(R^{3}\)).

Dada la función \(z=x^{2}+y^{2}\),
dando valor constante a z,
generamos un circunferencia en el plano.
Me puedo imaginar que la forma es un paraboloide,
porque va de una circunferencia chica,
a una más grande,
a medida que z crece.

\textbf{Las mas conocidas.}
Isobaras, líneas sobre le mapa, misma presión,
isotermas, líneas sobre el mapa, misma temperatura,
líneas equipotenciales,
mapas topográficos.

Dada función de 3 variables,
puedo graficar dominio como superficie.
Se puede imaginar forma de dominio en \(R^{3}\).
\textbf{Solo dominio}.
