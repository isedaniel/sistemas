\section{Primera clase}

\subsection{Conceptos previos}

\textbf{Espacio métrico.}
Todo espacio que pueda calcular una distancia hablo de espacio métrico.

\textbf{Distancia.}
La distancia entre dos puntos P y Q es un real \textit{no negativo},
denotado \(|P-Q|\),
que cumple con las propiedades:

\begin{align*}
    |P-Q| = 0 \implies P = Q \\
    |P-Q| + |P-R| \geq |Q-R| \\
    |P-Q| = |Q-P|  
\end{align*}

\textbf{Espacio Euclídeo n-dimensional.}
Se denota \(R^{n}\),
y es el conjunto de todos los puntos de un espacio n dimensional,
cuyos elementos se denominan n-uplas.
Ejemplo, \(R^{2}\) indica espacio \textit{bidimensional},
plano,
formado por todos los pares ordenados de números reales.
Ejemplo 2, \(R^{3}\) es el espacio tridimensional,
formado por \textit{ternas} de números reales.
(Funciones de 2 puedo graficar.
De más no.)

Definimos \textbf{distancia euclídea} en \(R^{n}\) 
entre \(P = (x_{1},x_2 \dots x_n)\) y \(Q = (y_{1},y_2 \dots y_n)\) al número:

\begin{align*}
    |P-Q| = \sqrt{(y_1 - x_1)^{2} + (y_2 - x_2)^{2} + \dots + (y_n + x_n)^{2}}
\end{align*}
