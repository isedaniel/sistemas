\section{Cuarta clase}

8 de abril, 2025

\subsection{Plano tangente}

Dada función \(z = f(x,y)\), función escalar,
con derivadas parciales contínuas en \((a,b)\).

Decimos que el plano tangente a la superficie en el punto \(P(a, b, f(a,b))\) es el plano que pasa por P y contiene  las rectas tangetes a las dos curvas:

\begin{align*}    
    C_1 \{z = f(x,y) ; x=a\} \\ C_2 \{z = f(x,y) ; y=b\}
\end{align*}

Ahora vamos a ver cómo encontrar un plano tangente.

Recordar que el normal es el producto verctorial de los dos vectores.

\subsection{Función implícita}

Una función está dada de manera explícita cuando y está en función de x.
Hablamos de función implícita cuando: la función se anula en al menos un punto,
posee derivada parcial contínua en torno del punto donde se hace 0 y,
por último,
la derivada parcial con respecto a la variable debe ser distinta de 0.