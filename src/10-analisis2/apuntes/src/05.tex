\section{Quinta clase}

15 de abril, 2025

\subsection{Repaso}

Actividad 13.

Diferencial de volumen, sobre volumen, devuelve error relativo.
Ahora, el diferencial del radio es el 4\% del radio.
Y el diferencial de la altura es el 2\% de la altura.
No lo puedo poner asi nomás.
Como lo incluyo en la expresión del diferencial con la dimensión,
podemos operar algebraicamente y obtener un resultado para el diferencial.

\subsection{Parcial}

Para hacer el parcial: resolver los prácticos.

La próxima clase es clase de consulta: venir con prácticos resueltos y dudas.
La clase dura lo que duren las dudas.

\subsection{Extremos libres locales}

Viendo un dibujo, ¿cómo identifico un máximo?
Porque localmente no hay valores mayores de Z.
Dado un entorno de (x,y), todos los valores de Z que devuelve la función son menores.

Ahora.
Teorema: condición necesaria de extremo relativo. 
Si \(\varphi\) tiene un máximo (o mínimo) relativo,
en un punto \((a,b)\),
y las derivadas parciales de primer orden \textit{existen} en ese punto,
entonces \(f_x(a,b) = 0\) y \(f_y(a,b) = 0\).
Se que tengo un extremo, no se si es máximo, o mínimo.

\subsection{Operando}

Queremos determinar puntos críticos. 
Primero, operamos derivadas parciales.
Igualamos derivadas parciales a 0.

Observemos que para los puntos del eje x, y = 0, entonces \(f(x,y) = -x^2 < 0\),
para todo \(x \neq 0\).

Si, en un entorno alrededor de un punto (a,b)
tenemos un punto en que una derivada es positiva y la otra negativa,
estamos frente a un punto de ensilladura.

Si lo pensamos en términos geométricos, las derivadas de primer orden
permiten obtener las rectas tangentes y determinar el plano tangente.
Un máximo o un mínimo tiene un entorno de la superificie por encima (o por abajo)
del plano tangente.
Sin embargo, en el punto silla el plano no es tangente: 
tiene una parte por debajo y otra parte por arriba.

Sin embargo, frente a un ejercicio no vamos a hacer todo esto:
sacamos primera derivada para saber y segunda para determinar si es máximo o mínimo.
Pero, en análisis de muchas variables tengo muchas derivadas:
por eso entra el determinante hessiano.

\subsection{Determinante Hessiano}

Determinante de la matriz hessiana, formada por las derivadas parciales de segundo orden de una función de varias variables.

