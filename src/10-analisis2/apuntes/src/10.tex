\section{Clase 27 de mayo}

\subsection{Integrales triples}

No puedo graficar lo que quiero integrar:
es un hiperplano, de cuatro dimensiones.
Pero puedo graficar recintos de integración.

Por Teorema de Fubini,
puedo resolver las integrales triples en el orden que quiera,
mismo critero que para integrales dobles.

Idea: tengo que poner dos variables en función de la otra.
Procedimiento:
\begin{enumerate}
    \item Despejo z en función de x e y
    \item Voy al piso xy, despejo y en función de x
    \item x queda en un intervalo
    \item Planteo las funciones y voy resolviendo de \(z\to y\to x\)
\end{enumerate}

También puedo tener volumen con vértices,
tomo vértices como vectores y:
\begin{enumerate}
    \item Producto vectorial para obtener normal
    \item Tomo un punto y genero expresión del plano
    \item Aplico metodología de arriba
\end{enumerate}

\subsection{Coordenadas cilíndricas}

Equivalente a coordenadas polares en integrales dobles.
Aquí se llama cilíndricas.

\begin{align*}
    \begin{cases}
        x = r \cos\theta \\
        y = r \sen\theta \\
        z = z
    \end{cases}
\end{align*}

El jacobiano es \(r\).

\subsection{Coordenadas esféricas}

Las uso cuando mi recinto es:
una esfera o un cono.
Sino uso cilíndricas.

Para determinar los ángulos:
corresponden a cada plano,
considerarlos ahí.
