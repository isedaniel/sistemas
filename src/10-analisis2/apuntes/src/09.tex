\section{Clase 20 de mayo}

\subsection{Repasamos TP de integrales dobles}

Vemos ejercicio 1q.
Integral doble con recinto rectangular.
Planteamos integral doble que se resuelve de forma iterativa.
Por convención, ponemos diferencial \(dx dy\).

\subsection{Integración por partes}

Repasamos integración por partes:
elegimos \(u\) y \(dv\) con regla ILPET:
irracional, logarítmica, polinómica, exponencial, trigonométrica.

Elegidas \(u\) y \(dv\):
calculamos \(du\) y \(v\).
Por último, planteamos \(\int udv = uv - \int v du\).

\subsection{Cambiar variables}

Hacemos unos ejercicios de repaso.
Vemos cómo (y cuándo) se hace el cambio a coordenadas polares.

Vamos a poder terminar la práctica de integrales dobles con esto.

El resto de transformaciones las vamos a conocer,
pero no se suelen aplicar.
Las que vamos a trabajar en profundidad son las coordenadas polares.

\subsection{Coordenadas polares}

Se hace cuando la región de integración es circular.
Apelamos a la transformación:

\begin{align*}
    \begin{cases}
        x = \rho\cos\theta    \\
        y = \rho\sin\theta    \\
        x^{2} + x^{2} = r^{2} \\
    \end{cases}
\end{align*}

El Jacobiano correspondiente a la transformación es operar determinante sobre un cojunto de derivadas parciales.

En general se integra primero el radio y después el ángulo.
Porque el ángulo suele ser numérico.

Si hiciéramos la integral iterativa común,
la que ya conocemos,
sería muy engorroso:
esa es la utilidad de la transformación en coordenadas polares.
Queda prácticamente una integral directa multiplicada por el resultado del jacobiano.