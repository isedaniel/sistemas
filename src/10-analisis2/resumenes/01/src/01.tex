Primer apunte con enfoque estructuras de datos y funciones.

\section{Unidad 1}

\subsection{Campo escalar}

\texttt{CampoEscalar (x,y) => z}

Es una función de dos variables \texttt{xy} que, 
dado un punto \((x,y)\),
devuelve un escalar \(z\).

La función puede estar definida de forma 
\texttt{Explícita f(x,y) = z} o \texttt{Implícita f(x,y,z) = 0}.

\subsection{Dominio}

\texttt{Dom CampoEscalar => \(\{(x,y) \in R^{2} / \text{\texttt{Condicion a cumplir}}\}\)}

Dominio recibe un CampoEscalar y devuelve el conjunto de puntos \((x,y)\)
que el \texttt{CampoEscalar} acepta.

Para determinar dominio:
\begin{enumerate}
    \item Establecemos restricciones
    \item Despejamos una de las variables
    \item Expresamos como conjunto
\end{enumerate}

\textbf{Ejemplo.}

Dado \texttt{CampoEscalar} con función 
\(z = \frac{\sqrt{x+y}}{x-2y}\).

Determinamos restricciones.

\begin{equation*}
    x+y \geq 0 \land x-2y \neq 0
\end{equation*}

Despejamos una de las variables, en este caso y.

\begin{equation*}
    y \geq -x \land y \neq \frac{x}{2}
\end{equation*}

Por lo tanto, \texttt{Dom CampoEscalar} =>

\begin{equation*}
    \{(x,y) \in R^{2} / y \geq -x \land y \neq \frac{x}{2}\}
\end{equation*}

\subsection{Definiciones de topología}

\subsubsection{Entorno}

\texttt{E(A,\(\delta\)) => \(\{x \in R^{n}: |x-A| < \delta\}\)}

Entorno recibe un punto \(A\) y un radio \(\delta\),
y devuelve un conjunto de puntos en \(R^{n}\),
ubicados a una distancia menor a \(\delta\).

\textbf{Ejemplos.}

En \(R\): entorno, con centro en \(A\), en recta real, con radio \(\delta\).

En \(R^{2}\): \textbf{disco}, con centro en \(A\), en plano real, con radio \(\delta\).

En \(R^{3}\): \textbf{esfera}, con centro en \(A\), en el espacio, con radio \(\delta\).

\subsubsection{Entorno reducido}

\texttt{EntornoReducido A, \(\delta\) => E(A,\(delta\) - A)}

Es igual al entorno,
pero devuelve el conjunto de puntos \textbf{sin incluir} \(A\).

\subsubsection{Clasificación de puntos}

Dado conjunto \(C\):

\begin{itemize}
    \item A es \textbf{punto interior} si está incluído en \(C\)
    \item A es \textbf{punto exterior} si está fuera de \(C\)
    \item A es \textbf{punto frontera} si está sobre la frontera de \(C\)
    \item A es \textbf{punto de acumulación} si su entorno es interior o frontera, pero A no pertenece a C.
\end{itemize}

\subsubsection{Clasificación de conjuntos}

Dado conjunto \(C\):

\begin{itemize}
    \item Es \textbf{abierto} si solo tiene puntos interiores.
    \item \textbf{Cerrado} si tiene interiores + frontera.
    \item \textbf{Acotado} si se puede incluir \textit{completamente} en un entorno, de cualquier radio,
    con \(A\) en origen.
\end{itemize}

\subsection{Gráfica de campo escalar}

\texttt{Graf CampoEscalar => superficie en \(R^{3}\)}

La gráfica devuelve todos los puntos \((x,y,z)\) tal que \(z = f(x,y)\).
Es una superficie en \(R^{3}\).

Para determinarla:

\begin{enumerate}
    \item Igualamos a 0 cada uno de sus componentes
    \item Vemos como se comporta el resto
\end{enumerate}

\subsection{Curva de nivel}

Intersección de \texttt{CampoEscalar} con cualquier plano \(z = k\),
paralelo al plano \texttt{xy}, siendo \(k\) una constante.

Para determinarla:

\begin{enumerate}
    \item Igualamos z a un valor
    \item Despejamos alguna de las variables que queda y obtenemos una en función de otra
    \item Viendo cómo se comporta se obtiene una idea de lo que hacer el \texttt{CampoEscalar}
\end{enumerate}

\subsection{Cuádricas, cónicas, cilindros}

