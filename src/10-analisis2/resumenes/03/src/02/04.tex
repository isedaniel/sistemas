\subsection{Cambio de variable en integrales dobles}

Cuando tengo recintos que son paralelogramos o trapecios rotados,
conviene aplicar una transformación lineal para resolver la integral.

Las condiciones para aplicar este método son:

\begin{align*}
    \begin{cases}
        \text{1. La función es contínua en R}                              \\
        \text{2. x e y tienen derivadas parciales contínuas en el recinto} \\
        \text{2. el jacobiano es no nulo en el recinto}                    \\
    \end{cases}
\end{align*}

Si se dan estas tres condiciones,
podemos expresar la integral doble del recinto R en términos del recinto \(R'\),
siguiendo:

\begin{align*}
    \int\int_R\,f(x,y)\,dA = \int\int_R'\,f(x(u,v),y(u,v))\left|\frac{\partial(x,y)}{\partial(u,v)}\right|\,du\,dv
\end{align*}

\subsubsection{Ejemplo 1}

Hallar área delimitada por:
\begin{align*}
    \begin{cases}
        x-2y=4 \\
        x-2y=0 \\
        x+y=4  \\
        x+y=1  \\
    \end{cases}
\end{align*}