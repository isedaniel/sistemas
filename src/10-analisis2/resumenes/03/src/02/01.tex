\setcounter{section}{3}

\section{Unidad 4}

\subsection{Integrales múltiples}

La integral definida es el \textit{volumen} del sólido \(S\),
desde la base en el recinto \(R\),
hasta la superficie \(z = f(x,y)\).

\begin{equation*}
    \int\int f(x,y)\,dA_R = V_S
\end{equation*}

Si \(R = [a,b] \times [c,d]\), la podemos expresar como:

\begin{equation*}
    \int_{c}^{d}\int_{a}^{b} f(x,y) \,dx\,dy
\end{equation*}

\subsection{Propiedades de las integrales múltiples}

Dadas \(f\) y \(g\) funciones de dos variables,
siendo ambas \textit{integrables} en el recinto \(R\):

1. La integración múltiple es distributiva respecto de la suma y resta de funciones:

\begin{equation*}
    \int\int [f \pm g] \,dA = \int\int [f] \;dA \pm \int\int [g] \;dA
\end{equation*}

2. Si tengo producto de función por \(k\) constante, 
\textit{puedo sacar la constante}:

\begin{equation*}
    \int\int [kf] \; dA = k \int\int f \; dA
\end{equation*}

3. La integral definida para el recinto \(R\) 
es igual a la suma de las integrales para 
los \textit{subrecintos} \(R_1\) y \(R_2\):

\begin{equation*}
    \int\int_R [f] \; dA = \int\int_{R_1} [f] \; dA + \int\int_{R_2} [f] \; dA
\end{equation*}

4. Si \(f(x,y)\) esta por encima de \(g(x,y)\) para todo \(R\), 
entonces la integral también va a estar por encima.
