Primer apunte con enfoque estructuras de datos y funciones.

\section{Unidad 1}

\subsection{Campo escalar}

\texttt{CampoEscalar (x,y) => z}

Es una función de dos variables \texttt{xy} que,
dado un punto \((x,y)\),
devuelve un escalar \(z\).

La función puede estar definida de forma
\texttt{Explícita f(x,y) = z} o \texttt{Implícita f(x,y,z) = 0}.

\subsection{Dominio}

\texttt{Dom CampoEscalar => \(\{(x,y) \in R^{2} / \text{\texttt{Condicion a cumplir}}\}\)}

Dominio recibe un CampoEscalar y devuelve el conjunto de puntos \((x,y)\)
que el \texttt{CampoEscalar} acepta.

Para determinar dominio:
\begin{enumerate}
    \item Establecemos restricciones
    \item Despejamos una de las variables
    \item Expresamos como conjunto
\end{enumerate}

\textbf{Ejemplo.}

Dado \texttt{CampoEscalar} con función
\(z = \frac{\sqrt{x+y}}{x-2y}\).

Determinamos restricciones.

\begin{equation*}
    x+y \geq 0 \land x-2y \neq 0
\end{equation*}

Despejamos una de las variables, en este caso y.

\begin{equation*}
    y \geq -x \land y \neq \frac{x}{2}
\end{equation*}

Por lo tanto, \texttt{Dom CampoEscalar} =>

\begin{equation*}
    \{(x,y) \in R^{2} / y \geq -x \land y \neq \frac{x}{2}\}
\end{equation*}

\subsection{Definiciones de topología}

\subsubsection{Entorno}

\texttt{E(A,\(\delta\)) => \(\{x \in R^{n}: |x-A| < \delta\}\)}

Entorno recibe un punto \(A\) y un radio \(\delta\),
y devuelve un conjunto de puntos en \(R^{n}\),
ubicados a una distancia menor a \(\delta\).

\textbf{Ejemplos.}

En \(R\): entorno, con centro en \(A\), en recta real, con radio \(\delta\).

En \(R^{2}\): \textbf{disco}, con centro en \(A\), en plano real, con radio \(\delta\).

En \(R^{3}\): \textbf{esfera}, con centro en \(A\), en el espacio, con radio \(\delta\).

\subsubsection{Entorno reducido}

\texttt{EntornoReducido A, \(\delta\) => E(A,\(delta\) - A)}

Es igual al entorno,
pero devuelve el conjunto de puntos \textbf{sin incluir} \(A\).

\subsubsection{Clasificación de puntos}

Dado conjunto \(C\):

\begin{itemize}
    \item A es \textbf{punto interior} si está incluído en \(C\)
    \item A es \textbf{punto exterior} si está fuera de \(C\)
    \item A es \textbf{punto frontera} si está sobre la frontera de \(C\)
    \item A es \textbf{punto de acumulación} si su entorno es interior o frontera, pero A no pertenece a C.
\end{itemize}

\subsubsection{Clasificación de conjuntos}

Dado conjunto \(C\):

\begin{itemize}
    \item Es \textbf{abierto} si solo tiene puntos interiores.
    \item \textbf{Cerrado} si tiene interiores + frontera.
    \item \textbf{Acotado} si se puede incluir \textit{completamente} en un entorno, de cualquier radio,
          con \(A\) en origen.
\end{itemize}

\subsection{Gráfica de campo escalar}

\texttt{Graf CampoEscalar => superficie en \(R^{3}\)}

La gráfica devuelve todos los puntos \((x,y,z)\) tal que \(z = f(x,y)\).
Es una superficie en \(R^{3}\).

Para determinarla:

\begin{enumerate}
    \item Igualamos a 0 cada uno de sus componentes
    \item Vemos como se comporta el resto
\end{enumerate}

\subsection{Curva de nivel}

Intersección de \texttt{CampoEscalar} con cualquier plano \(z = k\),
paralelo al plano \texttt{xy}, siendo \(k\) una constante.

Para determinarla:

\begin{enumerate}
    \item Igualamos z a un valor
    \item Despejamos alguna de las variables que queda y obtenemos una en función de otra
    \item Viendo cómo se comporta se obtiene una idea de lo que hacer el \texttt{CampoEscalar}
\end{enumerate}

\subsection{Cuádricas, cónicas, cilindros}

\subsubsection{Cilindro de z arbitrario}

Forma: \(x^{2} + y^{2} = r^{2}\)

Es una circunferencia en \(R^{2}\)

\subsubsection{Plano}

\(y + 2z = 2\), recta en \(R^{2}\), plano con \(x\) arbitraria en \(R^{3}\).

\subsubsection{Cilindro}

Toda superficie con líneas paralelas a una generatriz,
curvas que recorren una curva plana.

\(ax^{2} + by^{2} = r^{2}\), igual a círculo,
pero con \(a \neq b\), tenemos elipse en el plano,
cilindro elíptico en el espacio.

\(z = x^{2}\), parábola en plano \(xz\),
cilindro parabólico en el espacio.

\subsubsection{Cuadricas. Elipsoide.}

\begin{equation*}
    \frac{x^{2}}{a^{2}} + \frac{y^{2}}{b^{2}} + \frac{z^{2}}{c^{2}} = r^{2}
\end{equation*}

Si \(a = b = c\), esfera, sino elipsoide.
Para graficar,
como siempre, igualamos cada una a 0 y vemos como queda en cada plano,
para darnos una idea.

\subsubsection{Paraboloide}

\begin{equation*}
    \frac{z}{c} = \frac{x^{2}}{a^{2}} + \frac{y^{2}}{b^{2}}
\end{equation*}

Atención a \texttt{Grado z => 1}.

\subsubsection{Cono}

Igual estructura al anterior, pero \texttt{Grado z => 2}:

\begin{equation*}
    \frac{z^{2}}{c^{2}} = \frac{x^{2}}{a^{2}} + \frac{y^{2}}{b^{2}}
\end{equation*}

\subsubsection{Hiperboloide de una hoja}

Igual al anterior, pero \(\exists\) \texttt{CoeficienteIndependiente}.

\subsubsection{Hiperboloide de una hoja}


\subsection{Límites Campos Escalares}

En una variable:

\texttt{Lim x, a, f(x) => L \(\iff\) f(x) tiende a L por izquierda y derecha}

En funciones de 2 variables \texttt{CampoEscalar} es la misma idea,
pero \texttt{Entorno (x,y) => } Disco,
por lo tanto no solo me acerco por izquierda y derecha.

\subsubsection{Propiedades}

\begin{enumerate}
    \item Distributivo respecto de la suma
    \item Respecto del producto
    \item Lim x cte x f = cte x L
    \item Distributivo respecto de cociente, siempre que denominador \(\neq\) 0
    \item Respecto de potencia
\end{enumerate}

\subsection{Cálculo de límites en dos variables}

\subsubsection{Límite doble o simultáneo}

Reemplazamos, operamos algebraicamente y llegamos a un resultado.

\subsubsection{Límite iterado}

Hacemos primero con una y después con otra:

\texttt{Lim x, a, f(x,y) => \(f_x\)}.

\texttt{Lim y, b, \(f_x\) => \(L_1\)}.

Hacemos el otro camino también,
que devuelve \(L_2\).

Si \(L_1 = L_2 = L\), suponemos \(\exists L\) hasta que se demuestre lo contrario (hay que seguir)

\subsubsection{Límites radiales, parabólicos, cúbicos, etc.}

Igual al iterado, pero con:

\begin{enumerate}
    \item \(y = mx\)
    \item \(y = ax^{2}\)
    \item \(y = ax^{3}\)
\end{enumerate}

Y sus respectivas variantes en función de \(x\).
Si aparece \(L\) distinto, concluimos que \(\nexists L\).

\subsubsection{Límites en coordenadas polares}


Se utiliza cuando aparece \(x^{2} + y^{2}\) en el denominador.

Sustituimos:
\begin{enumerate}
    \item \(x = \rho \cos\theta\)
    \item \(y = \rho \sin\theta\)
    \item \(x^{2} + y^{2} = \rho^{2}\)
\end{enumerate}

Operamos algebraicamente y buscamos la función \(F(\rho)\),
es decir, que quede en función de \(\rho\) y no de \(\theta\).

Operamos \(\lim_{\rho\to0} F(\rho)\).

Como \(f(\rho\cos\theta,\rho\sin\theta) < F(\rho) \implies \lim_{(x,y)\to(0,0)}f(x,y) = L\).

\textbf{Ejemplo.}

Dado \(\lim_{(x,y)\to(9,9)}\frac{x^{3} + x^{2} - 2y^{3} + y^{2}}{x^{2} + y^{2}}\).

\texttt{Lim \(\to\) (0,0)} y además \texttt{\(x^{2} + y^{2}\)} en denominador \(\implies\) calculamos con coordenadas polares:

Sustituimos:

\begin{equation*}
    f(\rho\cos\theta,\rho\sin\theta) = \frac{\rho^{3}\cos^{3}\theta + \rho^{2}\cos^{2}\theta - 2\rho^{3}\sin^{3}\theta + \rho^{2}\sin^{2}\theta}{\rho^{2}}
\end{equation*}

Operamos algebraicamente, sacamos \(\rho^{2}\) factor común para que quede una sola función:

\begin{equation*}
    \frac{\cancel{\rho^{2}}(\rho\cos^{3}\theta + \cos^{2}\theta - 2\rho\sin^{3}\theta + \sin^{2}\theta)}{\cancel{\rho^{2}}}
\end{equation*}

Simplificamos \(\cos^{2}\theta + \sin^{2}\theta = 1\)

\begin{equation*}
    \rho\cos^{3}\theta - 2\rho\sin^{3}\theta + 1
\end{equation*}

Operamos el \(\lim_{\rho\to0}\)

\begin{equation*}
    \lim_{\rho\to0}\rho\cos^{3}\theta - 2\rho\sin^{3}\theta + 1 = \boxed{1}
\end{equation*}

Llegamos a un valor, para comprobarlo buscamos \(F(\rho)\),
igualando expresión de \(L\) a 0:

\begin{align*}
    \rho\cos^{3}\theta - 2\rho\sin^{3}\theta + 1     & = 1 \\
    \rho\cos^{3}\theta - 2\rho\sin^{3}\theta + 1 - 1 & = 0 \\
    \rho\cos^{3}\theta - 2\rho\sin^{3}\theta         & = 0 \\
\end{align*}

Sobre la expresión resultante aplicamos módulo:

\begin{align*}
    |\rho\cos^{3}\theta - 2\rho\sin^{3}\theta|                 & \leq |\rho\cos^{3}\theta| + |2\rho\sin^{3}\theta|            \\
    |\rho\cos^{3}\theta| + |2\rho\sin^{3}\theta|               & = |\rho|\cdot|\cos^{3}\theta| + |2\rho|\cdot|\sin^{3}\theta| \\
    |\rho|\cdot|\cos^{3}\theta| + |2\rho|\cdot|\sin^{3}\theta| & \leq |\rho| + |2\rho|                                        \\
    \boxed{F(\rho) = 3\rho}
\end{align*}

Operamos límite tendiendo a 0 de \(F(\rho)\)

\begin{align*}
    \lim_{\rho\to0} 3\rho = \boxed{0}
\end{align*}

Como límite tiende a 0, por lo tanto, límite es igual al 1 encontrado con anterioridad.

\subsection{Continuidad de funciones de dos variables}

\subsubsection{Contínua en un punto}

\texttt{Continua f(x,y), (x0,y0) => \(\{True, False\}\)}

Continua recibe \texttt{CampoEscalar} y un punto \texttt{P = (x0,y0)},
devuelve continuidad verdadera si:

\begin{enumerate}
    \item \texttt{f(x0,y0) => z0}
    \item \texttt{lim (x,y), (x0,y0), f(x,y) => L}
    \item \texttt{L == x0}
\end{enumerate}

\subsubsection{Función contínua}

\texttt{Continua f(x,y) => \(\{True, False\}\)}

Que sea contínua depende del polinomio del \texttt{CampoEscalar},
si no tiene discontinuidades en su dominio es contínua.

\textbf{Ejemplo.}
\(\sqrt{x+y}\) es contínua en su dominio \(y \geq -x\).