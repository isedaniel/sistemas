\section{Unidad 1}

\subsection{Campo escalar}

Es, básicamente, una función de dos variables que,
dado un punto \((x,y)\),
devuelve un escalar \(z\).

La función puede estar definida de forma:
\begin{itemize}
    \item Explícita: \(f(x,y) = z\)
    \item Implícita: \(f(x,y,z) = 0\)
\end{itemize}

\subsection{Dominio de un campo escalar}

\(Dom \, f(x,y) = \{(x,y) \in R^{2} / \, \text{Condición}\}\)

El dominio de un campo escalar es el conjunto de puntos para los cuales el campo
escalar está definido.

\subsubsection{Procedimiento}

\begin{enumerate}
    \item Determinamos las restricciones (puntos para los cuales la función
          no está definida)
    \item Despejamos una de las variables para cada una de las restricciones
    \item Expresamos como definición de un conjunto
\end{enumerate}

\subsubsection{Ejemplo}

Dado \(z = \frac{\sqrt{x+y}}{x-2y}\):

1. Determinamos restricciones.

\begin{equation*}
    x+y \geq 0 \quad \land \quad x-2y \neq 0
\end{equation*}

2. Despejamos una de las variables, por ejemplo, \(y\).

\begin{equation*}
    y \geq -x \quad \land \quad y \neq \frac{x}{2}
\end{equation*}

3. Expresamos como conjunto.

\begin{equation*}
    \text{Dom } f(x,y) = \{(x,y) \in R^{2} / y \geq -x \quad \land \quad y \neq \frac{x}{2}\}
\end{equation*}

Si quisiera representar gráficamente, grafico las fronteras que se deducen de
las desigualdades y marco en consecuencia: para arriba y frontera si es mayor o
igual, etc.

\subsection{Definiciones de topología}

Conceptos que se usan a lo largo de la materia y es importante conocer.

\subsubsection{Entorno}

\(E(A,\delta) = \{x \in R^{n}: |x-A| < \delta\}\)

Dado un punto \(A\) y un radio \(\delta\), llamamos entorno al conjunto de
puntos \(\in R^{n}\) ubicados a una distancia menor a \(\delta\).

\textbf{Ejemplos.}

\begin{itemize}
    \item En \(R\): \textit{entorno}, con centro en \(A\), sobre la recta real,
          con radio \(\delta\).
    \item En \(R^{2}\): \textit{disco}, con centro en \(A\), sobre el
          \textit{plano} real, con radio \(\delta\).
    \item En \(R^{3}\): \textit{esfera}, con centro en \(A\), en el
          \textit{espacio}, con radio \(\delta\).
\end{itemize}


\subsubsection{Entorno reducido}

Es igual al entorno,
pero el conjunto de puntos resultante \textbf{no incluye a} \(A\):

\(E_R(A,\delta) = \{x \in R^{n}: |x-A| < \delta \quad \land \quad x \neq A\}\)


\subsubsection{Clasificación de puntos}

Dado un conjunto \(C\):

\begin{itemize}
    \item A es \textbf{punto interior} si está incluído en \(C\)
    \item A es \textbf{punto exterior} si está fuera de \(C\)
    \item A es \textbf{punto frontera} si está sobre la frontera de \(C\)
    \item A es \textbf{punto de acumulación} si su entorno es interior o
          frontera, pero A no pertenece a C (Creo que se podría decir que el entorno
          reducido de A pertenece a C).
\end{itemize}

\subsubsection{Clasificación de conjuntos}

Dado un conjunto \(C\), decimos que:

\begin{itemize}
    \item Es \textbf{abierto} si solo tiene puntos interiores y no tiene
          frontera.
    \item Es \textbf{Cerrado} si tiene puntos interiores y frontera.
    \item Es \textbf{Acotado} si se puede incluir \textit{completamente} en un
          entorno, de cualquier radio, con \(A\) en origen.
\end{itemize}

\subsection{Gráfica de campo escalar}

La representación gráfica del campo escalar \(z = f(x,y)\).
Es una superficie en \(R^{3}\).

Para determinarla:

\begin{enumerate}
    \item Igualamos a 0 cada uno de sus componentes
    \item Vemos como se comporta el resto
\end{enumerate}

\subsection{Curva de nivel}

Intersección de un campo escalar con cualquier plano \(z = k\), paralelo al
plano \(xy\), siendo \(k\) una constante.

Para determinar una curva de nivel:
\begin{enumerate}
    \item Igualamos z a un valor \(k\)
    \item Despejamos alguna de las variables restantes y obtenemos una función
          de una variable
    \item Analizamos el comportamiento de esa función
\end{enumerate}

\subsection{Cuádricas, cónicas, cilindros}

\subsubsection{Cilindro de z arbitrario}

Forma: \(x^{2} + y^{2} = r^{2}\)

Gráficamente forma una circunferencia en \(R^{2}\),
de radio \(r\).

\subsubsection{Plano}

\(y + 2z = 2\), recta en \(R^{2}\),
y un plano con \(x\) arbitraria en \(R^{3}\).

\subsubsection{Cilindro}

Se denomina cilindro a una superficie generada a partir de una línea recta,
llamada \textit{generatriz},
que se mueve manteniéndose paralela a una dirección,
mientras recorre una curva fija, que se llama \textit{directriz}.

La expresión \(ax^{2} + by^{2} = r^{2}\),
con \(a \neq b\), describe una elipse en el plano,
que, con \(z\) arbitraria,
obtendríamos un cilindro elíptico en el espacio.

\(z = x^{2}\) describe una parábola en el plano \(xz\),
una curva que funciona como \textit{directriz}.
En el espacio, con \(y\) arbitraria, obtenemos un cilindro parabólico.

\subsubsection{Cuadricas}

Una cuádrica es una superficie definida por una ecuación de segundo grado en las
variables \(x, y, z\), de forma general:

\begin{equation*}
    ax^{2} + by^{2} + cz^{2} + dxy + eyz + fzx + gx + hy + jz + k = 0
\end{equation*}

Siendo \(a, b, c, d, f, g, h, j, k\) constantes.
Los términos en \(xy,\,yz\,,\,xz\) indican rotaciones.
Los términos lineales indican traslaciones.

\subsubsection{Elipsoide}

Caso particular de cuádrica, que sigue la expresión:

\begin{equation*}
    \frac{x^{2}}{a^{2}} + \frac{y^{2}}{b^{2}} + \frac{z^{2}}{c^{2}} = r^{2}
\end{equation*}

Sus trazas\footnote{Las curvas que resultan de su intersección con un plano \(z = cte\)}
forman elipses.
Si tenemos \(a = b = c\), el resultado es un caso particular de elipsoide:
la esfera.
Para darnos una idea del gráfico podemos igualar cada uno de los componentes
\(x, y, z\) a 0, obteniendo elipses para los planos \(xy\), etc.


\subsubsection{Paraboloide}

Es otro tipo de cuádrica, que sigue la expresión:

\begin{equation*}
    \frac{z}{c} = \frac{x^{2}}{a^{2}} + \frac{y^{2}}{b^{2}}
\end{equation*}

Atención al \(\text{grado }z = 1\).
Con \(z = cte\), las trazas son elipses;
mientras que, \(x = 0\) o \(y = 0\) devuelve parábolas.

\subsubsection{Cono}

Otra cuádrica, con expresión:

\begin{equation*}
    \frac{z^{2}}{c^{2}} = \frac{x^{2}}{a^{2}} + \frac{y^{2}}{b^{2}}
\end{equation*}

Notar que es estructuralmente igual a la expresión anterior, salvo que el
\(\text{Grado }z = 2\).
Las trazas con \(z = cte\) son elipses y las trazas \(x = 0\) e \(y = 0\) son
lineales.

\subsubsection{Hiperboloide de una hoja}

Igual al anterior, pero con \textit{coeficiente independiente}:.

\begin{equation*}
    \frac{x^{2}}{a^{2}} + \frac{y^{2}}{b^{2}} - \frac{z^{2}}{c^{2}} = \boxed{1}
\end{equation*}

El \textit{término negativo} indica el eje que funciona como \textit{eje de
    simetría}, en el ejemplo, el eje \(z\). \(z = k\) son elipses; \(x=0\) e \(y=0\)
son \textit{hipérbolas}.

\subsubsection{Hiperboloide de dos hojas}

Sigue la ecuación:

\begin{equation*}
    -\frac{x^{2}}{a^{2}} - \frac{y^{2}}{b^{2}} + \frac{z^{2}}{c^{2}} = \boxed{1}
\end{equation*}

Notar que es igual a la cuádrica anterior, con la salvedad que tiene
\textit{dos} términos negativos. En este caso, es el término positivo el que
marca el \textit{eje de simetría}.

\subsection{Límites de Campos Escalares}

Recordamos la definición de límite para funciones de \textit{una variable}:

\begin{equation*}
    \lim_{x\to a} f(x) = L \iff \text{ tiende a L por izquierda y derecha}
\end{equation*}

En funciones de 2 variables la idea es la misma, pero considerando que el
\textit{entorno} de \(a\) es un \textit{disco}, por lo que no me acerco
\textit{solo} por izquierda y derecha.

\subsubsection{Propiedades}

Dadas \(g(x,y) = z\) y \(f(x,y) = z\):

\begin{enumerate}
    \item El límite es distributivo respecto de la suma:
          \begin{equation*}
              \lim \left[g + f\right] = \lim g + \lim f = L + M
          \end{equation*}

    \item Es distributivo respecto del producto:
          \begin{equation*}
              \lim \left[g \cdot f\right] = \lim g(x,y) \cdot \lim f(x,y) = L\cdot M
          \end{equation*}

    \item Límite de constante es la constante:
          \begin{equation*}
              \lim k\cdot f = k\cdot\lim f = kL
          \end{equation*}

    \item Distributivo respecto de cociente, siempre que denominador \(\neq\) 0:
          \begin{equation*}
              \lim \left[f/g\right] = L/M \iff M \neq 0
          \end{equation*}

    \item Respecto de potencia
          \begin{equation*}
              \lim [f]^{m/n} = L^{m/n}
          \end{equation*}
\end{enumerate}


\subsection{Límite doble o simultáneo}

Es el límite en que las dos variables tienden juntas al punto en cuestión:
\begin{equation*}
    \lim_{(x,y)\to(a,b)} f(x,y) = L
\end{equation*}

Para \textit{probar la existencia} de un límite deberíamos acercarnos desde
todos los caminos posibles,
pero esta tarea es \textit{imposible}.
Adicionalmente,
demostrar existencia por definicion es, por lo general, muy difícil.

Por eso,
resulta más sencillo probar la \textit{no existencia} del límite.
Con este objetivo nos aproximamos por varios caminos,
si alguno resulta en una \(L\) distinta concluimos no existencia.
Si por todos los caminos llegamos al mismo límite,
podemos \textit{sospechar} existencia,
aunque \textit{no la podemos asegurar}.


\subsection{Cálculo de límites en funciones de dos variables}

\subsubsection{Salvando indeterminación por medios algebraicos}

Primero, sustituimos en la función y vemos si tiende hacia un valor.

\textbf{Ejemplo.}

\begin{align*}
    \lim_{(x,y) \to (0,1)} \frac{x - xy + 3}{x^{2}y + 5xy - y^{3}} & = \\
    \frac{0 - 0\cdot1 + 3}{0^{2} + 5\cdot0\cdot1 - 1^{3}}          & = \\
    \frac{3}{-1} = \boxed{-3}                                          \\
\end{align*}

Si encontramos una indeterminación,
tratamos de salvarla por medios algebraicos:

\begin{align*}
    \lim_{(x,y) \to (0,0)}\frac{x^{2} - xy}{\sqrt{x}-\sqrt{y}}
\end{align*}

Para salvar la indeterminación tenemos que \textit{racionalizar el denominador},
multiplicando numerador y denominador por el \textit{conjugado del denominador}:

\begin{align*}
    \lim_{(x,y) \to (0,0)}\frac{x^{2} - xy}{\sqrt{x}-\sqrt{y}}      & =           \\
    \lim_{(x,y) \to (0,0)}\frac{x^{2} - xy}{\sqrt{x}-\sqrt{y}} \cdot
    \frac{\sqrt{x} + \sqrt{y}}{\sqrt{x} + \sqrt{y}}                 & =           \\
    \lim_{(x,y) \to (0,0)}
    \frac{x \cancel{(x - y)} (\sqrt{x} + \sqrt{y})}{\cancel{x - y}} & =           \\
    (0)(\sqrt{0} + \sqrt{0})                                        & = \boxed{0}
\end{align*}

En caso que la indeterminación no se salve por medios algebraicos,
tratamos de calcular los límites acercándonos por distintas curvas.
En cuanto alguno de diferente,
\textit{aseguramos la no existencia del límite}.

\subsubsection{Límite iterado}

Calculamos el límite para una sola de las variables,
considerando la otra como constante.
Seguidamente, calculamos por la segunda variable.

\begin{align*}
    \lim_{(x,y) \to (a,b)} f =
    \begin{cases}
        \lim_{y \to b}\left[ \lim_{x \to a} f \right] = L_1 \\
        \lim_{a \to b}\left[ \lim_{y \to b} f \right] = L_2 \\
    \end{cases}
\end{align*}

Si \(L_1 = L_2 = L\),
supondremos que \(L\) existe hasta que se demuestre lo contrario.

\textbf{Ejemplo.}

\begin{align*}
    \lim_{(x,y) \to (0,0)} \frac{x + y}{x - y} \\
\end{align*}

No podemos salvar la \textit{indeterminación por medios algebraicos},
hacemos primero un límite y luego el otro:

\begin{align*}
    \lim_{x \to 0} \frac{x + y}{x - y} & = \frac{y}{-y} = \boxed{-1}
\end{align*}

Probamos con el otro:

\begin{align*}
    \lim_{y \to 0} \frac{x + y}{x - y} & = \frac{x}{x} = \boxed{1}
\end{align*}

Obtenemos resultados distintos mediante los límites iterados,
por lo tanto,
podemos \textit{asegurar} el límite de esta función tendiendo a \((0,0)\)
no existe.

\subsubsection{Límites radiales, parabólicos, cúbicos, etc.}

Funcionan como un acercamiento al límite desde distintas curvas:

\begin{itemize}
    \item Recta que pasa por el origen: \(y = mx\)
    \item Recta que pasa por \((a,b)\): \(y = m(x - a) + b\)
    \item Parábola: \(y = ax^{2}\)
    \item Cúbica: \(x = ay^{3}\)
    \item Etc.
\end{itemize}

Se opera como si fuese un límite iterado,
pero con una de las variables tendiendo a la función elegida.
Si el límite queda en función de la variable introducida,
concluimos que el límite no existe:

\begin{align*}
    \lim_{(x,y) \to (0,0)} \frac{xy}{x^{2} + y^{2}}
\end{align*}

Si sustituimos, tenemos una \textit{indeterminación}.
No podemos salvarla algebraicamente.
Los límites iterados,
por la estructura del numerador,
dan 0 los dos.

Probamos acercándonos con \(y=mx\):

\begin{align*}
    \lim_{x \to 0} \left( \lim_{y \to mx} \frac{xy}{x^{2} + y^{2}} \right) \\
    \lim_{x \to 0} \frac{x \cdot mx}{x^{2} + (mx)^{2}}                     \\
    \lim_{x \to 0} \frac{\cancel{x^{2}}\cdot m}{\cancel{x^{2}} (1 + m)}    \\
    \boxed{\frac{m}{1 + m}}
\end{align*}

Como la expresión queda en función de \(m\),
concluimos que el límite con \((x,y) \to (0,0)\) \textit{no existe}.

\subsubsection{Límites en coordenadas polares}

Se utiliza cuando la expresión tiene \(x^{2} + y^{2}\) en el denominador.

Sustituimos:
\begin{align*}
    \begin{cases}
        x = \rho \cos\theta \\
        y = \rho \sin\theta \\
        x^{2} + y^{2} = \rho^{2}
    \end{cases}
\end{align*}

Operamos tratando de que la expresión quede en función de \(\rho\)
y no de \(\theta\),
es decir,
\(F(\rho)\).
Para buscar \(F(\rho)\):
1. igualamos la expresión a la que llegamos a 0,
y 2. tomamos su módulo.

Luego, operamos \(\lim_{\rho\to0} F(\rho)\).

Ya que
\(f(\rho\cos\theta,\rho\sin\theta) < F(\rho)\)
concluimos que \(\lim_{(x,y)\to(0,0)}f(x,y) = L\).

\vspace{.5cm}
\textbf{Ejemplo.}

Dado:

\begin{align*}
    \lim_{(x,y)\to(9,9)}\frac{x^{3} + x^{2} - 2y^{3} + y^{2}}{x^{2} + y^{2}}
\end{align*}

Como tenemos \(x^{2} + y^{2}\) en el denominador,
recurrimos al cálculo con coordenadas polares.
Sustituimos \(x = \rho\cos\theta\),
\(y = \rho\sen\theta\) y \(x^{2} + y^{2} = \rho^{2}\):

\begin{align*}
    \frac{(\rho\cos\theta)^{3} + (\rho\cos\theta)^{2} - 2(\rho\sen\theta)^{3} +
    (\rho\sen\theta)^{2}}{\rho^{2}} \\
    \frac{\rho^{3}\cos^{3}\theta + \rho^{2}\cos^{2}\theta - \cdots}{\rho^{2}}
\end{align*}

Sacamos \(\rho^{2}\) factor común para simplificar el denominador:

\begin{equation*}
    \frac{\cancel{\rho^{2}}(\rho\cos^{3}\theta + \cos^{2}\theta - 2\rho\sin^{3}\theta + \sin^{2}\theta)}{\cancel{\rho^{2}}}
\end{equation*}

Sabemos por identidad trigonométrica que
\(\cos^{2}\theta + \sin^{2}\theta = 1\),
por lo tanto:

\begin{equation*}
    \rho\cos^{3}\theta - 2\rho\sin^{3}\theta + 1
\end{equation*}

Si operamos \(\lim_{\rho\to0}\):

\begin{equation*}
    \lim_{\rho\to0}\rho\cos^{3}\theta - 2\rho\sin^{3}\theta + 1 = \boxed{1}
    \quad\forall\quad \theta \in [0,2\pi)
\end{equation*}

Es decir,
con independencia del valor de \(\theta\),
el límite sería 1.

Para buscar \(F(\rho)\), igualamos la función a 0:

\begin{align*}
    \rho\cos^{3}\theta - 2\rho\sin^{3}\theta + 1                       & = 1 \\
    \rho\cos^{3}\theta - 2\rho\sin^{3}\theta + \cancel{1} - \cancel{1} & = 0 \\
    \rho\cos^{3}\theta - 2\rho\sin^{3}\theta                           & = 0 \\
\end{align*}

Operamos módulo y vamos acotando a su límite superior:

\begin{align*}
    |\rho\cos^{3}\theta - 2\rho\sin^{3}\theta|                 & \leq |\rho\cos^{3}\theta| + |2\rho\sin^{3}\theta|            \\
    |\rho\cos^{3}\theta| + |2\rho\sin^{3}\theta|               & = |\rho|\cdot|\cos^{3}\theta| + |2\rho|\cdot|\sin^{3}\theta| \\
    |\rho|\cdot|\cos^{3}\theta| + |2\rho|\cdot|\sin^{3}\theta| & \leq |\rho| + |2\rho|                                        \\
    \boxed{F(\rho) = 3\rho}
\end{align*}

Entonces, a lo sumo va a valer \(3\rho\).
Operamos límite tendiendo a 0 de \(F(\rho)\)

\begin{align*}
    \lim_{\rho\to0} 3\rho = \boxed{0}
\end{align*}

Como límite tiende a 0,
por lo tanto,
límite es igual al 1 encontrado con anterioridad.

Si al operar coordenadas polares llegamos a una expresión en función de
\(\theta\),
concluimos que el límite \textit{no existe}.

\subsection{Continuidad de funciones de dos variables}

\subsubsection{Continuidad en un punto}

Decimos que una función de dos variables es continua en \((a,b)\) si:

\begin{enumerate}
    \item Se puede evaluar en \((a,b)\): \(f(a,b) = c\)
    \item Existe límite en \((a,b)\): \(\lim_{(x,y) \to (a,b)} = L\)
    \item Estos son iguales: \(L = c\)
\end{enumerate}

Es decir,
las \textit{mismas} condiciones de continuidad que una \textit{función de una
    variable}.

\subsubsection{Función continua}

Decimos que una función de dos variables es continua si no tiene
discontinuidades en su dominio.

\textbf{Ejemplo.}
\(\sqrt{x+y}\) tiene una discontinudad en su dominio, puesto que todo punto
\(x+y<0\) será indeterminado, no cumpliendo con la condición de continuidad en
el punto.

Para que sea continua podemos acotar su dominio:
\(\sqrt{x+y} \quad/\quad x+y \geq 0\).

\vspace{.5cm}
\textbf{Ejemplo.}
