\section{Unidad 1}

\subsection{Campo escalar}

Es, básicamente, una función de dos variables que,
dado un punto \((x,y)\),
devuelve un escalar \(z\).

La función puede estar definida de forma:
\begin{itemize}
    \item Explícita: \(f(x,y) = z\)
    \item Implícita: \(f(x,y,z) = 0\)
\end{itemize}

\subsection{Dominio de un campo escalar}

\(Dom \, f(x,y) = \{(x,y) \in R^{2} / \, \text{Condición}\}\)

El dominio de un campo escalar es el conjunto de puntos para los cuales el campo 
escalar está definido.

\subsubsection{Procedimiento}

\begin{enumerate}
    \item Determinamos las restricciones (puntos para los cuales la función
    no está definida)
    \item Despejamos una de las variables para cada una de las restricciones
    \item Expresamos como definición de un conjunto
\end{enumerate}

\subsubsection{Ejemplo}

Dado \(z = \frac{\sqrt{x+y}}{x-2y}\):

1. Determinamos restricciones.

\begin{equation*}
    x+y \geq 0 \quad \land \quad x-2y \neq 0
\end{equation*}

2. Despejamos una de las variables, por ejemplo, \(y\).

\begin{equation*}
    y \geq -x \quad \land \quad y \neq \frac{x}{2}
\end{equation*}

3. Expresamos como conjunto.

\begin{equation*}
    \text{Dom } f(x,y) = \{(x,y) \in R^{2} / y \geq -x \quad \land \quad y \neq \frac{x}{2}\}
\end{equation*}

Si quisiera representar gráficamente, grafico las fronteras que se deducen de 
las desigualdades y marco en consecuencia: para arriba y frontera si es mayor o 
igual, etc.

\subsection{Definiciones de topología}

Conceptos que se usan a lo largo de la materia y es importante conocer.

\subsubsection{Entorno}

\(E(A,\delta) = \{x \in R^{n}: |x-A| < \delta\}\)

Dado un punto \(A\) y un radio \(\delta\), llamamos entorno al conjunto de 
puntos \(\in R^{n}\) ubicados a una distancia menor a \(\delta\). 

\textbf{Ejemplos.}

\begin{itemize}
    \item En \(R\): \textit{entorno}, con centro en \(A\), sobre la recta real, 
    con radio \(\delta\).
    \item En \(R^{2}\): \textit{disco}, con centro en \(A\), sobre el 
    \textit{plano} real, con radio \(\delta\).
    \item En \(R^{3}\): \textit{esfera}, con centro en \(A\), en el 
    \textit{espacio}, con radio \(\delta\).
\end{itemize}


\subsubsection{Entorno reducido}

Es igual al entorno,
pero el conjunto de puntos resultante \textbf{no incluye a} \(A\):

\(E_R(A,\delta) = \{x \in R^{n}: |x-A| < \delta \quad \land \quad x \neq A\}\)


\subsubsection{Clasificación de puntos}

Dado un conjunto \(C\):

\begin{itemize}
    \item A es \textbf{punto interior} si está incluído en \(C\)
    \item A es \textbf{punto exterior} si está fuera de \(C\)
    \item A es \textbf{punto frontera} si está sobre la frontera de \(C\)
    \item A es \textbf{punto de acumulación} si su entorno es interior o 
    frontera, pero A no pertenece a C (Creo que se podría decir que el entorno 
    reducido de A pertenece a C).
\end{itemize}

\subsubsection{Clasificación de conjuntos}

Dado un conjunto \(C\), decimos que:

\begin{itemize}
    \item Es \textbf{abierto} si solo tiene puntos interiores y no tiene 
    frontera.
    \item Es \textbf{Cerrado} si tiene puntos interiores y frontera.
    \item Es \textbf{Acotado} si se puede incluir \textit{completamente} en un 
    entorno, de cualquier radio, con \(A\) en origen.
\end{itemize}

\subsection{Gráfica de campo escalar}

La representación gráfica del campo escalar \(z = f(x,y)\).
Es una superficie en \(R^{3}\).

Para determinarla:

\begin{enumerate}
    \item Igualamos a 0 cada uno de sus componentes
    \item Vemos como se comporta el resto
\end{enumerate}

\subsection{Curva de nivel}

Intersección de un campo escalar con cualquier plano \(z = k\), paralelo al 
plano \(xy\), siendo \(k\) una constante.

Para determinar una curva de nivel:
\begin{enumerate}
    \item Igualamos z a un valor \(k\)
    \item Despejamos alguna de las variables restantes y obtenemos una función 
    de una variable
    \item Analizamos el comportamiento de esa función
\end{enumerate}

\subsection{Cuádricas, cónicas, cilindros}

\subsubsection{Cilindro de z arbitrario}

Forma: \(x^{2} + y^{2} = r^{2}\)

Gráficamente forma una circunferencia en \(R^{2}\),
de radio \(r\).

\subsubsection{Plano}

\(y + 2z = 2\), recta en \(R^{2}\), 
y un plano con \(x\) arbitraria en \(R^{3}\).

\subsubsection{Cilindro}

Se denomina cilindro a una superficie generada a partir de una línea recta,
llamada \textit{generatriz}, 
que se mueve manteniéndose paralela a una dirección,
mientras recorre una curva fija, que se llama \textit{directriz}.

La expresión \(ax^{2} + by^{2} = r^{2}\),
con \(a \neq b\), describe una elipse en el plano,
que, con \(z\) arbitraria, 
obtendríamos un cilindro elíptico en el espacio.

\(z = x^{2}\) describe una parábola en el plano \(xz\),
una curva que funciona como \textit{directriz}.
En el espacio, con \(y\) arbitraria, obtenemos un cilindro parabólico.

\subsubsection{Cuadricas}

Una cuádrica es una superficie definida por una ecuación de segundo grado en las 
variables \(x, y, z\), de forma general:

\begin{equation*}
    ax^{2} + by^{2} + cz^{2} + dxy + eyz + fzx + gx + hy + jz + k = 0
\end{equation*}

Siendo \(a, b, c, d, f, g, h, j, k\) constantes.
Los términos en \(xy,\,yz\,,\,xz\) indican rotaciones.
Los términos lineales indican traslaciones.

\subsubsection{Elipsoide}

Caso particular de cuádrica, que sigue la expresión:

\begin{equation*}
    \frac{x^{2}}{a^{2}} + \frac{y^{2}}{b^{2}} + \frac{z^{2}}{c^{2}} = r^{2}
\end{equation*}

Sus trazas\footnote{Las curvas que resultan de su intersección con un plano \(z = cte\)}
forman elipses.
Si tenemos \(a = b = c\), el resultado es un caso particular de elipsoide: 
la esfera.
Para darnos una idea del gráfico podemos igualar cada uno de los componentes 
\(x, y, z\) a 0, obteniendo elipses para los planos \(xy\), etc.


\subsubsection{Paraboloide}

Es otro tipo de cuádrica, que sigue la expresión:

\begin{equation*}
    \frac{z}{c} = \frac{x^{2}}{a^{2}} + \frac{y^{2}}{b^{2}}
\end{equation*}

Atención al \(\text{grado }z = 1\).
Con \(z = cte\), las trazas son elipses;
mientras que, \(x = 0\) o \(y = 0\) devuelve parábolas.

\subsubsection{Cono}

Otra cuádrica, con expresión:

\begin{equation*}
    \frac{z^{2}}{c^{2}} = \frac{x^{2}}{a^{2}} + \frac{y^{2}}{b^{2}}
\end{equation*}

Notar que es estructuralmente igual a la expresión anterior, salvo que el 
\(\text{Grado }z = 2\).
Las trazas con \(z = cte\) son elipses y las trazas \(x = 0\) e \(y = 0\) son 
lineales.

\subsubsection{Hiperboloide de una hoja}

Igual al anterior, pero con \textit{coeficiente independiente}:.

\begin{equation*}
    \frac{x^{2}}{a^{2}} + \frac{y^{2}}{b^{2}} - \frac{z^{2}}{c^{2}} = \boxed{1}
\end{equation*}

El \textit{término negativo} indica el eje que funciona como \textit{eje de 
simetría}, en el ejemplo, el eje \(z\). \(z = k\) son elipses; \(x=0\) e \(y=0\)
son \textit{hipérbolas}.

\subsubsection{Hiperboloide de dos hojas}

Sigue la ecuación:

\begin{equation*}
    -\frac{x^{2}}{a^{2}} - \frac{y^{2}}{b^{2}} + \frac{z^{2}}{c^{2}} = \boxed{1}
\end{equation*}

Notar que es igual a la cuádrica anterior, con la salvedad que tiene 
\textit{dos} términos negativos. En este caso, es el término positivo el que 
marca el \textit{eje de simetría}.

\subsection{Límites de Campos Escalares}

Recordamos la definición de límite para funciones de \textit{una variable}:

\begin{equation*}
    \lim_{x\to a} f(x) = L \iff \text{ tiende a L por izquierda y derecha}
\end{equation*}

En funciones de 2 variables la idea es la misma, pero considerando que el 
\textit{entorno} de \(a\) es un \textit{disco}, por lo que no me acerco 
\textit{solo} por izquierda y derecha.

\subsubsection{Propiedades}

Dadas \(g(x,y) = z\) y \(f(x,y) = z\):

\begin{enumerate}
    \item El límite es distributivo respecto de la suma:
    
    \(\lim \left[g(x,y) + f(x,y)\right] = \lim g(x,y) + \lim f(x,y)\)
    \item Es distributivo respecto del producto:
    
    \(\lim \left[g(x,y) \cdot f(x,y)\right] = \lim g(x,y) \cdot \lim f(x,y)\)
    \item Límite de constante es la constante:
    
    \(\lim k\cdot f(x,y) = k\cdot\lim f(x,y)\)
    \item Distributivo respecto de cociente, siempre que denominador \(\neq\) 0:
    

    \item Respecto de potencia
\end{enumerate}


\subsection{Cálculo de límites en dos variables}

Dependiendo de la expresión, podemos recurrir a diferentes formas de resolver un 
límite con dos variables:

\subsubsection{Límite doble o simultáneo}

Cuando no tenemos una indeterminación simplemente reemplazamos, operamos y 
llegamos a un resultado.

\subsubsection{Límite iterado}

Cuando existe una indeterminación, un posibilidad es calcular el límite para una
de las variables y luego para la siguiente. 
Entonces, \(\lim_{(x,y) \to (a,b)} f(x,y)\):

\begin{align*}
    \lim_{x \to a} f(x,y) & = f_x \\
    \lim_{y \to b} f_x & = \boxed{L_1}
\end{align*}

Luego, operamos en sentido inverso:

\begin{align*}
    \lim_{y \to b} f(x,y) & = f_y \\
    \lim_{x \to a} f_y & = \boxed{L_2}
\end{align*}

Si \(L_1 = L_2 = L\), suponemos que el límite existe hasta que se demuestre lo 
contrario. En límites en dos variables, nunca podemos estar completamente 
seguros.

\subsubsection{Límites radiales, parabólicos, cúbicos, etc.}

Procedemos de la misma forma que el límite iterado, pero poniendo una variable 
en función de la otra. Esto implica acercarnos al punto desde una recta, una 
parábola, etc:

\begin{enumerate}
    \item \(y = mx\)
    \item \(y = ax^{2}\)
    \item \(y = ax^{3}\)
\end{enumerate}

Se opera con \(f(x) = y\) y sus respectivas variantes en \(f(y) = x\):

\begin{enumerate}
    \item \(x = my\)
    \item \(x = ay^{2}\)
    \item \(x = ay^{3}\)
\end{enumerate}

Si aparece un \(L\) distinto, concluimos que no existe. Como mencionamos en el 
apartado anterior, no podemos confirmar existencia, sino \textit{no existencia}.

\subsubsection{Límites en coordenadas polares}

Se utiliza cuando la expresión tiene \(x^{2} + y^{2}\) en el denominador.

Sustituimos:
\begin{enumerate}
    \item \(x = \rho \cos\theta\)
    \item \(y = \rho \sin\theta\)
    \item \(x^{2} + y^{2} = \rho^{2}\)
\end{enumerate}

Operamos buscando que quede \(F(\rho)\),
es decir, que quede en función de \(\rho\) y no de \(\theta\).

Luego, operamos \(\lim_{\rho\to0} F(\rho)\).

Como \(f(\rho\cos\theta,\rho\sin\theta) < F(\rho) \implies \lim_{(x,y)\to(0,0)}f(x,y) = L\).

\textbf{Ejemplo.}

Dado \(\lim_{(x,y)\to(9,9)}\frac{x^{3} + x^{2} - 2y^{3} + y^{2}}{x^{2} + y^{2}}\).

\texttt{Lim \(\to\) (0,0)} y además \texttt{\(x^{2} + y^{2}\)} en denominador \(\implies\) calculamos con coordenadas polares:

Sustituimos:

\begin{equation*}
    f(\rho\cos\theta,\rho\sin\theta) = \frac{\rho^{3}\cos^{3}\theta + \rho^{2}\cos^{2}\theta - 2\rho^{3}\sin^{3}\theta + \rho^{2}\sin^{2}\theta}{\rho^{2}}
\end{equation*}

Operamos algebraicamente, sacamos \(\rho^{2}\) factor común para que quede una sola función:

\begin{equation*}
    \frac{\cancel{\rho^{2}}(\rho\cos^{3}\theta + \cos^{2}\theta - 2\rho\sin^{3}\theta + \sin^{2}\theta)}{\cancel{\rho^{2}}}
\end{equation*}

Simplificamos \(\cos^{2}\theta + \sin^{2}\theta = 1\)

\begin{equation*}
    \rho\cos^{3}\theta - 2\rho\sin^{3}\theta + 1
\end{equation*}

Operamos el \(\lim_{\rho\to0}\)

\begin{equation*}
    \lim_{\rho\to0}\rho\cos^{3}\theta - 2\rho\sin^{3}\theta + 1 = \boxed{1}
\end{equation*}

Llegamos a un valor, para comprobarlo buscamos \(F(\rho)\),
igualando expresión de \(L\) a 0:

\begin{align*}
    \rho\cos^{3}\theta - 2\rho\sin^{3}\theta + 1     & = 1 \\
    \rho\cos^{3}\theta - 2\rho\sin^{3}\theta + 1 - 1 & = 0 \\
    \rho\cos^{3}\theta - 2\rho\sin^{3}\theta         & = 0 \\
\end{align*}

Sobre la expresión resultante aplicamos módulo:

\begin{align*}
    |\rho\cos^{3}\theta - 2\rho\sin^{3}\theta|                 & \leq |\rho\cos^{3}\theta| + |2\rho\sin^{3}\theta|            \\
    |\rho\cos^{3}\theta| + |2\rho\sin^{3}\theta|               & = |\rho|\cdot|\cos^{3}\theta| + |2\rho|\cdot|\sin^{3}\theta| \\
    |\rho|\cdot|\cos^{3}\theta| + |2\rho|\cdot|\sin^{3}\theta| & \leq |\rho| + |2\rho|                                        \\
    \boxed{F(\rho) = 3\rho}
\end{align*}

Operamos límite tendiendo a 0 de \(F(\rho)\)

\begin{align*}
    \lim_{\rho\to0} 3\rho = \boxed{0}
\end{align*}

Como límite tiende a 0, 
por lo tanto, 
límite es igual al 1 encontrado con anterioridad.

\subsection{Continuidad de funciones de dos variables}

\subsubsection{Cotinuidad en un punto}

Decimos que una función de dos variables es continua en \((a,b)\) si:

\begin{enumerate}
    \item \(f(a,b) = c\)
    \item \(\lim_{(x,y) \to (a,b)} = L\)
    \item \(L = c\)
\end{enumerate}

Es decir,
las \textit{mismas} condiciones de continuidad que una \textit{función de una 
variable}.

\subsubsection{Función continua}

Decimos que una función de dos variables es continua si no tiene 
discontinuidades en su dominio.

\textbf{Ejemplo.}
\(\sqrt{x+y}\) tiene una discontinudad en su dominio, puesto que todo punto 
\(x+y<0\) será indeterminado, no cumpliendo con la condición de continuidad en 
el punto. 

Para que sea continua podemos acotar su dominio:
\(\sqrt{x+y} \quad/\quad x+y \geq 0\).