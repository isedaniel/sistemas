\section{Unidad 2}

\subsection{Derivadas parciales}

Similar a derivada de una variable, que sería: 

\texttt{Derivada f(x) x0 => Pendiente de la recta tangente a la función en x0}

Derivada Parcial considera el cambio en una de las variables, manteniendo constante la otra.

\texttt{DerivadaParcial f(x,y) (x0,y0) \(\{x,y\}\) => pendiente de tangente \(\cap\) x = x0}

Hay entonces dos tipos de derivada parcial:

\begin{align*}
    \frac{\partial f}{\partial x} = f_x & y \frac{\partial f}{\partial y} = f_y \\
\end{align*}

Es decir,
manteniendo función y punto constante, 
\texttt{DerivadaParcial} puede devolver 2 pendientes:
una de la curva que se forma en la intersección con el plano \(x = x_0\),
y otro de la intersección con el plano \(y = y_0\).

Operativamente es igual a la derivada de una variable: operamos considerando a la otra constante.

\subsection{Reglas de derivación}

\begin{enumerate}
    \item Si \(a\) constante, derivada parcial en x de función \(f\): \((af)_x = af_x\)
    \item Distributiva respecto de suma: \((f \pm g)_x = f_x \pm g_x\)
    \item Regla del producto: \((f\cdot g)_x = f_x\cdot g + f\cdot g_x\)
    \item Regla del cociente: \(\left(\frac{f}{g}\right)_x = \frac{f_x\cdot g - f\cdot g_x}{g^{2}}\)
    \item Regla de la cadena: \((g^{\circ}f)_x = g(f)_x \cdot f_x\)
\end{enumerate}

\subsection{Derivada parcial por definición}

\subsubsection{Cuando no está definida en (a,b)}

\begin{align*}
    f_x = \lim_{x \to a}\frac{f(x,b) - f(a,b)}{x} \\
    f_y = \lim_{y \to b}\frac{f(a,y) - f(a,b)}{y} \\
\end{align*}

\subsubsection{Cuando está definida por trazos}

\subsubsection{Cuando puede que una exista y la otra no}

\subsubsection{En funciones de varias variables: derivabilidad \(\nRightarrow\) continuidad}

\subsection{Derivadas sucesivas}

Derivadas de \(2^{\circ}\) orden en varias variables:

\((f_x)_x = f_{xx}\).

\subsection{Teorema de Schwarz}

Si existen \(f_x\),
\(f_y\),
\(f_{xy}\),
entorno de punto \(P=(x_0,y_0)\),
y \(f_{xy}\) es contínua en \(P\),
\textbf{existe} \(f_{yx}\) y \(f_{yx}|_P = f_{xy}|_P\).

Es decir,
la derivadas cruzadas son iguales donde la función es contínua.

\subsection{Matrices especiales}

\subsubsection{Matriz jacobiana}

Matriz cuyas filas son las derivadas parciales de 1er orden de las funciones incluídas.

\subsubsection{Matriz Hessiana}

Matriz cuyas filas son derivadas parciales de 2do orden de \textbf{una} función.
Una función tiene \(2^{n}\) derivadas cruzadas si tiene 2 variables.

\subsection{Regla de la cadena}

Si \(z\) es campo escalar en función de xy, \(f(x,y)\),
, a su vez x e y son funciones de t, \(x(t), y(t)\),
se puede hacer una composición de funciones, quedando \(z(t)\).

Se puede componer y derivar \(\frac{dz}{dt}\) o, por regla de la cadena:

\begin{equation*}
    \frac{dz}{dt} = \frac{\partial z}{\partial x}\cdot\frac{dx}{dt} + \frac{\partial z}{\partial y}\cdot\frac{dy}{dt}
\end{equation*}

\subsection{Derivada direccional}

