\section{Unidad 2}

\subsection{Derivadas parciales}

Si operamos derivada a una función de una varible obtenemos otra función,
que describe la pendiente de la primera.
La derivada en funciones de dos variables sigue la misma idea, 
pero con diferencias en el cálculo.

Aquí aparece el concepto de \textit{derivada parcial}: 
operamos derivada de una de las variables y consideramos constante la otra.

\begin{align*}
\frac{\partial f}{\partial x} = f_x \\
\frac{\partial f}{\partial y} = f_y \\
\end{align*}

\subsection{Propiedades y reglas de la derivada parcial}

\begin{enumerate}
    \item Con \(a\) constante: 
    \begin{equation*}
        (a \cdot f)_x = a\cdot f_x
    \end{equation*}
    \item Distributiva respecto de la suma y resta: 
    \begin{equation*}
        (f \pm g)_x = f_x \pm g_x
    \end{equation*}
    \item Regla del producto: 
    \begin{equation*}
        (f\cdot g)_x = f_x\cdot g + f\cdot g_x
    \end{equation*}
    \item Regla del cociente: 
    \begin{equation*}
        \left(\frac{f}{g}\right)_x = \frac{f_x\cdot g - f\cdot g_x}{g^{2}}
    \end{equation*}
    \item Regla de la cadena: 
    Derivada parcial de \(g\), con \(f\) como está,
    por derivada parcial de \(f\):
    \begin{equation*}
        (g(f))_x = g(f)_x \cdot f_x
    \end{equation*}
\end{enumerate}

\subsection{Cálculo de derivada parcial por definición}

\subsubsection{Cuando no está definida en (a,b)}

\begin{align*}
    f_x = \lim_{x \to a}\frac{f(x,b) - f(a,b)}{x} \\
    f_y = \lim_{y \to b}\frac{f(a,y) - f(a,b)}{y} \\
\end{align*}

\subsubsection{Cuando está definida por trazos}

\subsubsection{Cuando puede que una exista y la otra no}

\subsubsection{En funciones de varias variables: derivabilidad \(\nRightarrow\) continuidad}

\subsection{Derivadas sucesivas}

Derivadas de \(2^{\circ}\) orden en varias variables:

\((f_x)_x = f_{xx}\).

\subsection{Teorema de Schwarz}

Si existen en torno al punto \(P\) \(f_x\),
\(f_y\) 
y \(f_{xy}\),
con \(f_{xy}\) continua en \(P\),
\textbf{existe} \(f_{yx}\) y \(f_{yx}|_P = f_{xy}|_P\).

En concreto,
las derivadas cruzadas son iguales si la función es continua.

\subsection{Matrices especiales}

\subsubsection{Matriz jacobiana}

Matriz de las derivadas parciales de una o más funciones.

\subsubsection{Matriz Hessiana}

Matriz de las derivadas parciales \textit{de 2do orden} 
de \textit{una} función.
Una función tiene \(2^{n}\) derivadas cruzadas si tiene 2 variables.

\subsection{Regla de la cadena}

Si \(f(x,y) = z\),
y podemos expresar x e y en función de t, 
es decir, \(x(t), y(t)\),
se puede hacer una composición \(z(t)\).

Se puede componer y derivar \(\frac{dz}{dt}\) o, por regla de la cadena:

\begin{equation*}
    \frac{dz}{dt} = \frac{\partial z}{\partial x}\cdot\frac{dx}{dt} + \frac{\partial z}{\partial y}\cdot\frac{dy}{dt}
\end{equation*}

\subsection{Derivada direccional}

