\section{Unidad 3}

\subsection{Funciones implícitas}

Una función \(f(x) = y\) está definida de forma implícita cuando sigue la
estructura \(f(x,y) = 0\).

\textbf{No toda} función explítica se puede definir de forma implícita.
Para que ello sea posible,
se deben cumplir las condiciones del \textbf{Teorema de Dini}.

\subsection{Teorema de Dini}

Dada \(f(x,y)\),
decimos que está definida implícitamente en \(P = (x_0,y_0)\) si:
\begin{enumerate}
    \item la función se anula en algún punto de su dominio: \(f|_P = 0\)
    \item posee derivadas parciales continuas en \(P\)
    \item la derivada parcial respecto de la variable dependiente es distinta
          de 0: \(f_y \neq 0\)
\end{enumerate}

Si se cumplen estas condiciones,
la derivada de la función será:

\begin{align*}
    \frac{dy}{dx} = -\frac{f_x}{f_y}
\end{align*}

\vspace{.5cm}
\textbf{Ejemplo.}

Verificar que \(f(x,y) = x^{2} - 3xy + x + y\) es una función implícita,
considerando \(P = (0,0)\).
Hallar \(dy/dx\) en dicho punto.

Primero comprobamos Dini:
\begin{enumerate}
    \item \(f(0,0) = 0\)
    \item \(f_x|_{(0,0)} = (2x - 3y + 1)|_{(0,0)} = \boxed{1} \qquad f_y|_{(0,0)} = (-3x + 1)|_{(0,0)} = \boxed{1}\),
          derivadas existen y son continuas en \(P\)
    \item \(f_y|_{(0,0)} = 1 \neq 0\)
\end{enumerate}

Se cumplen las tres condiciones del \textbf{teorema de Dini},
por lo tanto la función está definida en \(P\),
siendo su derivada:

\begin{align*}
    \frac{dy}{dx} = \frac{3y - 2x - 1}{1 - 3x}
\end{align*}

Si la evaluamos en \(P\): \(dy/dx|_P = -1\).

\subsection{Teorema de Dini en campos escalares}

El teorema se puede extender para campos escalares,
expresados \(f(x,y,z) = 0\),
en un punto \(P= (x_0,y_0,z_0)\):
\begin{enumerate}
    \item \(f(x_0,y_0,z_0) = 0\)
    \item Existen \(f_x|_P \quad f_y|_P \quad f_z|_P\) y son continuas en \(P\)
    \item \(f_z|_P \neq 0\)
\end{enumerate}

Dadas esas condiciones,
podemos determinar derivadas parciales siguiendo la expresión:

\begin{align*}
    \frac{\partial z}{\partial x} = -\frac{f_x}{f_z} \qquad
    \frac{\partial z}{\partial y} = -\frac{f_y}{f_z}
\end{align*}

Este teorema se puede generalizar,
a su vez,
para \(n\)-variables independientes.

\subsection{Sistemas de Funciones Implícitas}

Un sistema de ecuaciones implícitas puede definir una curva \(C\) en \(R^{3}\):
\begin{align*} C:
    \begin{cases}
        F(x,y,z)=0 \\
        G(x,y,z)=0
    \end{cases}
    \text{con } y = y(x) \text{ y } z = z(x)
\end{align*}

Las condiciones de existencia de la curva en \(P=(x_0,y_0,z_0)\) se desprenden del
\textbf{teorema de Cauchy Dini}:
\begin{enumerate}
    \item \(F(P) = 0 \quad \land \quad G(P) = 0\)
    \item Derivadas parciales de \(F\) y \(G\) existen y son continuas en \(P\)
    \item Determinante del jacobiano \(\neq 0\)
\end{enumerate}

Con estas condiciones podemos asegurar la existencia de
\(y = y(x)\) y \(z = z(x)\) diferenciables,
y la curva \(C\) se puede parametrizar siguiendo:
\begin{align*}
    \alpha(x) = (x0, \,y(x), \,z(x))
\end{align*}

El \textbf{vector tangente} a \(C\) se calcula:
\begin{align*}
    \alpha'(x) = (1, \,y'(x), \,z'(x))
\end{align*}

O como el producto vectorial de los gradientes de las superficies \(F\) y \(G\):
\begin{align*}
    \alpha'(x) = \nabla F|_P \times \nabla G|_P
\end{align*}

Sabemos que \(F = 0\) y \(G = 0\)
definen implícitamente \(y = y(x)\) y \(z = z(x)\).
Ademas, hemos determinado que son diferenciables.
Por lo tanto:

\begin{align*}
    \begin{cases}
        dF = F'_x \; dx + F'_y \; dy + F'_z \; dz = 0 \\
        dG = G'_x \; dx + G'_y \; dy + G'_z \; dz = 0
    \end{cases}
\end{align*}

Despejamos derivada parcial respecto de \(x\):

\begin{align*}
    \begin{cases}
        F'_y \; dy + F'_z \; dz = -F'_x \; dx \\
        G'_y \; dy + G'_z \; dz = -G'_x \; dx
    \end{cases}
\end{align*}

Si dividimos ambos miembros por \(dx\):

\begin{align*}
    \begin{cases}
        F'_y \; \frac{dy}{dx} + F'_z \; \frac{dz}{dx} = -F'_x \\
        G'_y \; \frac{dy}{dx} + G'_z \; \frac{dz}{dx} = -G'_x
    \end{cases}
\end{align*}

Para encontrar las derivadas parciales aplicamos \textbf{Regla de Cramer}:

\begin{align*}
    det \text{ del sistema } = J = \begin{vmatrix}
                                       F'_y & F'_z \\
                                       G'_y & G'_z
                                   \end{vmatrix}
\end{align*}

Por lo tanto:

\begin{align*}
    y'(x) = \frac{\begin{vmatrix}
                          -F'_x & F'_z \\
                          -G'_x & G'_z
                      \end{vmatrix}}{\begin{vmatrix}
                                         F'_y & F'_z \\
                                         G'_y & G'_z
                                     \end{vmatrix}}
    \qquad\text{ y }\qquad
    z'(x) = \frac{
        \begin{vmatrix}
            F'_y & -F'_x \\
            G'_y & -G'_x
        \end{vmatrix}}{\begin{vmatrix}
                           F'_y & F'_z \\
                           G'_y & G'_z
                       \end{vmatrix}}
\end{align*}

\vspace{.5cm}
\textbf{Ejemplo 1.}

Sea \(C\) curva en \(R^{3}\),
definida implícitamente por el sistema:

\begin{align*} C:
    \begin{cases}
        F(x,y,z) = z^{3} + 2x^{2} - 4yx + 1 = 0 \\
        G(x,y,z) = 2y - 3x^{2}y - 4z - 4 = 0
    \end{cases}
    \text{ con } z = z(x) \text{ e } y = y(x)
\end{align*}

Hallar y'(x) y z'(x).

\textit{Resolución.}

Primero buscamos las derivadas parciales:

\begin{table*}[h!]
    \centering
    \begin{tabular}{lll}
        $F'_x = 4x - 4y$ & $F'_y = -4x$        & $F'_z = 3z^{2}$ \\
        $G'_x = - 6xy$   & $G'_y = 2 - 3x^{2}$ & $G'_z = -4 $    \\
    \end{tabular}
\end{table*}

Componemos los jacobianos:

\begin{align*}
    J_x =
    \begin{vmatrix}
        -4x        & 3z^{2} \\
        2 - 3x^{2} & -4     \\
    \end{vmatrix} = 16x - 6z^{2} + 9x^{2}z^{2}
\end{align*}

\begin{align*}
    J_y =
    \begin{vmatrix}
        4y - 4x & 3z^{2} \\
        6xy     & -4     \\
    \end{vmatrix} = 16x - 16y - 18xyz^{2}
\end{align*}

\begin{align*}
    J_z =
    \begin{vmatrix}
        -4x        & 4y - 4x \\
        2 - 3x^{2} & 6xy     \\
    \end{vmatrix} = 8x - 8y -12x^{2}y - 12x^{3}
\end{align*}

\vspace{.5cm}
\textbf{Ejemplo 2.}

Hallar ecuación de recta tangente y plano normal a la curva \(C\),
dada por la intersección de dos superficies en el punto \(P=(0,0,2)\):

\begin{align*} C =
    \begin{cases}
        F(x,y,z): x\sen y + z\cos(xy) = 2 \\
        G(x,y,z): e^{y} + \sen(zx) = 1
    \end{cases}
\end{align*}

Primero,
probamos que las funciones estén definidas de forma implícita por teorema de
Cauchy Dini:

\begin{enumerate}
    \item \(F|_P = 0 + 2 - 2 = \boxed{0}\) y \(G|_P = 1 + 0 - 1 = \boxed{0}\)
    \item Determinamos derivadas parciales:
          \begin{itemize}
              \item[] \(F'_x = \sen y - zy\sen(xy)\)
              \item[] \(F'_y = x\cos y - zx\sen(xy)\)
              \item[] \(F'_z = \cos(xy)\)
              \item[] \(G'_x = z\cos(zx)\)
              \item[] \(G'_y = e^{y}\)
              \item[] \(G'_z = x\cos(zx)\)
          \end{itemize}
          Todas son continuas y existen porque son polinomios de
          funciones trigonométricas y exponenciales.
    \item Evaluamos jacobiano
          \(J_x = \begin{vmatrix}
              0 & 1 \\
              1 & 0
          \end{vmatrix} = \boxed{-1 \neq 0}\)
\end{enumerate}

Se cumplen las tres condiciones,
por lo tanto \(C\) está definida implícitamente.

Para hallar la recta tangente y el plano normal necesitamos 
el \textbf{vector tangente} en \(P\):

\begin{align*}
    \alpha'(x) = \nabla F|_P \times \nabla G|_P \\
\end{align*}

Determinamos gradientes en \(P\):

\begin{align*}
    \nabla F|_P = \langle 0, 0, 1 \rangle \\
    \nabla G|_P = \langle 2, 1, 0 \rangle
\end{align*}

Operamos producto vectorial:

\begin{align*}
    \nabla F|_P \times \nabla G|_P =
    \begin{vmatrix}
        \hat{i} & \hat{j} & \hat{k} \\
        0       & 0       & 1       \\
        2       & 1       & 0       \\
    \end{vmatrix} =
    -\hat{i} + 2\hat{j} + 0\hat{k} = \boxed{\langle -1, 2, 0 \rangle}
\end{align*}

Con el vector tangente podemos expresar la recta tangente de forma vectorial:

\begin{align*}
    r: (x, y, z) = (0, 0, 2) + \lambda(-1, 2, 0)
\end{align*}

Y el plano tangente lo expresamos:

\begin{align*}
    \pi&: (-1, 2, 0) \cdot (x - 0, y - 0, z - 2) \\
    \pi&: \boxed{2y - x = 0}
\end{align*}

\subsection{Extremos libres y condicionados}

