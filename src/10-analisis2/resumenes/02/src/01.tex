Segundo apunte, seguimos enfoque "input/output"

\section{Unidad 4}

\subsection{Integrales múltiples}

La integral definida para un recinto R,
es el volumen \(V\) del sólido \(S\), 
con base \(R = [a,b] \times [c,d]\),
hasta la superficie \(z = f(x,y)\).

\begin{equation*}
    \int\int f(x,y)\,dA = V
\end{equation*}

Es decir, si \(R\) tiene como límites \([a,b] \times [c,d]\):

\begin{equation*}
    \int_{c}^{d}\int_{a}^{b} f(x,y) \,dx\,dy
\end{equation*}

\subsection{Propiedades}

Dadas dos funciones de dos variables,
\(f(x,y)\) y \(g(x,y)\),
con ambas \textit{integrables} en el recinto \(R\):

1. La integración múltiple es distributiva respecto de la suma y resta de funciones: 

\begin{equation*}
    \int\int [f(x,y) \pm g(x,y)] \,dA = \int\int f(x,y) \,dA \pm \int\int g(x,y) \,dA
\end{equation*}

2. Si tengo producto de función por constante (\(k\)), \textit{puedo sacar la constante}:

\begin{equation*}
    \int\int [kf(x,y)] \,dA = k \int\int f(x,y) \,dA
\end{equation*}

3. La integral definida para el recinto \(R\) es igual a la suma de las integrales para los \textit{subrecintos} \(R_1\) y \(R_2\):

\begin{equation*}
    \int\int_R [f(x,y)] \,dA = \int\int_{R_1} [f(x,y)] \,dA + \int\int_{R_2} [f(x,y)] \,dA
\end{equation*}

4. Si \(f(x,y)\) esta por encima de \(g(x,y)\) para todo \(R\), entonces la integral también va a estar por encima 
