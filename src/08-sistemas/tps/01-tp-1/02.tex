\documentclass[12pt]{article}
\usepackage[a4paper, margin=2.54cm]{geometry}

% español
\usepackage[spanish]{babel}

% imágenes
%\usepackage{graphicx}
%\graphicspath{{img}}

% fuentes de conjuntos numéricos
\usepackage{amsfonts}

% símbolos
\usepackage{amsmath, amssymb}

% gráficos
\usepackage{tikz}

% plots
%\usepackage{pgfplots}
%\pgfplotsset{width=10cm, compat=1.9}

% averiguar
\setlength{\jot}{8pt}
\setlength{\parindent}{0cm}

% espacio entre párrafos
\usepackage[skip=10pt plus1pt]{parskip}

% cancelar términos
\usepackage{cancel}

% links
%\usepackage[colorlinks=true, 
%    urlcolor=blue]{hyperref}

% shapes
\usetikzlibrary{
  babel, 
  positioning, 
  shapes.multipart,
}

% incluir pdfs
%\usepackage{pdfpages}

\title{Materia\\Tipo}
\author{Daniel Ise}
\date{}

\begin{document}

\begin{tikzpicture}[
  trinode/.style={
    circle split,
    draw,
    path picture={\draw (path picture bounding box.center)--(path picture bounding box.south);}
  }
]

\node[trinode] (in) {Inicio\nodepart{lower} 0\ \ 0};

\node[trinode, right=of in] (B) {B\nodepart{lower} 9\ \ 99};

\node[trinode, above right=of 2] (3) {99\nodepart{lower} 99\ \ 99};

\draw[->] (in)--(2) node[midway, right]{A(1)};
\draw[->,dashed] (2)--(3);
\end{tikzpicture}

\end{document}

