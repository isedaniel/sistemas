\documentclass[12pt]{article}
\usepackage[a4paper, margin=2.54cm]{geometry}
\usepackage[spanish]{babel}

% imágenes
%\usepackage{graphicx}
%\graphicspath{{img}}

% fuentes de conjuntos numéricos
\usepackage{amsfonts}

% math
\usepackage{amsmath, amssymb}

% gráficos y plots
\usepackage{tikz}
%\usepackage{pgfplots}
%\pgfplotsset{width=10cm, compat=1.9}
\usetikzlibrary{babel}

\setlength{\jot}{8pt}
\setlength{\parindent}{0cm}

% espacio entre párrafos
\usepackage[skip=10pt plus1pt, indent=12pt]{parskip}

% cancelar términos
\usepackage{cancel}

% links
%\usepackage[colorlinks=true,
%    urlcolor=blue]{hyperref}

% shapes
%\usetikzlibrary{shapes.geometric}

% incluir pdfs
%\usepackage{pdfpages}

\begin{document}

\thispagestyle{empty}

\begin{center}
	\vspace*{.5cm}
	\includegraphics[scale=.6]{~/Pictures/udemm-logo.png}\\
	\vspace{.2cm}
	\Large
	\textbf{Facultad de Ingeniería}\\
	\textbf{Ingeniería en Sistemas}\\
	\vspace{2cm}

	\Huge
	Análisis de Sistemas I\\
	Trabajo Práctico N\(^\circ\) 2 \\
	\vfill

	\raggedright
	\Large
	Docentes:
	\begin{itemize}
		\item[] Mg. Margarita Castronuovo \\
	\end{itemize}
	Alumno:
	\begin{itemize}
		\item[] Daniel Ise
	\end{itemize}
	Legajo:
	\begin{itemize}
		\item[] 28547
	\end{itemize}
	Fecha:
	\begin{itemize}
		\item[] Noviembre, 2024
	\end{itemize}
\end{center}

\pagebreak

1.- Objetivo del proyecto. Descripción Básica del problema. Requerimientos Funcionales. Restricciones. Objeto de negocio

2.- Identificación de Actores.

3.- Diagrama de Contexto.

4.- Identificar 7 Casos de Uso en general.

5.- Diagrama de Casos de Uso.

6.- Descripción resumida. Plantilla Básica. Elegir dos Casos de Uso y realizar la plantilla detallada de los Casos de Uso elegidos.

7.- Realizar el Modelo de Datos.

Presentación: En fecha y horario de parcial, con coloquio personalizado.

\pagebreak

\end{document}
