\documentclass[12pt]{article}
\usepackage[a4paper, margin=2.54cm]{geometry}

% español
\usepackage[spanish]{babel}

% imágenes
%\usepackage{graphicx}
%\graphicspath{{img}}

% fuentes de conjuntos numéricos
\usepackage{amsfonts}

% símbolos
\usepackage{amsmath, amssymb}

% gráficos
%\usepackage{tikz}

% plots
%\usepackage{pgfplots}
%\pgfplotsset{width=10cm, compat=1.9}

% averiguar
\setlength{\jot}{8pt}
\setlength{\parindent}{0cm}

% espacio entre párrafos
\usepackage[skip=10pt plus1pt]{parskip}

% cancelar términos
\usepackage{cancel}

% links
%\usepackage[colorlinks=true, 
%    urlcolor=blue]{hyperref}

% shapes
%\usetikzlibrary{shapes.geometric}

% incluir pdfs
%\usepackage{pdfpages}

\title{Análisis de sistemas\\
       Prof. Margarita Castronuovo\\
       Apuntes de clase
}
\author{Daniel Ise}
\date{Noviembre de 2024}

\begin{document}

\maketitle

\tableofcontents

\section{Clase 12 de octubre}

\subsection{Funcionalidades}

Cada funcionalidad tiene un conjunto de \textbf{interfaces},
vinculada a las relaciones que el \textbf{cliente} mantiene con el \textbf{sistema}.

\textbf{Casos de uso.}
Permite gestionar esas funciones que realiza el \textbf{cliente}.
Cada caso de uso es un \textbf{requerimiento},
que tiene uno o más formatos en relación a una plantilla de diseño,
que vincula lo que hace el cliente con lo que hace el sistema.

En cada \textbf{iteración} se va refinando,
de forma progresivamente detallada,
las partes de la plantilla.

\section{Clase 19 de octubre}

\subsection{Caso de ejemplo}

Está el caso práctico para revisar en el campus.
17 pp.
Hay que generar casos de uso y modelo de datos.
Con eso estamos.

\subsection{Trabajo final}



\end{document}

