\documentclass[12pt]{article}
\usepackage[a4paper, margin=2.54cm]{geometry}

% español
\usepackage[spanish]{babel}

% imágenes
%\usepackage{graphicx}
%\graphicspath{{img}}

% fuentes de conjuntos numéricos
\usepackage{amsfonts}

% símbolos
\usepackage{amsmath, amssymb}

% gráficos
%\usepackage{tikz}

% plots
%\usepackage{pgfplots}
%\pgfplotsset{width=10cm, compat=1.9}

% averiguar
\setlength{\jot}{8pt}
\setlength{\parindent}{0cm}

% espacio entre párrafos
\usepackage[skip=10pt plus1pt]{parskip}

% cancelar términos
\usepackage{cancel}

% links
%\usepackage[colorlinks=true, 
%    urlcolor=blue]{hyperref}

% shapes
%\usetikzlibrary{shapes.geometric}

% incluir pdfs
%\usepackage{pdfpages}

\title{Análisis de sistemas\\Apunte de clase}
\author{Daniel Ise}
\date{5 de octubre de 2024}

\begin{document}

\maketitle

\tableofcontents

\section{Objetivo de la materia}

Utilizar \textbf{casos de uso} para describir las \textbf{interfaces} de un 
sistema, vinculada a un ciclo de vida.
El objetivo es desarrollar un \textbf{proyecto}.

\section{Proceso unificado de desarrollo}

Resulta de la combinación del ciclo de vida en cascada
y en espiral.
Es evolutivo, \textbf{iterativo} e \textbf{incremental}.
Dirigido por \textbf{casos de uso}.
Centrado en la \textbf{arquitectura}.

\textbf{Unified modeling language.}
Es una notación.
Precisa de un proceso de desarrollo, es decir un ciclo de vida,
que especifique la \textit{secuencia} de actividades a realizar.

\section{Ingeniería de requisitos}

Direcciona el proceso de \textit{elicitación},
-que es el paso de información de forma fluida-,
definición,
modelado,
análisis,
especificación
y validación de los requerimientos de un sistema,
separando el \textit{qué} del \textit{cómo} del diseño.

\textbf{Requisito.}
Es una \textit{capacidad} que el sistema debe conformar. 
Es un acuerdo entre el cliente y el desarrollador acerca de 
lo que el sistema debe hacer.
Lleva a comprender de manera clara la necesidad del usuario.

Permite el \textit{flujo de trabajo fundamental},
orientando el desarrollo hacia el sistema \textit{correcto}.

\textbf{Problemas para identificar requisitos.}
\begin{itemize}
    \item No son obvios y pueden tener muchas fuentes
    \item Pueden ser difíciles de enunciar en palabras
    \item Existen muchos tipos distintos, de distinto nivel de detalle
    \item Pueden crecer a un número inmanejable
    \item Las partes interesadas y responsables pueden ser muchas,
    con distintos intereses
    \item Se interrelacionan entre sí 
    \item Puede haber dificultades de lenguaje en la comunicación
\end{itemize}

Estos problemas impactan directamente sobre el costo.

\section{Tipos de requisitos}

\textbf{Requisitos de usuarios.}
Necesidades expresadas por los usuarios.

\textbf{Requisitos del sistema.}
Componentes que el sistema debe tener para proveer sus funcionalidades.

\textbf{Requisitos funcionales.}
Especifican una acción que debe ser capaz de realizar el sistema,
sin tener en cuenta las restricciones físicas.
Describe la funcionalidad provista por el sistema.

\textbf{Requisitos no funcionales.}
Restricciones físicas sobre un requisitos funcional, de rendimiento,
plataforma, fiabilidad.
No refieren a funciones que entrega el sistema sino a sus propiedades emergentes:
fiabilidad, respuesta, capacidad de almacenamiento.
Refiere a la infraestructura en sí.
Vinculado al Hardware y las redes.

\section{Requisitos y casos de uso}

\textbf{Casos de uso.}
Técnica para la especificación de requisitos funcionales propuesta 
por Jacobson (1992).

Dan respuesta a la pregunta: 
¿qué debe hacer el sistema para los usuarios?

Es una secuencia de eventos entre el actor y el sistema.

\textbf{Actor.}
Interactúá con el sistema,
persona,
organización,
programa o sistema.
Estimula al sistema con algún evento o recibe información.
Para cada actor se definer una \textbf{interfaz}.

El actor puede ser:
\begin{itemize}
    \item \textbf{Primario.} 
    Utiliza las funciones principales.
    \item \textbf{Secundario.}
    Efectúan tareas administrativas o de mantenimiento.
\end{itemize}

\textbf{Diagrama de contexto.}
Permite determinar las fronteras del sistema,
incluyendo clientes,
administradores,
otros sistemas que interactúan.

\end{document}
