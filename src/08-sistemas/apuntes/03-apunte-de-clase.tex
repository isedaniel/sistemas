\documentclass{article}
\usepackage[a4paper, margin=2.54cm]{geometry}

% español
\usepackage[spanish]{babel}

% imágenes
%\usepackage{graphicx}
%\graphicspath{{img}}

% fuentes de conjuntos numéricos
\usepackage{amsfonts}

% símbolos
\usepackage{amsmath, amssymb}

% gráficos
%\usepackage{tikz}

% plots
%\usepackage{pgfplots}
%\pgfplotsset{width=10cm, compat=1.9}

% averiguar
\setlength{\jot}{8pt}
\setlength{\parindent}{0cm}

% espacio entre párrafos
\usepackage[skip=8pt plus1pt]{parskip}

% cancelar términos
\usepackage{cancel}

% links
%\usepackage[colorlinks=true, 
%    urlcolor=blue]{hyperref}

% shapes
%\usetikzlibrary{shapes.geometric}

% incluir pdfs
%\usepackage{pdfpages}

\title{Análisis de sistemas\\Apunte de clase}
\author{Daniel Ise}
\date{7 de septiembre de 2024}

\begin{document}

\maketitle

\tableofcontents

\section{Planificación y programación de proyectos}

\textbf{Planificación.}
Etapa previa a la \textbf{programación}.
Se elabora el \textbf{planeamiento}.
Establece el \textbf{tiempo} del proyecto.

\textbf{Revisión.}
Ya durante la etapa de \textbf{ejecución},
se hace una verificación de qué pasa con el \textbf{tiempo},
que había sido pautado en la etapa de \textbf{planificación}.

\textbf{Proyecto.}
Una sucesión de \textbf{tareas} \textit{interrelacionadas},
que \textit{guardan} un \textbf{orden},
y \textit{buscan alcanzar} un \textbf{objetivo}.

\textbf{Planeamiento.}
Establece \textit{qué} debe hacerse
y \textit{en qué} secuencia.
En esta materia aparece como \textit{dada}.

\textbf{Programación.}
Determinar cuándo debe hacerse,
acotar en el \textbf{tiempo} \textit{lo planeado}.

\textbf{Control.}
Verificar el \textbf{cumplimiento} de
\textit{lo planeado}
y \textit{lo programado}.

\section{Herramientas para la administración de proyectos}

\textbf{Diagrama de Gantt.}
Viene de \textbf{ingeniería civil}.
Una de las técnicas más \textbf{simples} para la administración de proyectos.
Consiste en la representación, por medio de barras proporcionales,
a la \textbf{duración} de las tareas.
Permite visualización rápida, simple y práctica.
\textbf{Deventajas}.
No permite determinar impacto del \textbf{atraso} de una tarea.
En proyectos \textit{complejos} se necesita más \textbf{información}.
El \textbf{camino crítico} nace como opción y \textit{complemento}.

\begin{center}
    \begin{tabular}{ c c c }
        Actividad & Duración & Precedencia \\
        \hline                             \\
        A         & 2        & -           \\
        B         & 3        & -           \\
        C         & 1        & B           \\
        D         & 1        & A, C        \\
        E         & 2        & D           \\
        F         & 1        & E           \\
        \hline                             \\
    \end{tabular}
\end{center}

\textbf{CPM.}
Critic Path Method.
Utiliza \textbf{camino crítico}.
\textit{Determina} \textbf{tiempos}.
Preguntas principales:

\begin{enumerate}
    \item Cuál es el \textbf{Tiempo total}.
    \item Cuáles son fechas de \textbf{inicio} y \textbf{terminación}.
    \item Qué actividades son \textbf{críticas} y deben terminarse \textbf{exactamente} dentro de lo programado.
\end{enumerate}

\begin{center}
    \begin{tabular}{ c c c c }
        Tarea & Actividad & Duración & Precedencia \\
        \hline                                     \\
        0-1   & A         & 2        & -           \\
        0-2   & B         & 3        & -           \\
        0-3   & C         & 1        & B           \\
        1-2   & D         & 1        & A, C        \\
        2-5   & E         & 2        & D           \\
        3-5   & F         & 1        & E           \\
        3-5   & G         & 1        & E           \\
        3-5   & H         & 1        & E           \\
        3-5   & I         & 1        & E           \\
        3-5   & J         & 1        & E           \\
        3-5   & K         & 1        & E           \\
        \hline                                     \\
    \end{tabular}
\end{center}

\textbf{Grafo.} Con la tabla se puede armar un grafo.
Se puede calcular una \textbf{fecha temprana}.
Y una \textbf{fecha tardía}.
Cuando hay diferencia entre temprana y tardía aparece un \textbf{margen}.
El \textbf{margen total} para cada tarea es un indicador.
Cuando da 0, esa actividad es \textbf{crítica}.
El \textbf{camino} por los \textbf{nodos críticos}

\textbf{Pert.}
Utiliza \textbf{camino crítico}.
\textit{Considera} \textbf{tiempo} y \textbf{tareas} \textit{inciertos}.

\end{document}
