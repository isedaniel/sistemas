\documentclass{article}
\usepackage[margin=2.54cm]{geometry}
%\usepackage{graphicx}               % imágenes
%\graphicspath{{img}}
\usepackage{amsfonts}               % fuentes de conjuntos numéricos
\usepackage{amsmath, amssymb}       % símbolos
%\usepackage{tikz}                   % gráficos
%\usepackage{pgfplots}               % plots
%\pgfplotsset{width=10cm, compat=1.9}
\setlength{\jot}{8pt}
\setlength{\parindent}{0cm}
\usepackage{parskip}                % espacio entre párrafos
\usepackage{cancel}                 % cancelar términos
%\usepackage[colorlinks=true, 
%    urlcolor=blue]{hyperref}        % links
%\usetikzlibrary{shapes.geometric}   % shapes
%\usepackage{pdfpages}               % incluir pdfs

\title{Clase de Análisis de Sistemas I}
\author{Daniel Ise}
\date{31 de agosto de 2024}

\begin{document}

\maketitle

\section*{Proceso unificado de desarrollo.}

\textbf{UML.} Unified Model Language.
Es un conjunto de herramientas de modelado.

\textbf{Ciclo de vida en cascada.} 
Tiene distintas etapas. 
Se hace cada una de manera sucesiva.
1. \textbf{Análisis de requisitos}.
2. \textbf{Diseño.}
3. \textbf{text}

\textbf{Ciclo de vida en espiral.} 
Se hace rápido un usable y se va colaborando con el cliente.

\textbf{Proceso Unificado.} 
Es la unión entre los dos. 
Proceso \textbf{iterativo} e \textbf{incremental}.
Cada \textbf{iteración}, de manera \textit{simultánea}, 
pone \textit{énfasis} diferenciado en cada una de las etapas.
Es un proceso de \textbf{desarrollo de software}.

\textbf{Paradigmas del PU.}

\begin{enumerate}
    \item Dirigido por \textbf{casos de uso}.
    \item Centrado en \textbf{arquitectura}.
    \item Iterativo incremental.
\end{enumerate}

\textbf{Dirigido por casos de uso.} 
Se centra en las \textbf{interfaces} visibles por el usuario.

\textbf{Centrado en arquitectura.}
Centrado en el \textbf{modelo de datos}, la \textbf{base de datos}. 
El software se construye sobre la base del tratamiento de los datos.

\textbf{Interativo e incremental.}
Una iteración es un \textbf{subproyecto}, 
de \textit{duración fija} (2 a 6 semanas).
\textbf{Resultado de cada iteración:} 
Un \textbf{subconjunto} de \textit{calidad} del sistema final.
Sistema ejecutable aunque incompleto.
Listo para que lo pruebe el cliente.
Cada iteración aborda nuevos \textbf{requerimientos}.
El sistema se amplía incrementalmente.


\end{document}
