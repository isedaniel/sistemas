\documentclass{article}
\usepackage[margin=2.54cm]{geometry}
%\usepackage{graphicx}               % imágenes
%\graphicspath{{img}}
\usepackage{amsfonts}               % fuentes de conjuntos numéricos
\usepackage{amsmath, amssymb}       % símbolos
%\usepackage{tikz}                   % gráficos
%\usepackage{pgfplots}               % plots
%\pgfplotsset{width=10cm, compat=1.9}
\setlength{\jot}{8pt}
\setlength{\parindent}{0cm}
\usepackage{parskip}                % espacio entre párrafos
\usepackage{cancel}                 % cancelar términos
%\usepackage[colorlinks=true, 
%    urlcolor=blue]{hyperref}        % links
%\usetikzlibrary{shapes.geometric}   % shapes
%\usepackage{pdfpages}               % incluir pdfs

\title{Clase de Análisis de Sistemas I}
\author{Daniel Ise}
\date{24 de agosto de 2024}

\begin{document}

\maketitle

\textbf{Sistema.} Conjunto de componentes (que pueden ser a su vez
\textbf{subsistemas}), que se \textit{interrelacionan} para lograr un
determinado \textbf{objetivo.}.
El concepto de \textbf{sistema/subsistema/suprasistema} es
\textit{relativo al observador}.
\textbf{Propiedades de los sitemas}. Tomadas de los sistemas en general, son
también aplicables a los \textbf{sistemas de información}. El
\textbf{conocimiento} de esas \textbf{propiedades} es de utilidad para el
\textbf{diseño} y \textbf{construcción} de \textbf{sistemas de información}.

\textbf{Metodologías de análisis de sistemas.} Todas tratan de lidiar con la 
\textbf{complejidad}.

\textbf{Descomposición funcional.} Comprender el \textbf{funcionamiento} de un 
sistema como un \textit{todo} puede resultar muy complejo. Por ello, la
\textbf{descomposición funcional}, es decir, la
\textit{división del sistema en partes}, es uno de los mecanismos que se
utilizan para dominar esa \textbf{complejidad}. La descomposición no debe de ser
arbitraria. Existen formatos de descomposición. Existen dos criterios generales
que auxilian en el proceso de descomposición: \textbf{cohesión} y 
\textbf{acomplamiento}.

\textbf{Cohesión.} Relación interna entre los componentes, cuya ligazón debe ser
alta.

\textbf{Acomplamiento.} Es la relación entre los procesos después de la 
descomposición, es recomendable que sea baja. 

\textbf{Características de los sistemas.} 

\textbf{Propósito.} objetivo, que tiene relación con el \textbf{ambiente} que lo
rodea. 

\textbf{Elementos.} Sus partes constitutivas, que a su vez pueden ser sistemas.

\textbf{Interacción.} Lugar y rol.

\textbf{Ambiente.} El entorno donde el sistema despliga su actividad. Engloba
todo lo que está fuera del control del sistema. Es, a su vez, \textit{dinámico}.
El ambiente \textit{actúa} sobre el sistema, proveyendo entradas y recibiendo su
salida.

\textbf{Límite.} Es la separación del sistema con el ambiente. Se puede pensar 
como una \textit{frontera}, por lo que puede ser un límite \textbf{abierto} o 
\textbf{cerrado}. En algunos sistemas, por ejemplo en física, son más fáciles de
delimitar que otros, por ejemplo los sistemas sociales como las organizaciones.

\textbf{Homeostasia.} Equilibrio dinámico entre los procesos internos.

\textbf{Entropía.} Tendencia al desgaste del sistema, lo que implica a su vez 
que el sistema requiere de \textbf{mantenimiento}. Un sistema de mayor calidad 
es más fácil de mantener.

\textbf{Teoría general de sistemas.} Es una forma sistemática de aproximarnos y
representar la realidad. Referir a un sistema implica pensar en una 
\textit{totalidad} cuyas \textbf{propiedades }no son atribuibles a la suma de 
las propiedades de sus \textbf{partes}.

\end{document}
