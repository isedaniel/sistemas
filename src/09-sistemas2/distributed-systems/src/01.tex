Apuntes de las clases de sistemas distribuidos
(https://www.youtube.com/watch?v=cQP8WApzIQQ&list=PLrw6a1wE39_tb2fErI4-WkMbsvGQk9_UB).

\section{Introducción}

Sistemas distribuidos,
básicamente,
es usar muchas computadoras conectadas a través de la red para resolver
problemas computacionales
(provisión de almacenamiento, grandes bases de datos, etc.)

Los sistemas distribuidos son fundamentales hoy para la infraestructura.

Muchas cosas se resuelven con sistemas distribuidos y generalmente son mejores.

Conceptos centrales:

Paralelización: hacer varias cosas a la vez.

Tolerancia a fallos: si falla una, que tenga reemplazo.

Razones físicas: están en muchos lugares

Seguridad: dividir la computación para que no afecte a otros sistemas
(\textit{aislar} procesos)

Aparecen los problemas de \textit{computación concurrente}.
Eso la hace difícil.

El hecho de tener muchas computadoras y encima sobre redes 
hace que aparezcan problemas raros.
También colabora con la dificultad.

Entonces, desafíos básicos: concurrencia, falla parcial,
rendimiento (que el sistema de la performance que queremos).

Antes era una disciplina \textit{académica}.
Hoy es fundamental: las páginas web son gigantes y tienen que resolver 
millones de peticiones.


