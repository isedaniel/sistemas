\section{Ingeniería hacia adelante}

En algunos casos, 
la reingeniería de un programa puede resultar imposible,
por múltiples motivos:
\begin{itemize}
    \item El \textbf{costo de mantenimiento y refactorización} 
    puede superar al costo de hacer una nueva aplicación
    \item El \textbf{rediseño de una arquitectura} puede garantizar un costo 
    de mantenimiento menor en el futuro
    \item La \textbf{existencia de un prototipo} en funcionamiento puede 
    aumentar la productividad y acelerar el desarrollo de la nueva versión
    \item La \textbf{experiencia}, tanto del equipo como de los usuarios,
    puede colaborar con el nuevo desarrollo 
    \item Las \textbf{herramientas automatizadas} pueden ayudar en el trabajo
\end{itemize}

En cualquier caso,
una empresa con gran cantidad de software puede llegar a mantener miles de 
módulos,
entre los cuales algunos serán candidatos a la reingeniería y otros a la 
ingeniería hacia adelante. 

Esta última implica necesariamente la aplicación de los principios y 
métodos de la ingeniería de software, con el propósito de obtener un producto 
mantenible en el futuro.

La ingeniería hacia adelante no es la simple readaptación de un programa antiguo:
muchas veces implica incoporar la experiencia de los usuarios 
en nuevos requerimientos y 
la extensión de las capacidades de la apliación de legado.