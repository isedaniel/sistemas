\section{Reestructuración del software}

En este caso, Pressman reconoce dos tipos de reestructuración: 
la reestructuración de código y la reestructuración de datos.

En el primer caso,
la reestructuración del código puede o no modificar la \textbf{arquitectura}
del programa,
puesto que si esta resulta del mantenimiento en el tiempo de buenas prácticas
de ingeniería de software,
antes que un proceso de reingeniería propiamente dicho alcanzaría con la 
reestructuración del código de alguno módulos o subsistemas puntuales.
En cualquier caso,
el autor sugiere algunas técnicas para abordar la reestructuración,
entre las que se destacan:
\begin{enumerate}
    \item La \textbf{simplificación lógica} de Warnier,
    que implica el modelado siguiendo la lógica booleana y,
    mediante la aplicación de una serie de reglas,
    llegar a una reestructuración lógica,
    conforme a las reglas de la programación estructurada.
    \item Por otra parte, el \textbf{mapeo de módulos} y \textbf{recursos}
    puede permitir una representación del flujo del programa y de su arquitectura,
    resultando en sugerencias para lograr un mínimo acoplamiento de los primeros 
    y, por ende, una simplificación de esta última.
\end{enumerate}

Por su parte, la \textbf{reestructuración de datos} implica la evaluación 
de todos los enunciados que incluyan:
\begin{itemize}
    \item definiciones de datos 
    \item descripciones de entrada/salida 
    \item descripciones de interfaz
\end{itemize}

El objetivo es reconstruir el flujo de datos en la aplicación.
Una vez completado este diagnóstico inicial,
la reestructuración se abocará a:
\begin{itemize}
    \item \textbf{Estandarizar} el registro de datos,
    clarificando definiciones,
    buscando consistencia.
    \item \textbf{Racionalizar} el nombre de los datos,
    garantizando el seguimiento de la nomenclatura apropiada.
    \item \textbf{Modificar} estructuras de datos existentes,
    que puede implicar desde el cambio de las estructuras de datos en el código 
    hasta el paso de un tipo de base de datos a otra.
\end{itemize}