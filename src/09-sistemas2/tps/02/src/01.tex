\section{Introducción}

Pressman inicia su capítulo sobre mantenibilidad y reingeniería 
refiéndose al marco general en que estas dos actividades se despliegan. 
Refiere para ello al trabajo de Lehman y otros,
que proponen una \textit{teoría unificada de la evolución del software},
y cuyos leyes (o principios) más relevantes se enumeran a continuación:

\begin{enumerate}
    \item \textbf{Cambio contínuo.}
    El contexto computacional evoluciona con el tiempo y el sistema se tiene que
    adaptar a ese cambio.
    \item \textbf{Complejidad creciente.}
    A medida que un sistema evoluciona, la complejidad tiende a crecer 
    naturalmente.
    \item \textbf{Conservación de la familiaridad.}
    Conforme un sistema evoluciona, la familiaridad del personal de desarrollo,
    ventas y operaciones debe mantenerse, pero esto no siempre es así,
    principalmente por el incremento de la complejidad.
    \item \textbf{Crecimiento contínuo.}
    Para mantener satisfecho al usuario, el sistema debe incorporar nuevas 
    funcionalidades.
    \item \textbf{Declive de la calidad.}
    Por la rotación del personal, 
    por la falta de buenas prácticas,
    o por la evolución de las mismas,
    un sistema puede sufrir un declive en su calidad a lo largo del tiempo.
\end{enumerate}

En ese marco, Pressman presenta algunos conceptos, 
metodologías, técnicas y propuestas concretas,
para lidiar con los problemas que emergen de la naturaleza misma de la 
evolución del software.
El presente trabajo trata de sintetizar algunos de sus aportes al respecto.
Para ello se desarrollan los conceptos de mantenibilidad,
soportabilidad y reingeniería.
En esta última nos vamos a detener particularmente,
ya que el autor propone un interesante \textit{modelo cíclico},
que detalla las diferentes etapas que atraviesa un proceso de reingeniería,
con los desafíos que supone y diferentes recomendaciones que surgen.