\section{Introducción}

Pressman inicia su capítulo sobre mantenibilidad y reingeniería 
refiéndose al marco general en que estas dos actividades se despliegan. 
Refiere para ello al trabajo de Lehman y otros,
que proponen una \textit{teoría unificada de la evolución del software},
y cuyos leyes (o principios) más relevantes se enumeran a continuación:

\begin{enumerate}
    \item \textbf{Cambio contínuo.}
    El contexto computacional evoluciona con el tiempo y el sistema se tiene que
    adaptar a ese cambio.
    \item \textbf{Complejidad creciente.}
    A medida que un sistema evoluciona, la complejidad tiende a crecer 
    naturalmente.
    \item \textbf{Conservación de la familiaridad.}
    Conforme un sistema evoluciona, la familiaridad del personal de desarrollo,
    ventas y operaciones debe mantenerse, pero esto no siempre es así,
    principalmente por el incremento de la complejidad.
    \item \textbf{Crecimiento contínuo.}
    Para mantener satisfecho al usuario, el sistema debe incorporar nuevas 
    funcionalidades.
    \item \textbf{Declive de la calidad.}
    Por la rotación del personal, 
    por la falta de mantenimiento de buenas prácticas de Ingeniería,
    o por la evolución de las mismas,
    un sistema puede sufrir un declive en su calidad.
\end{enumerate}

El presente trabajo trata de sintetizar los aportes de Pressman y otros
en los campos de la mantenibilidad y la reingeniería.
Se desarrollan estos y otros conceptos estrechamente relacionados,
como los de soportabilidad o ingeniería inversa y
por último se presenta la propuesta metodológica 
del autor para afrontar procesos de reingeniería,
con sus correspondientes recomendaciones en la toma de decisiones
respecto de cada uno de sus componentes,
desde la producción de documentación
hasta la reescritura de código,
teniendo en cuenta los diferentes niveles de abstracción.