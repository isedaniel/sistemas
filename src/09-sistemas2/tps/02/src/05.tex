\section{Análisis de inventarios}

El análisis de inventarios consiste en el mantenimiento regular y 
actualizado de un listado de todas las piezas de software mantenidas por la 
organización.

Esta base de datos debe incluir, entre otros, tamaño de la aplicación,
tiempo de servicio, mantenibilidad, soportabilidad 
e importancia en los procesos de la empresa.

De esta forma, ponderando todos estos aspectos,
el análisis de inventarios hará emerger los potenciales candidatos a 
reingeniería, asignando los recursos -siempre escasos-
a aquellas aplicaciones que sean más críticas para los procesos más importantes 
de la empresa.

Pressman recomienda también la actualización regular de este listado,
puesto que la importancia para la empresa de las diferentes aplicaciones 
pueden variar con el tiempo,
y una aplicación que fue secundaria en un momento determinado de la empresa sea,
en el futuro,
una aplicación fundamental para un proceso o desafío nuevo que esta asuma.