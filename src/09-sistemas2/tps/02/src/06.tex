\section{Reestructuración de documentos}

Pressman señala que la documentación débil e incompleta suele ser una
característica compartida por los sistemas de legado.
Adicionalmente, componer o actualizar una documentación puede ser una actividad 
difícil y que consuma muchos recursos,
por lo cual resulta fundamental elegir con mucho criterio y pragmatismo el 
curso de acción.

Frente a esta situación, el autor sugiere posibilidades:
\begin{enumerate}
    \item Un programa que no es crítico, que se mantiene relativamente estático
    -es decir, sus características han permanecido estables durante mucho tiempo-
    o que se aproxima al final de su vida útil -y será reemplazado o dado de baja pronto-
    no merece los recursos de la organización y, probablemente, lo mejor sea 
    destinarlos a otras aplicaciones.
    \item Un sistema o un conjunto de subsistemas require de documentación,
    aunque no es urgente contar con ella: frente a esta situación el autor 
    propone un abordaje de \textbf{documentar cuando toque}, es decir,
    ir documentando a medida que trabajamos con cada uno de los módulos,
    de manera tal que, a lo largo del tiempo, la organización vaya 
    contando con documentación actualizada, útil y estable.
    \item Un sistema crítico para la empresa: en este caso, no hay otra opción a 
    desarrollar o actualizar una documentación apropiada y exhaustiva,
    aunque probablmente uno pueda distinguir subsistemas más importantes dentro 
    de un sistema importante para la organización y, en consecuencia,
    adoptar un abordaje iterativo e incremental en el desarrollo de la 
    documentación.
\end{enumerate}

En esta etapa no hay recetas \textit{a priori}, 
y el curso de acción tomado dependerá del análisis 
que el equipo de ingeniería haga en cada caso.