\section{Conclusiones}

Una organización con años de experiencia en su rubro,
con fuerte implementación de sistemas de información y desarrollo propio,
puede llegar a mantener miles de aplicaciones y módulos,
y que pueden ser de importancia crítica para sus procesos empresariales.

En determinado momento,
el costo del mantenimiento de apliaciones de legado,
que pueden tener décadas de servicio,
puede superar al costo de desarrollo de aplicaciones nuevas.

Esta situación demuestra la importancia fundamental que tienen las buenas 
prácticas de ingeniería de software,
en lo que respecta al diseño e implementación de aplicaciones,
teniendo en cuenta tanto la mantenibilidad como la soportabilidad desde 
un principio: una aplicación comprensible en una aplicación mantenible.

Cuando la inestabilidad aparece, la reingeniería es una herramienta que 
toda organización tiene que considerar,
apelando a todas sus etapas de manera cíclica:

El análisis de inventarios,
para detectar a tiempo las necesidades de reingeniería más críticas y urgentes;
la reestructuración de documentos, con un enfoque pragmático,
para aprovechar adecuandamente los recursos escasos;
la ingeniería inversa para \textit{redescubrir} el diseño de aplicaciones 
pobremente documentadas;
la reestructuración, tanto del código como de los datos,
reduciendo el código incomprensible,
el acople de módulos y las incosistencias en bases de datos;
y, en los casos en que amerite,
recurrir a la ingeniería hacia adelante,
reemplazando módulos antiguos a la luz de la experiencia obtenida,
por los equipos de desarrollo así como por la realimentación proveniente 
de los destinatarios de todo sistema: sus usuarios.