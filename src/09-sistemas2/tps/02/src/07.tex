\section{Ingeniería Inversa}

El término ingeniería inversa se origina en el mundo del Hardware,
y refiere comunmente al desensamblado 
y análisis de un producto de la competencia,
con el objetivo de obtener información sobre el mismo.
Una proceso de ingenieria inversa exitosa resultaba en 
sugerencias sobre el diseño y fabricación de un producto determinado.

Este concepto, trasladado al mundo del Software,
implica el análisis del código fuente de un producto,
para elaborar una representación del mismo 
en un \textit{nivel de abstracción superior}.
Un proceso de ingeniería inversa exitoso,
en este contexto,
redunda en una \textbf{recuperación del diseño} del software,
en particular sobre su diseño de datos,
arquitectura y procedimientos.

En concreto, la ingeniería inversa parte del propio \textbf{código fuente}
de la aplicación,
para desarrollar especificaciones sobre:
\begin{enumerate}
    \item \textbf{Estructuras de datos}, 
    definiendo clases de objetos,
    agrupando variables relacionadas, 
    identificando tipos de datos abstractos.
    \item \textbf{Estructura de la base de datos}, 
    construyendo modelos de objetos, 
    determinando claves candidatas,
    refinando clases tentantivas,
    descubriendo asocaciones entre las entidades,
    llegando en última instancia a un mapeo de la base de datos.
    \item \textbf{Procesamiento},
    teniendo en cuenta la funcionalidad global del sistema,
    creando diagramas de bloques que representen la interacción entre 
    abstracciones funcionales,
    usando herramientas automatizadas para comprender el código
    de forma semántica
    \item \textbf{Interfaces de usuario},
    especificando estructura y comportamiento de la interfaz,
    reconciendo acciones básicas a procesar -como clicks, atajos de teclado-,
    y las respuestas a esas acciones.
\end{enumerate}