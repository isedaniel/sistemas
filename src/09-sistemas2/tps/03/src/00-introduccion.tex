\section{Introducción}

Este documento presenta una propuesta para la implementación 
de un sistema informático,
cuyo principal objetivo es colaborar con la toma de decisiones de una empresa
dedicada a la venta de productos de computación,
en particular en lo que refiere a sus compras y ventas.

Aunque este sistema se haya diseñado para el rubro de la informática,
bien podría generalizarse algunas de sus propuestas para otras empresas 
dedicadas a la comercialización de otro tipo de productos,
en particular aquellas que se dediquen a la venta de productos importados
\footnote{Este diseño se basa en mi experiencia trabajando en la empresa familiar, 
dedicada a la venta de equipos informáticos e insumos de impresión.}.

En la anterior iteración fundamentábamos la necesidad de este sistema 
en el contexto de la economía argentina,
a la que caracterizábamos, 
en aquel momento, 
como inflacionaria 
y con dificultades para la importación.

Aunque la inflación haya bajado 
la economía argentina enfrenta aun niveles de inflación relativamente altos
-tanto para el mundo como a nivel regional-.
Adicionalmente,
uno de los factores más importantes para calcular los costos,
en esta y en toda industria con alto componente de importación,
es el precio del dólar.
Aunque este se encuentre \textit{relativamente} estable,
sus fluctuaciones continúan, inclusive de forma intradiaria,
lo que impacta directamente sobre la rentabilidad de este tipo de empresa.

Por otra parte,
en los últimos años han aparecido nuevos competidores:
pensemos, por ejemplo, en el servicio puerta a puerta de Amazon,
o el recientemente lanzado servicio de venta desde Estados Unidos,
incorporado a Mercado Libre.
Este contexto, junto al lento pero sostenido crecimiento de las importaciones 
en el país,
refuerzan la necesidad de contar con la información del mercado adecuada,
en el momento adecuado.
Bajar costos y 
-en paralelo- crecer en competitividad son dos desafíos 
fundamentales de este período.

Por ello, 
el sistema diseñado propone acompañar a la empresa,
desde la compra de un producto a proveedor,
pasando por la fijación del precio,
hasta la emisión de la factura electrónica correspondiente.
La idea central es ofrecer información sintética y oportuna 
sobre los precios de distintos productos,
colaborando con la toma de decisiones y,
de esa forma,
bajando costos y mejorando la competitividad.

El primer desafío consiste en identificar 
al mejor proveedor para un determinado producto.
Para ello,
el sistema recabará datos de precios,
recurriendo a las interfaces de programación de aplicaciones (APIs)
de aquellos proveedores que ofrezcan el servicio,
así como al \textit{scrapeo web} para aquellos que no lo ofrezcan.
De esta manera,
el encargado de compras de la empresa contará con la información necesaria para 
negociar las mejores condiciones y bajar los costos.

El segundo desafío se vincula con la competitividad,
e implica identificar el precio de mercado de un producto.
Sabemos que con una economía en proceso de apertura,
pero aun relativamente cerrada,
los precios de los productos importados pueden sufrir de gran dispersión.
Por ello,
el sistema recurrirá a las APIs (o \textit{scrapeo web}, según corresponda),
de la competencia.
Esta información,
sumada a la información provista por los proveedores,
permitirá al encargado de ventas fijar precios competitivos,
que cuiden a su vez la rentabilidad y el modelo de negocios de la empresa.

Por último,
para gestionar la venta presencial,
el sistema será capaz de conectarse con la 
API de la Agencia de Recaudación y Control Aduanero (ARCA),
con el objetivo de emitir la factura digital correspondiente.

De esta forma,
el sistema pretende acompañar a la empresa comercial en todo su proceso de valor,
desde la compra del producto hasta la concreción de la venta.