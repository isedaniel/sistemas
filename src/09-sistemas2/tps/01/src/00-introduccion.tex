\section{Introducción}

El presente proyecto pretende describir un sistema de gestión de una Pyme dedicada a la venta de equipos informáticos
\footnote{Este diseño se basa en mi experiencia de más de 15 años de trabajo en la Pyme familiar}.
Aunque los ejemplos concretos serán pensados para la venta de equipos de computación, 
el sistema se podría generalizar para empresas cuyos productos tengan una importante proporción de insumos importados.
Afirmamos esto porque,
en una economía inflacionaria y con dificultades para importar,
como la Argentina de los últimos años,
puede haber una dispersión de precios importante,
tanto en costos y como en precios finales de los productos.
Como consumidores, 
muchas veces cuesta conocer el precio de mercado de los productos. 
Esta característica se agrava para el caso de los productos importados,
cuya escasez relativa puede hacer que el mismo producto valga 2 o 3 veces más dependiendo del lugar de compra.

Por ello, 
el sistema diseñado propone acompañar a la empresa desde la compra a proveedor hasta la emisión de la factura.
La idea es vincular los precios ofrecidos por distintos proveedores que,
tal y como lo percibe el cliente final,
también tienen grandes variaciones,
colaborando con la toma de decisiones de la Pyme y asegurando un mejor precio.
Adicionalmente,
el sistema recabará datos de la competencia,
recurriendo tanto a APIs como al \textit{scrapeo} web,
tratando de establecer el precio de mercado un producto.
En un contexto de alta dispersión, 
acompañado de una baja de la inflación,
tener un precio competitivo es muy importante.
Con estos datos,
tanto los encargados de compras como de ventas podrán establecer las mejores estrategias de precios,
disminuyendo costos y cuidando a los clientes en simultáneo.

Por último,
el sistema deberá ser capaz de conectarse con la API de la ahora denominada
Agencia de Recaudación y Control Aduanero (ARCA),
para emitir la factura digital correspondiente a la venta presencial.