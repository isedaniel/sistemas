\section{Introducción}

El presente proyecto propone un sistema de información
orientado a mejorar la toma de decisiones de una empresa comercial,
en particular en lo que refiere a la compra y venta de productos.

Aunque,
en el marco del presente proyecto,
se diseñará pensando en la venta de equipos informáticos,
el sistema podría generalizarse para otro rubros,
en particular para empresas que se dediquen a la venta de productos importados
\footnote{El diseño se basa en mi experiencia trabajando en la empresa familiar, 
dedicada a la venta de equipos informáticos e insumos de impresión.}.

En la anterior iteración fundamentábamos la necesidad de este sistema 
en el contexto de la economía argentina,
a la que caracterizábamos, 
en aquel momento, 
como inflacionaria 
y con dificultades para la importación.

Aunque la inflación haya bajado 
la economía argentina enfrenta aun niveles de inflación relativamente altos,
con un precio del dólar fluctuando con cierta importancia,
a veces de forma intradiaria.

Por otra parte,
el lento pero sostenido crecimiento de las importaciones,
así como la llegada de otros actores al comercio minorista,
(pensemos, por ejemplo, en el servicio puerta a puerta de Amazon)
refuerzan la necesidad de contar con la información del mercado adecuada,
en el momento adecuado.
Bajar costos y crecer en competitividad son dos desafíos fundamentales en este contexto.

Por ello, 
el sistema diseñado propone acompañar a la empresa comercial,
desde la compra de un producto a proveedor,
la fijación del precio teniendo en cuenta a la competencia,
y la emisión de la factura electrónica correspondiente.
La idea central es ofrecer un conocimiento rápido de los precios de distintos productos,
colaborando con la toma de decisiones de la empresa, bajando costos y mejorando la competitividad.

Para responder a la primera necesidad
(comparar costos para elegir al mejor proveedor por producto)
el sistema recabará datos de precios de proveedores,
recurriendo a las interfaces de programación de aplicaciones (APIs)
de aquellos proveedores que ofrezcan el servicio,
así como al \textit{scrapeo web} para aquellos que no lo ofrezcan.
De esta manera,
el encargado de compras de la empresa contará con la información necesaria para negociar de la mejor manera.

En segundo término,
tenemos que identificar el precio de mercado del producto.
Sabemos que con una economía en proceso de apertura,
pero aun relativamente cerrada,
los precios de los productos importados pueden sufrir de gran dispersión.
Por ello,
el sistema recurrirá a las APIs (o \textit{scrapeo web}, según corresponda),
de la competencia,
ofreciendo al encargado de ventas la posibilidad de fijar precios con esta información en mente.

Por último,
una vez realizada la venta,
el sistema será capaz de conectarse con la API de la Agencia de Recaudación y Control Aduanero (ARCA),
a los efectos de emitir la factura digital correspondiente.

De esta forma,
el sistema pretende acompañar a la empresa desde la compra del producto hasta la concreción de la venta.