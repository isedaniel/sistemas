\section{Objetivo general}

La idea central es ofrecer un conocimiento rápido de los precios de distintos productos,
colaborando con la toma de decisiones de la empresa, bajando costos y mejorando la competitividad.

\begin{enumerate}
	\item Acceder a la información suministrada por los proveedores en listas de precios y páginas web, 
	en particular números de parte, descripciones y precios de los productos.
	\item Recabar información en portales de venta online, como Mercado Libre, sobre el precio de mercado de los productos 
	\item Almacenar los precios fijados por el encargado de ventas, para que sean empleados por el personal de ventas.
	\item Permitir la emisión de factura electrónica desde el sistema, integrándolo a la API de ARCA.
\end{enumerate}

\section{Objetivos específicos}

Para responder a la primera necesidad
(comparar costos para elegir al mejor proveedor por producto)
el sistema recabará datos de precios de proveedores,
recurriendo a las interfaces de programación de aplicaciones (APIs)
de aquellos proveedores que ofrezcan el servicio,
así como al \textit{scrapeo web} para aquellos que no lo ofrezcan.
De esta manera,
el encargado de compras de la empresa contará con la información necesaria para negociar de la mejor manera.

En segundo término,
tenemos que identificar el precio de mercado del producto.
Sabemos que con una economía en proceso de apertura,
pero aun relativamente cerrada,
los precios de los productos importados pueden sufrir de gran dispersión.
Por ello,
el sistema recurrirá a las APIs (o \textit{scrapeo web}, según corresponda),
de la competencia,
ofreciendo al encargado de ventas la posibilidad de fijar precios con esta información en mente.

Por último,
una vez realizada la venta,
el sistema será capaz de conectarse con la API de la Agencia de Recaudación y Control Aduanero (ARCA),
a los efectos de emitir la factura digital correspondiente.
