
\section{Descripción detallada de casos de uso}

\subsection{Consultar precios disponibles para un producto}

\textbf{Resumen:}
Dada una descripción o número de parte
-que es un nomenclador único de productos-,
el encargado de compras recibirá un listado de precios de dicho producto,
que será descargado de las páginas web e interfaces de programación de los diversos proveedores,
según el caso.

\textbf{Precondición:} 
el encargado de compras ingresa parte de la descripción/número de parte en el sistema para iniciar una búsqueda

\textbf{Escenario principal:}
\begin{enumerate}
	\item El encargado de compras ingresa a la opción búsqueda.
	\item El sistema solicita ingreso de número de parte o descripción del producto.
	\item El encargado de compras ingresa cualquiera de los dos datos, completos o en parte.
	\item El sistema muestra coincidencias al encargado de compras, para que elija cuál era el artículo que buscaba.
	\item El encargado de compras confirma el producto que buscaba.
	\item El sistema descarga la información necesaria, 
	recurriendo en los casos en que sea posible a las interfaces de programación de los proveedores (API),
	al \textit{Web Scrapping} donde no sea posible. Una vez recopilada la información, 
	se presenta al encargado por nombre de proveedor y precio.
	\item El encargado toma las decisiones correspondientes e inicia el proceso de compra por canales habituales.
\end{enumerate}

\textbf{Poscondición:}
El encargado de compras dispone de información suficiente en tiempo real para hacer una compra minimizando costos.

\textbf{Flujo alternativo:}

\textbf{A1} La descripción no coincide con un producto cargado en base de datos 

La secuencia comienza en el punto 3.

\begin{enumerate}
	\item[4.] El sistema informa que no existen coincidencias con productos en base de datos,
	y solicita cargar número de parte íntegramente para buscar por nomenclador único.
	\item[5.] El usuario carga número de parte íntegramente.
	\item[6.] El sistema muestra coincidencias tomadas directamente de proveedor,
	de ser el producto buscado se incorpora a base de datos,
	para facilitar búsqueda por descripción en el futuro.
\end{enumerate}

El escenario vuelve a punto 6.

\subsection{Emitir factura electrónica}

\textbf{Resumen:}
Personal de venta, 
luego de la brindar atención y asesoramiento al comprador minorista,
requiere de la información del sistema
-descripción del producto, cantidad, disponibilidad y precio-
para emitir una factura electrónica.
Esto puede hacerse sincronizando al sistema con la API de la ex AFIP,
ahora ARCA.

\textbf{Actores:} Personal de ventas (primario), Sistema de ARCA (secundario).

\textbf{Precondición:} 
Personal de ventas ha concluido con la atención y asesoramiento correspondiente,
el cliente minorista decide proceder con la compra,
por lo cual el personal de ventas solicita iniciar el procedimiento para emitir factura electrónica.

\textbf{Escenario principal:}
\begin{enumerate}
	\item Personal de ventas ingresa a la sección de emisión de factura electrónica
	\item El sistema solicita ingresar identificación de cliente, mediante CUIT o código interno alfanumérico.
	\item Personal de ventas ingresa identificación de cliente o nomenclador por categoría, como CF para consumidor final.
	\item El sistema, de acuerdo a la identificación del cliente, determina si la boleta será tipo A o tipo B.
	\item El sistema, luego de determinar el tipo de factura, solicita ingresar el primer producto.
	\item Personal de ventas ingresa el producto, por número de parte, código de barras o descripción.
	\item El sistema muestra las coincidencias en base de datos, esperando confirmación.
	\item Personal de ventas confirma la coincidencia e ingresa cantidad.
	\item El sistema aguarda la carga de nuevo item, en dicho caso vuelve a punto 6.
	\item Personal de ventas indica que finaliza la carga de productos, solicitando autorización de comprobante.
	\item El sistema envía petición a servicio de ARCA, incluyendo: CUIT de cliente, cantidad de productos, Iva y precio final.
	\item Servidor de ARCA confirma autorización de boleta, devolviendo número de comprobante electrónico.
	\item El sistema solicita al personal de ventas confirme para imprimir el comprobante o exportarlo a PDF.
	\item Personal de ventas confirma y obtiene el comprobante en el formato deseado.
\end{enumerate}

\textbf{Poscondición:}
Personal de ventas tiene el comprobante correspondiente a la compra del cliente y puede finalizar el proceso de compra entregando comprobante y producto.

\textbf{Flujo alternativo:}

\textbf{A1} El personal de ventas ingresa el CUIT del cliente, pero este no está cargado en base de datos

Secuencia comienza en punto 3

\begin{enumerate}
	\item[4.] El sistema hace una petición al sistema de ARCA de los datos del contribuyente, 
	empleando el CUIT para identificarlo.
	\item[5.] ARCA devuelve los datos del contribuyente.
	\item[6.] El sistema los incorpora a la base de datos.
\end{enumerate}

Vuelve al escenario principal en punto 4

\textbf{A2} Personal de ventas ingresa un producto, pero este no existe en base de datos

Secuencia comienza en 6.

\begin{enumerate}
	\item[7.] El sistema no encuentra coincidencias, por lo que pide reingreso al personal de ventas.
	\item[8.] Personal de ventas prueba con descripción, código de barras o ingreso manual de nombenclador.
\end{enumerate}

Vuelve a flujo normal en punto 8.
