\documentclass[11pt]{article}
\usepackage[a4paper, margin=2.54cm]{geometry}
\usepackage[spanish]{babel}

% imágenes
\usepackage{graphicx}
%\graphicspath{{img}}

% fuentes de conjuntos numéricos
\usepackage{amsfonts}

% math
\usepackage{amsmath, amssymb}

% gráficos y plots
\usepackage{tikz}
%\usepackage{pgfplots}
%\pgfplotsset{width=10cm, compat=1.9}
\usetikzlibrary{babel}

\setlength{\jot}{8pt}
\setlength{\parindent}{0cm}

% espacio entre párrafos
\usepackage[skip=10pt plus1pt, indent=12pt]{parskip}

% cancelar términos
\usepackage{cancel}

% links
%\usepackage[colorlinks=true,
%    urlcolor=blue]{hyperref}

% shapes
%\usetikzlibrary{shapes.geometric}

% incluir pdfs
%\usepackage{pdfpages}

\begin{document}

\thispagestyle{empty}

\begin{center}
	\vspace*{.5cm}
	\includegraphics[scale=.6]{~/Pictures/udemm-logo.png}\\
	\vspace{.2cm}
	\Large
	\textbf{Facultad de Ingeniería}\\
	\textbf{Ingeniería en Sistemas}\\
	\vspace{2cm}

	\Huge
	Análisis de Sistemas I\\
	Trabajo Práctico N\(^\circ\) 2 \\
	\vfill

	\raggedright
	\Large
	Docentes:
	\begin{itemize}
		\item[] Mg. Margarita Castronuovo \\
	\end{itemize}
	Alumno:
	\begin{itemize}
		\item[] Daniel Ise
	\end{itemize}
	Legajo:
	\begin{itemize}
		\item[] 28547
	\end{itemize}
	Fecha:
	\begin{itemize}
		\item[] Noviembre, 2024
	\end{itemize}
\end{center}

\pagebreak

\section{Introducción}

El presente proyecto pretende describir un sistema de gestión de una Pyme dedicada a la venta de equipos informáticos
\footnote{Este diseño se basa en mi experiencia de más de 15 años de trabajo en la Pyme familiar}.
Aunque los ejemplos concretos serán pensados para la venta de equipos de computación, 
el sistema se podría generalizar para empresas cuyos productos tengan una importante proporción de insumos importados.
Afirmamos esto porque,
en una economía inflacionaria y con dificultades para importar,
como la Argentina de los últimos años,
puede haber una dispersión de precios importante,
tanto en costos y como en precios finales de los productos.
Como consumidores, 
muchas veces cuesta conocer el precio de mercado de los productos. 
Esta característica se agrava para el caso de los productos importados,
cuya escasez relativa puede hacer que el mismo producto valga 2 o 3 veces más dependiendo del lugar de compra.

Por ello, 
el sistema diseñado propone acompañar a la empresa desde la compra a proveedor hasta la emisión de la factura.
La idea es vincular los precios ofrecidos por distintos proveedores que,
tal y como lo percibe el cliente final,
también tienen grandes variaciones,
colaborando con la toma de decisiones de la Pyme y asegurando un mejor precio.
Adicionalmente,
el sistema recabará datos de la competencia,
recurriendo tanto a APIs como al \textit{scrapeo} web,
tratando de establecer el precio de mercado un producto.
En un contexto de alta dispersión, 
acompañado de una baja de la inflación,
tener un precio competitivo es muy importante.
Con estos datos,
tanto los encargados de compras como de ventas podrán establecer las mejores estrategias de precios,
disminuyendo costos y cuidando a los clientes en simultáneo.

Por último,
el sistema deberá ser capaz de conectarse con la API de la ahora denominada
Agencia de Recaudación y Control Aduanero (ARCA),
para emitir la factura digital correspondiente a la venta presencial.

\section{Objetivos}

\begin{enumerate}
	\item Acceder a la información suministrada por los proveedores en listas de precios y páginas web, 
	en particular números de parte, descripciones y precios de los productos.
	\item Recabar información en portales de venta online, como Mercado Libre, sobre el precio de mercado de los productos 
	\item Almacenar los precios fijados por el encargado de ventas, para que sean empleados por el personal de ventas.
	\item Permitir la emisión de factura electrónica desde el sistema, integrándolo a la API de ARCA.
\end{enumerate}

\section{Identificación de los actores}

\subsection*{Actores primarios}
\begin{enumerate}
	\item Encargado de compras
	\item Encargado de ventas
	\item Personal de ventas
\end{enumerate}

\subsection*{Actores secundarios}
\begin{enumerate}
	\item Proveedores 
	\item Portales de venta online 
	\item Sistemas de ARCA
\end{enumerate}

\section{Diagrama de contexto}

\begin{figure}[ht]
	\vspace{20pt}
	\centering
	\vspace{15pt}
	\includegraphics[width=.9\textwidth]{src/00-diagrama-contexto}
	\caption{Diagrama de contexto}
	\vspace{15pt}
\end{figure}

\section{Casos de uso}

\subsection*{Encargado de compras}
\begin{enumerate}
	\item Consultar precios disponibles para un producto, 
	de acuerdo a su número de parte, 
	para optar por el proveedor que ofrece mejores condiciones.
	\item Seguir el stock disponible,
	previendo necesidades de corto y mediano plazo.
\end{enumerate}

\subsection*{Encargado de ventas}
\begin{enumerate}
	\item Conocer los precios de recompra,
	para que esa información permita definir un precio al consumidor final.
	\item Conocer los precios de la competencia,
	para tener un precio competitivo que fidelice al cliente.
	\item Conocer el margen actual para un producto,
	de manera tal de ofrecer mejores precios a los clientes importantes.
\end{enumerate}

\subsection*{Personal de ventas}
\begin{enumerate}
	\item Consultar precios de los productos para informar al consumidor final.
	\item Emitir factura electrónica,
	autorizada por ARCA, aprovechando la base de datos del sistema.
\end{enumerate}

\section{Diagrama de casos de uso}

\begin{figure}[ht]
	\vspace{20pt}
	\centering
	\vspace{15pt}
	\includegraphics[width=\textwidth]{src/01-diagrama-casos-uso}
	\caption{Diagrama de casos de uso}
	\vspace{15pt}
\end{figure}

\pagebreak

\section{Descripción resumida de casos de uso}

\subsection{Consultar precios disponibles para un producto}

Dado un número de parte,
que es un nomenclador único de productos,
el encargado de compras contará con un listado de precios de dicho producto,
que será descargado de las páginas web e interfaces de programación de los diversos proveedores,
según el caso.

\textbf{Actores:} Encargado de compras (primario), proveedores (secundario).

\subsection{Seguir el stock disponible, previendo necesidades de corto y mediano plazo.}

Dado un ingreso de producto en sucursal,
el encargado de compras debe disponer de un formulario que permita dar de alta el producto en la base de datos propia,
de ser necesario,
y actualizar el stock.

\textbf{Actores:} Encargado de compras (primario).

\subsection{Conocer los precios de recompra}

El encargado de ventas puede necesitar la información del costo de recompra de un producto,
para reaccionar a un cambio en el precio de mercado.
En ocasiones, un precio al cliente final puede caer y el encargado de ventas debe ser capaz de reaccionar a ello.
Este precio se renovará directamente, 
como lo hace cuando el encargado de compras quiere saber qué proveedor ofrece mejores condiciones.

\textbf{Actores:} Encargado de ventas (primario), proveedores (secundario).

\subsection{Conocer los precios de la competencia}

Para poner un precio competitivo,
que cuide a los clientes al tiempo que protege las finanzas de la empresa,
el encargado de ventas debe de disponer del precio de mercado casi en tiempo real.
Para ello, se descargará el mismo, empleando el número de parte,
de portales de venta como ser Mercado Libre y otros.
Con esta información, 
el encargado de ventas definirá los precios de lista de los productos,
que será el precio al consumidor final.

\textbf{Actores:} Encargado de ventas (primario), portales de venta online (secundario).

\subsection{Calcular el margen actual para un producto}

Dado el precio de recompra y el precio de lista establecido por el encargado de compras,
la empresa obtiene un margen por la venta unitaria de los productos.
Sin embargo,
en ocasiones es necesario ofrecer descuentos, disminuyendo dicho margen,
para ofrecer un precio competitivo a clientes institucionales, revendedores u otras empresas.

\textbf{Actores:} Encargado de ventas (primario), proveedores (secundario), portales de venta online (secundario).

\subsection{Consultar precio de un producto}

El personal de venta atiende al comprador minorista y debe consultar el último precio disponible,
decidido ya por el encargado de ventas.
Estos precios pueden cambiar varias veces en una semana y el personal de venta debe estar sincronizado con dichos cambios.

\textbf{Actores:} Personal de ventas (primario), proveedores (secundario).

\subsection{Emitir factura electrónica}

El personal de venta, 
luego de la brindar atención y asesoramiento al comprador minorista,
puede necesitar utilizar la información del sistema
-descripción del producto, cantidad, disponibilidad y precio-
para emitir una factura electrónica.
Esto puede hacerse sincronizando al sistema con la API de la ex AFIP,
ahora ARCA.

\textbf{Actores:} Personal de ventas (primario), Sistema de ARCA (secundario).

\pagebreak

\section{Descripción detallada de casos de uso}

\subsection{Consultar precios disponibles para un producto}

\textbf{Resumen:}
Dada una descripción o número de parte
-que es un nomenclador único de productos-,
el encargado de compras recibirá un listado de precios de dicho producto,
que será descargado de las páginas web e interfaces de programación de los diversos proveedores,
según el caso.

\textbf{Precondición:} 
el encargado de compras ingresa parte de la descripción/número de parte en el sistema para iniciar una búsqueda

\textbf{Escenario principal:}
\begin{enumerate}
	\item El encargado de compras ingresa a la opción búsqueda.
	\item El sistema solicita ingreso de número de parte o descripción del producto.
	\item El encargado de compras ingresa cualquiera de los dos datos, completos o en parte.
	\item El sistema muestra coincidencias al encargado de compras, para que elija cuál era el artículo que buscaba.
	\item El encargado de compras confirma el producto que buscaba.
	\item El sistema descarga la información necesaria, 
	recurriendo en los casos en que sea posible a las interfaces de programación de los proveedores (API),
	al \textit{Web Scrapping} donde no sea posible. Una vez recopilada la información, 
	se presenta al encargado por nombre de proveedor y precio.
	\item El encargado toma las decisiones correspondientes e inicia el proceso de compra por canales habituales.
\end{enumerate}

\textbf{Poscondición:}
El encargado de compras dispone de información suficiente en tiempo real para hacer una compra minimizando costos.

\textbf{Flujo alternativo:}

\textbf{A1} La descripción no coincide con un producto cargado en base de datos 

La secuencia comienza en el punto 3.

\begin{enumerate}
	\item[4.] El sistema informa que no existen coincidencias con productos en base de datos,
	y solicita cargar número de parte íntegramente para buscar por nomenclador único.
	\item[5.] El usuario carga número de parte íntegramente.
	\item[6.] El sistema muestra coincidencias tomadas directamente de proveedor,
	de ser el producto buscado se incorpora a base de datos,
	para facilitar búsqueda por descripción en el futuro.
\end{enumerate}

El escenario vuelve a punto 6.

\subsection{Emitir factura electrónica}

\textbf{Resumen:}
Personal de venta, 
luego de la brindar atención y asesoramiento al comprador minorista,
requiere de la información del sistema
-descripción del producto, cantidad, disponibilidad y precio-
para emitir una factura electrónica.
Esto puede hacerse sincronizando al sistema con la API de la ex AFIP,
ahora ARCA.

\textbf{Actores:} Personal de ventas (primario), Sistema de ARCA (secundario).

\textbf{Precondición:} 
Personal de ventas ha concluido con la atención y asesoramiento correspondiente,
el cliente minorista decide proceder con la compra,
por lo cual el personal de ventas solicita iniciar el procedimiento para emitir factura electrónica.

\textbf{Escenario principal:}
\begin{enumerate}
	\item Personal de ventas ingresa a la sección de emisión de factura electrónica
	\item El sistema solicita ingresar identificación de cliente, mediante CUIT o código interno alfanumérico.
	\item Personal de ventas ingresa identificación de cliente o nomenclador por categoría, como CF para consumidor final.
	\item El sistema, de acuerdo a la identificación del cliente, determina si la boleta será tipo A o tipo B.
	\item El sistema, luego de determinar el tipo de factura, solicita ingresar el primer producto.
	\item Personal de ventas ingresa el producto, por número de parte, código de barras o descripción.
	\item El sistema muestra las coincidencias en base de datos, esperando confirmación.
	\item Personal de ventas confirma la coincidencia e ingresa cantidad.
	\item El sistema aguarda la carga de nuevo item, en dicho caso vuelve a punto 6.
	\item Personal de ventas indica que finaliza la carga de productos, solicitando autorización de comprobante.
	\item El sistema envía petición a servicio de ARCA, incluyendo: CUIT de cliente, cantidad de productos, Iva y precio final.
	\item Servidor de ARCA confirma autorización de boleta, devolviendo número de comprobante electrónico.
	\item El sistema solicita al personal de ventas confirme para imprimir el comprobante o exportarlo a PDF.
	\item Personal de ventas confirma y obtiene el comprobante en el formato deseado.
\end{enumerate}

\textbf{Poscondición:}
Personal de ventas tiene el comprobante correspondiente a la compra del cliente y puede finalizar el proceso de compra entregando comprobante y producto.

\textbf{Flujo alternativo:}

\textbf{A1} El personal de ventas ingresa el CUIT del cliente, pero este no está cargado en base de datos

Secuencia comienza en punto 3

\begin{enumerate}
	\item[4.] El sistema hace una petición al sistema de ARCA de los datos del contribuyente, 
	empleando el CUIT para identificarlo.
	\item[5.] ARCA devuelve los datos del contribuyente.
	\item[6.] El sistema los incorpora a la base de datos.
\end{enumerate}

Vuelve al escenario principal en punto 4

\textbf{A2} Personal de ventas ingresa un producto, pero este no existe en base de datos

Secuencia comienza en 6.

\begin{enumerate}
	\item[7.] El sistema no encuentra coincidencias, por lo que pide reingreso al personal de ventas.
	\item[8.] Personal de ventas prueba con descripción, código de barras o ingreso manual de nombenclador.
\end{enumerate}

Vuelve a flujo normal en punto 8.

\section{Modelo de datos}

Para plantear el modelo de datos,
adoptamos el enfoque de listar las diferentes tablas que puede necesitar cada uno de los actores.

El encargado de compras puede necesitar:
\begin{itemize}
	\item Precio de producto por proveedor 
	\item Productos por stock 
	\item Precio histórico por producto
\end{itemize}

El encargado de ventas puede necesitar:
\begin{itemize}
	\item Precio de producto por competidor
	\item Precio histórico de ventas por cliente
	\item Margen de ganancia por venta
\end{itemize}

El personal de ventas puede necesitar:
\begin{itemize}
	\item Precio por producto
	\item Productos por stock 
\end{itemize}

Las entidades serían entonces:
\begin{itemize}
	\item Producto
	\item Cliente 
	\item Proveedor 
	\item Competidor 
\end{itemize}

Las relaciones:
\begin{itemize}
	\item Producto-proveedor 
	\item Producto-cliente 
	\item Producto-competidor 
\end{itemize}

Dadas estas entidades y relaciones,
la primera iteración del modelo de datos entidad-relación propuesta se presenta a continuación:

\begin{figure}[ht]
	\vspace{20pt}
	\centering
	\vspace{15pt}
	\includegraphics[width=\textwidth]{src/02-modelo-entidad-relacion}
	\caption{Modelo entidad-relación}
	\vspace{15pt}
\end{figure}

\end{document}
