\section{Clase 19 de mayo}

\subsection{Por mail}

Enviar trabajo, nombre y fecha de la exposición que vamos a hacer.
Elegimos un tema, ponemos un día y exponemos.
Dentro de las 3 semanas previas al parcial.
Tiene que ser antes del parcial.

Además,
hay que mandarle a la profe nuestros números de telefono,
para la creación de un grupo de Whatsapp,

\subsection{Proyecto final}

En cuanto a la siguiente iteración del trabajo final,
debemos:
\begin{itemize}
    \item Refinar el proyecto, corregir errores
    \item Introducir diagrama de interacción
    \item Diagrama de secuencia
    \item Diagrama de Comunicación
    \item Mejorar introducción, visión y misión de la organización y del proyecto 
\end{itemize}

Alcanza con poner un diagrama de secuencia de sistema para cada caso de uso.
Debajo de cada caso de uso,
ponemos un pedacito del diagrama de clase,
referido al caso de uso,
y el diagrama de secuencia del caso de uso.

Extraemos los mensajes del caso de uso,
nombramos los mensajes (que van y vienen, del actor a la interfaz y viceversa),
estos nombres son \textit{independientes} de la implementación,
es como un pseudocódigo pero no implica nada en relación a la implementación final.
Es una traducción del caso de uso,
de forma verbal,
al diagrama de secuencia de sistema.
