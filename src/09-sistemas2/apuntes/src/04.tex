\section{Cuarta clase}

7 de abril, 2025

\subsection{Instancias extend e instancias include}

La relación \texttt{include} significa que un caso de uso \textit{necesariamente} utliza la funcionalidad de otro caso de uso para completarse.

En contraste, una relación \texttt{extend} implica que un caso de uso \textit{puede} agregar funcionalidad de otro.
Es decir, el caso de uso puede funcionar perfectamente por si solo, pero puede extenderse en ciertas circunstancias.

\subsection{Relación include}

La relación \texttt{include} cuenta entre sus \textbf{ventajas} que las descripciones son más cortas.
Además, se identifica funcionalidad común, descubriendo componentes en la implementación.

Entre sus \textbf{desventajas}: hace los diagramas más difíciles de leer (cuando hay muchos, como en herencia).

\subsection{Cuándo usar include}

Cuando un caso incorpora \textit{explícitamente} el comportamiento de otro.

Para evitar \textit{repeticiones}.

Cuando un conjunto de casos de uso comparten \textit{funcionalidad}.

Cuando un caso es extenso y difícil de leer (se lo divide para que sea más manejable).

Hay un conjunto de idas y vueltas que se repiten:
se crea un caso de uso include común.

\subsection{Relación extend}

Cuando un caso de uso incluye \textit{opcionalmente} otro caso de uso. 
\textbf{Atención.} En esquema, la flecha va de secundario a principal (apunta al \textit{caller}).

Si se cumple una condición, el caso de uso base llama al caso de uso secundario.

\subsection{Cuando usar extend}

Cuando se desea describir una variación del comportamiento normal de un caso de uso.

Para agrupar conjuntos de eventos que se ejecutan opcionalmente.

Cuando los flujos alternativos se tornan grandes y difíciles de leer: 
creamos casos de uso que engloban esos flujos alternativos.
