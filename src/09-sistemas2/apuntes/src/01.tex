\section{Primera Clase}

17 de marzo, 2025.

\subsection{Introducción a la materia}

Vamos a seguir trabajando sobre el proyecto de análisis 1.

\textbf{Prácticos de muestra.}
Hay prácticos de ejemplo para que nos demos una idea de lo que se pretende en la materia.

\subsection{Contenidos de la materia}

\textbf{Unidad 1.}
Repaso de lo visto. Ciclos de vida. Metodologías.
UML. 
Las dos herramientas que trabajamos son los \textbf{casos de uso} y \textbf{diagramas de clase}.

Ambas herramientas tienen teoría, pero lo más importante es el \textit{uso}, la \textit{práctica}.

\textbf{Caso de uso.}
Tiene un aspecto \textbf{gráfico} y un aspecto \textbf{descriptivo}.

\textbf{Diagramas de clase.}
Diagramas de \textit{entidad-relación},
con algunos aspectos más -gráficamente hablando.
Tiene dos subherramientas.

\textbf{Diagramas de comunicación.}

\textbf{Diagramas de secuencia.}

Primero explicamos \textit{qué} hace el usuario,
con estas herramientas.
Desde el \textit{qué} vamos al \textit{cómo},
que sería la parte del código.

Vamos a desarrollar una documentación más robusta,
iterando sobre la primera.

\textbf{Diagramas de actividad.}
No lo vamos a ver en profundidad.

Algo vamos a ver de patrones de diseño,
metodologías ágiles.

\subsection{Trabajo final}

Grupos de no más de 2 alumnos.

\subsection{Trabajos de investigación}

Vamos a tener que investigar un poco acerca de ingeniería de software.
Vamos a presentarlo en un coloquio. Esto es \textit{aparte} del proyecto.
Hay mucha bibliografía, 
sobre herramientas que están -o no-
en el desarrollo de la materia
(es decir, podemos trabajar otras cosas no incluídas en el programa).
La nota del trabajo de investigación pondera para la nota final.

Hay que pensarlo para la segunda parte de la materia.

\subsection{Trabajos prácticos}

Hay que presentarlos en tiempo y forma.

\subsection{Parciales}

Hay dos parciales y sus recuperatorios.

\subsection{Bibliografía}

Hay dos libros fundamentales.
Larman (2003). \textit{Una introducción al analisis y diseño orientado a objetos y al proceso unificado}. 
Y Booch, Rumbaugh y Jacobson (2006). \textit{El lenguaje Unificado de Modelado}.

Hay herramientas que implementan estas metodologías.

Lo más importante, en definitiva,
es el diseño de la aplicación.
Vamos a ver alguna forma de plasmar el prototipo de manera documental.
