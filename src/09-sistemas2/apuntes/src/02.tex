\section{Segunda clase}

31 de marzo, 2025

\subsection{Pautas para avance de iteración}

Vamos a hacer una nueva iteración sobre el diseño de la aplicación.

Con Proceso Unificado de Modelado.
Con Herramientas UML: casos de uso y diagramas de clases.

Hacemos un refinamiento de los casos de uso.
Lo mejoramos en base a los errores encontrados.

En esta materia desarrollamos el prototipo de forma documental.

Mejoramos en base a los errores encontrados,
volviendo a la primera etapa del objeto de negocio,
avanzamos \textit{refinando}:
\begin{itemize}
    \item Elaboramos caso de uso detallado. Para todas las interfaces.
    \item Nombre del caso de uso.
    \item Descripción resumida.
    \item Precondición.
    \item Escenario principal.
    \begin{itemize}
        \item Si hay un error de ingreso: flujo alternativo y retorno a escenario principal
    \end{itemize}
    \item Poscondición: saliendo con éxito del escenario principal, cómo seguimos.
\end{itemize}

Después viene algo más:
\begin{itemize}
    \item Casos de Uso Include y extend.
    \item Realizar el Diagrama de Clase.
    \item Mejorar el entorno textual de la descripción de la aplicación: incluir visión, misión de la organización, rol de los actores o stakeholders.
\end{itemize}

Esto es lo que habría que trabajar para el primer parcial (4 semanas, hay que meterle nomás).

Para el segundo parcial vamos a profundizar todavía más.

Tenemos que mirar de nuevo esto,
iterar sobre el trabajo en base a estos señalamientos,
e ir mandando a la profe.
En base a estas iteraciones sale el proyecto final.
