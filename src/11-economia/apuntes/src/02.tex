\section{Segunda clase}

27 de marzo, 2025.

\subsection{Repaso}

Definiciones. Economía positiva y normativa.
Recursos.
Cuado sinóptico de necesidades humanas.
Seguimos en tipos de bienes.

\subsection{Tipos de bienes}

\textbf{Definición de bien.}
Aquello que satisface, directa o indirectamente, deseos o necesidades humanas.

Según su carácter:
\begin{itemize}
    \item Libres: ilimitados (o muy abundates) y no son propiedad de nadie: aire.
    \item Económicos: son escasos en relación con los deseos y necesidades humanas.
    La economía se ocupa de su estudio.
\end{itemize}

Según su naturaleza:
\begin{itemize}
    \item De capital: 
    No atiende directamente las necesidades humanas:
    máquinas.
    \item De consumo:
    Se destinan a la satisfacción directa de necesidades:
    pizza.
    \begin{itemize}
        \item Duraderos
        \item No duraderos
    \end{itemize}
\end{itemize}

Según su función:
\begin{itemize}
    \item Intermedios:
    Deben sufrir algún tipo de transformación antes de convertirse en bienes de consumo o capital: harina.
    \item Finales:
    Ya han sufrido las transformaciones necesarias para su uso o consumo: pan.
\end{itemize}

\subsection{Modelo económico}

Al tratar de influir sobre la actividad económica,
los economistas se ocupan de relaciones causa-efecto.
Y para atacar este tema recurren a un razonamiento teórico
y análisis de datos de las variables económicas.
Por eso los economistas utilizan \textbf{modelos económicos}.

Un \textbf{modelo} es una simplifación y abstracción de la realidad que,
siguiendo supuestos, argumentos y conclusiones,
explica una determinada proposición.
Se supone que el comportamiento de los individuos va a ser racional.
Esa racionalidad garantiza un criterio estable,
lo cual permite tomar una decisión lógica ante cada situación.

Causa y efecto.

Variables exógenas: precios, salarios \(\to\) Estructura y relaciones del modelo: preferencias y gustos, hipótesis: maximización de la satisfacción \(\to\) Variábles endógenas: cantidad consumida de los bienes

\subsection{Instrumentos del análsis económico}

Conjunto de conceptos, técnicas y procedimientos que ayudan a encarar y resolver problemas económicos.

Técnicas de análisis:
\begin{itemize}
    \item Historia
    \item Estadísticas
    \item Teorías tue describen los fenómenos que se pretenden explicar.
\end{itemize}

Procedimientos empleados en economía:
\begin{itemize}
    \item Verbal
    \item Matemático
    \item Geométrico
    \item O una mezcla de todos
\end{itemize}

\subsection{Modelos y datos}

Los modelos económicos describen las relaciones existentes entre las variables.

Los datos miden las variables económicas,
y permiten así analizar las relaciones predecidas por los modelos económicos.
Así pues,
los datos económicos generalmente son cifras que ofrecen información.
Aclaración: los datos económicos deben ser confeccionados con información fiable,
para así llegar a afirmaciones o resoluciones de problemas en forma lógica.

\subsection{Medición de las variables económicas}

Ejemplo: en base a precios,
en base a cantidad,
en base a combinación de los dos.

\subsection{Número índices}

Muchas veces la economía decide comparar datos sin hacer incapié en las unidades precisas.
Para efectuar este tipo de comparaciones se utilizan \textit{números indices},
que expresan los datos en relación a un \textit{valor base} temporal.

Ejemplo: la tasa de inflación se basa en la comparación de un número índice de este mes con 
el número índice del mes anterior.

Para ahondar en el tema, lo que se hace es 
obtener el índice del mes pasado y compararlo con el índice del otro mes,
divido el índice del otro mes.

\begin{equation*}
    \frac{IPC_{n} - IPC_{n-1}}{IPC_{n-1}}
\end{equation*}

El índice es un número, no una tasa.
Operando con los índices obtenemos una tasa.

\subsection{Trabajo práctico}

De Wikipedia: verificamos qué hicieron Adam Smith, Karl Marx y John Maynard Keynes.
Una carilla de cada uno. No vale copypaste. Cultura general.

\subsection{Funciones del Estado}

Pregunta de examen.