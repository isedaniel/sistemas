\section{Quinta clase}

24 de abril, 2025.

\subsection{Elasticidad}

El concepto de elasticidad se utiliza en economía para analizar en términos cuantitativos
cómo se ajusta el mercado en las variaciones de los determinantes de la demanda y la oferta.
Indica la sensibilidad.

\begin{equation*}
    \text{Ingreso total} = \text{precio} \times \text{cantidad demandada}
\end{equation*}

Indica la facturación, no la ganancia.
Es bruto, no \textit{neto}.

El sentido del cambio del ingreso total cuando varía el precio,
depende de la \textit{sensibilidad} de la cantidad demandada,
y esto precisamente se expresa mediante el concepto de elasticidad de la demanda:

Elasticidad-precio de la demanda:
mide el grado en que la cantidad demandada (\(Q\)) responde a un cambio en el precio (\(P\)).

\begin{equation*}
    E_P = \frac{\text{- Variación porcentual de la cantidad demandada}}{\text{Variación porcentual del precio}}
\end{equation*}

Fórmula generalista:

\begin{equation*}
    E_P = \frac{\Delta Q}{\Delta P}
\end{equation*}

\begin{table}[H]
    \centering
    \begin{tabular}{cccc} % {|l|c|r|p{3cm}|} define cuatro columnas con diferentes alineaciones
        \hline
        Situación & Precio & Cantidad demandada & Ingreso total \\
        \hline
        Inicial   & 100    & 300                & 30.000        \\
        Caso 1    & 80     & 340                & 27.200        \\
        Caso 2    & 75     & 390                & 29.250        \\
        \hline
    \end{tabular}
\end{table}

Lo relevante es que una elasticidad alta indica un elevado grado de respuesta de la cantidad demandada a la variación en el precio,
y una elasticidad baja indica una escasa sensibilidad a la variación del precio.

Ejemplo de elasticidad de demanda de productos agropecuarios.

\begin{table}[h!]
    \centering
    \begin{tabular}{lr}
        \hline
        Producto      & Elasticidad \\
        \hline
        Trigo         & 0.03        \\
        Papa          & 0.16        \\
        Leche         & 0.23/0.35   \\
        Carne vacuna  & 0.30/0.48   \\
        Pollo         & 0.75        \\
        Carne porcina & 0.81/0.84   \\
        \hline
    \end{tabular}
\end{table}

En el ejemplo del trigo,
aunque varíe su precio,
la gente va a comprar igual \(Q_{PAN}\).

\begin{equation*}
    E_P = -\frac{\frac{\Delta Q}{Q_i}}{\frac{\Delta P}{P_i}}
\end{equation*}

Nota sobre sl gino negativo:
dado que la curva de demanda tiene una inclinación negativa,
las variaciones de \(P\) y \(Q\) son en sentido contrario,
con lo cual el cociente tendría un signo negativo.
Sin embargo, en la práctica, se considera que esta fórmula se multiplica por \(-1\)
y se trabaja en positivo.

\begin{figure}[h!]
    \centering
    \begin{tikzpicture}
        \begin{axis}[
                xlabel=Cantidad,
                ylabel=Precio,
                axis lines=left,
                xmin=0, xmax=10,
                ymin=0, ymax=10,
            ]
            \addplot[domain=0:8, samples=100] {8 - 3/10*x};
        \end{axis}
    \end{tikzpicture}
    \caption{Demanda elástica}
\end{figure}

Como es mayor a 1, es elástica, generalmente bienes que se sustituyen entre si.

\begin{figure}[h!]
    \centering
    \begin{tikzpicture}
        \begin{axis}[
                xlabel=Cantidad,
                ylabel=Precio,
                axis lines=left,
                xmin=0, xmax=10,
                ymin=0, ymax=10,
            ]
            \addplot[domain=0:8, samples=100] {8 - x};
        \end{axis}
    \end{tikzpicture}
    \caption{Demanda unitaria}
\end{figure}

La elasticidad es 1. Cambia 1:1.

\begin{figure}[h!]
    \centering
    \begin{tikzpicture}
        \begin{axis}[
                xlabel=Cantidad,
                ylabel=Precio,
                axis lines=left,
                xmin=0, xmax=10,
                ymin=0, ymax=10,
            ]
            \addplot[domain=0:8, samples=100] {8 - 2*x};
        \end{axis}
    \end{tikzpicture}
    \caption{Demanda inelástica}
\end{figure}

\begin{figure}[h!]
    \centering
    \begin{tikzpicture}
        \begin{axis}[
                xlabel=Cantidad,
                ylabel=Precio,
                axis lines=left,
                xmin=0, xmax=10,
                ymin=0, ymax=10,
            ]
            \addplot[domain=0:8, samples=100] {10/x};
        \end{axis}
    \end{tikzpicture}
    \caption{Elasticidad arco}
\end{figure}

Calcular elasticidad: pregunta de examen.

\subsection{Elasticidad cruzada de la demanda}

La cantidad demandada de un bien no solo muestra sensibilidad ante los cambios en el precio del propio bien,
sino también ante alteraciones de los precios de ciertos productos que están estrechamente relacionados con el bien.

\begin{equation*}
    \text{Elasticidad cruzada del bien i con respecto al bien j} = \frac{\text{Variacion porcentual de la cantidad demandada del bien i}}{\text{Variacion porcentual del precio del bien j}}
\end{equation*}

La elasticidad-precio cruzada de la demanda mide la influencia de una variación del precio de un bien en la cantidad demandada de otro.

\subsection{Elasticidad de la demanda e ingreso total}

\begin{equation*}
    \text{Ingreso total} = \text{precio} \times \text{cantidad demandada}
\end{equation*}

Lo que se intenta demostrar es que si \(E = 1\),
se consigue el ingreso total máximo.

\begin{table}[h!]
    \centering
    \begin{tabular}{ccccc}
        \hline
        Precio & Cantidad & Elasticidad & Relaciones \(E_P \text{ vs } I_T\) & Ingreso Total (\(I_T\)) \\
        \hline
        45     & 100      & 9.0         &                                    & 4.500                   \\
        40     & 200      & 4.0         & \(E_P > 1\)                        & 8.000                   \\
        30     & 400      & 1.5         &                                    & 12.000                  \\
        25     & 500      & 1.0         & \(E_P = 1\)                        & 12.500                  \\
        20     & 600      & 0.67        &                                    & 12.000                  \\
        10     & 800      & 0.25        &                                    & 8.000                   \\
        5      & 900      & 0.09        & \(E_P < 1\)                        & 4.500                   \\
        2      & 950      & 0.05        &                                    & 1.900                   \\
        0      & 1000     & 0           &                                    & 0                       \\
        \hline
    \end{tabular}
\end{table}

En la tabla se observa que en \(E=1\) se maximiza el ingreso.

\subsection{Elasticidad-precio de la oferta}

Indica como responde los mercados a los cambios de la renta.
Es la variación experimentada por la cantidad ofrecida de un bien cuando varía su precio en \(1\%\),
manteniéndose constante el resto de factores que afectan a la cantidad ofrecida.

\begin{equation*}
    \text{Elasticidad de la oferta} = \frac{\text{Cambio porcentual en la cantidad}}{\text{Cambio porcentual en el precio}}
\end{equation*}

En cuanto a la respuesta del mercado,
cuanto más elástica sea la oferta,
más fácil resultará a los vendedores incrementar la producción ante el aumento del precio.

La elasticidad de la oferta depende de la \textit{capacidad de reacción} de los productores
ante alteraciones en el precio,
que también estará ligada (o condicionada)
por las características de los procesos productivos,
y por la necesidad de factores específicos,
como por ejemplo, emplear más personal.
Es decir,
todo esto condiciona de manera determinante a la elasticidad de la oferta.
En resumen,
a corto plazo,
la elasticidad de la oferta tiende a \(E<1\),
porque no hay tiempo de reacción para producir.
Y luego tiende a ser \(E>1\).

\subsection{Calculamos elasticidad demanda}

(Pregunta de examen).

Si un aumento en el precio de \$10 a \$20 reduce la cantidad demandada (\(Q\)) de 600 a 400,
calcular \(E_P\) de la demanda.

R: \(\frac{3}{5}\)

\begin{align*}
    E = \frac{\frac{-200}{600}}{\frac{20 - 10}{10}} \\
    E = \frac{1}{3} \cdot \frac{10}{10}
\end{align*}

\subsection{Utilidad}

Es la percepción subjetiva de placer o satisfacción que una persona experimenta como consecuencia
de consumir un bien o un servicio.

En cuanto a la evolución de la utilidad,
a medida que aumenta la cantidad consumida de un bien,
el incremento de utilidad total que proporciona la última unidad es cada vez menor,
con lo cual la \textbf{utilidad marginal} es decreciente.

\begin{empheq}[box=\fcolorbox{red}{white}]{equation*}
    \text{Marginal} = \text{Adicional}
\end{empheq}

\subsection{Ley de utilidad marginal decreciente}

Establece que,
a medida que aumenta la cantidad consumida de un bien,
la utilidad marginal de ese bien tiende a disminuir.

\begin{table}[h!]
    \centering
    \begin{tabular}{ccc}
        \hline
        Cantidad Consumida & Utilidad Total & Utilidad Marginal \\
        \hline
        0                  & 0              & -                 \\
        1                  & 100            & 100               \\
        2                  & 180            & 80                \\
        3                  & 240            & 60                \\
        4                  & 280            & 40                \\
        5                  & 300            & 20                \\
        \hline
    \end{tabular}
\end{table}