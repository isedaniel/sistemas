\section{Cuarta clase}

10 de abril, 2025.

\subsection{Curva y la función demanda}

La demanda se conceptualiza desde el punto de vista del comprador.

Cuanto mas barato es un producto, \textit{ceteris paribus}, más se compra.
Es decir,
manteniendo el resto de factores constante (renta, preferencias, estacionalidad, etc.),
la cantidad demandanda es inversamente proporcional al precio.

La curva \textit{decreciente} de demanda relaciona cantidad demandada y precio.
Al reducir el precio,
aumenta la cantidad de demandada.

A cada precio \(P_A\) corresponde una cantidad \(Q_A\),
que los demandantes están dispuestos a adquirir.

La curva de la demanda se va a confeccionar de acuerdo a la tabla antedicha,
y va a mostrar las cantidades de algún determinado bien que serán demandadas durante un período,
por una población específica, a cada uno de los posibles precios.

La función demanda se va a expresar de acuerdo al precio del bien demandado \(A\),
a la renta, a los gustos de los consumidores y a los precios relativos de otros bienes:

\begin{equation*}
    Q_A = D(P_A, Y, P_B, G)
\end{equation*}

Donde \(Q_A\) es la cantidad demandada del bien A en un período concreto,
\(P_A\) el precio del bien A,
\(Y\) el ingreso de los consumidores,
\(P_B\) los precios del resto de bienes en el mismo período
y \(G\) los gustos de los consumidores.

\begin{center}
    \begin{tikzpicture}
        \begin{axis}[
                xlabel=Cantidad,
                ylabel=Precio,
                axis lines=left,
                xmin=0, xmax=10,
                ymin=0, ymax=10,
            ]
            \addplot[domain=0:8, samples=100] {8 - x};
        \end{axis}
    \end{tikzpicture}
\end{center}

La \textbf{función de demanda} es la relación entre la cantidad demandada de un bien y su precio,
manteniendose constantes el resto de factores: renta, gustos, precios del resto de bienes, etc.
Es decir,
ceteris paribus.

\subsection{Alteración de los otros factores}

¿Qué pasa si cambia uno de los factores permaneciendo constate el precio?
Ya no estamos en una situación ceteris paribus.

Una alteración de cualquier factor diferentes del precio del bien
desplazará toda la curva hacia la derecha o la izquierda,
y es lo que llama \textit{cambios en la demanda}.

\begin{equation*}
    Y = -\frac{4}{3}x + 8
\end{equation*}

Graficamos:

\begin{center}
    \begin{tikzpicture}
        \begin{axis}[
                xlabel=Cantidad,
                ylabel=Precio,
                axis lines=left,
                xmin=0, xmax=10,
                ymin=0, ymax=10,
            ]
            \addplot[domain=0:8, samples=100] {8 - 4/3*x};
        \end{axis}
    \end{tikzpicture}
\end{center}

\subsection{Oferta}

Se conceptualiza desde el punto de vista del vendedor.

Al igual que en el caso de la demanda,
tabién existen un conjunto de factores que determinan la oferta del vendedor.
Estos factores se resumen en: la tecnología, los precios de factores productivos (tierra, capital, trabajo),
y el precio del bien que se desea ofrecer.

\subsection{Tabla de la oferta}

Tambien bajo la condición de ceteris paribus,
la tabla de la oferta relaciona el precio de un bien y las cantidades que un empresario está dispuesto a ofrecer a ese precio,
para un período de tiempo determinado.

Mientras la tabla de la demanda muestra el comportamiento de los consumidores,
la tabla de la oferta señala el comportamiento de los productores.

\subsection{Ley de los rendimientos decrecientes}

El crecimiento de la curva de la oferta puede establecerse diciendo que,
sin por ejemplo se desea una mayor producción de algún bien,
habrá que ir añadiendo mayores cantidades de mano de obra,
apelando a la \textbf{ley de los rendimientos decrecientes},
resultará que el costo necesario para elevar la producción \textit{en una unidad} será cada vez mayor.

Tabla:
\begin{table}[h!] % El [h!] es una sugerencia de ubicación "aquí mismo"
    \centering
    \begin{tabular}{c c} % {|l|r|} define dos columnas, una alineada a la izquierda (l) y otra a la derecha (r), con líneas verticales
        \hline
        \( P_A\) & \(Q_A\) \\ % Encabezados de las columnas en negrita
        \hline
        2        & 0       \\ % Filas de datos, separadas por &
        4        & 2       \\
        6        & 4       \\
        8        & 6       \\
        \hline
    \end{tabular}
\end{table}

\subsection{Curva y función oferta}

Es la representación gráfica de la tabla de oferta.
Muestra las cantidades del bien que se ofreceran a la venta,
en un periodo de tiempo específico,
para diversos precios de mercado.
En este caso la curva tiene pendiente positiva,
debido a que,
al aumentar el precio de un bien,
manteniendose constantes el resto de factores,
la cantidad ofrecida de dicho bien crecerá.

También la cantidad ofrecida dependerá de otras variables,
siendo la función oferta la siguiente:

\begin{equation*}
    Q_A = O(P_A, P_B, r, K)
\end{equation*}

Donde \(Q_A\) es la cantidad ofrecida del bien A para un período concreto,
\(P_A\) el precio del bien \(A\),
\(P_B\) el precio del resto de bienes,
\(r\) el precio de los factores de producción,
y \(K\) el estado de la tecnología.

En este análisis se considera también la condición \textit{ceteris paribus},
es decir, que todas las variables permanecen constantes,
excepto por la cantidad ofrecida \(Q_A\) y el precio \(P_A\).

La función de oferta es la relación entre la cantidad ofrecida de un bien y su precio,
considerando todos los demás factores constantes. Gráficamente:

\begin{center}
    \begin{tikzpicture}
        \begin{axis}[
                xlabel=Cantidad,
                ylabel=Precio,
                axis lines=left,
                xmin=0, xmax=10,
                ymin=0, ymax=10,
            ]
            \addplot[domain=0:8, samples=100] {2 + x};
        \end{axis}
    \end{tikzpicture}
\end{center}

\subsection{Equilibrio de mercado}

Cuando se ponen en contacto el consumidor y el productor 
con sus respectivos planes de consumo y producción
(sus curvas de demanda y oferta),
solamente habrá un punto de coincidencia entre los dos,
y será en el punto de corte entre ambas curvas.
A este precio de coincidencia se lo denomina \textbf{precio de equilibrio},
y la cantidad ofrecida y demandada a ese precio se le llama \textbf{cantidad de equilibrio}.

Así pues,
el precio de equilibrio es aquel para el cual la cantidad demandada es igual a la cantidad ofrecida.
Y esa cantidad se denomina cantidad de quilibrio.

El punto de corte de ambas curvas se denomina \textbf{punto de equilibrio}.

Por encima del punto de equilibrio se denomina \textbf{zona de excedente}.
Por debajo, \textbf{zona de escasez}.

En resumen, en la situación de equilibro se igualan cantidades ofrecidas y demandadas,
y es cuando se sienten cómodos ambos jugadores.

\subsection{Desplazamientos de las curvas de demanda y oferta}

Esto sucederá cuando la condición ceteris paribus se interrumpe.

\subsubsection{Desplazamientos de la curva de demanda}

En realidad esto es frecuente, lo que motivará los desplazamientos de la curva de demanda.
En nuestro análisis vamos a ver lo que sucede con los factores de la renta de los consumidores,
los precios de los demás bienes y los gustos.

\textbf{a.} La renta de los consumidores.

\textbf{Caso a:} Si la renta aumenta,
los individuos pueden consumir más cualquiera que sea el precio,
por lo que la curva de la demanda se desplazará a la derecha.
Tengamos en cuenta que aún el precio del bien \(A\)
permanece constante.

\begin{center}
    \begin{tikzpicture}
        \begin{axis}[
                xlabel=Cantidad,
                ylabel=Precio,
                axis lines=left,
                xmin=0, xmax=10,
                ymin=0, ymax=10,
            ]
            \addplot[domain=0:8, samples=100] {8 - 4/3*x};
            \addplot[domain=0:10, samples=100] {10 - 4/3*x};
        \end{axis}
    \end{tikzpicture}
\end{center}

\textbf{Caso b:}
Sucede a veces que cuando el individuo gana más sus patrones de consumo del producto \(A\)
varíen y adquiera una menor cantidad el bien \(A\). Ejemplo: papas.
Y se incremente su consumo de otros bienes, como por ejemplo carne,
aumentando la cantidad demandada del nuevo producto.
Con lo cual la curva de las papas se desplaza hacia la izquierda.

\textbf{b.} Precios de los bienes relacionados

El analisis de este caso va a ser similar al caso b de la renta de los consumidores
(lo que sucedio con las papás y la carne).

Las alteraciones en el precio de un bien pueden ocasionar también desplazamientos en la curva de demanda
de otro bien. Por ejemplo, si aumenta el combustible, la gente tendrá un menor interés
en comprar bienes relacionados como vehículos,
y por lo tanto la curva de la demanda se desplazará hacia la izquierda.

Del mismo modo, el consumo del GNC aumentará,
desplazándose la curva del GNC hacia la derecha.

\textbf{c.} Gustos y preferencias de los consumidores

Los gustos pueden modificarse a lo largo del tiempo,
o bien por campañas publicitarias,
shocks (como pandemias),
etc.

Si los gustos se alteran en el sentido de desear demandar una mayor cantidad de un determinado producto,
la curva de demanda se desplazará hacia la derecha,
mientras que si la modificación de las preferencias es en sentido contrario,
la curva de demanda se desplazará hacia la izquierda.

\subsection{Cambios en la demanda y punto de equilibrio}

Ante el movimiento de una de las curvas,
se modificará consecuentemente la posición del punto de equilibrio,
con lo cual se arriba a una nueva situación de equilibrio.

Si el desplazamiento de la curva demanda \(D_0\) es hacia la derecha \(D_1\),
el precio de equilibrio y la cantidad demandada aumentarán,
siempre considerando que la oferta no se desplazó
(por ejemplo, que el Estado le de un incentivo al productor para que eso no suceda).

\subsection{Desplazamientos de la curva de oferta}

En este caso los factores que estaba constantes eran el precio de los factores productivos,
el precio de bienes relacionados y la tecnología.

\textbf{a.} Precio de los factores productivos

En términos gráficos, el deseo de producir más para cualquier nivel de precios 
implica un desplazamiento de la curva de oferta hacia la derecha.
Si se reducen los costos de producción,
por ejemplo baja la luz,
los productores producirán más al precio anterior.

\textbf{b.} Precio de los bienes relacionados 

Si el precio del maíz disminuye,
es probable que los agricultores reduzcan su producción y decidan cultivar trigo.
En este caso, la curva del trigo se desplaza a la derecha y la del maíz hacia la izquierda.

\textbf{c.} Tecnología

Una mejora de la tecnología contribuye a reducir costos de producción,
incrementando rendimientos,
lo que hará que los empresarios ofrezcan más productos a cualquier precio,
desplazando la curva de la oferta hacia la derecha.

\subsection{Cambios en la oferta y punto de equilibrio}

En cuanto al equilibrio, en este caso se considera la demanda fija,
moviéndose las curvas de oferta.

\subsection{Desplazamientos de ambas curvas}

Tal cual analizamos hasta recién,
cuando una de las curvas permanece fija
(sea la de demanda o de oferta),
se puede observar los efectos sobre precios y cantidades de equilibrio,
que serán \textit{predecibles}.

Sin embargo,
cuando se desplazan ambas,
los efectos no son \textit{perfectamente} predecibles.

Quedamos en demanda y elasticidad.