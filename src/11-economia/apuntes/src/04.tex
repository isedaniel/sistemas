\section{Clase 10 de abril}

\subsection{Curva y función demanda}

Cuanto mas barato es un producto, 
\textit{ceteris paribus},
más se compra.
Es decir,
manteniendo el resto de factores constante 
(renta, preferencias, estacionalidad, etc.),
la cantidad demandanda es inversamente proporcional al precio.

La curva \textit{decreciente} de demanda relaciona cantidad demandada y precio.
Al reducir el precio,
aumenta la cantidad de demandada.

A cada precio \(P_A\) corresponde una cantidad \(Q_A\),
que los demandantes están dispuestos a adquirir.

La curva de la demanda se va a confeccionar de acuerdo a la tabla antedicha,
y va a mostrar las cantidades de determinado bien,
que serán demandadas durante un período de tiempo,
por una población específica,
a cada uno de los posibles precios.

La función demanda se va a expresar de acuerdo al precio del bien demandado \(A\),
a la renta, 
a los gustos de los consumidores 
y a los precios relativos de otros bienes:

\begin{equation*}
    Q_A = D(P_A, Y, P_B, G)
\end{equation*}

Donde \(Q_A\) es la cantidad demandada del bien A en un período concreto,
\(P_A\) el precio del bien A,
\(Y\) el ingreso de los consumidores,
\(P_B\) los precios del resto de bienes en el mismo período
y \(G\) los gustos de los consumidores.

\begin{center}
    \begin{tikzpicture}
        \begin{axis}[
                xlabel=Cantidad,
                ylabel=Precio,
                axis lines=left,
                xmin=0, xmax=10,
                ymin=0, ymax=10,
            ]
            \addplot[domain=0:8, samples=100] {8 - x};
        \end{axis}
    \end{tikzpicture}
\end{center}

La \textbf{función de demanda} es la relación entre la cantidad demandada de un bien y su precio,
manteniendose constantes el resto de factores: renta, gustos, precios del resto de bienes, etc.
Es decir,
ceteris paribus.

\subsection{Alteración de los otros factores}

¿Qué pasa si cambia uno de los factores permaneciendo constate el precio?
Ya no estamos en una situación ceteris paribus.

Una alteración de cualquier factor diferentes del precio del bien
desplazará toda la curva hacia la derecha o la izquierda,
y es lo que llama \textit{cambios en la demanda}.

\begin{equation*}
    Y = -\frac{4}{3}x + 8
\end{equation*}

Graficamos:

\begin{center}
    \begin{tikzpicture}
        \begin{axis}[
                xlabel=Cantidad,
                ylabel=Precio,
                axis lines=left,
                xmin=0, xmax=10,
                ymin=0, ymax=10,
            ]
            \addplot[domain=0:8, samples=100] {8 - 4/3*x};
        \end{axis}
    \end{tikzpicture}
\end{center}

\subsection{Oferta}

Se conceptualiza desde el punto de vista del vendedor.

Al igual que en el caso de la demanda,
tabién existen un conjunto de factores que determinan la oferta del vendedor.
Estos factores se resumen en: 
tecnología, 
precios de factores productivos 
(tierra, capital, trabajo)
y el precio del bien ofrecido.

\subsection{Tabla de la oferta}

Tambien bajo la condición \textit{ceteris paribus},
la tabla de oferta relaciona el precio de un bien y las cantidades que una empresa está dispuesta a ofrecer a ese precio,
para un período de tiempo determinado.

Mientras la tabla de la demanda muestra el comportamiento de los consumidores,
la tabla de la oferta señala el comportamiento de los productores.

\subsection{Ley de los rendimientos decrecientes}

El crecimiento de la curva de la oferta puede establecerse diciendo que,
si se desea una mayor producción de un bien,
habrá que ir añadiendo mayores cantidades de mano de obra y,
apelando a la \textbf{ley de los rendimientos decrecientes},
resultará que el costo necesario para elevar la producción \textit{en una unidad} será cada vez mayor.

Tabla:
\begin{table}[H]
    \centering
    \begin{tabular}{c c} 
        \hline
        \( P_A\) & \(Q_A\) \\
        \hline
        2        & 0       \\
        4        & 2       \\
        6        & 4       \\
        8        & 6       \\
        \hline
    \end{tabular}
\end{table}

\subsection{Curva y función oferta}

Es la representación gráfica de la tabla de oferta.
Muestra las cantidades del bien que se ofreceran a la venta,
en un periodo de tiempo específico,
para diversos precios de mercado.
En este caso la curva tiene pendiente positiva,
debido a que,
al aumentar el precio de un bien,
manteniendose constantes el resto de factores,
la cantidad ofrecida de dicho bien crecerá.

También la cantidad ofrecida dependerá de otras variables,
siendo la función oferta la siguiente:

\begin{equation*}
    Q_A = O(P_A, P_B, r, K)
\end{equation*}

Donde \(Q_A\) es la cantidad ofrecida del bien A para un período concreto,
\(P_A\) el precio del bien \(A\),
\(P_B\) el precio del resto de bienes,
\(r\) el precio de los factores de producción,
y \(K\) el estado de la tecnología.

En este análisis se considera también la condición \textit{ceteris paribus},
es decir, 
que todas las variables permanecen constantes,
a excepción de la cantidad ofrecida \(Q_A\) y el precio \(P_A\).

La función de oferta es la relación entre la cantidad ofrecida de un bien y su precio,
considerando todos los demás factores constantes. 
Gráficamente:

\begin{center}
    \begin{tikzpicture}
        \begin{axis}[
                xlabel=Cantidad,
                ylabel=Precio,
                axis lines=left,
                xmin=0, xmax=10,
                ymin=0, ymax=10,
            ]
            \addplot[domain=0:8, samples=100] {2 + x};
        \end{axis}
    \end{tikzpicture}
\end{center}

\subsection{Equilibrio de mercado}

Cuando interactúan consumidor y productor,
con sus respectivos planes de consumo y producción
(sus curvas de demanda y oferta),
habrá un punto de coincidencia entre los dos,
y será en el punto de corte entre ambas curvas.
A este precio de coincidencia se lo denomina \textbf{precio de equilibrio},
y la cantidad ofrecida y demandada a ese precio se le llama \textbf{cantidad de equilibrio}.

Así,
el precio de equilibrio es aquel para el cual la cantidad demandada es igual a la cantidad ofrecida.
Esa cantidad se denomina cantidad de quilibrio.

La región por encima del punto de equilibrio se denomina \textbf{zona de excedente}.
Por debajo, \textbf{zona de escasez}.

En resumen, 
en la situación de equilibro se igualan cantidades ofrecidas y demandadas,
y es cuando se sienten cómodos ambos jugadores.

\subsection{Desplazamientos de las curvas de demanda y oferta}

Esto sucederá con la interrupción de la condición \textit{ceteris paribus}.

\subsection{Desplazamientos de la curva de demanda}

En realidad esto es frecuente,
lo que motivará los desplazamientos de la curva de demanda.
Vamos a ver lo que sucede con un cambio en la renta de los consumidores,
los precios de los demás bienes y los gustos.

\subsubsection{a. Cambios en la renta de los consumidores}

\textbf{Caso A: renta aumenta, precio constante, patrón de consumo constante} 

Los individuos pueden consumir más,
cualquiera que sea el precio,
por lo que la curva de la demanda se desplazará a la derecha.
Tengamos en cuenta que aún el precio del bien \(A\)
permanece constante.

\begin{center}
    \begin{tikzpicture}
        \begin{axis}[
                xlabel=Cantidad,
                ylabel=Precio,
                axis lines=left,
                xmin=0, xmax=10,
                ymin=0, ymax=10,
            ]
            \addplot[domain=0:8, samples=100] {8 - 4/3*x};
            \addplot[domain=0:10, samples=100] {10 - 4/3*x};
        \end{axis}
    \end{tikzpicture}
\end{center}

\textbf{Caso B: renta aumenta, patrón de consumo cambia}

El individuo gana más y sus patrones de consumo del producto \(A\) varían,
adquiriendo menos del bien \(A\),
incrementando el consumo de otros bienes: 
ej. menos papas, más carne.
Con ello, la curva de demanda de papas se desplaza hacia la izquierda.

\subsubsection{b. Cambio en precios de los bienes relacionados}

Este caso es similar al caso B anterior
(lo sucedido con las papas y la carne).

Las alteraciones en el precio de un bien 
pueden ocasionar desplazamientos en la curva de demanda de otro bien. 
Por ejemplo, 
si aumenta el combustible, 
la gente tendrá un menor interés
en comprar bienes relacionados como vehículos,
y por lo tanto la curva de la demanda se desplazará hacia la izquierda.

Del mismo modo, 
el consumo del GNC aumentará,
desplazándose la curva del GNC hacia la derecha.

\subsubsection{c. Cambio en gustos y preferencias de los consumidores} 

Los gustos pueden modificarse a lo largo del tiempo,
por campañas publicitarias,
shocks (como la pandemia),
etc.

Si los gustos incrementan el deseo de un bien,
la curva de demanda se desplazará hacia la derecha,
mientras que si la modificación de las preferencias es en sentido contrario,
la curva de demanda se desplazará hacia la izquierda.

\subsection{Cambios en la demanda y punto de equilibrio}

Ante el movimiento de una de las curvas,
la posición del punto de equilibrio cambiará.

Si la curva demanda \(D_0\) se desplaza hacia la derecha \(D_1\),
tanto el precio de equilibrio como la cantidad demandada aumentarán,
siempre considerando que la oferta no se desplazó
(por ejemplo, 
que el Estado mediante un impuesto desincentive al productor).

\subsection{Desplazamientos de la curva de oferta}

En el caso de la oferta,
los factores que consideramos constantes eran:
el precio de los factores de producción,
el precio de bienes relacionados y la tecnología.

\subsubsection{a. Precio de los factores productivos}

Si se reducen los costos de producción,
por ejemplo,
por una baja en el precio de la energía,
los productores pueden producir más al mismo precio,
desplazando la curva de oferta hacia la derecha.

\subsection{b. Precio de los bienes relacionados}

Si el precio del maíz disminuye,
es probable que los agricultores reduzcan su producción,
optando por un cultivo alternativo, como el trigo.
En este caso, 
la curva del trigo se desplaza a la derecha y la del maíz hacia la izquierda.

\subsubsection{c. Tecnología}

Una mejora de la tecnología contribuye a reducir costos de producción,
incrementando rendimientos,
e incentivando a los productores a ofrecer más producto,
desplazando la curva de oferta hacia la derecha.

\subsection{Cambios en la oferta y punto de equilibrio}

En cuanto al equilibrio, 
en este caso se considera la demanda fija,
moviéndose las curvas de oferta.

\subsection{Desplazamientos de ambas curvas}

Cuando una de las curvas permanece fija
(sea la de demanda o de oferta),
se puede observar los efectos sobre precios y cantidades de equilibrio,
que serán \textit{predecibles}.

Sin embargo,
cuando se desplazan ambas,
los efectos no son \textit{perfectamente} predecibles.
