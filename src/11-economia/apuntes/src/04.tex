\section{Cuarta clase}

10 de abril, 2025.

\subsection{Curva y la función demanda}

La demanda se conceptualiza desde el punto de vista del comprador.

Cuanto mas barato es un producto, \textit{ceteris paribus}, más se compra.
Es decir,
manteniendo el resto de factores constante (estacionalidad, etc.),
la cantidad demandanda es inversamente proporcional al precio.

La curva \textit{decreciente} de demanda relaciona cantidad demandada y precio.
Al reducir el precio,
aumenta la cantidad de demandada. 
A cada precio \(P_A\) corresponde una cantidad \(Q_A\),
que los demandantes están dispuestos a adquirir.
La curva de la demanda se va a confeccionar de acuerdo a la tabla antedicha,
y va a mostrar las cantidades de algún determinado bien que serán demandadas durante un período,
por una población específica, a cada uno de los posibles precios.

La función demanda se va a expresar de acuerdo al precio del bien demandado \(A\),
a la renta, a los gustos de los consumidores y a los precios relativos de los bienes sustitutos.

\begin{equation*}
    Q_A = D(P_A, Y, P_B, G)
\end{equation*}

Donde \(Q_A\) es la cantidad demandada del bien A en un período concreto,
\(P_A\) el precio del bien A,
\(Y\) el ingreso de los consumidores,
\(P_B\) los precios del resto de bienes en el mismo período
y \(G\) los gustos de los consumidores.

\begin{center}
    \begin{tikzpicture}
        \begin{axis}[
            xlabel=Cantidad,
            ylabel=Precio,
            axis lines=left,
            xmin=0, xmax=10,
            ymin=0, ymax=10,
            ]
            \addplot[domain=0:8, samples=100] {8 - x};
        \end{axis}
    \end{tikzpicture}
\end{center}

La \textbf{función de demanda} es la relación entre la cantidad demandada de un bien y su precio,
manteniendose constantes el resto de factores: renta, gustos.

\subsection{Alteración de los otros factores}

¿Qué pasa si cambian esos factores?
Ya no estamos en una situación ceteris paribus.

¿Qué pasa si cambia uno de los factores permaneciendo constate el precio?

Una alteración de cualquier factor difrente del precio del bien desplazará toda la curva hacia la derecha o la izquierda,
y es lo que llama cambios en la demanda.

\begin{equation*}
    Y = -\frac{4}{3}x + 8
\end{equation*}

Hacemos gráfico de esta función:

\begin{center}
    \begin{tikzpicture}
        \begin{axis}[
            xlabel=Cantidad,
            ylabel=Precio,
            axis lines=left,
            xmin=0, xmax=10,
            ymin=0, ymax=10,
            ]
            \addplot[domain=0:8, samples=100] {8 - 4/3*x};
        \end{axis}
    \end{tikzpicture}
\end{center}

\subsection{Oferta}

Desde el punto de vista del vendedor.
De la misma manera que en el caso de la demanda,
tabién existe un conjunto de factores para determinar la oferta del vendedor.
Estos factores son: la tecnolkogía, los precios de factores rpoductivos (tierra, capital, trabajo)
y el precio del bien que se desea ofrecer.

\subsection{Tabla de la oferta}

Tambien bajo la condición e ceteris paribus, la tabla de la oferta va a relacionar
el precio de un bien y las cantidades

Para elevar la producción en una unidad será cada vez mayor.

Tabla:
\begin{equation*}
    P_A: 2 4 6 8
    O_A: 0 2 4 6
\end{equation*}

\subsection{Curva y función oferta}

Muestra las cantidades del bien en un periodo especifico,
para diversos precios de mercado.
En este caso la curva tiene pendiente positiva,
debido a que,
al aumentar el precio de un bien,
manteniendose constantes el resto de factores,
la cantidad ofrecida de dicho bien crecerá.

También la cantidad ofrecida dependerá de otras variables,
siendo la función oferta la siguiente:

\begin{equation*}
    Q_A = O(P_A, P_B, r, K)
\end{equation*}

Donde \(Q_A\) es la cantidad ofrecida del bien A para un período determinado,
\(P_A\) el precio del bien \(A\),
\(P_B\) el precio del resto de bienes,
\(r\) el precio de los factores de producción,
y \(K\) el estado de la tecnología.

En este análisis se considera también la condición \textit{ceteris paribus},
es decir, que todas las variables permanecen constantes,
excepto por la cantidad ofrecida \(Q_A\) y el precio \(P_A\).

La función de oferta es la relación entre la cantidad ofrecida de un bien y su precio,
considerando todos los demás factores constantes.

\begin{center}
    \begin{tikzpicture}
        \begin{axis}[
            xlabel=Cantidad,
            ylabel=Precio,
            axis lines=left,
            xmin=0, xmax=10,
            ymin=0, ymax=10,
            ]
            \addplot[domain=0:8, samples=100] {2 + x};
        \end{axis}
    \end{tikzpicture}
\end{center}

\subsection{Equilibrio de mercado}

Cuando se ponen en contacto el consumidor y el productor con sus respectivos planes de consumo y producción