\section{Clase 26 de junio}

Examen: jueves 10 a las 19.
Modalidad: misma que la anterior.

\subsection{Teoría de la producción}

Entendemos que las actividades productivas (fábricas, explotación agrícola,
etc.) lo hacen eficientemente, es decir,
con el menor costo posible.
O sea,
obtener el máximo nivel de producción con una dada cantidad de factores.
Cuando hablamos de factores, por ejemplo,
podríamos mencionar horas hombre (HH), cantidad de empleados,
líneas de montaje, etc.

\subsection{Función de producción}

Especifica la cantidad máxima de producción que puede obtenerse con una cantidad
dada de factores,
dependiendo de los conocimientos técnicos,
por ejemplo,
en Occidente,
para cavar un pozo de \(2\times15\) metros se utiliza maquinaria adecuada,
personal capacitado y quizá un supervisor.
En cambio,
en Corea del Norte,
para realizar ese mismo trabajo se utilizan 50 personas con pico y pala.

Por lo tanto,
desde el punto de vista económico,
se utiliza la \textit{función producción} para describir la capacidad productiva
de una empresa.

\subsection{Producto total, marginal y medio}

\textbf{Producto total}, es la cantidad total de producción en unidades físicas.
Ejemplo, en 3 horas hice 3000 tapitas. En 4 horas hice 4000 tapitas.

\textbf{Producto marginal},
es la producción \textit{adicional} que se obtiene por unidad de factor,
por ejemplo,
por una unidad de hora hombre se obtienen 1553 tapitas.

\textbf{Producto medio} es la producción total dividida por el total de unidades
del factor,
es decir,
su promedio.

\vspace{.5cm}
\begin{table}[H]
    \centering
    \begin{tabular}{cccc}
        \hline
        Factor & Producto total & Producto marginal & Producto medio \\
        0      & 0              & -                 & 0              \\
        1      & 2000           & 2000              & 2000           \\
        2      & 3000           & 1000              & 1500           \\
        3      & 3500           & 500               & 1167           \\
        4      & 3800           & 300               & 950            \\
        5      & 3900           & 100               & 780            \\
        \hline
    \end{tabular}
\end{table}
\vspace{.5cm}

\subsection{Ley de los rendimientos decrecientes}

Establece que cuando se añade una cantidad \textit{adicional} de un factor,
manteniendo los demás fijos,
se obtienen una cantidad adicional del producto,
pero cada vez más pequeña.
Es decir, 
el producto marginal de ese factor disminuye a medida que aumenta la cantidad de
ese factor.
Quiere decir que al comienzo lo que se debe hacer es todo.
O sea,
la disponibilidad de maquinarias, instalaciones, etc., son muy amplias 
y a medida que la capcidad de explotación se va intensificando,
el ritmo de expansión se hace más lento,
hasta llegar a un punto de saturación,
lo que implica que debería incorporarse más unidades pero de otros factores.
Ejemplo: comprar una nueva línea de montaje.


\vspace{.5cm}
\begin{figure}[H]
    \centering
    \caption{Curva de los rendimientos decrecientes}
    \vspace{.25cm}
    \begin{tikzpicture}
        \begin{axis}[
                axis lines=middle,
                xmin=0, xmax=10,
                ymin=0, ymax=4,
                xtick={2,4,6,8},
                ytick={0,1,2,3},
                domain=1:10,
                samples=100,
                grid=both,
                grid style={dashed, gray!50},
            ]
            % función
            \addplot[
                color=blue,
                thick,
                smooth
            ] {ln(x)+1};
        \end{axis}
    \end{tikzpicture}
\end{figure}
\vspace{.5cm}

\subsection{Costos: fijos, variables y totales}

\textbf{Costos fijos.}
Son los que permanecen constantes dada cierta capacidad física de la empresa,
y no tienen relación con el aumento o disminución de la producción.
Ejemplo:
alquiler de la fábrica,
seguro,
ciertos sueldos como el de la seguridad.
Aún si no existiese producción alguna,
estos gastos existirían igualmente.

Los costos fijos (\(CF\)) permanecen constantes en el corto plazo,
hasta que aumenta la capacidad física de la empresa.
Por ejemplo, 
si se construyen más galpones y se adquieren más maquinarias,
estos costos aumentarán.

\textbf{Costos variables.}
Tienen relación directa con el volumen de producción.
Ejemplo: produzco más, gasto más energía, sueldo, insumos.

\textbf{Costo total} es la suma de \(CF\) y \(CV\).

\begin{equation*}
    CT = CF + CV
\end{equation*}

Ejemplo, si alquilar una fábrica vale \$10,
energía \$5, sueldos \$5, seguro \$3, materia prima \$2,
calcular \(CF\), \(CV\) y \(CT\).

\begin{align*}
    CF = 10 + 3 = 13 \\
    CV = 5 + 5 + 2 = 12 \\
    CT = \boxed{25}
\end{align*}

Con \(V\) ventas, y \(B\) beneficio:

\begin{equation*}
    V - CV = CF + B \implies B = V - CT
\end{equation*}

\subsection{Inflación}

La inflación se puede identificar como el crecimiento continuo y generalizado de 
los precios de bienes y servicios.
Este crecimiento se mide bajo la evolución de algún índice\footnote{Índice \(\neq\) porcentaje}.
Tal cual lo indicado, 
la inflación se define como el aumento del nivel general de precios
y este se expresa mediante índices de precios, 
los cuales,
si se comparan sobre la base del mes anterior o del año anterior,
se determina el porcentaje de inflación del período considerado.

El índice de precios al consumidor (IPC) 
representa el costo de una canasta de bienes y servicios.
La inflación medida por el IPC es la tasa de variación porcentual 
que experimenta este índice en el período de tiempo considerado.

\begin{align*}
    I_{2024} = \frac{IPC_{2024}-IPC{2023}}{IPC_{2024}}
\end{align*}

De esta manera pasamos de un índice a porcentaje.

El IPC no es un índice de costo de vida.
Para determinar el IPC se hace un relevamiento de precios 
de ciertos productos de la canasta.
Hasta hoy son 9, a saber:
\begin{itemize}
    \item Alimentos y bebidas
    \item Indumentaria
    \item Vivienda y servicios básicos
    \item Equipamiento y mantenimiento del hogar
    \item Atención médica y gastos para la salud 
    \item Transporte y comunicaciones
    \item Esparcimiento
    \item Educación
    \item Otros bienes y servicios 
\end{itemize}

A su vez,
esta canasta se subdivide.