\section{Clase 8 de mayo}

\subsection{Examen}

Horario: 19.30.
Estudiar todo.

\subsection{Competencia perfecta e imperfecta}

Mercado: 
es el marco donde se desarrolla la oferta y la demanda,
presencial o no presencial.

Un mercado perfectamente competitivo es aquel en el cual 
todas las empresas son demasiado pequeñas
como para influir en el precio de mercado.
Eso significa que,
independientemente del precio vigente en el mercado,
los vendedores pueden vender toda su producción.
Un competidor, en ese marco,
no tiene razón para vender a un precio inferior,
porque perdería dinero,
ni tampoco elevaría su precio por encima del precio de mercado,
puesto que no vendería nada,
otro competidor ocuparía su lugar.

\subsection{Competencia imperfecta}

Si una empresa puede influir significativamente en el precio de mercado 
de su producto,
se considera como competidor imperfecto.
En una industria, 
siempre que los vendedores puedan controlar en alguna medida 
el precio de sus productos,
existe competencia imperfecta.
No implica que una empresa controle absolutamente el precio de su producto.
El competidor imperfecto tiene al menos 
un margen de maniobra para fijar su precio.

\subsection{Tipos de competidores imperfectos}

\subsubsection{Monopolio}

Un vendedor.
Es un caso extremo,
donde hay un único vendedor que controla absolutamente 
todo el mercado de una industria.

Actualmente es difícil encontrar un verdadero caso de monopolio,
generalmente sucede si interviene el estado para proteger al vendedor.

Generalmente, se da en compañias de servicios.

\subsubsection{Oligopolio}

Pocos vendedores.
2 a 10 empresas.
En el oligopolio cada empresa puede influir en el precio de mercado.
En el sector aéreo, por ejemplo,
la decisión de una línea aérea de bajar sus tarifas puede desencadenar una guerra de precios,
que provoque una reducción de las tarifas de todos sus competidores.

\subsubsection{Oligopolio colusorio}

(Pregunta de examen)

Cuando cooperan entre si las empresas.
En un mercado pequeño de oligopolio,
las compañias pueden optar por cooperar o no entre si.
Si no cooperan, esa conducta provoca una guerra de precios.
Cuando las empresas cooperan, 
tratan de minimizar la competencia entre ellas,
practicando de esta manera la colusión.
Dicho de otra manera,
se describe una situación en que las empresas fijan conjuntamente sus precios,
niveles de producción
o participación en el mercado.

\subsubsection{Resumen}

Monopolio: único productor,
producto sin bueno sustitutos,
generalmente regulados por el estado (edenor, servicios públicos).

Oligopolio: pocos productores,
diferencias escasas o nulas entre los productos,
colusión.

\subsubsection{Competencia monopolística}

Muchos productores. Muchas diferencias reales entre los productos.
Ejemplo: comercios al por menor, pizzería.

\subsection{Dumping}

Vender por debajo de ganancia,
para destruir la industria local ganando mercado.

Existe dumping cuando se verifica que un producto se vende a otro país a un precio inferior
al que ese mismo producto se vende en el mercado interno del país productor/exportador 
(valor normal: \(V_N\)).

Sucede a veces que una vez que la empresa tenga penetración en otro país y funda a sus competidores,
dicha empresa controlará la industria y aumentará los precios,
percibiendo ganancias extraordinarias.

Es decir,
al principio va a pérdida,
penetra el mercado
y finalmente eleva sus precios.

Es una práctica de comercio desleal.
