\section{Clase 20 de marzo}

\subsection{Economía}

Economía es una ciencia que estudia cómo individuos y sociedades administran recursos escasos 
para satisfacer sus necesidades.
Como toda ciencia,
combina aspectos objetivos y subjetivos.

La economía se ocupa de los problemas vinculados a satisfacción de las necesidades,
apelando a la administración de recursos escasos,
con el objetivo de producción y distribución de bienes y servicios.

Economía es la ciencia que estudia la asignación más conveniente 
de los recursos escasos de una sociedad 
para la obtención de un conjunto ordenado de objetivos.

\subsection{Economía positiva y economía normativa}

\textbf{Pregunta de examen.}

\textbf{Economía positiva.}
Se define como
la ciencia que busca explicaciones \textit{objetivas} del funcionamiento de los fenómenos económicos.
Se ocupa de lo que es o podría ser.
Está precedida por la asepsia científica.
Se dedica a establecer proposiciones del tipo:
dadas tales circunstancias,
tendrán lugar tales acontecimientos.

La \textbf{economía normativa} ofrece prescripciones para la acción
basada en juicios de valor personal y subjetivo
(de lo que \textit{debería} ser).
Desde el punto de vista normativo, 
la economía genera prescripciones
sobre el sistema económico.

\subsection{Necesidades}

\textbf{Necesidad humana.}
Es la sensación de carencia de algo,
unida al deseo de satisfacerla.

\subsubsection{Tipos de necesidades}

\begin{itemize}
    \item Según su origen:
    \begin{enumerate}
        \item Necesidades del individuo: 
        \begin{enumerate}
            \item Naturales (comer)
            \item Sociales (fiestas).
        \end{enumerate}
        \item Necesidades de la sociedad 
        \begin{enumerate}
            \item Colectivas (transporte) 
            \item Públicas (seguridad)
        \end{enumerate}
    \end{enumerate}
    \item Según su naturaleza 
    \begin{enumerate}
        \item Necesidades primarias: de las que depende la conservación de la vida (alimentos)
        \item Necesidades secundarias: mejoran la calidad de vida (turismo)
    \end{enumerate}
\end{itemize}
