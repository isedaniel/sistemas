\section{Clase 20 de marzo}

\subsection{Economía}

La Economía es una ciencia que estudia 
cómo individuos y sociedades administran recursos escasos 
para satisfacer sus necesidades.
Como toda ciencia,
combina aspectos objetivos y subjetivos.

La economía se ocupa de problemas vinculados a satisfacción de necesidades
mediante la administración de recursos escasos,
con el objetivo de producir y distribuir bienes y servicios.

Economía es la ciencia que estudia la asignación más conveniente 
de los recursos escasos de una sociedad 
para la obtención de un conjunto ordenado de objetivos.

\subsection{Economía positiva}

\textbf{(Pregunta de examen)}

La economía positiva se ocupa de buscar explicaciones \textit{objetivas} 
para los fenómenos económicos.
Se hace preguntas por \textit{lo que es} o \textit{lo que podría ser}.
Está presidida por la asepsia científica.
Establece proposiciones del tipo:
dadas tales circunstancias,
tendrán lugar tales acontecimientos.

\subsection{Economía normativa}

La economía normativa ofrece prescripciones para la acción,
basadas en juicios de valor personal y subjetivo.
Se pregunta \textit{cómo deberían ser} las cosas.
Desde el punto de vista normativo, 
la economía genera prescripciones
sobre el sistema económico.

\subsection{Necesidades}

Una necesidad es una sensación de carencia de algo.
Va unida al deseo de satisfacerla.

\subsection{Tipos de necesidades}

\begin{itemize}
    \item Según su origen:
    \begin{enumerate}
        \item Necesidades del individuo: 
        \begin{enumerate}
            \item Naturales (comer)
            \item Sociales (fiestas).
        \end{enumerate}
        \item Necesidades de la sociedad 
        \begin{enumerate}
            \item Colectivas (transporte) 
            \item Públicas (seguridad)
        \end{enumerate}
    \end{enumerate}
    \item Según su naturaleza 
    \begin{enumerate}
        \item Necesidades primarias: de las que depende la conservación de la vida (alimentos)
        \item Necesidades secundarias: mejoran la calidad de vida (turismo)
    \end{enumerate}
\end{itemize}
