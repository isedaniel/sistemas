\section{Primera clase}

20 de marzo, 2025.

\section{Presentación}

Materia asincrónica.
No hace falta ir a clase.
Se toma lo que se da en clase.
Dos parciales.
Promocionable, 
con promedio 7 o más,
6 mínimo.
Primero, preguntas para desarrollar.
Segundo parcial, \textit{multiplechoice}.
Poniéndose las pilas se promociona.

\subsection{Economía}

Es una ciencia.
En tanto ciencia, tiene que hacer un estudio de la realidad.
Pregunta de examen.
No sarasa.
Hay que poner tal cual lo dice el profe.

La economía se ocupa de los problemas vinculados a satisfacción de las necesidades,
apelando a la administración de recursos escasos,
con el objetivo de producción y distribución de bienes y servicios.

Economía es la ciencia que estudia la asignación más conveniente 
de los recursos escasos de una sociedad 
para la obtención de un conjunto ordenado de objetivos.

\subsection{Economía positiva y economía normativa}

\textbf{Pregunta de examen.}

\textbf{Economía positiva.}
Se define como
la ciencia que busca explicaciones \textit{objetivas} del funcionamiento de los fenómenos económicos.
Se ocupa de lo que es o podría ser.
Está precedida por la asepsia científica.
Se dedica a establecer proposiciones del tipo:
dadas tales circunstancias,
entonces tendrán lugar tales acontecimientos.

La \textbf{economía normativa} ofrece prescipciones para la acción
basada en juicios de valor personal y subjetivo
(de lo que \textit{debería} ser).
Desde el punto de vista normativo, 
la economía normativa genera prescripciones
sobre el sistema económico

\subsection{Necesidades}

\textbf{Necesidad humana.}
Es la sensación de carencia de algo,
unida al deseo de satisfacerla.

\textbf{Tipos de necesidades.}

Segun de quién surgen:

\begin{enumerate}
    \item Necesidades del individuo: Naturales (comer), Sociales (fiestas).
    \item Necesidades de la sociedad: colectivas (transporte), pública (seguridad)
\end{enumerate}

Según su naturaleza:

\begin{enumerate}
    \item Necesidades primarias: de las que depende la conservación de la vida (alimentos)
    \item Necesidades secundarias: (turismo)
\end{enumerate}