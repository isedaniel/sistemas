\section{Clase 29 de mayo}

\subsection{Valor agregado, producción, producto, ingreso}

A fin de poder ilustrar claramente la relación entre el valor agregado,
el producto y la producción,
analizaremos dos ejemplos simples,
donde se verá la valuación de la producción de un bien
pasando por cada sector de la economía
(primario, secundario, terciario).

\vspace{.5cm}
\begin{table}[H]
    \centering
    \caption{\\Estructura de costos del sector agrícola en la producción de trigo\\
        (\textit{en miles de pesos})}
    \vspace{.5cm}
    \begin{tabular}{lr}
        \hline
        Materias Primas               &     \\
        \hspace{.3cm} Semillas        & 25  \\
        \hspace{.3cm} Combustibles    & 15  \\
        \hspace{.3cm} Envases         & 10  \\
        Subtotal                      & 50  \\
        \hline
        Valor Agregado Bruto          &     \\
        \hspace{.3cm} Salarios        & 30  \\
        \hspace{.3cm} Renta del Suelo & 5   \\
        \hspace{.3cm} Beneficio       & 15  \\
        Subtotal                      & 50  \\
        \hline
        Valor de Producción Bruto     & 100 \\
    \end{tabular}
\end{table}
\vspace{.5cm}

Entonces, lo podemos expresar como:

\begin{equation*}
    V_{PB} = M_P + V_{AB}
\end{equation*}

Donde \(V_{PB}\) es el valor de producción bruto,
\(M_P\) es el valor de las materias primas y
\(V_{AB}\) es el valor agregado bruto.

\textbf{Nota.}
Si se restara al beneficio la \textit{depreciación}
de los activos fijos \((A_F)\),
se obtiene el Valor Neto de Producción.

\textbf{Depreciación.}
Desgaste sufrido por los bienes de capital.
Es decir,
máquinas y equipos se desgastan en el proceso productivo,
disminuyendo cada año en el patrimonio de las empresas.

Los valores netos resultan de restar depreciación a los valores brutos.

Habiendo analizado el cuadro 1,
luego el trigo pasará a las siguientes etapas productivas
y comercialziación.

En esta segunda esta el trigo va a ser la matería prima porque será un insumo
en esta nueva industria.
Por ejemplo, en el sector indutrial se necesita el trigo para fabricar la
harina que luego será pan.

Tener en cuenta también que cada una de las etapas incurrirá en gastos de
comercialización, para así completar así el proceso productivo.

\vspace{.5cm}
\begin{table}[H]
    \centering
    \caption{\\Desplazamiento de un bien a través de la Economía\\
        (\textit{en miles de pesos})}
    \vspace{.5cm}
    \begin{tabular}{lr}
        \hline
        Agro (trigo):                        &     \\
        Materias Primas                      & 50  \\
        Valor Agregado Bruto                 & 50  \\
        Valor de Producción Bruto del Agro   & 100 \\
        \hline                                     \\
        \hline
        Industria (harina y pan)             &     \\
        Materias primas                      &     \\
        \hspace{.3cm}Productos del agro      & 100 \\
        \hspace{.3cm}Combustibles            & 10  \\
        \hspace{.3cm}Insumos industriales    & 20  \\
        Subtotal                             & 130 \\
        \hline
        Valor Agregado Bruto                 &     \\
        \hspace{.3cm}Salarios                & 40  \\
        \hspace{.3cm}Beneficios              & 40  \\
        Subtotal                             & 80  \\
        \hline
        Valor de Producción Industrial Bruto & 210 \\
    \end{tabular}
\end{table}
\vspace{.5cm}

\textbf{Producto generado por el gobierno}

En el caso del gobierno,
tanto el valor agregado bruto (remuneraciones),
como el valor de producción bruto,
no son fácilmente visibles,
pues los bienes y servicios que pudieran resultar de su gestión no se transan
en el mercado.
Y lo que hay que considerar son los gastos salariales.

\subsection{Producto Interno Bruto}

(Pregunta de examen)

Cuando se habla de la producción de un país,
se suma la producción bruta total (o valor de producción bruto)
de cada uno de los sectores.
Pero no se puede decir que esa sea la riqueza generada en un país por año,
porque se estaría computando varias veces la misma cosa.

Por lo tanto la riqueza creada en un país en un período dado es la suma de los
valores agregados brutos, por todas las empresas, de todos los sectores.

Con lo cual,
el PIB es la sumatoria de los valores agregados brutos
por cada uno de los sectores,
más las remuneraciones del gobierno.

Para ser más precisos,
a su vez habría que sumarle los gastos de comercialización.
Y según cómo sea la definición a la que queremos llegar,
también los impuestos.

\textbf{Impuestos}

El gobierno cobra impuestos a empresas y personas para financiar su actividad.

Hay impuestos directos e indirectos.
Los indirectos son incorporados al precio del producto: IVA.
Los directos no se incorporan al precio del producto.
Se aplican sobre el beneficio en una empresa o a las personas.

\textbf{PBI a costo de factores:} sin impuestos indirectos.

\textbf{PBI a precio de mercado:} con impuestos indirectos.

Ejemplo de valor de producción bruto con impuestos:

\vspace{.5cm}
\begin{table}[H]
    \centering
    \caption{\\Cálculo del PIB industrial a precios de mercado\\
        (\textit{en miles de pesos})}
    \vspace{.5cm}
    \begin{tabular}{lr}
        \hline
        Empresa Industrial                 &     \\
        Materias Primas                    & 130 \\
        Valor Agregado Bruto               & 80  \\
        Impuestos indirectos               & 22  \\
        Gastos de comercialización         & 10  \\
        \hline
        Valor de Producción Bruto (\(PM\)) & 242 \\
    \end{tabular}
\end{table}
\vspace{.5cm}

Como se observa en el cuadro 3,
se agrega el impuesto indirecto -como el IVA- y gastos de comercialización
y obtenemos el Valor Bruto de Producción a \textit{precios de mercado}.

\vspace{.5cm}
\begin{table}[H]
    \centering
    \caption{\\Valor agregado a costo de factores y a precios de mercado\\
        (\textit{en miles de pesos})}
    \vspace{.5cm}
    \begin{tabular}{lrr}
        \hline
                                            & Agro & Industria \\
        Valor Agregado a costo de factores  & 50   & 80        \\
        Impuestos indirectos                & 0    & 22        \\
        \hline
        Valor Agregado a precios de mercado & 50   & 102       \\
    \end{tabular}
\end{table}
\vspace{.5cm}

Habiendo visto todo esto,
recordemos el concepto de depreciación,
con lo cual:

\begin{equation*}
    PIB = PIN + D
\end{equation*}

Donde \(PIB\) es producto interno bruto,
\(PIN\) es producto interno neto y 
\(D\) depreciación.

Tanto PIB como PIN puede ser a costo de factores o a precio de mercado.

El PIB es la sumatoria de los valores agregados + las remuneraciones del 
gobierno. Para ser más preciso, se suma también los gastos de comercialización.

\begin{equation*}
    PBI = C_P + G + I + Ex - Im
\end{equation*}

Donde \(C_P\) es consumo privado,
\(G\) es gasto público,
\(I\) es inversión privada,
\(E_x\) exportación
e \(I_m\) importación.
\(E_x - I_m\) es balanza comercial.