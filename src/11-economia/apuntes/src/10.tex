\section{Clase 12 de junio}

\subsection{Incoterms}

Son reglas para la implementación de los términos comerciales fijados por la 
cámara de comercio internacional. 
La palabra incoterms en una contracción,
que significa International Commercial Terms
(Términos de comercio internacional).

Los incoterms se centran en la obligación de entrega desde el punto de vista del
\textit{vendedor} (el exportador).

Los incoterms regulan:
\begin{itemize}
      \item La distribución de documentos
      \item Las condiciones de entrega de la mercadería
      \item La distribución de costos de la operación
      \item La distribución de riesgos de la operación 
\end{itemize}

Pero no regulan:
\begin{itemize}
      \item La legislación aplicable
      \item Las formas de pago de la operación 
\end{itemize}

Así pues, 
los incoterm definen y reparten claramente las obligaciones, 
los gastos, 
los riesgos del transporte internacional 
y el seguro tanto en el exportador como el importador.

Así pues, el incortem seleccionado,
entre el exportador y el importador,
determinará quien pagará el costo de cada segmento del transporte,
quién será responsable de cargar y descargar la mercadería,
y quién lleva el riesgo de la pérdida en un momento dado durante el envío
(sea local o internacional)
y se clasifican en dos grupos fundamentales:
\begin{enumerate}
      \item Grupo de salida
      \item Grupo de llegada
\end{enumerate}

\textbf{Grupo de salida.}
Empiezan con la letra E, F, C.

\textbf{Grupo de llegada.}
Empiezan con la letra D.

A comienzos del siglo XX,
se estableció la necesidad de la reglamentación del contrato de compraventa 
internacional,
y fue así que se crearon en 1936,
habiendo varias revisiones o endosos posteriores.

\subsection{Ejemplos de los principales términos incoterms}

\textbf{Grupo de salida.}
\begin{itemize}
      \item EXW: Exwork el vendedor entrega el producto en la puerta de su fábrica. 
      De ahí en adelante el resto de gastos corren por cuenta del comprador.
      \item FAS: Free Alongside Ship (franco al costado del buque).
      La entega de la mercadería se realiza cuando es colocada por el vendedor 
      al costado del buque en el puerto de carga. Desde ahí, todos los costos 
      y riesgos quedan a cargo del comprador.
      \item FOB: Free On Board (franco abordo).
      El vendedor tiene la obligación de cargar la mercadería a bordo del buque 
      en el puerto de embarque y el comprador de encarga del resto 
      (Seleccionar el buque, contratarlo, pagar el flete marítimo,
      el seguro, los gastos de descarga, aduana y flete en final).
      \item CFR: Cost and Freight (Costo y Flete).
      El vendedor paga los gastos de transporte y otros para que la mercadería 
      llegue al puerto convenido en destino. Incluye también el despacho 
      (aduana) de la mercadería de exportación. Pero el seguro del buque va a 
      cargo del comprador.
      \item CIF: Cost, Insurance and Freight (Costo, seguro y flete).
      Igual al anterior, pero el vendedor se hace cargo del seguro 
      internacional.
\end{itemize}

\textbf{Grupo de llegada.}
\begin{itemize}
      \item DAF: Delivered at frontier (entregado en frontera)
      El vendedor cumple su obligación de entregar la mercadería hasta un punto
      convenido de la frontera antes de rebasar la aduana.
      \item DDP: Delivered duty paid (entregado con pago de derechos)
      En este término el vendedor realiza la entrega de la mercadería al 
      comprador ya con el despacho terminado de importación en la puerta del 
      lugar convenido del país importador (el vendedor se encarga de todo).
\end{itemize}

En el transporte internacional los recibos de embarque de las mercaderías toman 
diferentes nombres de acuerdo al medio de transporte.
\begin{enumerate}
      \item Conocimiento de embarque o Bill of Lading (BL). Es el documento por 
      medio del cual se instrumenta el contrato de transporte de mercadería 
      por agua. No indica el precio de la mercadería, es simplemente un recibo,
      no es una factura. El BL es entregado por el transportador, por lo cual 
      acredita el contrato de transporte.
      \item Vía aérea, Air Waybill (AW). (Igual al primero)
      \item  Guías Aéreas (Air Waybill): 
      \item Carta de Porte: 
      se utiliza en transporte terrestre internacional. 
      Recibo de carga con la descripción de la mercadería.
\end{enumerate}