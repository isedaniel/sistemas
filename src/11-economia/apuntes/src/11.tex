\section{Clase 19 de junio}

\subsection{Mercados Financieros}

En países con mercados financieros modernos,
las personas tienen la posibilidad de elegir
entre varios tipos de activos diferentes,
que van desde el más simple -que es el dinero-,
hasta los bonos, acciones, fondos comunes de inversión, etc.

\subsection{Bonos}

Pueden ser emitidos tanto por el estado como por los privados.

En general,
podemos definirlo como una obligación financiera,
donde el emisor contrae una promesa de pago a futuro
documentada en un certificado,
en el cual se indica monto, plazo, moneda y secuencia de pago.

Cuando un inversionista compra un bono se puede decir que le está prestando su
dinero al estado o a la empresa emisora,
y en retorno a este préstamo el emisor promete pagarle al inversionista ciertos
intereses durante la vida del bono,
para que el capital invertido sea reinvertido a dicha tasa cuando llegue a la
maduración o vencimiento.

\begin{equation*}
    M = C + I
\end{equation*}

Donde $M$ es monto,
$C$ es capital
e $I$ el interés.
No es tasa de interés $(i)$,
sino \textit{monto} del interés.

Las claves para elegir un bono van a depender de muchas variables:
su maduración o vencimiento, tasa de interés, impuestos, etc.

\subsection{Bonos basura}

Son bonos de baja calificación.
Tienen un alto rendimiento.
Es decir, tienen una reputación dudosa
cuyo nivel de riesgo sobrepasa los límites de una inversión común.
En contrapartida tienen un rendimiento elevado,
por encima del promedio del mercado,
lo que tienta por las buenas ganancias.
A mayor riesgo,
mayor rendimiento.

\subsection{Clasificación de Bonos}

\begin{enumerate}
    \item Por cantidad de pagos:
          \begin{enumerate}
              \item Bonos de cupón 0:
                    corresponde a un solo pago que se denomina valor nominal,
                    pero que se compran más barato que lo que dice
                    el valor nominal.
                    Ejemplo: valor nominal 100, valor de compra 80.
              \item Bonos a plazo fijo:
                    prometen efectuar muchos pagos y cada cuota se llama cupón.
                    Se especifican en el bono 2 tipos de rendimiento:
                    el primero es rendimiento corriente, que es
                    \textit{cupon/valor de compra},
                    el segundo es redimiento cupón \textit{cupon/valor nominal}.
          \end{enumerate}
    \item Por duración:
          \begin{enumerate}
              \item Corto plazo: Menos de 1 año.
              \item Mediano plazo: Entre 1 y 10 años.
              \item Largo plazo: Más de 10 años.
              \item Nota:
                    en Estados Unidos se denominan: 
                    a corto plazo: \emph{letras del tesoro}
                    y en general son bonos de cupón 0; 
                    a mediano plazo: son los denominados
                    \emph{obligaciones del tesoro} y son bonos a plazo fijo; 
                    a largo plazo:
                    \emph{bonos del tesoro}, también a plazo fijo.
          \end{enumerate}
\end{enumerate}

\subsection{Bolsa - Acciones}

Las empresas,
cuando necesitan financiamiento emiten un bono, 
un \emph{pagaré},
tal cual explicado en párrafos anteriores, 
o piden un préstamo en algún banco,
o emiten acciones, 
consiguiendo capital genuino si comprometerse a devolver esos fondos 
al inversionista y, por lo tanto, no adquieren una deuda. 
El inversionista se convierte en propietario de una parte de la compañia, 
y su objetivo es que su capital se incremente 
en la medida en que la compañia también crezca. 
Algunas acciones pagan dividendos, 
que pueden ser trimestrales, 
semestrales o anuales. 
Sin embargo, 
no todas pagan dividendos 
y se concentran todos los ingresos en hacer crecer el valor de la acción que,
en definitiva, 
la ganancia del inversionista resulta del crecimiento del precio de la acción.

La transacción de las acciones se realiza en la \emph{bolsa}, 
que es un mercado, 
donde se encuentran compradores y vendedores,
inversionistas y empresas. 
La bolsa cumple una función en el crecimiento de una economía, 
ya que canaliza el ahorro hacia la inversión productiva.

\subsection{Ejemplos de bolsas mundiales}

En Estados Unidos se comercializan las acciones 
en lo que se denomina Wall Street. 
No obstante, 
hay que tener en cuenta los índices accionarios o bursátiles, 
que indican el crecimiento o no del mercado accionario. 
El índice más conocido es el Dow Jones, 
representativo de las 30 mayores compañias industriales de Estados Unidos, 
y se calcula sumando los precios de sus acciones y dividiendo por una constante.
Algunas de estas 30 empresas son:

\begin{enumerate}
    \item American Express
    \item Boeing
    \item Caterpillar
    \item Chevron
    \item Citygroup
    \item Disney
    \item Dupont
    \item General Electric
    \item General Motors
    \item Coca Cola
    \item McDonald's
    \item Exxon
    \item IBM
    \item Merck
    \item Union Carbide
    \item Wallmart
\end{enumerate}

Como el índice Dow Jones ha quedado obsoleto, 
atento a que es un promedio de empresas industriales -las 30 \emph{bluechips}-, 
no representa las transacciones diarias en Estados Unidos. 
Por lo cual se han creado nuevos índices, a saber:

Standard \& Poor's 500, 
compuesto por 400 acciones industriales, 
20 del sector transporte, 
40 de servicios públicos 
y 40 financieras.

Amex Composite, 
índice de todas las acciones 
que se venden en un mercado llamado American Stock Exchange.

También está el NASDAQ 100, 
que representa las acciones que se venden en la bolsa o el mercado NASDAQ, 
no en Wall Street,
y generalmente representan acciones viculadas a la tecnología, 
como Adobe, Microsoft o Cisco.

\subsection{Otras bolsas de valores en el mundo}

En Japón, NIKKEI; en Alemania, DAX; Francia, CAC; Reino Unido, FTSE; Argentina,
Merval; Brasil, Bovespa.
