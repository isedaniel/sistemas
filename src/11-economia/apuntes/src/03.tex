\section{Clase 3 de abril}

\subsection{Funciones del estado}

\textbf{(Pregunta de examen)}

\begin{enumerate}
    \item Establecer el marco legal para la economía de mercado.
    \item Ofrecer y comprar bienes, servicios y realizar transferencias.
    \item Establecer impuestos.
    \item Estabilizar la economía.
    \item Redistribuir la renta.
    \item Procurar la eficiencia económica.
\end{enumerate}

\subsection{Establecer el marco legal para la economía de mercado}

El estado regula el funcionamiento de la economía mediante un conjunto de leyes,
normas y reglamentaciones, como por ejemplo, leyes que definen la propiedad privada,
condiciones de los contratos, regulación de sociedades, etc.

Por otro lado, hay agencias de regulación estatales que verifican la existencia de monopolios,
otras que protegen al consumidor final, etc.

\subsection{Ofrecer y comprar bienes, servicios y realizar transferencias}

Existe una serie de bienes y servicios que el Estado está en mejores condiciones de ofrecer que los particulares,
por ejemplo, la defensa, la seguridad, el correo -donde haga falta-.

Por otro lado, en cuanto a la compra de bienes,
puede ser la compra de edificios,
muebles, equipos informáticos,
que serviran para el normal funcionamiento de la actividad pública.

Transferencias: pagos por los cuales los que reciben no dan en contraprestación ningún bien o servicio,
ejemplo: plan social, pensiones, seguridad social.

\subsection{Establecer impuestos}

El sector público, \textit{para financiar} sus gastos se ve obligado a establecer impuestos.
Es decir, el objetivo primordial principal de los impuestos es cubrir el gasto público.

Sin embargo, también existen ciertos impuestos para desalentar la producción de determinados bienes,
estableciendo impuestos adicionales que elevan el precio del producto en cuestión,
ejemplo: tabaco.

Los impuestos pueden ser proporcionales, progresivos o regresivos.
Proporcional: el IVA. 
Un progresivo: es cuando a medida que aumenta la renta,
el impuesto es porcentualmente mayor. 
Y el impuesto regresivo es si detrae un porcentaje de la renta cada vez menor a medida que la renta aumenta.

También existen los directos e indirectos.
Los indirectos, como el IVA, recauda sobre los bienes y servicios y, por tanto,
afectan indirectamente al contribuyente.
En cambio, el directo afecta directamente a la renta, ejemplo: IIGG.

\subsection{Tratar de estabilizar la economía}

Evitar que las variables macro (nivel de empleo, nivel de precios) tengan fluctuaciones excesivas.
Para ello se suele poner en práctica ciertas políticas de estabilización,
que tienen por objetivo \textit{suavizar los ciclos económicos}.

Dado un grafico sinusoidal,
el extremo positivo se llama auge,
el extremo negativo depresión,
el segmento que va de auge a depresión se denomina recesión,
el segmento que va de depresión a auge se denomina recuperación.

\subsection{Redistribuir la renta}

El estado puede emplear los impuestos, el gasto público, para redistribuir la renta.
Sucede muchas veces que no resulta absolutamente equitativa y que cierto grupo de personas
viva en una situación de opulencia y otras no tengan lo suficiente para subsistir.
Y es ahí donde el estado debe intervenir para tomar ciertas medidas y acudir a ciertos instrumentos para modificar esa situación.

\subsection{Procurar la eficiencia económica}

La intervención del estado se debe a que en ocasiones los mercados no asignan los recursos eficientemente.
En esos casos, el mismo estado debe mejorar esos resultados.
El argumento general en favor de la intervención del estado es la existencia de \textit{fallos de mercado}:
competencia imperfecta, bienes públicos y otras externalidades.

Bienes públicos: bienes que no son rentables como un faro.

\subsection{Demanda}

Desde el punto de vista del comprador.
Supongamos una serie de \textit{factores} determinantes de las \textit{cantidades} (Q)
que los \textit{consumidores} desean adquirir de cada bien por unidad de tiempo,
tales como las preferencias, el ingreso, 
estacionalidad, los precios de los demás bienes y, sobre todo,
el precio del propio bien en cuestión.

Si consideramos constantes todos estos factores, salvo el precio del propio bien,
podemos hablar de una tabla de demanda de un bien A,
para un consumidor determinado,
cuando consideramos la relación que existe entre la cantidad demandada (Q)
y el precio de ese bien (P).

Tablita:

\begin{tabular}{ c c }
    \(Q_A\) & \(P_A\) \\
    1& 8 \\
    2& 6 \\
    3& 4 \\
    4& 2 \\
\end{tabular}

Bajo la condición de \textit{ceteris paribus},
para un precio del bien A determinado,
la suma de las demandas individuales nos dará la demanga global o de mercado de ese bien.

\subsection{Ley de la demanda}

Es la relación inversa existente entre el precio de un bien y la cantidad demandada,
es decir, al aumentar el precio disminuye la cantidad demandada, y viceversa,
cuando el precio disminuye la cantidad demandada se incrementa.

Para la próxima: \textbf{curva y la función demanda}.