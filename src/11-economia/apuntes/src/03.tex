\section{Clase 3 de abril}

\subsection{Funciones del estado}

\textbf{(Pregunta de examen)}

\begin{enumerate}
    \item Establecer el marco legal para la economía de mercado.
    \item Ofrecer y comprar bienes, servicios y realizar transferencias.
    \item Establecer impuestos.
    \item Estabilizar la economía.
    \item Redistribuir la renta.
    \item Procurar la eficiencia económica.
\end{enumerate}

\subsection{Establecer el marco legal para la economía de mercado}

El estado regula el funcionamiento de la economía mediante un conjunto de leyes,
normas y reglamentaciones, como por ejemplo, leyes que definen la propiedad privada,
condiciones de los contratos, regulación de sociedades, etc.

Por otro lado, hay agencias de regulación estatales que verifican la existencia de monopolios,
otras que protegen al consumidor final, etc.

\subsection{Ofrecer y comprar bienes, servicios y realizar transferencias}

Existe una serie de bienes y servicios que el Estado está en mejores condiciones de ofrecer
que los particulares,
por ejemplo,
la defensa,
la seguridad,
el correo -donde el comercial no llega-,
etc.

Por otro lado,
el estado compra bienes,
como edificios, muebles, equipos informáticos,
que servirán para el normal funcionamiento de la actividad pública,
esto es,
para la provisión de los bienes y servicios mencionados.

Por último,
el estado hace transferencias,
que son pagos sin contraprestación por parte del receptor,
ejemplo:
pensiones, seguridad social.

\subsection{Establecer impuestos}

El sector público establece impuestos \textit{para financiar} sus gastos.
El objetivo principal de los impuestos es cubrir el gasto público.

Además,
algunos impuestos se puede utilizar para desincentivar la producción de determinados bienes,
elevando el precio del producto en cuestión,
por ejemplo: impuestos excepcionales al tabaco.

Los impuestos pueden ser proporcionales, progresivos o regresivos.
\begin{itemize}
    \item Proporcional: el IVA.
    \item Progresivo: el impuesto es porcentualmente mayor
          a medida que aumenta la renta del contribuyente.
    \item Regresivo: es proporcionalmente menor a medida que
          aumenta la renta del contribuyente.
\end{itemize}

Tabién se puede clasificar en directos e indirectos.
Los indirectos,
como el IVA,
recaudan sobre bienes y servicios,
afectando indirectamente al contribuyente.
En cambio,
un impuesto directo afecta directamente a la renta,
como el impuesto a las ganancias.

\subsection{Tratar de estabilizar la economía}

Evitar que las variables macro
(nivel de empleo, nivel de precios)
tengan fluctuaciones excesivas.
Para ello se suele poner en práctica ciertas políticas de estabilización,
que tienen por objetivo \textit{suavizar los ciclos económicos}.

Dado un grafico sinusoidal,
el extremo positivo se llama auge,
el extremo negativo depresión,
el segmento que va de auge a depresión se denomina recesión,
el segmento que va de depresión a auge se denomina recuperación.

\subsection{Redistribuir la renta}

El estado puede emplear impuestos y gasto público,
en la redistribución de la renta.
Puede suceder que un parte de la población
viva en una situación de opulencia,
mientras otra parte no tenga lo suficiente para subsistir.
Frente a esta situación,
el estado debe intervenir, tomar ciertas medidas
y acudir a ciertos instrumentos para modificar la situación.

\subsection{Procurar la eficiencia económica}

La intervención del estado puede deberse a que,
en ciertas ocasiones,
los mercados no asignan los recursos de forma eficiente.
En esos casos,
el estado debe mejorar esos resultados.
El estado interviene por la existencia de \textit{fallos de mercado},
tales como:
\begin{itemize}
    \item Competencia imperfecta: monopolio por falla de mercado.
    \item Bienes públicos: bienes que no son rentables, ej. infraestructura, faro.
    \item Externalidades: precio relativo que cambia fuertemente por una shock externo,
    ej. Guerra que encarece productos.
\end{itemize}

\subsection{Demanda}

El análisis de la demanda se hace desde la perspectiva del comprador.
Se concentra en una serie de \textit{factores} determinantes de las \textit{cantidades} (Q)
que los \textit{consumidores} desean adquirir de cada bien, 
tales como las preferencias, 
el ingreso,
estacionalidad, 
los precios de los demás bienes y, sobre todo,
el precio del propio bien.

Considerando \textit{constantes} todos estos factores salvo el precio,
podemos confeccionar una tabla de demanda de un bien A,
para un consumidor determinado,
considerando la relación entre la cantidad demandada (Q)
y el precio del bien (P).

\begin{tabular}{ c c }
    \(Q_A\) & \(P_A\) \\
    1       & 8       \\
    2       & 6       \\
    3       & 4       \\
    4       & 2       \\
\end{tabular}

Bajo la condición \textit{ceteris paribus},
para un precio determinado del bien A,
la suma de las demandas individuales nos dará la demanda de mercado de ese bien.

\subsection{Ley de la demanda}

Es la relación inversa entre el precio de un bien y la cantidad demandada,
es decir,
un aumento en el precio implica una disminución en la cantidad demandada,
y viceversa,
cuando el precio disminuye la cantidad demandada se incrementa.
