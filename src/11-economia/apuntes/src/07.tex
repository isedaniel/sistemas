\section{Clase 22 de mayo - Segunda mitad}

\subsection{Ingreso nacional}

(Introducción al Producto Interno Bruto)

Los economistas suelen utilizar información estadística
para evaluar la evolución de la economía.
En general, la información económica se expresa en unidades monetarias.
Básicamente,
se trata de hacer un relevamiento y registro de una serie de magnitudes,
de variables,
referidas a transacciones económicas en un período dado.

\subsection{Sectores productivos de la economía}

Se denomina sector al conjunto de unidades productivas que actúan en la economía,
que tienen en común ciertas características referidas al tipo de bienes que producen
o a las tecnologías que utilizan:

\vspace{.5cm}
\begin{table}[H]
    \centering
    \begin{tabular}{ll}
        \hline
        Sector primario   & Agricultura, minería pesca \\
        Sector secundario & Industria manufacturera    \\
        Sector terciario  & Servicios                  \\
        \hline
    \end{tabular}
\end{table}
\vspace{.5cm}

A su vez cada uno de los sectores puede subdividirse en varios otros,
dando lugar así a una clasificación más detallada.
Por ejemplo,
en el sector secundario puede haber distintas ramas industriales,
como la alimentaria, la química,
la metalúrgica,
etc.

También estas subdivisiones se pueden,
a su vez,
dividir en otros componentes,
como la alimentaria en vitivinícola,
aceitero,
lácteos,
conservas,
etc.

\subsection{Factores de producción y remuneración percibida}

\vspace{.5cm}
\begin{table}[H]
    \centering
    \begin{tabular}{ll}
        Factores & Remuneración     \\
        \hline
        Trabajo  & Salario o sueldo \\
        Capital  & Interés          \\
        Tierra   & Renta            \\
        Empresa  & Beneficios       \\
        \hline
    \end{tabular}
\end{table}
\vspace{.5cm}

El factor genera o recibe una remuneración.

Para producir intervienen cuatro elementos o factores.
La producción resulta de la articulación del trabajo, el capital, la tierra y la empresa.
En la práctica,
muchas veces suelen confundirse las remuneraciones de alguno de los factores entre sí.
¿El dueño de una empresa recibe beneficio o salario?

Por ejemplo,
si hablamos de la tierra,
el dueño al explotarla está recibiendo dos tipos de beneficios en uno:
la ganancia por cultivar y la renta que ganaría si hubiese dado en alquiler.
El empresario recibe dos rentas como beneficio de la empresa:
una como si tuviera un sueldo que le paga su propia empresa
y la otra el adicional por su riesgo empresarial.

\subsection{Clasificación de bienes disponibles}

Los bienes pueden clasificarse según el uso que se haga de ellos en el proceso económico,
como por ejemplo,
bienes finales y bienes intermedios.

\begin{itemize}
    \item Finales
          \begin{itemize}
              \item No durables
              \item Durables
                    \begin{itemize}
                        \item De consumo
                        \item De producción
                    \end{itemize}
          \end{itemize}
    \item Intermedios
\end{itemize}

\textbf{Bienes durables de producción:} Son aquellos que se destinan a producir otros bienes.

\textbf{Bienes intermedios:}
Son aquellos que no se usan ni para producir ni para consumir
en el estado en que se encuentran,
sino que son utilizados en el proceso productivo
destinado a producir otros bienes.
Ejemplo:
Para el pan se utiliza harina.
El pan es un bien final.
El trigo y la harina,
en este caso,
es un bien intermedio.