\section{Clase 22 de mayo - Segunda mitad}

\subsection{Ingreso nacional: introducción al Producto Interno Bruto}

Los economistas suelen utilizar información estadística
para evaluar y monitorear la evolución de la economía.
En general, 
la información económica se expresa en unidades monetarias.
Básicamente,
se trata de hacer un relevamiento y registro sistemático 
de una serie de magnitudes, de variables,
referidas a transacciones económicas en un período dado.

\subsection{Sectores productivos de la economía}

Se denomina \textbf{sector} al conjunto de unidades productivas 
que actúan en la economía
y que tienen en común el tipo de bienes que producen 
o a las tecnologías que utilizan:

\vspace{.5cm}
\begin{table}[H]
    \centering
    \begin{tabular}{ll}
        \hline
        Sector primario   & Agricultura, minería, pesca \\
        Sector secundario & Industria manufacturera    \\
        Sector terciario  & Servicios                  \\
        \hline
    \end{tabular}
\end{table}
\vspace{.5cm}

A su vez cada uno de los sectores puede subdividirse en varios otros,
dando lugar así a una clasificación más detallada.
Por ejemplo,
en el sector secundario puede haber distintas ramas industriales,
como la alimentaria, la química, la metalúrgica, etc.

Estas subdivisiones se pueden dividir,
nuevamente,
en otros componentes,
como la alimentaria en vitivinícola, aceiteros, lácteos, etc.

\subsection{Factores de producción y remuneración percibida}

\vspace{.5cm}
\begin{table}[H]
    \centering
    \begin{tabular}{ll}
        Factores & Remuneración     \\
        \hline
        Trabajo  & Salario o sueldo \\
        Capital  & Interés          \\
        Tierra   & Renta            \\
        Empresa  & Beneficios       \\
        \hline
    \end{tabular}
\end{table}
\vspace{.5cm}

En todo proceso productivo intervienen estos cuatro factores.
La producción resulta de la coordinación del trabajo, 
el capital, la tierra y la empresa.
Cada factor genera o recibe una remuneración.

En la práctica,
las formas de remuneración pueden confundirse.
Por ejemplo,
si el dueño de la tierra produce con ella,
estaría recibiendo dos tipos de beneficios:
la ganancia por cultivar y la renta que ganaría si la hubiese dado en alquiler.
El empresario que administra su propia empresa recibe dos tipos de remuneración:
un salario por su trabajo,
-que de otra manera debería pagarlo a alguien más-
y un beneficio por los riesgos que asume.

\subsection{Clasificación de bienes disponibles}

Los bienes puden clasificarse de muchas formas.
Hemos visto algunas de estas clasificaciones con anterioridad.
En este punto,
podemos clasificarlos por el \textit{uso que se hace de ellos}
en el proceso productivo,
distinguiendo bienes finales y bienes intermedios.

\begin{itemize}
    \item Finales
          \begin{itemize}
              \item No durables
              \item Durables
                    \begin{itemize}
                        \item De consumo
                        \item De producción
                    \end{itemize}
          \end{itemize}
    \item Intermedios
\end{itemize}

Algunos ejemplos:

\textbf{Bienes no durables de consumo:}
alimentos, se agotan en su uso.

\textbf{Bienes durables de consumo:}
ropa, muebles, no se agotan en su uso pero tienen como destino principal el 
satisfacer una necesidad.

\textbf{Bienes durables de producción:} 
una máquina,
tampoco se agotan en su uso,
por ello son durables,
pero su principal objetivo es la producción de otros bienes.

\textbf{Bienes intermedios:}
Son aquellos que,
en su estado actual,
no pueden usarse para producir ni para consumir.
En cambio,
son insumos que se emplean en el proceso productivo para elaborar otros bienes.
Por ejemplo:
Para el pan se utiliza harina.
El pan es un bien final.
El trigo y la harina,
en este ejemplo,
son bienes intermedios.
