\section{Segunda clase, 21 de agosto}

\subsection{Programación orienta a objetos}

En programación estructurada:
\begin{itemize}
    \item Vamos paso a paso
    \item Controlando el flujo puntillosamente
\end{itemize}

En POO:
\begin{itemize}
    \item Nos orientamos al problema a resolver
    \item Ejemplo: gestión de barcos en un puerto
    \item Hay una entidad que se llama barcos, otra grúa
    \item Hay conceptos que forman parte del \textbf{dominio} del problema
    \item Vamos buscando un \textbf{diseño}
    \item Teniendo en cuenta esos \textbf{elementos/objetos}
    \item Que existen en nuestra \textbf{representación}
\end{itemize}

\subsection{Objeto}

Un array:
\begin{itemize}
    \item Tiene varios elementos
    \item De un mismo tipo
\end{itemize}

Un objeto:
\begin{itemize}
    \item Puede tener mucha información
    \item No necesariamente mismo tipo
    \item En esto no se diferencia aún de \texttt{struct}
    \item Tiene variables, pero además tiene \textit{operaciones}, que en 
    POO se denominan \textbf{métodos}
    \item Ayuda a entender el código, la intención del objeto en el modelo
\end{itemize}

\subsection{Público:}

\begin{itemize}
    \item Aquello a lo que cualquiera puede llegar a acceder
    \item La \textbf{interfaz} hacia el interior del objeto
\end{itemize}

\subsection{Privado}

\begin{itemize}
    \item Lo que el objeto hace internamente
\end{itemize}