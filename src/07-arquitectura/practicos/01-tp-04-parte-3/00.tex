\documentclass[12pt]{article}
\usepackage[a4paper, margin=2.54cm]{geometry}
\usepackage[spanish]{babel}

% imágenes
%\usepackage{graphicx}
%\graphicspath{{img}}

% fuentes de conjuntos numéricos
\usepackage{amsfonts}

% math
\usepackage{amsmath, amssymb}

% gráficos y plots
\usepackage{tikz}
%\usepackage{pgfplots}
%\pgfplotsset{width=10cm, compat=1.9}
\usetikzlibrary{babel}

\setlength{\jot}{8pt}
\setlength{\parindent}{0cm}

% espacio entre párrafos y sangría
\usepackage[skip=10pt plus1pt, indent=12pt]{parskip}

% cancelar términos
\usepackage{cancel}

% links
%\usepackage[colorlinks=true,
%    urlcolor=blue]{hyperref}

% shapes
%\usetikzlibrary{shapes.geometric}

% incluir pdfs
%\usepackage{pdfpages}

\usepackage[style=apa]{biblatex}
\addbibresource{bib.bib}
\usepackage{csquotes}

\begin{document}

\thispagestyle{empty}

\begin{center}
	\vspace*{.5cm}
	\includegraphics[scale=.6]{~/Pictures/udemm-logo.png}\\
	\vspace{.2cm}
	\Large
	\textbf{Facultad de Ingeniería}\\
	\textbf{Ingeniería en Sistemas}\\
	\vspace{2cm}

	\Huge
	Arquitectura de computadoras 				\\
	Trabajo Práctico N\(^\circ\) 4 - Parte III  \\
	\vfill

	\raggedright
	\Large
	Docentes:
	\begin{itemize}
		\item[] Lic. Claudio M. Biffani
	\end{itemize}
	Alumno:
	\begin{itemize}
		\item[] Daniel Ise
	\end{itemize}
	Legajo:
	\begin{itemize}
		\item[] 28547
	\end{itemize}
	Fecha:
	\begin{itemize}
		\item[] Noviembre, 2024
	\end{itemize}
\end{center}

\pagebreak

\tableofcontents

\pagebreak

\section{Consignas}

\begin{enumerate}
	\item Desarrollar el concepto de Acceso Directo a Memoria (DMA).
	\item Desarrollar el concepto de Mapeo de Memoria o I/O Mapeado en memoria.
	\item Ejemplificar el caso de uso más apropiado para cada caso.
	\item Referenciar la información utilizada en formato APA.
\end{enumerate}

\pagebreak

\section{Respuestas}

\subsection{Acceso Directo a Memoria}

El Acceso Directo a Memoria (o DMA, por sus siglas en inglés),
refiere a la capacidad de algunos sistemas informáticos de
otorgar permisos de lectoescritura de la memoria principal
a un dispositivo de entrada/salida,
con independencia del Microprocesador.
Los sistemas que cuentan con canales dedicados al DMA pueden lidiar con
dispositivos cuya velocidad o cantidad de datos ingresadas sean difíciles de
manejar para el procesador.
Esto permite alcanzar un acceso de mayor velocidad y,
en simultáneo, liberar parte del trabajo que,
de otra manera,
debería realizar la CPU\parencite{directmemoryaccess}.

Como contrapartida,
dedicar parte de la memoria de manera exclusiva a un dispositivo de entrada/salida
requiere de soluciones de Hardware -como circuitos específicos para dicho propósito-,
así como previsiones en el Software para el mismo,
a modo de evitar que ocurran problemas de sincronización
-acceder a una versión vieja de un dato grabado en memoria, por ejemplo-,
que pueden complicar inclusive la estabilidad del sistema.
En suma, el DMA supone desafíos a la administración de memoria \parencite{corbetetal2005linux}.

Existen principalmente dos modalidades para evitar este problema: la primera es
la exclusividad de acceso a la memoria para el periferico, hasta que finalice el
proceso, dejando al CPU ocuparse del resto del programa hasta recibir la señal
de \textit{interrupción} por parte del dispositivo \parencite{harvey1991dmafundamentals}.
Esto permite computación en paralelo,
con sus consecuentes puntos positivos y negativos,
concernientes tanto a la mayor velocidad como creciente complejidad,
que deberán de ser administradas del lado del software por el desarrollador.
Por otra parte, se puede detener el proceso hasta tanto
el dispositivo haya finalizado la transferencia de datos,
que reduce la complejidad del software
pero significando un crecimiento del tiempo para finalizar el proceso
\parencite{santamaria1993electronica}.

Existen casos de uso en los cuales el DMA no sería deseable, por ejemplo,
cuando se desea hacer trabajo \textit{on the fly} con los datos que el dispositivo
va introduciendo al sistema. En ese caso, antes que un \textit{overhead} del procesador,
el hecho de que los datos introducidos por el dispositivo de entrada/salida se vayan
trabajando implican un ahorro de tiempo a futuro \parencite{harvey1991dmafundamentals}.

Entre los dispositivos que pueden recurrir al DMA encontramos placas de video,
placas de red o discos duros, entre otros. Inclusive, el DMA puede ser empleado
por los diferentes núcleos de un mismo procesador para compartir información
entre sí. En dicho caso, los circuitos integrados dentro del mismo dispositivo
suelen permitir el intercambio de datos en la memoria de trabajo \parencite{directmemoryaccess}.

\subsection{Entrada/Salida Mapeada en Memoria}

La Entrada/Salida mapeada en memoria (MMIO, por sus siglas en inglés)
es otra técnica para interactuar con dispositivos de entrada/salida. 
En este caso, se asigna al dispositivo una
dirección de memoria, como si se tratase de memoria del sistema. Mediante esta
técnica el procesador puede acceder a los dispositivos de entrada/salida como si
se tratase de memoria física en el sistema, reduciendo la cantidad de instrucciones
necesarias para interactuar con el mismo \parencite{wikientradasalidamapeada}.
Esto fue particularmente importante en los inicios de la computación, en la
medida en que reducía la complejidad necesaria para los procesadores.

La Entrada/Salida mapeada en memoria puede resultar más fácil de implementar y
simple, en la medida en que utilizando un conjunto de intrucciones limitada 
del procesador, es posible interactuar con los dispositivos de entrada/salida \parencite{stallings2016computer}.

Como contrapartida,
en los casos en que las direcciones de memoria puede enfrentar limitaciones,
como era el caso en su momento con las primeras computadoras en incluir esta tecnología,
y como lo es hoy en ciertas industrias,
como en sistemas embebidos,
la entrada/salida mapeada en memoria puede suponer dificultades en un registro de memoria limitado,
por lo que es una condición a tener en cuenta para su uso \parencite{wikimemorymappedio}.

\printbibliography[heading=bibnumbered]

\end{document}
