\documentclass[12pt]{article}
\usepackage[a4paper, margin=2.54cm]{geometry}
\usepackage[spanish]{babel}

% imágenes
%\usepackage{graphicx}
%\graphicspath{{img}}

% fuentes de conjuntos numéricos
\usepackage{amsfonts}

% math
\usepackage{amsmath, amssymb}

% gráficos y plots
\usepackage{tikz}
%\usepackage{pgfplots}
%\pgfplotsset{width=10cm, compat=1.9}
\usetikzlibrary{babel}

\setlength{\jot}{8pt}
\setlength{\parindent}{0cm}

% espacio entre párrafos
\usepackage{parskip}

% cancelar términos
\usepackage{cancel}

% links
%\usepackage[colorlinks=true,
%    urlcolor=blue]{hyperref}

% shapes
%\usetikzlibrary{shapes.geometric}

% incluir pdfs
%\usepackage{pdfpages}

\begin{document}

\thispagestyle{empty}

\begin{center}
  \vspace*{.5cm}
  \includegraphics[scale=.6]{~/Pictures/udemm-logo.png}\\
  \vspace{.2cm}
  \Large
  \textbf{Facultad de Ingeniería}\\
  \textbf{Ingeniería en Sistemas}\\
  \vspace{2cm}

  \Huge
  Arquitectura de computadoras\\
  Trabajo Práctico N\(^\circ\) 3\\
  \vfill

  \raggedright
  \Large
  Docente:
  \begin{itemize}
    \item[] Lic. Claudio M. Biffani \\
  \end{itemize}
  Alumno:
  \begin{itemize}
    \item[] Daniel Ise
  \end{itemize}
  Legajo:
  \begin{itemize}
    \item[] 28547
  \end{itemize}
  Fecha:
  \begin{itemize}
    \item[] Octubre, 2024
  \end{itemize}
\end{center}

\pagebreak

\tableofcontents

\pagebreak

\section{Consignas}

\begin{enumerate}
  \item Según lo visto en clase, listar y describir los distintos tipos de
        memoria, con un ejemplo de apliación práctica
  \item Desarrollar los conceptos de SWAP, paginación y segmentación de
        memoria.
\end{enumerate}

\section{Respuestas}

\subsection{Tipos de memoria}

\subsubsection{}

\end{document}
