\documentclass[12pt]{article}
\usepackage[a4paper, margin=2.54cm]{geometry}
\usepackage[spanish]{babel}

% imágenes
%\usepackage{graphicx}
%\graphicspath{{img}}

% fuentes de conjuntos numéricos
\usepackage{amsfonts}

% math
\usepackage{amsmath, amssymb}

% gráficos y plots
\usepackage{tikz}
%\usepackage{pgfplots}
%\pgfplotsset{width=10cm, compat=1.9}
\usetikzlibrary{babel}

\setlength{\jot}{8pt}
\setlength{\parindent}{0cm}

% espacio entre párrafos
\usepackage{parskip}

% cancelar términos
\usepackage{cancel}

% links
%\usepackage[colorlinks=true,
%    urlcolor=blue]{hyperref}

% shapes
%\usetikzlibrary{shapes.geometric}

% incluir pdfs
%\usepackage{pdfpages}

\begin{document}

\thispagestyle{empty}

\begin{center}
  \vspace*{.5cm}
  \includegraphics[scale=.6]{~/Pictures/udemm-logo.png}\\
  \vspace{.2cm}
  \Large
  \textbf{Facultad de Ingeniería}\\
  \textbf{Ingeniería en Sistemas}\\
  \vspace{2cm}

  \Huge
  Arquitectura de computadoras\\
  Trabajo Práctico N\(^\circ\) 3\\
  \vfill

  \raggedright
  \Large
  Docente:
  \begin{itemize}
    \item[] Lic. Claudio M. Biffani \\
  \end{itemize}
  Alumno:
  \begin{itemize}
    \item[] Daniel Ise
  \end{itemize}
  Legajo:
  \begin{itemize}
    \item[] 28547
  \end{itemize}
  Fecha:
  \begin{itemize}
    \item[] Octubre, 2024
  \end{itemize}
\end{center}

\pagebreak

\tableofcontents

\pagebreak

\section{Consignas}

\begin{enumerate}
  \item Según lo visto en clase, listar y describir los distintos tipos de
        memoria, con un ejemplo de apliación práctica.
  \item Desarrollar los conceptos de SWAP, paginación y segmentación de
        memoria.
\end{enumerate}

\section{Respuestas}

\subsection{Tipos de memoria}

\subsubsection{Memoria estática de acceso aleatorio}

La SRAM (Static Random-Access Memory)
es un tipo de memoria de acceso aleatorio (RAM),
volátil
(es decir, que pierde su información en cuanto se corta el suministro eléctrico).

En contraste con la memoria dinámica de acceso aleatorio,
no requiere de refresco periódico,
resultando en una velocidad mayor y un menor tiempo de respuesta.

Como contrapartida, su costo y dificultad de fabricación es
generalmente superior al de la memoria dinámica de acceso aleatorio.

Esta características la convierten en la memoria ideal para ser empleada
como caché y registro interno de una Unidad Central de Procesamiento (CPU),
empleada como almacenamiento temporal y colaborando con un incremento en
la velocidad de procesamiento.

\begin{figure}[h]
  \centering
  \includegraphics[width=.8\textwidth]{TP-03-unidad-iv-ise/sram.jpg}
  \caption{Memoria estática de acceso aleatorio de 8 bits}
\end{figure}

\subsubsection{Memoria dinámica de acceso aleatorio}

Conocida por su acrónimo DRAM (Dynamic Random-Access Memory),
es un tipo de memoria de acceso aleatorio,
basada por lo general en capacitores y transistores.
El capacitor puede estar cargado o descargado,
por lo que puede representar los dos estados del bit: 0 y 1.
Sin embargo, como la carga eléctrica del capacitor tiende a
decaer, requiere de ciclos periódicos de refresco.

Es una memoria volátil, que requiere del suministro eléctrico para
retener la información.
Se utiliza normalmente como memoria principal en computadoras y
placas de video.

\begin{figure}[h]
  \centering
  \includegraphics[width=.8\textwidth]{TP-03-unidad-iv-ise/dram.jpg}
  \caption{Memoria dinámica de acceso aleatorio de 128 KiB}
\end{figure}

\subsubsection{Memoria asociativa}

También conocida como memoria de contenido direccionable
(Content-Addressable Memory, CAM),
es un tipo de memoria de alta velocidad,
enfocada en aplicaciones que requieren de grandes velocidades de búsqueda.

En lugar de utilizar una dirección en memoria, las búsquedas
se realizan directamente contra una tabla de datos almacenados,
retornando las direcciones en que encuentre la información.

Podríá pensarse como la implementación en \textit{hardware}
de lo que se conoce como un arreglo o \textit{array} de datos en el
desarrollo de \textit{software}.

Típicamente se emplea en dispositivos de red.

\subsubsection{Memoria de solo lectura}

También conocida por sus siglas ROM
(Read-only memory),
es un tipo de memoria no volátil,
utilizada tanto en computadoras como en diferentes tipos de dispositivos
electrónicos.

Su función consiste en almacenar el \textit{firmware},
es decir, del \textit{software} de bajo nivel encargado del control del
\textit{hardware}.

Se utiliza en muchos dipositivos electrónicos,
almacenando desde el comportamiento de diferentes programas de lavado de un
lavarropa, hasta las instrucciones de arranque de una computadora,
como es el caso de la BIOS.

\begin{figure}[h]
  \centering
  \includegraphics[width=.8\textwidth]{TP-03-unidad-iv-ise/rom.jpg}
  \caption{Memoria de solo lectura, perteneciente a un cartucho de un videojuego}
\end{figure}

\subsubsection{Memoria flash}

Es una memoria electrónica de almacenamiento no volátil,
de fácil borrado y reprogramado.

Existen principalmente dos tipos:
\begin{itemize}
  \item Memorias NOR
  \item Memorias NAND
\end{itemize}

Sus nombres se derivan de las puertas lógicas NOR y NAND.
Al permitir la lectura y escritura en múltiples posiciones de manera simultánea,
permiten altas velocidades de funcionamiento.

Se utilizan principalmente para memorias de almacenamiento interno,
pendrives, discos de estado sólido y el las BIOS y UEFI actuales.

\begin{figure}[h]
  \centering
  \includegraphics[width=.8\textwidth]{TP-03-unidad-iv-ise/flash.jpg}
  \caption{Dispositivo de almacenamiento con interfaz USB basado en tecnologia flash}
\end{figure}

\subsection{SWAP}

El término SWAP refiere a la práctica de emplear espacio de disco cuando 
la memoria principal del sistema se encuentra llena.
Es un término común a los sistemas Unix y basados en Unix.

El sistema operativo Unix gestiona la memoria de manera tal que,
frente a un pico de demanda, 
parte de los datos empleados con menor frecuencia se trasladen al almacenamiento
de la computadora, liberando así memoria Ram para las aplicaciones que lo 
requieran.

Como contrapartida, 
los datos almacenados en disco serán accedidos a una velocidad considerablemente
inferior una vez que sean requeridos.

\subsection{Paginación de memoria}

La paginación de memoria es una modalidad de gestión de memoria principal
a la que recurre el sistema operativo para incrementar la eficiencia en el 
uso de la misma.

Consiste básicamente en la división de la memoria principal (RAM) en 
\textit{páginas} de tamaño fijo,
por lo general,
4 u 8 KB.
El sistema operativo mantiene una tabla de páginas,
que mapea el uso de la memoria física,
reduciendo así la fragmentación y mejorando el uso.
A medida que las aplicaciones requieren una mayor cantidad de memoria,
el sistema puede mover algunas de las páginas inactivas a la partición SWAP.

Entre las ventajas de esta modalidad encontramos una mayor eficiencia en 
la utilización de memoria, el aislamiento de los procesos en páginas, que 
mejora la seguridad, y una menor fragmentación.

Entre sus desventajas contamos con una mayor sobrecarga del sistema,
que debe gestionar tablas de páginas y traducir direcciones.

\subsection{Segmentación de memoria}

Por su parte, la segmentación de memoria es otra modalidad de gestión de 
la memoria principal de un sistema informático. Esta consiste en la 
división del espacio físico de la memoria en segmentos de tamaño dinámico,
que representan diferentes tipos de estructuras de datos.

Ello repercute en un uso eficiente de la memoria, 
al permitir fragmentos de diferente tamaño, aunque suele incurrir en el 
costo de una mayor fragmentación.

Los sistemas operativos modernos,
entre los que contamos Linux y Windows,
suelen priorizar la paginación por encima de la segmentación,
por su simplicidad, eficiencia y velocidad de acceso,
aunque ello bo excluye que determinados usos específicos recurren a la 
segmentación.

\end{document}
