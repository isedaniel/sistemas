\documentclass[12pt]{article}
\usepackage[a4paper, margin=2.54cm]{geometry}
\usepackage[spanish]{babel}

% imágenes
\usepackage{graphicx}
\graphicspath{{img}}

\usepackage{svg}

% fuentes de conjuntos numéricos
\usepackage{amsfonts}

% math
\usepackage{amsmath, amssymb}

% gráficos y plots
\usepackage{tikz}
%\usepackage{pgfplots}
%\pgfplotsset{width=10cm, compat=1.9}
\usetikzlibrary{babel}

\setlength{\jot}{8pt}
\setlength{\parindent}{0cm}

% espacio entre párrafos
\usepackage[skip=10pt plus1pt, indent=12pt]{parskip}

% cancelar términos
\usepackage{cancel}

% links
%\usepackage[colorlinks=true,
%    urlcolor=blue]{hyperref}

% shapes
%\usetikzlibrary{shapes.geometric}

% incluir pdfs
%\usepackage{pdfpages}

% bibliografía
\usepackage[style=apa]{biblatex}
\addbibresource{bib.bib}
\usepackage{csquotes}

\begin{document}

\thispagestyle{empty}

\begin{center}
	\vspace*{.5cm}
	\includegraphics[scale=.6]{~/Pictures/udemm-logo.png}\\
	\vspace{.2cm}
	\Large
	\textbf{Facultad de Ingeniería}\\
	\textbf{Ingeniería en Sistemas}\\
	\vspace{2cm}

	\Huge
	Arquitectura de Computadores\\
	Trabajo Práctico N\(^\circ\) 5\\
	\vfill

	\raggedright
	\Large
	Docentes:
	\begin{itemize}
		\item[] Lic. Claudio Biffani \\
	\end{itemize}
	Alumno:
	\begin{itemize}
		\item[] Daniel Ise
	\end{itemize}
	Legajo:
	\begin{itemize}
		\item[] 28547
	\end{itemize}
	Fecha:
	\begin{itemize}
		\item[] Noviembre, 2024
	\end{itemize}
\end{center}

\pagebreak

\section{Consignas}

\begin{enumerate}
	\item Desarrollar concepto de Bus USB:
	\begin{enumerate}
		\item Breve historia
		\item Características técnicas
		\item Evolución a través del tiempo
		\item Ejemplo de multiplexación de buses
	\end{enumerate}
\end{enumerate}

\pagebreak

\section{Respuestas}

El Universal Serial Bus,
traducido como Bus de Serie Universal y más conocido como USB, 
es una especificación propuesta inicialmente a mediado de los noventa, 
a instancias del USB Implementers Forum,
enfocada principalmente en la conexión de diversos dispositivos de entrada/salida,
proveyendo de forma simultánea alimentación e intercambio de datos \parencite{wikienusb}.

La USB Implementers Forum, Inc. (en adelante, USB-IF) es una organización sin fines de lucro,
que agrupa a las corporaciones que propusieron inicialmente el estándar USB,
entre las que se cuentan a actores del tamaño de Intel, Microsoft, HP y Apple \parencite{usbifabout}. 
Su principal objetivo consiste en la promoción y el desarrollo del estándar USB.

Pensado inicialmente como una propuesta para estandarizar la conexión de diversos dispositivos de entrada/salida,
entre los que se incluyen impresoras, tabletas de digitalización, teclados, joystick y otros,
luego del lanzamiento de su segunda iteración en 1998,
la versión 1.1,
su adopación tuvo tal velocidad que desplazó a todo el resto de conectores de la industria.
Así, conectores serie, paralelos y PS/2, 
entre otros, 
se convirtieron en conectores de nicho,
dejando al USB como el nuevo estándar defacto \parencite{wikiusb}.

Un cable USB consiste básicamente en dos canales: 
uno para alimentación y otro para el intercambio de datos.
Aunque las sucesivas versiones fueron incrementando la cantidad de canales y,
con ello,
la capacidad de los cables de este estándar de alimentar y comunicar dispositivos,
la lógica subyacente sigue siendo la misma.

Desde el punto de vista del sistema,
sigue un modelo de anfitrión-dispositivo,
donde un equipo funciona como anfitrión -inicialmente, 
una computadora, 
aunque hoy múltiples dispositivos cumplen esa función-
y un dispositivo de entrada-salida se ubica en la otra posición \parencite{wikienusb}.

En lo que respecta a los conectores,
han habido diferentes iteraciones con los años,
entre las que se destacan:

El USB 1.0,
iteración inicial lanzada en 1996,
prácticamente sin implementación en la industria.
Su máxima velocidad teórica son 1.5Mbits/s.
El conector inicialmente lanzado fue el,
a día de hoy quizá más conocido,
USB Tipo A.

\begin{figure}[h]
	\vspace{20pt}
	\centering
	\includesvg[width=.2\textwidth]{./img/usb-tipo-a.svg}
	\caption{Conector USB Tipo A}
	\vspace{15pt}
\end{figure}

En 1998 se lanza el USB 1.1,
segunda iteración del estándar,
que lleva la máxima velocidad de transmisión de datos a 12Mbits/s.
Hoy ambos renombrados a \textit{BasicSpeed},
como nombre comercial impulsado por la USB-IF.

En el 2000 llega el USB 2.0,
con conectores tanto tipo A como tipo B.

\begin{figure}[h]
	\vspace{20pt}
	\centering
	\includesvg[width=.2\textwidth]{./img/usb-tipo-b.svg}
	\caption{Conector USB Tipo B}
	\vspace{15pt}
\end{figure}

Esta versión lleva la transmisión de datos a los 480 Mbits/s.
Renombrada hoy con el nombre comercial de \textit{High-Speed},
con el USB 2.0 se lanzan también los conectores \textit{miniusb},
como el Mini-B.

\begin{figure}[h]
	\vspace{20pt}
	\centering
	\includesvg[width=.2\textwidth]{./img/mini-usb-b.svg}
	\caption{Conector USB Mini B}
	\vspace{15pt}
\end{figure}

Como ocurriría con el USB 1.0 y 1.1,
estos conectores no recibirían mucha adopción y serían rápidamente reemplazados
por los más populares Micro USB.

\begin{figure}[h]
	\vspace{20pt}
	\centering
	\includesvg[width=.2\textwidth]{./img/micro-usb.svg}
	\caption{Conector Micro USB A}
	\vspace{15pt}
\end{figure}

En 2008 el USB-IF presentaba el estándar más utilizado a día de hoy:
el USB 3.0. Con él, las velocidades de transmisión alcanzarían los 5, 10 y 20Gbit/s,
en su primera, segunda y tercera generación, 
hoy renombradas Gen 1, Gen 2 y Gen 2x2.
Con el USB 3.0 llegaría también el conector que parece ser el futuro de todas las conexiones USB:
el USB C.

\begin{figure}[h]
	\vspace{20pt}
	\centering
	\includegraphics[width=.8\textwidth]{usb-c}
	\caption{Conector Micro USB A}
	\vspace{15pt}
\end{figure}

Entre sus características más destacadas, además del crecimiento del número de canales,
y con ellos de la provisión de energía y velocidad de transmisión,
hay que destacar un incremento en la calidad de vida de los usuarios con la reversibilidad del enchufe.

La interfaz USB4, aún de baja implementación a nivel de popularidad,
fue presentada en 2019,
con una segunda versión en 2022,
cuyas velocidades de transmisión alcanzan los 40 y 80 Gbit/s.
Por el momento,
el conector USB Tipo C parece ser el único que sería empleado en adelante,
dando un paso más en la estandarización de conexión de distintos tipos de dispositivos a los sistemas de computación.

\printbibliography[heading=bibnumbered]

\end{document}
