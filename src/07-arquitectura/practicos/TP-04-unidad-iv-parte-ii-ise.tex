\documentclass[12pt]{article}
\usepackage[a4paper, margin=2.54cm]{geometry}
\usepackage[spanish]{babel}

% imágenes
%\usepackage{graphicx}
%\graphicspath{{img}}

% fuentes de conjuntos numéricos
\usepackage{amsfonts}

% math
\usepackage{amsmath, amssymb}

% gráficos y plots
\usepackage{tikz}
%\usepackage{pgfplots}
%\pgfplotsset{width=10cm, compat=1.9}
\usetikzlibrary{babel}

%\setlength{\jot}{8pt}
%\setlength{\parindent}{0cm}

% espacio entre párrafos
\usepackage[skip=10pt plus1pt, indent=12pt]{parskip}

% cancelar términos
\usepackage{cancel}

% links
%\usepackage[colorlinks=true,
%    urlcolor=blue]{hyperref}

% shapes
%\usetikzlibrary{shapes.geometric}

% incluir pdfs
%\usepackage{pdfpages}

\usepackage[style=apa]{biblatex}
\addbibresource{TP-04-bib.bib}
\usepackage{csquotes}


\begin{document}

\thispagestyle{empty}

\begin{center}
  \vspace*{.5cm}
  \includegraphics[scale=.6]{~/Pictures/udemm-logo.png}\\
  \vspace{.2cm}
  \Large
  \textbf{Facultad de Ingeniería}\\
  \textbf{Ingeniería en Sistemas}\\
  \vspace{2cm}

  \Huge
  Arquitectura de computadoras\\
  Trabajo Práctico N\(^\circ\) 4\\
  \vfill

  \raggedright
  \Large
  Docente:
  \begin{itemize}
    \item[] Lic. Claudio M. Biffani \\
  \end{itemize}
  Alumno:
  \begin{itemize}
    \item[] Daniel Ise
  \end{itemize}
  Legajo:
  \begin{itemize}
    \item[] 28547
  \end{itemize}
  Fecha:
  \begin{itemize}
    \item[] Octubre, 2024
  \end{itemize}
\end{center}

\pagebreak

\tableofcontents

\pagebreak

\section{Consignas}

\begin{enumerate}
  \item Según el desarrollo del TP unidad IV ampliar el concepto de memoria virtual, su el funcionamiento y componentes necesarios para su aplicación, ejemplifique.
  \item Desarrollar el concepto de memoria externa.
\end{enumerate}

\section{Respuestas}

\subsection{Memoria virtual}

En el contexto de la computación, 
se denomina memoria virtual a la técnica de gestión de memoria,
de la cual se encarga el sistema operativo,
tanto para el software como para el usuario,
incrementando la cantidad de memoria disponible,
utilizando los recursos físicos \parencite{wiki_es_memoria_virtual}.

En términos concretos, 
el sistema operativo emplea una combinación de hardware y software,
mapeando la memoria utilizada por el programa,
que se denomina \textit{memoria virtual},
vinculándola a direcciones físicas de memoria \parencite{wiki_en_virtual_memory}.

La memoria virtual facilita el desarrollo de aplicaciones abstrayendo la 
fragmentación de la memoria física, delegando la gestión de la misma sobre 
el núcleo del sistema operativo \parencite{wiki_en_virtual_memory}.

\subsection{Memoria externa}

La memoria externa o secundaria consiste en un tipo de memoria no volátil,
esto es,
que no pierde la información con la desconexión de la energía,
cuya velocidad y precio son relativamente menores a la memoria RAM,
por lo que suponen el complemento ideal a la memoria de mayor velocidad 
para el almacenamiento de grandes cantidades de datos \parencite{wiki_en_secondary_storage}.

La implementación moderna típica de este tipo de almacenamiento son los 
discos rígidos y, recientemente, los discos de estado sólido \parencite{wiki_en_secondary_storage}.

\printbibliography[heading=bibnumbered]

\end{document}
