\documentclass{article}
\usepackage[margin=2.54cm]{geometry}
%\usepackage{graphicx}               % imágenes
%\graphicspath{{img}}
\usepackage{amsfonts}               % fuentes de conjuntos numéricos
\usepackage{amsmath, amssymb}       % símbolos
%\usepackage{tikz}                   % gráficos
%\usepackage{pgfplots}               % plots
%\pgfplotsset{width=10cm, compat=1.9}
\setlength{\jot}{8pt}
\setlength{\parindent}{0cm}
\usepackage{parskip}                % espacio entre párrafos
\usepackage{cancel}                 % cancelar términos
\usepackage[colorlinks=true, 
    urlcolor=blue]{hyperref}        % links
%\usetikzlibrary{shapes.geometric}   % shapes
%\usepackage{pdfpages}               % incluir pdfs

\title{Arquitectura de computadoras\\Segunda clase}
\author{Daniel Ise}
\date{13 de agosto de 2024}

\begin{document}

\maketitle

\section*{Cronograma.}

Actividad de clase. Se las hace en la clase. Tienen un cierre de entrega y un 
cierre de entrega definitivo antes de la clase siguiente. 

Van a ser simples de resolver.

Después hay prácticos cada 15 días. Esos tienen fecha definitiva antes del 
parcial.

Las actividades semanales y trabajos prácticos ponderan con el parcial.

\section*{Conceptos generales.}

\textbf{Dato.} Valor que representa una característica de un objeto del mundo
real.

\textbf{Información.} Conjunto de datos ordenados, procesados por un algoritmo,
es el sentido de los datos.

\textbf{Ejemplos.} Auto. Rojo $\rightarrow$ datos. 5 autos rojos en la 
concesionaria $\rightarrow$ información.

La información es más valiosa que los datos, porque los datos de por si no 
dicen nada.

\textbf{Cómo se genera la información.} Con un proceso. Tengo entrada de datos,
le hago un procesamiento y a la salida tengo información.

\textbf{Generación de computadoras.}

\textbf{Primera generación.} 1940-1956. 

\textbf{Tubo de vacío.} Utilizados como tecnología principal para circuitos y
tambores magnéticos para memoria. Tamaño: habitaciones enteras. Consumo de 
energía: alto. Mucho calor. Programación: te utilizaba lenguaje de máquina.
Velocidad: procesamiento lento, aunque superior al humano. Confiabilidad: fallas
frecuentes.

\textbf{Segunda generación. 1956 - 1963}

\textbf{Transistores.} Reemplazan al tubo de vacío. Permiten a los ordenadores
ser más chicos, rápidos y confiables. Menor consumo de energía y generación de
calor. Lenguaje de programación: aparece ensamblador y los primeros lenguajes de
alto nivel: FORTRAN y COBOL. Mayor velocidad de procesamiento. Crece el uso 
comercial, la aplicación militar y gubernamental. 

\textbf{Tercera generación. 1964 - 1971.}

\textbf{Circuito integrado.} Igual al transistor pero en un chip de silicio.
Mayor reducción de tamaño, con mayor potencia. Multiprogramación: capcidad de
ejecutar múltiples programas. Mayor confiabilidad. Menos fallas de hardware. 
Permite el crecimiento del desarrollo de software y sistemas operativos más 
sofisticados. Mejora la interacción con el usuario: teclado, monitor. 
IBM System/360.

\textbf{Cuarta generación. 1971 - Presente}

\textbf{Microprocesador.} Introducción del microprocesador, colocando CPU en 
un solo chip. Computadora personal: nace la PC, computadora accesible al 
público en general. VLSI/ULSI: integración a gran escala, chips más complejos
y potentes. Interfaz gráfica de usuario, facilitando uso de computadoras. 
Redes: crece internet, llevando al mundo a una mayor conexión. Movilidad: 
aumento de la portabilidad con laptop, tablet, smartphone. IBM PC.

\textbf{Pelis.} Jobs. Sillicon cowboys.

\textbf{Arquitectura de Von Neumann.} Modelo de diseño de computadoras, que 
describe un sistema en el cual una única estructura de almacenamiento (llamada
memoria) contiene tanto intrucciones de procesamiento (programas) como los 
datos.

Esta arquitectura es la base de la mayoría de computadoras modernas. Fue 
propuesta por John von Neumann en 1945. En lugar de cablear en hardware el 
programa, se hace a la computadora \textit{programable.}

Componentes de la arquitectura. 

\begin{enumerate}
    \item Unidad central de procesamiento (CPU).
     \begin{itemize}
        \item Unidad aritmético lógica
        \item Unidad de control
        \begin{itemize}
            \item Registro de instrucción
            \item Contador de programa
        \end{itemize}
     \end{itemize}
    \item Memoria Ram. 
    \item Dispositivos de entrada y salida. 
    \item Buses de conexión.
\end{enumerate}

\textbf{Programa almacenado.} Un programa son instrucciones secuenciales para el
tratamiento de los datos. Antes los programas estaban \textit{hardcodeados}, es
decir, eran diseñadas de fábrica para ejecutar un único programa. Estaba
pre-definida en el hardware. Para cambiar un programa había que reconfigurar
físicamente a la computadora. 

\textbf{IAS Machine.} Elementos clave:
\begin{enumerate}
    \item Almacenamiento en memoria
    \item Unidad de control
    \item Secuencia de ejecución
    \item Programa modificable
\end{enumerate}

\textbf{Ciclo de instrucción.}
\begin{enumerate}
    \item Buscar instrucción en la memoria principal
    \item Decodificar instrucción: buscar y obtener datos
    \item Ejecución de la instrucción: almacenamiento de datos.
    \item Repetición del ciclo.
\end{enumerate}

\end{document}