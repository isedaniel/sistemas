\documentclass[12pt]{article}
\usepackage[a4paper, margin=2.54cm]{geometry}

% español
\usepackage[spanish]{babel}

% imágenes
%\usepackage{graphicx}
%\graphicspath{{img}}

% fuentes de conjuntos numéricos
\usepackage{amsfonts}

% símbolos
\usepackage{amsmath, amssymb}

% gráficos
%\usepackage{tikz}

% plots
%\usepackage{pgfplots}
%\pgfplotsset{width=10cm, compat=1.9}

% averiguar
\setlength{\jot}{8pt}
\setlength{\parindent}{0cm}

% espacio entre párrafos
\usepackage[skip=10pt plus1pt]{parskip}

% cancelar términos
\usepackage{cancel}

% links
%\usepackage[colorlinks=true, 
%    urlcolor=blue]{hyperref}

% shapes
%\usetikzlibrary{shapes.geometric}

% incluir pdfs
%\usepackage{pdfpages}

\title{Arquitectura de computadoras\\Apunte de clase}
\author{Daniel Ise}
\date{8 de octubre de 2024}

\begin{document}

\maketitle

\tableofcontents

\section{Clasificación de memorias}

\textbf{Memoria de acceso secuencial.}
Se accede de adelante para atrás,
como en una cinta.

\textbf{Memoria de acceso aleatorio.}
Se alojan los programas para su ejecución.

\textbf{Según operaciones que aceptan.}
\begin{itemize}
    \item Solo lectura
    \item Lectura / Escritura
\end{itemize}

\textbf{Según persistencia.}
\begin{itemize}
    \item Persistente, como la BIOS
    \item Volátil, como la memoria Ram
\end{itemize}

\section{Memorias de semiconductores}

\textbf{Características generales.}
Basadas en transistores. 
Alta velocidad de acceso.
Tamaño compacto.
En el TP especificar tipo de electrónica asociada.

\textbf{Tipos.}
\begin{itemize}
    \item SRAM, Static Random Access Memory
    \item DRAM, Dinamic Random Access Memory, la ram común
    \item Memoria asociativa, desarrollar en TP
    \item ROM, Read Only Memory
    \item FLASH, evolución de la Rom
\end{itemize}

\section{Memoria segmentada}

\textbf{Concepto de segmento.}
Gradualmente el tamaño de la memoria ha ido creciendo,
así como sus usos por parte de diferentes programas,
con lo que surge la segmentación de la memoria.
Consiste en dividir la memoria en segmentos de diverso tamaño.

\textbf{Fundamento de la memoria segmentada.}
Facilita la gestión de la memoria por parte del programador,
lo que incrementa su productividad.

\textbf{Paginación.}
Divide la memoria en tamaños iguales.

Desarrollar diferencias entre segmentación y paginación de la memoria.

\end{document}
