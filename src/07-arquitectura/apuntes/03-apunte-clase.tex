\documentclass{article}
\usepackage[a4paper, margin=2.54cm]{geometry}

% español
\usepackage[spanish]{babel}

% imágenes
%\usepackage{graphicx}
%\graphicspath{{img}}

% fuentes de conjuntos numéricos
\usepackage{amsfonts}

% símbolos
\usepackage{amsmath, amssymb}

% gráficos
%\usepackage{tikz}

% plots
%\usepackage{pgfplots}
%\pgfplotsset{width=10cm, compat=1.9}

% averiguar
\setlength{\jot}{8pt}
\setlength{\parindent}{0cm}

% espacio entre párrafos
\usepackage[skip=8pt plus1pt]{parskip}

% cancelar términos
\usepackage{cancel}

% links
%\usepackage[colorlinks=true, 
%    urlcolor=blue]{hyperref}

% shapes
%\usetikzlibrary{shapes.geometric}

% incluir pdfs
%\usepackage{pdfpages}

\title{Arquitectura de Computadoras\\Apunte de clase}
\author{Daniel Ise}
\date{3 de septiembre de 2024}

\begin{document}

\maketitle

\section*{Sistema BCD}

Se suma 6 porque cada 6 se repite la estructura binaria.

\section*{Representación de Punto fijo con negativos complementados a la base}

Representación numérica es fundamental en sistema de cómputo.
En Punto fijo se almacena un número fijo de dígitos decimales.
Se usa en sistemas de precisión fija, como aplicaciones financieras.
Ejemplo: sistema de 4 dígitos, número 12.34 se representa sin cambios.

\textbf{Ventajas.}
Simple, rápido, adecuado para números pequeños..

\textbf{COnstrucción parte decimal.}
Multiplicamos 0.34 por 2 repetidamente para convertir fraccionaria a binario.

\section*{Álgebra de Boole}

\textbf{Estructura algebraica.}
Conjunto no vacío con una o más \textbf{operaciones} definidas en él.

\textbf{Operandos.}
Las operaciones aceptan dos operandos: \(S = {0,1}\).

\textbf{Operaciones.}
En álgebra de Boole tenemos 2 operaciones:
\textbf{suma} y \textbf{producto}.
Los resultados de las operaciones son:

\begin{center}
    Suma en álgebra de Boole\\
    \hfill \\
    \begin{tabular}{ c c c }
        A & B & Resultado \\
        \hline
        0 & 0 & 0         \\
        1 & 0 & 1         \\
        0 & 1 & 1         \\
        1 & 1 & 1         \\
        \hline
    \end{tabular}
\end{center}

\begin{center}
    Producto en álgebra de Boole\\
    \hfill \\
    \begin{tabular}{ c c c }
        A & B & Resultado \\
        \hline
        0 & 0 & 0         \\
        1 & 0 & 0         \\
        0 & 1 & 0         \\
        1 & 1 & 1         \\
        \hline
    \end{tabular}
\end{center}

La \textbf{suma} es \textit{equivalente} al \textbf{Or} en lógica.
La \textbf{multiplicación} es \textit{equivalente} al \textbf{And} en lógica.

\end{document}
