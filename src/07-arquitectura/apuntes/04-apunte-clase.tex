\documentclass{article}
\usepackage[a4paper, margin=2.54cm]{geometry}

% español
\usepackage[spanish]{babel}

% imágenes
%\usepackage{graphicx}
%\graphicspath{{img}}

% fuentes de conjuntos numéricos
\usepackage{amsfonts}

% símbolos
\usepackage{amsmath, amssymb}

% gráficos
%\usepackage{tikz}

% plots
%\usepackage{pgfplots}
%\pgfplotsset{width=10cm, compat=1.9}

% averiguar
\setlength{\jot}{8pt}
\setlength{\parindent}{0cm}

% espacio entre párrafos
\usepackage[skip=8pt plus1pt]{parskip}

% cancelar términos
\usepackage{cancel}

% links
%\usepackage[colorlinks=true, 
%    urlcolor=blue]{hyperref}

% shapes
%\usetikzlibrary{shapes.geometric}

% incluir pdfs
%\usepackage{pdfpages}

\title{Arquitectura de computadoras\\Apunte de clase}
\author{Daniel Ise}
\date{10 de septiembre de 2024}

\begin{document}

\maketitle

\tableofcontents

\section{Circuitos}

\subsection{Defininicones previas.}

\textbf{Reloj (clock).} 
Mecanismo que emite una señal oscilando entre dos valores,
alto y bajo (esto es, 0 y 1), 
contando con tiempos de transición,
lo que permite generar \textbf{ciclos}. 
Los ciclos se miden en \textbf{hertz (Hz)}.

\textbf{Compuertas.}
Son circuitos integrados monolíticos,
que permiten implementar familias lógicas.

\section{Circuitos lógicos aritméticos con compuerta}

\subsection{Circuitos lógicos aritméticos con compuerta combinacionales}

\textbf{Definición.}
Dependen de solo de \textbf{entradas} 
para \textit{determinar} \textbf{salidas}.

\textbf{Ejemplos.} 
Sumador de 1-bit.
Multiplexor.

\subsection{Circuitos lógicos aritméticos con compuerta secuenciales}

\textbf{Definición.}
Los secuenciales dependen de entradas pasadas,
es decir,
almancenan \textbf{estado}.

\textbf{Ejemplos.}
Contadores.

\textbf{Flip-flop.}
Tipo de circuito secuencial biestable,
es decir, con dos estados estables (0 y 1),
puede cambiar entre estos dos estados,
dependiendo de condiciones de entrada.
Almancena \textbf{1 bit} de información.
Es la \textbf{unidad básica de almacenamiento} en
electrónica digital.
\textit{Mantiene} su estado hasta que recibe señal de reloj,
o señal de entrada específica.

\textbf{Implementación en Hardware.} 

\textbf{Set-Reset Flip-Flop.}

\textbf{text}

\end{document}

