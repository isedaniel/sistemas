\section*{Actividad 1}
\textbf{Explique la relación que existe entre el álgebra, la matemática discreta y la computación.}

El \textbf{álgebra} es una de las principales ramas de la matemática, abocada al estudio de estructuras abstractas que surgen de la combinación de elementos, siguiendo un conjunto de reglas. En un principio, la disciplina se abocó al estudio de los diferentes sistemas numéricos y sus elementos -los números. Es por ello que se la solía considerar como una generalización de la aritmética. Para lograr la abstracción y la generalización, el álgebra emplea letras, que se operan como variables, identificadores susceptibles de tomar diferentes valores. 

Por su parte, la \textbf{matemática discreta} se puede considerar como una subdisciplina del álgebra, enfocada en el sistema numérico de los números naturales ($\mathbb{N}$), tomando también otras áreas de la matemática, como la combinatoria, la probabilidad, la geometría y la teoría de grafos. 

Al enfocarse en los números naturales, la matemática discreta se relaciona directamente con las áreas de la \textbf{informática}, en la medida en que la información se almacena y se transmite de forma discreta -en bits, empleando el sistema binario-, así como se hacen operaciones de conteo, se emplean bases de datos relacionales y se analizan algoritmos, que pueden ser definidos como procesos que siguen un número finito de pasos. 

La matemática discreta proporciona la base matemática para el tratamiento de las estructuras de datos, los algoritmos, las bases de datos y los sistemas operativos. También, al igual que el álgebra, provee herramientas para formalizar con rigurosidad los conceptos en el campo de la computación.
