\section*{Actividad 1}
\textbf{Explique la relación que existe entre el álgebra, la matemática discreta y la computación.}

El \textbf{álgebra} es una de las principales ramas de la matemática, abocada al estudio de estructuras abstractas que surgen de la combinación de elementos, siguiendo un conjunto de reglas. En un principio, la disciplina se abocó al elemento de los diferentes sistemas numéricos: los números. Es por ello que se la solía considerar como una generalización de la aritmética. Para lograr la abstracción, el álgebra emplea letras, que se pueden interpretar como variables, números desconocidos susceptibles de tomar diferentes valores. 

Por su parte, la \textbf{matemática discreta} puede considerarse, en parte, un subconjunto del álgebra, abocada al tratamiento de problemas relativos al conjunto de los números naturales, así como toma problemas tradicionalmente vinculados a otras áreas de la matemática, como la combinatoria, la probabilidad, la geometría y la teoría de grafos. 

Por su foco en el  conjunto de los números naturales, la matemática discreta se relaciona directamente con las áreas de la \textbf{informática} y las comunicaciones, en la medida en que la información se almacena y se transmite de forma discreta -recurriendo al sistema binario-, así como se hacen operaciones de conteo, se emplean bases de datos relaciones y se analizan procesos algoritmos, que pueden ser pensados como procesos de un número finito de pasos. 

La matemática discreta proporciona la base matemática para el tratamiento de las estructuras de datos, los algoritmos, las bases de datos y los sistemas operativos. También, al igual que el álgebra, colabora con la capacidad de formalizar rigurosamente los conceptos en el campo de la computación.
