\section*{Actividad 3}
\textbf{Definición de número natural, operaciones y propiedades.}

En matemática, un \textbf{número natural} es cualquiera de los números empleados para contabilizar la cantidad de elementos en un conjunto. Se denotan con la letra $\mathbb{N}$.

Las operaciones matemáticas que se definen en el conjunto de los números naturales son la suma y la multiplicación.

La suma y multiplicación son operaciones conmutativas y asociativas, es decir, el orden en que se operan los números no altera el resultado final.

Además, la suma y la multiplicación cumplen con la propiedad de clausura, esto es, para todo números naturales $x$ e $y$, la suma y multiplicación de $x$ e $y$ siempre da un número natural.

Los números naturales están totalmente ordenados. Ello implica que, para todo número $x$ e $y$, si afirmo que $x \le y$, entonces existe un número $z$ que haría posible $x + z = y$. 

En el sistema de los números naturales existen también los algoritmos de la resta y la división, aunque estas operaciones no cumplen con la propiedad de clausura, es decir, una resta o división de números naturales puede ser indeterminada dentro del sistema.

El algoritmo de la división puede expresarse como, dados dos números naturales $x$ e $y$, con $y \neq 0$, existen dos números naturales, que denominaremos $q$ y $r$, cociente y resto respectivamente, de manera que $a = (b \cdot q) + r$.

