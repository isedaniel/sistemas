\section*{Actividad 2}
\textbf{Definición de número y sistema de numeración.}

Un \textbf{sistema de numeración} es un conjunto de símbolos y reglas que, en conjunto, permiten construir todos los \textbf{números} -que respresentan al elemento de ese sistema de numeración- de manera válida.

Un sistema de numeración se puede expresar siguiendo la expresión:

\begin{align*}
N = (S, R)
\end{align*}

Donde:

\begin{itemize}
 \item[] $N$ es el sistema de numeración considerado (que puede ser binario, decimal, hexadecimal, octal, etc.). 
 \item[] $S$ es el conjunto de símbolos que el sistema permite (por ejemplo, en el sistema hexadecimal se permiten los simbolos $S =\{0, 1, 2 \dots 9, A, B, C, D, E, F\}$. 
 \item[] $R$ es un conjunto de reglas que permiten determinar los números y operaciones válidas en el marco de un sistema determinado (por ejemplo, en el sistema decimal la construcción de un número es posicional).
\end{itemize}

Aunque las reglas sean diferentes para cada sistema, todos tienen una en común: la construcción de un número válido requiere emplear los símbolos pertenecientes al conjunto $S$. Todo otro símbolo queda descartado y constituiría un número inválido, siempre en el marco de un determinado sistema numérico. 

Existen dos grandes categorías de sistemas numéricos: los sistemas \textbf{posicionales} y los \textbf{no posicionales}. Su diferencia principal radica que el valor de los símbolos numéricos en un sistema posicional viene dado no solo por el símbolo en sí, sino por su posición en el número. En un sistema no posicional el valor de los símbolos suele seguir reglas más complejas. 

