\section*{Actividad 9}
\textbf{Que entiende por sistema de numeración binario, operaciones básicas (suma, resta, multiplicación, divisón). Ejemplificar.}

El \textbf{sistema de numeración binario} es un sistema numérico que utiliza únicamente dos símbolos: $S=\{0,1\}$. Constituye el sistema fundamental para la computación digital, ya que se basa en la ausencia o presencia de señales eléctricas para la representación de valores.

Es un sistema posicional de base 2, es decir, cada posición en un número binario representa una potencia de 2. Por ejemplo, el número $1010$ representa, en número decimales, la operación $0b0110 = (0 \cdot 2^{3} + 1 \cdot 2^{2} + 1 \cdot 2^{1} + 0 \cdot 2^{0})$. Por lo cual $0b0110 = 6$. 

La suma, resta, multiplicación y división de binarios sigue reglas muy similares a la del sistema decimal. Al tratarse de sistemas posicionales, veremos que la forma de operar en cada uno esencialmente no varía. Lo único que cambia es el \textit{acarreo} entre columnas. 

En un sistema de base 2, cuando se opera una \textbf{suma} en la primera columna, si esta da como resultado un número mayor a 1, se anota 0 y se acarrea 1 hacia la columna siguiente. Por ejemplo, si sumamos $0111 + 0010$:

\begin{align*}
0111 + 0010 &= 1001
\end{align*}

La \textbf{resta} en el sistema binario es en parte igual a la resta decimal: 

\begin{align*}
1 - 1 &= 0\\
1 - 0 &= 1\\
0 - 0 &= 0
\end{align*}

El caso especial se encuentra cuanto hay que operar una resta de tipo $0 - 1$, que no se puede hacer directamente, por lo que simplemente se toma la columna de la izquierda y se opera. Por ejemplo, en la resta $10 - 01$, $0 - 1$ no se puede hacer, pero tomando la columna de la izquierda tenemos $10 - 01$, en decimales $2 - 1$, por lo que ahora si es posible operar, siendo el resultado $1$.

La \textbf{multiplicación} binaria es igual a la multiplicación decimal:

\begin{align*}
1 \cdot 1 &= 1\\
1 \cdot 0 &= 0\\
0 \cdot 1 &= 0\\
0 \cdot 0 &= 0
\end{align*}

Todo número multiplicado por 0 da 0, al igual que en el sistema decimal. Una vez que operamos la multiplicación por cada una de las cifras, operamos una suma binaria, teniendo especial atención a cuidar el posicionamiento. Por ejemplo:

\begin{align*}
110 \cdot 10 &= (0000) + (1100)\\
(0000) + (1100) &= 1100
\end{align*}

Por último, la \textbf{división} binaria es igual a la división decimal. Se opera de izquierda a derecha, tomando tantas cifras del dividendo como tenga el divisor. Se escribe un número 1, se multiplica por el divisor y se resta a las cifras consideradas del dividendo. Por ejemplo:

\begin{align*}
\frac{1101}{0010} &= 0110 + 0001
\end{align*}

En este caso 0110 (6 en decimal) equivale al cociente y 0001 (1 en decimal) al resto.
