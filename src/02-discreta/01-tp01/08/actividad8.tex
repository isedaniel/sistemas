\section*{Actividad 8}
\textbf{Intervalos abiertos, semiabiertos y cerrados}

Un \textbf{intervalo} es un subconjunto de la recta real, $I \subset \mathbb{R}$, que es conexo, es decir, que \textit{no se puede expresar como unión disjunta de otros intervalos}. Un intervalo satisface que, dados dos números $a, b \in I$, siendo $a$ y $b$ sus \textit{extremos}, si $a < x < b$, entonces $x \in I$.

La clasificación de intervalos en abiertos, semiabiertos y cerrados depende de la inclusión o no de sus extremos. 

Decimos que un intervalo es \textbf{abierto} cuando sus extremos no están incluidos en el intervalo. Siendo su notación $(a, b)$, donde $a$ y $b$ son sus extremos, el intervalo abierto define al conjunto $(a, b)=\{x \in \mathbb{R} : a < x < b\}$.

Un intervalo se define como \textbf{cerrado} cuando denotan al conjunto de los números de la recta real entre ellos, incluidos ellos. De manera que la notación $[a, b]$ denota al conjunto $[a, b] = \{x \in \mathbb{R} : a \leq x \leq b\}$.

Por último, un intervalo \textbf{semiabierto} designa la pertenencia de uno de los extremos y la exclusión del otro. En la notación se utiliza al paréntesis para señalar al extremo que no se incluye, que puede ser el de la izquierda o el de la derecha. Si tenemos, por ejemplo, al intervalo $(a, b]$, definimos su conjunto como $(a, b] = \{x \in \mathbb{R} : a < x \leq b\}$. Es decir, $x$ puede ser cualquier número real entre $a$ y $b$, salvo $a$.
