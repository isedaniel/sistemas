\section*{Actividad 6}
\textbf{Conectores lógicos: pertinencia, existencia, implicación, doble implicación, mayor, menor, igualdad.}

La relación de \textbf{pertenencia} se denota mediante el símbolo $\in$ y, al enunciar $x \in A$ estamos diciendo que $x$ es un elemento de $A$.

En una proposición, se habla de \textbf{existencia} de un elemento dentro de un conjunto utilizando el símbolo del cuantificador existencial: $\exists$. De esta forma, la proposición $\exists x \in A$ puede leerse como existe al menos un elemento $x$ dentro del conjunto $A$.

Por su parte, la relación de \textbf{implicación}, señalada con el símbolo $p \rightarrow q$, se puede leer como "Si p, entonces q". Una relación de implicación, siendo esta verdadera, implica que $q$ es \textbf{condición necesaria}, más no suficiente, de $p$. 

La \textbf{doble implicación}, por otro lado, es un operador lógico binario que, relacionando los enunciados $p$ y $q$, de forma $p \leftrightarrow q$, puede leerse como "$p$ si y solo si $q$". Una bicondicional es verdadera cuando $p$ y $q$ comparten valores de verdad, es decir, uno es verdadero cuando el otro lo es, y si uno es falso, el otro también será falso. De esta forma $p \leftrightarrow q$ sería equivalente a $p \rightarrow q \wedge q \rightarrow p$. 

En cuanto a la relación de \textbf{menor} o \textbf{mayor}, en matemática denotan la posición que un número $x$ tiene en relación a otro número $y$. De esta forma, si decimos que $x < y$, estamos afirmando que $x$ se encuentra hacia la izquierda de $y$ en una recta de número reales. Por el contrario, al afirmar que $x > y$, estamos diciendo que $x$ se ubica hacia la derecha de $y$ en la recta de los números reales. La \textbf{igualdad} entre ambos implicaría que $x$ e $y$ hacen referencia al mismo número.
