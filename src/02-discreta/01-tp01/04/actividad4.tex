\section*{Actividad 4}
\textbf{Otro conjunto numérico, por ejemplo, el de los números enteros.}

Un \textbf{número entero} es un elemento del conjunto numérico que contiene a los números naturales ($\mathbb{N}$), sus opuestos -los números negativos- y el cero. Se denota con el símbolo $\mathbb{Z}$. Todos los números negativos son menores al cero que, a su vez, es menor a los números positivos. Simbólicamente, los números negativos se representan anteponiendo el signo menos delante del número, por ejemplo, $-5$. Si no se escribe signo, se asume que el número es positivo.

Los números enteros pueden representarse en la recta numérica. Los números negativos se encuentran hacia la izquierda del cero y los positivos hacia la derecha. 

Los números enteros pueden sumarse, restarse, multiplicarse y dividirse. Adicionalmente, se añaden algunas reglas para el tratamiento de los signos.

Con la incorporación de los números negativos, el sistema numérico de los enteros permite extender la utilidad de los números naturales en la contabilidad de determinados fenómenos, por ejemplo, la temperatura bajo cero o la altura bajo el nivel del mar.

La suma y multiplicación de números enteros posee la propiedad asociativa y conmutativa, al igual que su paralelo para el sistema de los números naturales. Pero el conjunto de los enteros incorpora una propiedad adicional: la del elemento opuesto. Para todo número entero $x$, existe un entero $-x$ que, sumado al primero, se obtiene $x - x = 0$.

La \textbf{regla de los signos} implica que la multiplicación y división de números enteros será positiva cuando el signo de los operandos sea igual y será negativa cuando el signo de los operandos sea diferente. 

La división de enteros, por su parte, no cuenta con las propiedades asociativa, conmutativa ni distributiva.
