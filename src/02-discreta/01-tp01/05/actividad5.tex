\section*{Actividad 5}
\textbf{Diferencia entre número irracional e irracional. Ejemplificar.}

Los \textbf{números racionales} son números que pueden expresarse como fracciones de números enteros, es decir, dados dos números enteros $x$ e $y$, un número racional $z$ se puede expresar como:

\begin{align*}
	z &= \frac{x}{y}
\end{align*}

El conjunto de los números racionales se denota con el símbolo $\mathbb{Q}$.

Por otra parte, los \textbf{números irracionales} no pueden ser expresados de forma exacta o periódica, por lo que no pueden ser expresados en forma de fracción. El nombre irracional hace referencia a la imposibilidad de representar este número como el cociente entre dos números enteros. Ejemplos de números irracionales son el número pi ($\pi$), el número e ($e$) o el número aúreo ($\varphi$).
