\section*{Actividad 7}
\textbf{Definición de ecuaciones, inecuaciones y campo de existencia}

Una \textbf{ecuación} es una igualdad matemática entre dos expresiones, que se denominan miembros. Los miembros se separan por el signo igual ($=$). Dentro de cada miembro aparecen tanto elementos conocidos como datos desconocidos, que se denominan incógnitas. 

Los valores conocidos pueden ser números, coeficientes, constantes, variables, e incluso objetos complejos, como funciones o vectores. Los elementos desconocidos de la ecuación, generalmente representados por letras, constituyen valores que se pretende hallar. 

Por ejemplo, dada la ecuación $3x - 1 = 9 - x$, se puede afirmar que el primer miembro es $3x - 1$, y el segundo $9 - x$. La parte desconocida de la ecucación está designada por la letra $x$, y la parte conocida está integrada por el coeficiente $3$ y las constantes $9$ y $1$. La igualdad se puede determinar como verdadera o falsa de acuerdo a valor que asuma la incógnita. 

Por otro lado, una \textbf{inecuación} es una desigualdad algebraica, en la que los miembros se relacionan por los signos $<,>,\leq, \geq$. Al igual que las ecuaciones, una inecuación puede ser verdadera o falsa de acuerdo a los valores que asuma la incógnita. El conjunto de valores que verifican la desigualdad se denominan soluciones.

Por último, se conoce como \textbf{dominio} de una función al conjunto de los posibles valores de entrada de una función. Por ejemplo, dada la función $f(x)=x^{2}$, el dominio de la función sería todos los números pertenecientes al conjunto de los números reales, o $D=\mathbb{R}$. En cambio, dada la función $g(x)=\sqrt{x}$, tendremos soluciones pertenecientes a los reales únicamente con valores de $x > 0$, por lo que el dominio de la función estaría dentro del conjunto de los reales positivos y el cero: $D=\mathbb{R}^{+}_{0}$.
