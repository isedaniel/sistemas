\section*{Actividad 10}
\textbf{Qué entiende por sistema de numeración octal, operaciones básicas (suma, resta, multiplicación, división). Ejemplificar.}

El \textbf{sistema de numeración octal} es un sistema numérico que utiliza ocho símbolos: $S=\{0,1,2,3,4,5,6,7\}$. Es empleado en algunas ocaciones en la computación digital para trabajar con \textit{bytes}, entendidos como conjuntos de 8 \textit{bits}, aunque tambien se recurre al sistema hexadecimal por la posibilidad de representar un byte con solo dos dígitos.

Es un sistema posicional de base 8, es decir, cada posición en un número octal representa una potencia de 8. Por ejemplo, el número $1010$ representa la operación $0o1010 = (1 \cdot 8^{3} + 0 \cdot 8^{2} + 1 \cdot 8^{1} + 0 \cdot 8^{0})$. Por lo cual $0o1010 = 512 + 8 = 520$. 

La suma, resta, multiplicación y división números octales es igual a las operaciones decimales, con la diferencia que supone el acarreo entre columnas en la base 8.

En una \textbf{suma} octal, si la operación en una columna da como resultado un número mayor a 7, se anota 0 y se acarrea 1 hacia la columna siguiente. Por ejemplo:

\begin{tabular}{cccc}
	  & 1 &   &   \\
	  &   & 7 & 7 \\
	+ &   & 1 & 0 \\
\hline
	  & 1 & 0 & 7 \\
\end{tabular}

La \textbf{resta} octal es igual a la resta decimal, con la salvedad que el acarreo entre columnas suma 8 en lugar de 10:

\begin{tabular}{cccc}
	  &   & 6 & 8 \\
	  &   & \cancel{7} & 2 \\
	- &   & 1 & 6 \\
\hline
	  &   & 5 & 4 \\
\end{tabular}

La \textbf{multiplicación} en el sistema octal es igual a la multiplicación decimal, con la salvedad que el acarreo se hace cuando un número supera 8:

\begin{tabular}{cccc}
	  & 1 & 1 &   \\
	  &   & 7 & 7 \\
	x &   &   & 2 \\
\hline
	  & 1 & 7 & 6 \\
\end{tabular}

En este caso, tenemos 77 octal, que es 63 decimal, multiplicado por 2 (igual en octal y decimal). El resultado es 176 siguiendo el acarreo octal, que sería 126 en decimal.

Finalmente, la \textbf{división} octal es igual a la división decimal. Por ejemplo:

\begin{align*}
	\frac{74}{6} &= 12
\end{align*}

En este caso 74 (60 en decimal) dividido por 6 (igual en decimal) da 12 (10 en decimal).
