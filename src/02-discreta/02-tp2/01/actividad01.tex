\section*{Actividad 1}

\textbf{Álgebra de conjuntos, definición de conjunto. }

En el campo de la matemática, un conjunto se puede definir como 
una colección de elementos. Se puede decir de un elemento (o miembro) 
que \textit{pertenece} a un conjunto si está incluído en él. 

Un miembro puede ser cualquier tipo de objeto: números, personas, letras, etc.

Suelen definirse a partir de la enunciación de una propiedad compartida
por sus elementos. Por ejemplo, el conjunto de los números naturales 
($\mathbb{N}$), o el conjunto de los números pares positivos $S = \{2n \forall n \in \mathbb{N}\}$ o el conjunto de los colores primarios (rojo, amarillo y azul).

En cuanto a su notación, suelen denominarse con letras mayúsculas, utilizando
llaves para definir o describir a sus elementos.

Un conjunto puede ser finito o infinito. Un conjunto finito puede ser definido
de manera extensiva.

\begin{align*}
	C &= \{ rojo, amarillo, azul\}\\
\end{align*}

Un conjunto infinito, en cambio, usa la definición intensiva, que especifica
una propiedad compartida por todos sus elementos. Este tipo de definición
