\documentclass[11pt]{article}
\usepackage[a4paper, margin=2.54cm]{geometry}

% español
\usepackage[spanish]{babel}

% imágenes
%\usepackage{graphicx}
%\graphicspath{{img}}

% fuentes de conjuntos numéricos
\usepackage{amsfonts}

% símbolos
\usepackage{amsmath, amssymb}

% gráficos
%\usepackage{tikz}

% plots
%\usepackage{pgfplots}
%\pgfplotsset{width=10cm, compat=1.9}

% shapes
%\usetikzlibrary{shapes.geometric}

% separación fórmulas en align
%\setlength{\jot}{8pt}

% espacio entre párrafos
%\usepackage[skip=10pt plus1pt, indent=12pt]{parskip}

% cancelar términos
\usepackage{cancel}

% links
%\usepackage[colorlinks=true, 
%    urlcolor=blue]{hyperref}

% incluir pdfs
%\usepackage{pdfpages}

% tipografía cheta
\usepackage{tgadventor}
\renewcommand*\familydefault{\sfdefault} %% Only if the base font of the document is to be sans serif
\usepackage[T1]{fontenc}

% interlineado: 1.0 simple, 1.3 (1.5), 1.6 (doble)
\linespread{1.3}

\title{Física II\\Apuntes Khan Academy}
\author{Daniel Ise}
\date{Primer Cuatrimestre, 2025}

\begin{document}

\maketitle

\tableofcontents

\pagebreak

\section{Carga eléctrica}

\subsection{Ley de Coulomb}

La fuerza ejercida entre cargas \(Q_1\) y \(Q_2\).

\vspace{1cm}
\begin{equation}
    F = k\frac{|Q_1Q_2|}{r^{2}}
\end{equation}
\vspace{1cm}

\begin{enumerate}
    \item Siendo F la fuerza entre las cargas (por tercera ley), medida en newton (N)
    \item \(k\) la \textbf{Constante de Coulomb} o \textbf{Constante electrostática},
    con valor aproximado 
    \begin{equation*}
        k_0\approx9\times10^{9} \frac{N\cdot m^{2}}{C^{2}}
    \end{equation*}
    \item \(Q_{1}\) y \(Q_{2}\) las cargas, medidas en coulomb (C)
    \item y \(r\) la distancia entre las mismas, medida en metros (m)
\end{enumerate}

La constante de Coulomb puede expresarse también como:

\vspace{1cm}
\begin{equation*}
    k_0 = \frac{1}{4\pi\epsilon_0}
\end{equation*}
\vspace{1cm}

Siendo \(\epsilon_0\) la \textbf{Constante de permitividad},
cuyo valor aproximado es: 
\begin{equation*}
    \epsilon_0 \approx 8.85 \times 10^{-12} \frac{C^{2}}{N\cdot m^{2}}
\end{equation*}.

\section{Campo eléctrico}

Es la fuerza potencial que una carga puede ejercer sobre una carga de prueba,
ubicada a un distancia \textit{r}.

\vspace{1cm}
\begin{equation}
    E = \frac{F}{Q_0} = k \frac{|Q|}{r^{2}}
\end{equation}
\vspace{1cm}

\begin{itemize}
    \item Siendo \(E\) el campo eléctrico, medido en \(\frac{N}{Q}\)
    \item \(F\) la fuerza que la carga \(Q\) puede potencialmente ejercer sobre la carga de prueba \(Q_0\)
    \item \(k\) la constante de Coulomb 
    \item \(r\) la distancia entre las cargas
\end{itemize}

\section{Flujo Eléctrico - Ley de Gauss}

Refiere al campo eléctrico que atraviesa una superficie de área \(A\).

\vspace{1cm}
\begin{equation*}
    \phi = E \cdot A \cdot \cos \theta
\end{equation*}
\vspace{1cm}

\begin{itemize}
    \item Siendo \(\phi\) el flujo eléctrico 
    \item \(E\) el campo eléctrico 
    \item \(A\) el área considerada
    \item \(\theta\) el ángulo entre el vector normal de la superficie y el vector del campo eléctrico
\end{itemize}

Cuando el vector normal de la superificie y el vector del campo eléctrico son paralelos,
\(\cos\theta = 1\), por lo cual la ecuación se reduce a:

\vspace{1cm}
\begin{equation}
    \phi = E\cdot A
\end{equation}
\vspace{1cm}

La ley de Gauss resulta de reemplazar \(E\) por \(k\frac{|Q|}{r^{2}}\) y,
a su vez,
\(k\) por \(\frac{1}{4\pi\epsilon_0}\).
Por otra parte,
el área considerada es la superficie de una esfera,
por lo cual \(A = 4\pi r^{2}\),
por lo cual llegamos a la expresión:

\vspace{1cm}
\begin{equation*}
    \phi = \frac{1}{4\pi\epsilon} \cdot \frac{|Q|}{r^{2}} \cdot 4\pi r^{2}
\end{equation*}
\vspace{1cm}

Simplificamos:

\vspace{1cm}
\begin{equation*}
    \phi = \frac{1}{\cancel{4\pi}\epsilon} \cdot \frac{|Q|}{\cancel{r^{2}}} \cdot \cancel{4\pi} \cancel{r^{2}}
\end{equation*}
\vspace{1cm}

Finalmente, llegamos a la expresión de la \textbf{Ley de Gauss}:

\vspace{1cm}
\begin{equation}
    \phi = \frac{|Q|}{\epsilon}
\end{equation}
\vspace{1cm}
\section{Segunda clase}

25 de marzo, 2025

\subsection{Repaso}
\section{Tercera clase, 25 de agosto}

\subsection{Control de la conexión}
\begin{itemize}
    \item Detalles protocolares orientados al control de la conexión 
    \item Fases:
    \begin{itemize}
        \item Establecimiento
        \item Transferencia
        \item Cierre
    \end{itemize}
    \item Característica principal: cada extremo identifica a la PDU 
    (unidad de datos del protocolo)
    \item Numeración se relaciona con tres funciones: 
    controlar orden, flujo y errores
\end{itemize}

\subsection{Técnica control de errores}
\begin{itemize}
    \item Detección 
    \item Retransmisión
\end{itemize}
\section{Cuarta clase}

10 de abril, 2025.

\subsection{Curva y la función demanda}

La demanda se conceptualiza desde el punto de vista del comprador.

Cuanto mas barato es un producto, \textit{ceteris paribus}, más se compra.
Es decir,
manteniendo el resto de factores constante (estacionalidad, etc.),
la cantidad demandanda es inversamente proporcional al precio.

La curva \textit{decreciente} de demanda relaciona cantidad demandada y precio.
Al reducir el precio,
aumenta la cantidad de demandada. 
A cada precio \(P_A\) corresponde una cantidad \(Q_A\),
que los demandantes están dispuestos a adquirir.
La curva de la demanda se va a confeccionar de acuerdo a la tabla antedicha,
y va a mostrar las cantidades de algún determinado bien que serán demandadas durante un período,
por una población específica, a cada uno de los posibles precios.

La función demanda se va a expresar de acuerdo al precio del bien demandado \(A\),
a la renta, a los gustos de los consumidores y a los precios relativos de los bienes sustitutos.

\begin{equation*}
    Q_A = D(P_A, Y, P_B, G)
\end{equation*}

Donde \(Q_A\) es la cantidad demandada del bien A en un período concreto,
\(P_A\) el precio del bien A,
\(Y\) el ingreso de los consumidores,
\(P_B\) los precios del resto de bienes en el mismo período
y \(G\) los gustos de los consumidores.

\begin{center}
    \begin{tikzpicture}
        \begin{axis}[
            xlabel=Cantidad,
            ylabel=Precio,
            axis lines=left,
            xmin=0, xmax=10,
            ymin=0, ymax=10,
            ]
            \addplot[domain=0:8, samples=100] {8 - x};
        \end{axis}
    \end{tikzpicture}
\end{center}

La \textbf{función de demanda} es la relación entre la cantidad demandada de un bien y su precio,
manteniendose constantes el resto de factores: renta, gustos.

\subsection{Alteración de los otros factores}

¿Qué pasa si cambian esos factores?
Ya no estamos en una situación ceteris paribus.

¿Qué pasa si cambia uno de los factores permaneciendo constate el precio?

Una alteración de cualquier factor difrente del precio del bien desplazará toda la curva hacia la derecha o la izquierda,
y es lo que llama cambios en la demanda.

\begin{equation*}
    Y = -\frac{4}{3}x + 8
\end{equation*}

Hacemos gráfico de esta función:

\begin{center}
    \begin{tikzpicture}
        \begin{axis}[
            xlabel=Cantidad,
            ylabel=Precio,
            axis lines=left,
            xmin=0, xmax=10,
            ymin=0, ymax=10,
            ]
            \addplot[domain=0:8, samples=100] {8 - 4/3*x};
        \end{axis}
    \end{tikzpicture}
\end{center}

\subsection{Oferta}

Desde el punto de vista del vendedor.
De la misma manera que en el caso de la demanda,
tabién existe un conjunto de factores para determinar la oferta del vendedor.
Estos factores son: la tecnolkogía, los precios de factores rpoductivos (tierra, capital, trabajo)
y el precio del bien que se desea ofrecer.

\subsection{Tabla de la oferta}

Tambien bajo la condición e ceteris paribus, la tabla de la oferta va a relacionar
el precio de un bien y las cantidades

Para elevar la producción en una unidad será cada vez mayor.

Tabla:
\begin{equation*}
    P_A: 2 4 6 8
    O_A: 0 2 4 6
\end{equation*}

\subsection{Curva y función oferta}

Muestra las cantidades del bien en un periodo especifico,
para diversos precios de mercado.
En este caso la curva tiene pendiente positiva,
debido a que,
al aumentar el precio de un bien,
manteniendose constantes el resto de factores,
la cantidad ofrecida de dicho bien crecerá.

También la cantidad ofrecida dependerá de otras variables,
siendo la función oferta la siguiente:

\begin{equation*}
    Q_A = O(P_A, P_B, r, K)
\end{equation*}

Donde \(Q_A\) es la cantidad ofrecida del bien A para un período determinado,
\(P_A\) el precio del bien \(A\),
\(P_B\) el precio del resto de bienes,
\(r\) el precio de los factores de producción,
y \(K\) el estado de la tecnología.

En este análisis se considera también la condición \textit{ceteris paribus},
es decir, que todas las variables permanecen constantes,
excepto por la cantidad ofrecida \(Q_A\) y el precio \(P_A\).

La función de oferta es la relación entre la cantidad ofrecida de un bien y su precio,
considerando todos los demás factores constantes.

\begin{center}
    \begin{tikzpicture}
        \begin{axis}[
            xlabel=Cantidad,
            ylabel=Precio,
            axis lines=left,
            xmin=0, xmax=10,
            ymin=0, ymax=10,
            ]
            \addplot[domain=0:8, samples=100] {2 + x};
        \end{axis}
    \end{tikzpicture}
\end{center}

\subsection{Equilibrio de mercado}

Cuando se ponen en contacto el consumidor y el productor con sus respectivos planes de consumo y producción
\section{Clase 16 de mayo}

Después de la parte estática viene la parte magnética.
La corriente produce un campo magnético.
Por un campo eléctrico tengo un campo magnético en torno a la corriente.

\subsection{Transformador}

Tengo una bobina que hace pasar una tensión.
De frente tengo una bobina más chica,
que pesca parte de la tensión.
Así hace bajar el voltaje.
Pero las bobinas no se toca.
Ese traslado es por el campo magnético.

La relación entre espiras y voltaje es:

\begin{equation*}
    \frac{v_1}{v_2} = \frac{n_1}{n_2}
\end{equation*}

Donde \(v\) es voltaje y \(n\) número de espiras.

\subsection{Campo magnético}

Tengo un campo.
Tiro una carga en coulombs.
Campo magnético se mide en tesla (T).

\begin{equation*}
    F = qvB\sen\theta
\end{equation*}

Donde \(F\) es fuerza,
\(q\) es carga,
\(v\) es velocidad,
\(B\) es campo magnético
y \(\theta\) es ángulo con respecto al vector del campo.

\subsection{Campo magnético}

\begin{equation*}
    \Delta B =  \frac{\mu_0I\Delta L \sen\theta}{4\pi r^{2}}
\end{equation*}

Donde \(\mu_0\) es permeabilidad del espacio libre,
que equivale a \(4\pi\times10^{-7}T\frac{m^{2}}{A}\);


\section{Ejercicios 23 de mayo}

\section{Clase 19 de mayo}

\subsection{Por mail}

Enviar trabajo, nombre y fecha de la exposición que vamos a hacer.
Elegimos un tema, ponemos un día y exponemos.
Dentro de las 3 semanas previas al parcial.
Tiene que ser antes del parcial.

Además,
hay que mandarle a la profe nuestros números de telefono,
para la creación de un grupo de Whatsapp.

\subsection{Proyecto final}

En cuanto a la siguiente iteración del trabajo final,
debemos:
\begin{itemize}
    \item Refinar el proyecto, corregir errores
    \item Introducir diagrama de interacción
    \item Diagrama de secuencia
    \item Diagrama de Comunicación
    \item Mejorar introducción, visión y misión de la organización y del proyecto 
\end{itemize}

Alcanza con poner un diagrama de secuencia de sistema para cada caso de uso.
Debajo de cada caso de uso,
ponemos un pedacito del diagrama de clase,
referido al caso de uso,
y el diagrama de secuencia del caso de uso.

Extraemos los mensajes del caso de uso,
nombramos los mensajes (que van y vienen, del actor a la interfaz y viceversa),
estos nombres son \textit{independientes} de la implementación,
es como un pseudocódigo pero no implica nada en relación a la implementación final.
Es una traducción del caso de uso,
de forma verbal,
al diagrama de secuencia de sistema.

\section{Reestructuración del software}

En este caso, Pressman reconoce dos tipos de reestructuración: 
la reestructuración de código y la reestructuración de datos.

En el primer caso,
la reestructuración del código puede o no modificar la \textbf{arquitectura}
del programa,
puesto que si esta resulta del mantenimiento en el tiempo de buenas prácticas
de ingeniería de software,
antes que un proceso de reingeniería propiamente dicho alcanzaría con la 
reestructuración del código de alguno módulos o subsistemas puntuales.
En cualquier caso,
el autor sugiere algunas técnicas para abordar la reestructuración,
entre las que se destacan:
\begin{enumerate}
    \item La \textbf{simplificación lógica} de Warnier,
    que implica el modelado siguiendo la lógica booleana y,
    mediante la aplicación de una serie de reglas,
    llegar a una reestructuración lógica,
    conforme a las reglas de la programación estructurada.
    \item Por otra parte, el \textbf{mapeo de módulos} y \textbf{recursos}
    puede permitir una representación del flujo del programa y de su arquitectura,
    resultando en sugerencias para lograr un mínimo acoplamiento de los primeros 
    y, por ende, una simplificación de esta última.
\end{enumerate}

Por su parte, la \textbf{reestructuración de datos} implica la evaluación 
de todos los enunciados que incluyan:
\begin{itemize}
    \item definiciones de datos 
    \item descripciones de entrada/salida 
    \item descripciones de interfaz
\end{itemize}

El objetivo es reconstruir el flujo de datos en la aplicación.
Una vez completado este diagnóstico inicial,
la reestructuración se abocará a:
\begin{itemize}
    \item \textbf{Estandarizar} el registro de datos,
    clarificando definiciones,
    buscando consistencia.
    \item \textbf{Racionalizar} el nombre de los datos,
    garantizando el seguimiento de la nomenclatura apropiada.
    \item \textbf{Modificar} estructuras de datos existentes,
    que puede implicar desde el cambio de las estructuras de datos en el código 
    hasta el paso de un tipo de base de datos a otra.
\end{itemize}
\section{Campo eléctrico neto en 2D}

Lo mismo,
atendiendo a que es radial.

Consideramos el efecto de cada carga por separado,
descomponiendo su efecto en ejes vertical y horizontal.

Usamos toda la trigonometría que sea necesaria.
\section{Clase 12 de junio}

\subsection{Incoterms}

Son reglas para la implementación de los términos comerciales
fijados por la cámara de comercio internacional. 
La palabra \textit{incoterms} es una contracción,
que significa International Commercial Terms
(Términos de comercio internacional).

Los incoterms se centran en la obligación de entrega desde el punto de vista del
\textit{vendedor} (el exportador).

Los incoterms regulan:
\begin{itemize}
      \item La distribución de documentos
      \item Las condiciones de entrega de la mercadería
      \item La distribución de costos de la operación
      \item La distribución de riesgos de la operación 
\end{itemize}

Pero \textbf{no} regulan:
\begin{itemize}
      \item La legislación aplicable
      \item Las formas de pago de la operación 
\end{itemize}

Así pues, 
los incoterm definen y reparten claramente las obligaciones, 
los gastos, 
los riesgos del transporte internacional 
y el seguro tanto en el exportador como el importador.

El incortem seleccionado entre el exportador y el importador
determinará quien pagará el costo de cada segmento del transporte,
quién será responsable de cargar y descargar la mercadería,
y quién lleva el riesgo de la pérdida en un momento dado durante el envío
(sea local o internacional).

Se clasifican en dos grupos fundamentales:
\begin{enumerate}
      \item Grupo de salida
      \item Grupo de llegada
\end{enumerate}

\textbf{Grupo de salida.}
Empiezan con la letra E, F, C.

\textbf{Grupo de llegada.}
Empiezan con la letra D.

A comienzos del siglo XX,
se estableció la necesidad de la reglamentación del contrato de compraventa 
internacional,
y fue así que se crearon en 1936,
habiendo varias revisiones o endosos posteriores.

\subsection{Ejemplos de los principales términos incoterms}

\textbf{Grupo de salida.}
\begin{itemize}
      \item EXW: Exwork el vendedor entrega el producto en la puerta de su fábrica. 
      De ahí en adelante el resto de gastos corren por cuenta del comprador.
      \item FAS: Free Alongside Ship (franco al costado del buque).
      La entega de la mercadería se realiza cuando es colocada por el vendedor 
      al costado del buque en el puerto de carga. Desde ahí, todos los costos 
      y riesgos quedan a cargo del comprador.
      \item FOB: Free On Board (franco abordo).
      El vendedor tiene la obligación de cargar la mercadería a bordo del buque 
      en el puerto de embarque y el comprador de encarga del resto 
      (Seleccionar el buque, contratarlo, pagar el flete marítimo,
      el seguro, los gastos de descarga, aduana y flete en destino).
      \item CFR: Cost and Freight (Costo y Flete).
      El vendedor paga los gastos de transporte y otros para que la mercadería 
      llegue al puerto convenido en destino. Incluye también el despacho 
      (aduana) de la mercadería de exportación. Pero el seguro del buque va a 
      cargo del comprador.
      \item CIF: Cost, Insurance and Freight (Costo, seguro y flete).
      Igual al anterior, pero el vendedor se hace cargo del seguro 
      internacional.
\end{itemize}

\textbf{Grupo de llegada.}
\begin{itemize}
      \item DAF: Delivered at frontier (entregado en frontera)
      El vendedor cumple su obligación de entregar la mercadería hasta un punto
      convenido de la frontera antes de rebasar la aduana.
      \item DDP: Delivered duty paid (entregado con pago de derechos)
      En este término el vendedor realiza la entrega de la mercadería al 
      comprador ya con el despacho terminado de importación en la puerta del 
      lugar convenido del país importador (el vendedor se encarga de todo).
\end{itemize}

En el transporte internacional,
los recibos de embarque de las mercaderías toman 
diferentes nombres de acuerdo al medio de transporte.
\begin{enumerate}
      \item Conocimiento de embarque o Bill of Lading (BL). Es el documento por 
      medio del cual se instrumenta el contrato de transporte de mercadería 
      por agua. No indica el precio de la mercadería, es simplemente un recibo,
      no es una factura. El BL es entregado por el transportador, por lo cual 
      acredita el contrato de transporte.
      \item Vía aérea, Air Waybill (AW). (Igual al primero)
      \item  Guías Aéreas (Air Waybill): 
      \item Carta de Porte: 
      se utiliza en transporte terrestre internacional. 
      Recibo de carga con la descripción de la mercadería.
\end{enumerate}

\end{document}