\section{Placa infinita cargada}

La densidad de carga de la placa viene dada por la expresión:

\vspace{3cm}
\begin{equation*}
    \eta = \frac{Q}{A}
\end{equation*}
\vspace{3cm}

Donde \(Q\) es carga,
\(A\) es área.

Si ponemos una carga a una distancia \(h\) de la placa,
las fuerzas que llegan de costado se cancelan mutuamente,
por lo que podemos concentrarnos únicamente en la componente vertical.