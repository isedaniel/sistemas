\section{Potencia eléctrica}

Cuando la electricidad pasa por una resistencia,
podemos notar que la temperatura aumenta.
¿Por qué sucede esto?
¿Cómo se puede calcular?

Cuando una corriente fluye a través de una resistencia,
hay un voltaje,
puesto el que el potencial eléctrico \(\Delta V\) cambia,
entre el punto inicial y el final de la resistencia.
Las cargas de están moviendo desde una región de alto potencial eléctrico 
hacia uno de menor potencial eléctrico.
Es decir,
hay una disminución en la cantidad de energía potencial eléctrica.

\begin{equation*}
    dV=\frac{dU_e}{q}
\end{equation*}

Por principio de conservación de la energía,
esa energía tendría que ir algún lugar.
Por ejemplo,
cuando cae la energía potencial gravitacional de un objeto,
aumenta su energía cinética.
Sin embargo,
en el caso de la corriente,
esto no sucede,
puesto que el flujo eléctrico mantiene su velocidad.
Lo que sucede es que \textit{calientan la resistencia}.

Mecánicamente,
esto se da porque las partículas golpean los átomos y moléculas de la 
resistencia, incrementando su vibración y,
por lo tanto,
su temperatura.

¿Hay alguna manera de calcular este cambio de temperatura?
Para ello hay que usar la ecuación de potencia:

\begin{equation*}
    P=\frac{E}{t}
\end{equation*}

La potencia es energía por tiempo,
por lo que la reescribimos:

\begin{equation*}
    P=\frac{\Delta U_e}{t}
\end{equation*}

Como energía potencial es \(U=qV\),
\(\Delta U = qV_2 - qV_1 = q(V_2 - V_1)\),
reescribiendo:

\begin{equation*}
    P = \frac{q}{t}(V_2 - V_1)
\end{equation*}

Sabemos que \(\frac{q}{t}\) es la corriente, por lo tanto:

\begin{equation*}
    P=I\Delta V
\end{equation*}

La unidad es \(\frac{J}{s} = W\), es decir,
\textit{watts} son \textit{joules por segundo},
y refiere a la cantidad de energía que se va a usar en un dispositivo eléctrico,
cualquiera sea la energía en la que se va a convertir la energía potencial 
eléctrica.
Funciona para todo tipo de dispositivo que convierta energía potencial 
eléctrica en otro tipo de energía,
sea térmica, sonido, etc.

Como la ley de Ohm es \(V=IR\),
la potencia se puede reescribir:

\begin{equation*}
    P=I^{2}R
\end{equation*}

Es decir,
de acuerdo a los datos que nos den en el problema,
podemos utilizar cualquiera de las otras expresiones,
sustituyendo el dato faltante en la expresión de pontecia,
mediante la ley de Ohm.