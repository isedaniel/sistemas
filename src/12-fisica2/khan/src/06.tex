\section{Dirección del campo eléctrico}

\begin{enumerate}
    \item Encontrar dirección del campo eléctrico
    \item Encontrar dirección de la fuerza
\end{enumerate}

Usamos,
por convención,
carga eléctrica positiva.
Como la carga de prueba es positiva,
el campo eléctrico apunta en la misma dirección que la fuerza eléctrica.

Si la carga que origina el campo es positiva,
la fuerza eléctrica apuntará hacia afuera,
porque las dos cargas positivas se repelen.
Por ello,
el campo eléctrico también estará apuntando hacia afuera.

Si movemos la carga de prueba positiva,
vemos que una carga positiva siempre tendrá dirección radialmente hacia afuera de la carga.

En cambio,
si la carga que crea el campo eléctrico es negativa,
la dirección de la fuerza eléctrica sera hacia el centro radial,
por lo que el campo eléctrico generado por una carga negativa será también hacia el centro.

Lo bueno de la idea del campo eléctrico es que no necesito saber su origen:
a eso se refiere con \textit{localidad},
que es una cualidad buscada en física.

Entonces,
si la carga de prueba es positiva,
la fuerza eléctrica ejercida sobre ella apunta en la misma dirección.
Si tenemos una carga de prueba negativa,
la fuerza eléctrica apunta en dirección contraria al campo eléctrico.

Por ello decimos:
1. encontramos dirección del campo eléctrico,
2. encontramos la dirección de la fuerza eléctrica.
