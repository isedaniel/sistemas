\section{Dirección del campo eléctrico}

\begin{enumerate}
    \item Encontrar dirección del campo eléctrico
    \item Encontrar dirección de la fuerza
\end{enumerate}

Por convención,
para definir un campo eléctrico usamos una carga de prueba positiva,
de manera tal que el campo eléctrico apunta 
en la misma dirección que la fuerza eléctrica.

En otras palabras,
si la carga de origen es positiva,
la fuerza eléctrica ejercida sobre la carga de prueba apuntará hacia afuera,
puesto que ambas son positivas.
El campo eléctrico apunta hacia afuera.

Si cambiamos la carga de prueba de posición,
vemos que el vector de fuerza tendrá siempre dirección radial hacia afuera 
de la carga de origen.

En cambio,
si la carga de origen es negativa,
la dirección de la fuerza sobre la carga de prueba será radial 
y hacia el centro del campo.
El campo eléctrico originado por una carga negativa apunta hacia su centro.

Lo potente en la idea del campo eléctrico es que puedo 
\textit{abstraer} su origen:
a eso se refiere con \textit{localidad},
que es una cualidad buscada en física.

Entonces,
si la carga de prueba es positiva,
la fuerza eléctrica ejercida sobre ella apunta en la misma dirección.
Si tenemos una carga de prueba negativa,
la fuerza eléctrica apunta en dirección contraria al campo eléctrico.

Por ello decimos:
1. encontramos dirección del campo eléctrico,
2. encontramos la dirección de la fuerza eléctrica.
