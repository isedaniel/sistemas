\section{Energía potencial eléctrica con campo variable}

Tenemos carga puntual positiva \(q_1=1C\).
La carga genera un campo eléctrico en su entorno,
que sigue la expresión \(\vec{E}=\frac{k\cdot q_1}{r^{2}}\),
con vectores radialmente hacia afuera del punto \(q_1\).

A medida que nos alejamos de \(q_1\),
la intensidad del campo eléctrico disminuye con el cuadrado de la distancia \(r\).

Para obtener la fuerza ejercida por \(\vec{E}\)
sobre una carga \(q_2\),
multiplicamos por \(q_2\).
Es decir, Ley de Coulomb.

Supongamos que la carga \(q_2\) está a \(10m\) de \(q_1\)
y nuestro objetivo es desplazarla \(5m\) en dirección a \(q_1\).
Queremos calcular el trabajo \(W\)
necesario para desplazarla.
Trabajo es fuerza por distancia pero,
como la fuerza de \(\vec{E}\) va variando,
tenemos que recurrir al cálculo.
Como el trabajo realizado cambia de forma infinitesimal con respecto a la distancia recorrida,
tomamos cambio infinitesimal en trabajo \(dW\) como proporcional
al cambio infinitesimal en el radio \(dr\).
Entonces:

\vspace{.5cm}
\begin{equation*}
    dW = -\frac{k\cdot q_1\cdot q_2}{r^{2}}\,dr
\end{equation*}
\vspace{.5cm}

Es decir,
el cambio infinitesimal en el trabajo \(W\) varía siguiendo la expresión,
a medida que \(r\) cambia infinitesimalmente, 
cambio expresado por \(dr\).

Agregamos el menos adelante de la expresión porque el trabajo que tenemos 
que hacer es contrario al vector \(\vec{E}\).
En electricidad la convención es:
con el campo positivo, contra el campo negativo.

Ahora,
queremos calcular el trabajo total,
eso implica la sumatoria de todas las porciones de \(dr\).
Es decir,
sumatoria de porciones infinitesimalmente chicas, 
estamos hablando de integrar.

\begin{align*}
    W & = \int_{10}^{5}-\frac{k\cdot q_1\cdot q_2}{r^{2}}\,dr \\
    & = -kq_1q_2 \int_{10}^{5} r^{-2}\,dr \\
    & = (-kq_1q_2) \left.(-\frac{1}{r})\right|_{10}^{5} \\
    & = \boxed{\frac{kq_1q_2}{10m}}
\end{align*}

Así llegamos a una expresión para el trabajo.