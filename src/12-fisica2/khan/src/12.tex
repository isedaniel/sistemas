\section{Energía potencial eléctrica con campo variable}

Tenemos carga puntual positiva,
\(q_1=1C\).
Como el campo \(\vec{E}=\frac{k\cdot q_1}{r^{2}}\),
la carga genera un campo eléctrico en su entorno,
con vectores radialmente hacia afuera de \(q_1\).

A medida que nos alejamos de \(q_1\),
la intensidad del campo eléctrico disminuye,
porque disminuyen con el cuadrado de la distancia \(r\).

Si tenemos una carga \(q_2\) positiva,
para obtener la fuerza ejercida por \(\vec{E}\)
tenemos que multiplicamos por \(q_2\) 
y dividimos por el cuadrado de la distancia \(r\).
Es decir, Ley de Coulomb para dos cargas.

Supongamos que la carga \(q_2\) está a \(10m\) 
y nuestro objetivo es desplazarla \(5m\)
en dirección a \(q_1\).
Queremos calcular el trabajo \(W\)
necesario para desplazarla.
Trabajo es fuerza por distancia pero,
como la fuerza de \(\vec{E}\) va variando,
tenemos que recurrir al cálculo.
Como el trabajo realizado cambia de forma infinitesimal con respecto a la distancia recorrida,
tomamos cambio infinitesimal en trabajo \(dW\) como proporcional
al cambio infinitesimal en el radio \(dr\).
Entonces:

\vspace{.5cm}
\begin{equation*}
    dW = -\frac{k\cdot q_1\cdot q_2}{r^{2}}\,dr
\end{equation*}
\vspace{.5cm}

Es decir,
el cambio infinitesimal en el trabajo \(W\) varía siguiendo la expresión,
a medida que \(r\) cambia infinitesimalmente, 
cambio expresado por \(dr\).

Agregamos el menos adelante de la expresión porque el trabajo que tenemos 
que hacer es contrario al vector \(\vec{E}\).

Ahora,
queremos calcular el trabajo total,
eso implica la sumatoria de todas las porciones infinitamente chicas 
de \(dr\).
Es decir,
sumatoria de porciones infinitesimalmente chicas, 
estamos hablando de integrar.

\begin{align*}
    W & = \int_{10}^{5}-\frac{k\cdot q_1\cdot q_2}{r^{2}}\,dr \\
    & = -kq_1q_2 \int_{10}^{5} r^{-2}\,dr \\
    & = (-kq_1q_2) \left.(-\frac{1}{r})\right|_{10}^{5} \\
    & = \boxed{\frac{kq_1q_2}{10}}
\end{align*}