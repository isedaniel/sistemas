\section{Magnitud del campo eléctrico}

Ya conocemos las convenciones para determinar la dirección 
de un campo eléctrico y de su fuerza eléctrica.
Ahora nos abocamos determinar su magnitud.

La fórmula de la magnitud resulta de insertar Ley de Coulomb 
en la expresión del campo eléctrico:

\vspace{.3cm}
\begin{equation}
    |E| = \frac{kQ_1}{r^{2}}
\end{equation}
\vspace{.3cm}

Donde \(k\) es la constante electrostática,
\(Q_1\) la carga que genera el campo eléctrico,
\(r\) la distancia desde \(Q_1\) al punto
del espacio sobre el cual queremos averiguar la influencia.

Esta ecuación se cumple si la carga es realmente pequeña,
es decir,
se puede considerar puntual.
Debería tener una distribución homogénea donde está.

Como toda magnitud, no tenemos dirección vectorial:
esta es radial hacia afuera si la carga de origen es positiva,
radial hacia el centro si es negativa.