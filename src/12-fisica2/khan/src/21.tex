\section{Resistividad y conductividad}

La resistencia,
despejándola de la Ley de Ohm, 
se puede escribir como sigue:

\begin{equation*}
    R=\frac{V}{I}
\end{equation*}

Ojo, 
esto no implica que aumentar voltaje aumente resistencia.
Si aumentamos el voltaje manteniendo la resistencia constante,
la corriente \(I\) aumentará proporcionalmente.

Para cambiar una resistencia hay que cambiar el material,
el tamaño o su forma.

Una corriente circula por un área \(A\),
de largo \(L\).
Si aumentamos \(L\), sería lógico pensar que \(R\) también aumenta,
puesto que tiene que atravesar una distancia mayor,
a través de un material que se resiste al paso de la electricidad.
¿Qué pasa si modifico \(A\)?
Si crece \(A\) parecería que \(I\) tiene más espacio para pasar,
por lo que será inversamente proporcional al área.
Por lo tanto:

\vspace{.5cm}
\begin{equation*}
    R=\frac{L}{A}
\end{equation*}
\vspace{.5cm}

Sin embargo, el material también determina la resistencia.
Para hablar de la resistencia de un material 
recurrimos a su \textit{resistividad} \((\rho)\),
por ejemplo,
\(\rho_{\text{cobre}}=1.68\times^{-8}\), 
mientras \(\rho_{\text{plástico}}=10^{13}\).
La resistividad se mide en \textit{ohms por metro} \((\Omega\cdot m)\).
Si tenemos en cuenta la resistividad, llegamos a la expresión:

\vspace{.5cm}
\begin{equation*}
    R=\rho\frac{L}{A}
\end{equation*}
\vspace{.5cm}

Por otra parte, la \textit{conductividad} \((\sigma)\) 
es la inversa de la resistividad,
y refiere a la facilidad de un material para permitir la circulación eléctrica,
y se expresa:

\vspace{.5cm}
\begin{equation*}
    \sigma = \frac{1}{\rho}
\end{equation*}
\vspace{.5cm}
