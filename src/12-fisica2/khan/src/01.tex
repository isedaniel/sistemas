\section{Concepto de carga}

Por convención, negativa y positiva, pero es nombre arbitrario.

En el nivel atómico \(\Rightarrow\) nivel fundamental de la carga.
Núcleo: protones y neutrones. De acuerdo al marco, carga positiva.

El concepto de carga es una convención también.
Que es una propiedad para hacer referencia al comportamiento eléctrico.

Alrededor del núcleo saltan y zumban electrones.

Esto permite explicar el efecto triboeléctrico:
cuando frotamos materiales como el globo y el pelo,
el globo acumula electrones.

Esto lo \textit{carga} negativamente y al pelo con carga positiva.
Por ello, las puntas (con carga positiva) se atraen con la carga negativa del globo.

El modelo de la carga explica comportamientos que vemos en el universo.

Como nos gusta cuantificar, tenemos la unidad elemental de la carga: \textit{e}.
Es la carga de un protón, que es igual a la del electrón, pero opuesta a la del protón.
Por eso, escribimos \textit{e} y \textit{-e}.

Al alejarnos de la escala atómica, la unidad de carga es el coulomb (C),
definido \(1 C \approx 6.24 \times 10^{18} e\).

Con el recíproco sabemos la carga del electrón: \(e \approx 1.6 \times^{-19} C\).