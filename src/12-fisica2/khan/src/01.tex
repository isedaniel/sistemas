\section{Concepto de carga}

Por convención,
la carga puede ser negativa o positiva,
pero en realidad es un nombre arbitrario.

Nivel fundamental de la carga es el nivel atómico.
En el núcleo tenemos protones y neutrones.
De acuerdo al marco propuesto: carga positiva.
Alrededor del núcleo saltan y zumban electrones.

Cuando frotamos dos materiales,
tales como un globo sobre el pelo,
el globo acumula electrones.
Esto lo \textit{carga} negativamente,
dejando al pelo con carga positiva.
Por ello,
las puntas del pelo (con carga positiva) se atraen con la carga negativa del globo.

El modelo de la carga explica este y otros comportamientos que vemos en el universo.

En física nos gusta cuantificar.
Tenemos la unidad elemental de la carga: \textit{e}.
Que es la carga de un protón,
de signo positivo,
de igual intensidad pero signo opuesto a la del electrón.
Por eso,
escribimos \textit{e} y \textit{-e}.

Al alejarnos de la escala atómica, 
la unidad de carga es el coulomb (C),
definido \(1 C \approx 6.24 \times 10^{18} e\).

Con el recíproco sabemos la carga del electrón: \(e \approx 1.6 \times^{-19} C\).