\section{Energía potencial eléctrica de cargas}

Tenemos dos cargas positivas,
por lo tanto,
estas se repelen.
Si suponemos que parten del reposo al cabo de un tiempo
este \textit{sistema de cargas} se encuentra en movimiento,
por el principio de conservación de la energía,
sabemos que un tipo de energía \textit{necesariamente} se ha transformado en energía cinética.

La energía transformada es la energía potencial eléctrica,
que sigue la expresión:

\vspace{.5cm}
\begin{equation*}
U_e = k\frac{q_1q_2}{r}    
\end{equation*}
\vspace{.5cm}

Es decir,
es una expresión similar a la de la ley de Coulomb,
pero con la distancia entre las cargas \(r\) sin elevar al cuadrado.

Si un sistema parte del reposo, 
la expresión que lo describe es:

\vspace{.5cm}
\begin{equation*}
    U_i = U_f + K    
\end{equation*}
\vspace{.5cm}

Una cuestión más a considerar:
la energía potencial eléctrica puede ser negativa.
En la medida en que se convierte en en energía cinética,
inclusive,
puede hacerse \textit{aún más} negativa.

Esto resalta la importancia de tener en cuenta los signos de las cargas.
A diferencia de lo que pasa en las expresiones de la ley de Coulomb y el campo eléctrico
(las cuales tratan de establecer el módulo de un vector),
en este caso estamos (como en todas las energías) frente a una magnitud escalar:
los signos son importantes y nos dan información sobre el estado del sistema.
