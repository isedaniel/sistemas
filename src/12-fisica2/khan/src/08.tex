\section{Campo eléctrico neto}

Resolver problemas de campo eléctrico se complejiza cuando hay más de una carga involucrada.

Por ejemplo, si tenemos una carga de \(8nC\) y otra de \(-8nC\),
¿cuál es el campo eléctrico neto a \(6m\) de ambas cargas?

Calculamos el campo de cada una,
haciendo de cuenta que la otra no existe.
Tenemos en cuenta la dirección de cada una
(y no la carga, atención, la dirección, que es radial).
Una vez tengamos el campo de cada una de las cargas sobre el punto considerado,
sumamos,
obteniendo campo eléctrico neto \(\vec{E}_N\).

Una carga eléctrica positiva,
por la condición radial de los vectores del campo eléctrico,
puede generar una contribución negativa.

Siempre atención a eso.