\section{Aislantes y conductores}

Los materiales se pueden clasificar,
a grandes rasgos,
en aislantes y conductores\footnote{Aunque hay más tipos de materiales, como los semiconductores y los superconductores, a grandes rasgos y para este nivel de física podemos quedarnos con estos dos}.

Primero, en común,
tienen \textbf{cantidad enorme} de átomos y moléculas,
ambos con carga positiva en el nucleo y carga negativa alrededor.
Además,
ambos,
siempre que sean sólidos,
los núcleos no se van a poder mover libremente,
solo se pueden mover los electrónes con carga negativa.

Aqui empieza la diferencia:
en los conductores los electrones se mueven con relativa facilidad,
por todo el material.
En los aislantes quizás pueden pasar de uno a otro,
pero no con libertad.

En los conductores los electrones no se van a mover solos:
necesitan algo que los impulse.

A simple vista,
podría parecer que los conductores son mas relevantes,
pero esto no es así.

En los aislantes,
aunque las cargas no floten,
puede interactuar eléctricamente,
por ejemplo agregando cargas eléctricas
(como en el caso del globo visto en el primer apartado).

En un conductor, si agregamos cargas extras, se acumulan en la parte exterior,
se mueven hacia al borde.

Materiales aislantes son vidrios, maderas, plásticos.
Las cargas no se mueven.

En los conductores incluimos metales como el oro, cobre, plata.
La carga flota y se mueve libremente,
por ello cuando se cargan se van hacia la superficie,
al repelerse entre sí.

\subsection{Carga por inducción}

Si acercamos un material cargado negativamente a uno neutro,
las cargas negativas del otro material, suponiendo que es conductor,
van a desplazarse al otro lado del material.
Si conectamos el otro material a tierra
(el suelo, que funciona como un resumidero infinito de electrones).
Si acerco el cargado al neutro,
pero este está conectado a tierra,
los electrones fluyen hacia la tierra,
por ello el segundo cilindro está cargado positivamente.
Si sacamos la conexión a tierra y alejamos el material cargado,
el otro objeto queda cargado negativamente \textit{por inducción}.