\section{Materiales aislantes y materiales conductores}

Los materiales se pueden clasificar,
a grandes rasgos,
en aislantes y conductores\footnote{Aunque hay más tipos de materiales, como los semiconductores y los superconductores, a grandes rasgos y para este nivel de física podemos quedarnos con estos dos}.

Tienen en común su \textbf{cantidad enorme} de átomos y moléculas,
ambos con carga positiva en el nucleo y carga negativa en los electrones alrededor.
Además en ambos,
siempre que sean sólidos,
los núcleos no se van a poder mover libremente,
solo se pueden mover los electrones.

Su diferencia empieza precisamente en el movimiento de los electrones:
en los conductores los electrones se mueven con relativa facilidad,
por todo el material.
En los aislantes pueden pasar de un átomo a otro,
pero no con libertad.

A simple vista,
podría parecer que los conductores son mas relevantes,
pero esto no es así.
Aunque las cargas no floten,
los aislantes puede interactuar eléctricamente,
por ejemplo agregando cargas eléctricas
(como en el caso del globo visto en el primer punto).

En un conductor,
si agregamos cargas extras,
estas se acumulan en la parte exterior,
se mueven hacia su superficie.
Como los electrones gozan de libertad,
al repelerse entre si tienden a distribuirse uniformemente 
en los puntos más alejados entre sí.
Esos puntos son la superficie del conductor.

Ejemplos de materiales aislantes son vidrios, maderas, plásticos.
En los conductores incluimos metales como el oro, cobre, plata.

\subsection{Carga de un conductor por inducción}

Si acercamos un material cargado negativamente a un conductor neutro,
sus cargas negativas tenderán a desplazarse al lado contrario al material.
Si conectamos el conductor a \textit{tierra}
-que podemos imaginar como un resumidero infinito-
los electrones fluirán hacia ella,
cargando al conductor positivamente.
Luego de retirar la conexión a tierra,
hemos cargado al conductor positivamente \textit{por inducción}.
