\section{Introducción a circuitos y Ley de Ohm}

La Ley de Ohm sigue la expresión:

\vspace{.3cm}
\begin{equation*}
    V=IR
\end{equation*}
\vspace{.3cm}

Donde \(V\) es voltaje, medido en \textit{volts} \((V)\),
\(I\) es corriente eléctrica, medida en \textit{ampere} \((A)\),
y \(R\) resistecia, medida en \textit{ohms} \((\Omega)\).

Podemos hacer una analogía con la circulación de agua a través de una tubería.
Si tenemos agua a una determinada altura,
en un tanque de agua cerrado por abajo,
por ejemplo,
ésta tendrá energía potencial.
La energía potencial gravitatoria del agua, 
que la impulsará una vez que abramos el tanque,
cumple una función similar a la del voltaje.
Si liberamos el agua y dejamos que fluya,
el agua que pasa por una sección de la tubería puede pensarse como análoga a la corriente eléctrica,
de hecho \textit{ampere} equivale justamente a \textit{coulombs/segundo}.
Por último,
la resistencia se podría pensar como una parte del tubo que es más fina que el resto.
Esto hará que el agua en toda la tubería circule más lentamente.
Una resistencia en un circuito eléctrico tiene la misma propiedad.

\begin{figure}[H]
    \centering
    \begin{circuitikz}[american voltages]
        % componentes:
        \draw 
            (0,0) to (6,0)
            to [battery1,l_=$V$] (6,4)
            to [R,l_=$R$] (0,4)
            to [switch] (0,0);
        \draw
            (6,1) to [short,*-,i=$I$] (6,0);
    \end{circuitikz}
\end{figure}

Aunque los electrones fluyen de la parte negativa a la positiva,
por convención,
se considera que la circulación es en sentido contrario,
del lado positivo hacia el negativo.
Esto es así porque los electrones se descubrienron siglo y medio después de trabajar con circuitos.