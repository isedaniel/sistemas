\section{Potencial eléctrico en un punto del espacio}

Potencial eléctrico refiere,
como en el concepto del campo eléctrico,
a una \textit{propiedad del espacio}.

Si tenemos una carga \(Q\),
en una región del espacio,
podemos determinar el potencial eléctrico \(V\) en torno a dichar carga,
siguiendo la expresión:

\vspace{.5cm}
\begin{equation*}
    V = k\frac{Q}{r}
\end{equation*}
\vspace{.5cm}

Donde \(Q\) es la carga que origina el potencial,
\(k\) la constante de Coulomb,
y \(r\) la distancia entre \(Q\) y el punto del espacio considerado.

Aunque el concepto de potencial eléctrico y voltaje tengan similitudes
(y, de hecho, se midan en las mismas unidades)
hablamos de dos conceptos diferentes.
Cuando nos referimos al \textbf{voltaje} hablamos de \textit{diferencia de potencial:} \(\Delta V\).
Es decir, el cambio que hay en el potencial entre un punto y otro del espacio.

Cabe resaltar,
por último,
que el potencial eléctrico se convierte en energía potencial cuando,
como vimos en el apartado anterior,
metemos otra carga en dicho punto del espacio.
Con una sola carga, tenemos potencial. Con dos, tenemos energía potencial eléctrica.
\(V\) son \textbf{joules} que obtenemos,
en un punto del espacio, 
de acuerdo a la cantidad de \textbf{coulombs} que metamos ahí.