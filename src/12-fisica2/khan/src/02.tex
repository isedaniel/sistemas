\section{Ley de Coulomb}

Visto concept de carga,
si son iguales: se repelen. 
Si son diferentes: se atraen.

La carga es una propiedad de la materia.

Para predecir la fuerza de atracción o repulsión recurrimos a la ley de Coulomb,
de 1785.

Dadas 2 cargas,
\(q_1\) y \(q_2\),
con distancia \(r^{2}\) entre ellas,
la ley de Coulomb nos dice que la magnitud de la fuerzaelectrostática entre ellas es:

\vspace{.3cm}
\begin{equation}
    F_e = \frac{|q_1\cdot q_2|}{r^{2}}
\end{equation}
\vspace{.3cm}

Es decir,
es proporcional a la fuerza de las cargas
(tomamos absoluto para que sea indistinto si es fuerza de repulsión o atracción),
y es inversamente proporcional al cuadrado de la distancia entre ellas.

Es decir,
es similar a la gravitación,
siendo la carga y la masa propiedades de la materia.
Ambas muestran una similitud en el comportamiento del cosmos.
Pero la carga es fuerte a nivel atómico y la gravedad fuerte a nivel macro.

Por último,
hay que recordar la constante \(k\),
la \textbf{constante electrostática},
que es \(k_e = 9 \times 10^{9} N \frac{m^{2}}{C^{2}}\),
(notar que por unidades, solo quedaría N después de operar).

