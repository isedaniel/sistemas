\section{Ley de Coulomb}

Para medir la magnitud de la fuerza electrostática
recurrimos a la Ley de Coulomb, propuesta en 1785.

Dadas 2 cargas,
\(q_1\) y \(q_2\),
posicionadas a una distancia \(r\),
la ley de Coulomb dice que la magnitud de la fuerza electrostática 
que ejercen una sobre otra sigue la expresión:

\vspace{.3cm}
\begin{equation}
    F_e = k\frac{|q_1\cdot q_2|}{r^{2}}
\end{equation}
\vspace{.3cm}

Como vemos,
la fuerza electrostática es proporcional a las cargas
e inversamente proporcional al cuadrado de la distancia entre ellas.

Como operamos valor absoluto al producto de las cargas,
su signo es indistinto:
aunque sean fuerzas de atracción o repulsión,
aquí solo estamos determinando su magnitud.
Para determinar su dirección, recordar:
fuerzas iguales se repelen,
fuerzas opuestas se atraen.

Notemos que es similar a la gravitación,
siendo la carga y la masa propiedades de la materia.
Ambas muestran una similitud en el comportamiento del cosmos.
La carga es fuerte a nivel atómico y la gravedad es preponderante a nivel astronómico.

\(k\) es la \textbf{constante electrostática}:
\(k_e = 9 \times 10^{9} N \frac{m^{2}}{C^{2}}\),
(por sus unidades, solo quedaría N después de operar).
