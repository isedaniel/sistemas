
\section{Cantidades eléctricas fundamentales}

Antes de continuar,
vamos a repasar,
reveer e introducir algunos conceptos fundamentales para 
trabajar con circuitos eléctricos.

\subsection{Carga}

La fuerza eléctrica es una fuerza de acción a distancia,
que observamos en la naturaleza. 
La \textbf{carga} es la \textit{fuente} de esta fuerza
y es una propiedad de la materia.

Al igual que la gravedad,
la fuerza eléctrica actúa a distancia,
con dos diferencias fundamentales:
\begin{itemize}
    \item Es mucho más intensa que la gravedad
    \item Puede atraer (si son cargas opuestas) 
    o repeler (si tenemos cargas iguales); la gravedad siempre atrae
\end{itemize}

\subsection{Conductores, aislantes, semiconductores}

Hablamos de \textbf{conductores} para referirnos a materiales que 
tienen disponibilidad de electrones débilmente ligados a su núcleo,
por lo que facilitan la circulación de electricidad.
Una carga relativamente pequeña 
puede poner en circulación una cantidad importante de electrones
dentro de un material conductor.

Los \textbf{aislantes}, por el contrario, 
cuentan con electrones fuertemente ligados a los núcleos de sus átomos,
dificultando su circulación.
Esto no significa que la circulación sea imposible,
pero requiere de grandes cantidades de fuerza eléctrica.

Por su parte, 
la particularidad de los \textbf{semiconductores} 
es que su conductividad puede ser alterada,
lo que los convierte en materiales fundamentales en el campo de la electrónica,
permitiendo construir productos de notable complejidad.

\subsection{Corriente}

La corriente es carga fluyendo.
En términos diferenciales,
como la corriente es cantidad de carga sobre tiempo, 
la podemos expresar como:

\begin{equation*}
    i = \frac{dq}{dt}
\end{equation*}

Donde \(i\) es corriente infinitesimal,
\(dq\) es el diferencial de carga y \(dt\) el diferencial de tiempo.

\subsection{Voltaje}

El voltaje entre dos puntos es la diferencia de energía potencial 
expermientada por una carga cuando se desplaza entre ellos:

\begin{equation*}
    V = \frac{\Delta U_e}{q}
\end{equation*}

Se puede tratar, análogamente,
como la fuerza gravitatoria:
la carga fluye desde una "colina de voltaje" hacia abajo,
cediendo su energía potencial en el camino,
al igual que un cuerpo que desciende de una colina 
cede su energía potencial gravitatoria.

\subsection{Potencia}

La potencia es la tasa a la cual se entrega energía potencial eléctrica:

\begin{equation*}
    P = \frac{dU}{dt}
\end{equation*}

Si observamos detenidamente, \(V = U/q\) e \(I = q/t\).
El producto de estas dos magnitudes devuelve la potencia:

\begin{equation*}
    P = \frac{dU}{dq} \cdot \frac{dq}{dt} = \frac{dU}{dt}
\end{equation*}

Por lo tanto:

\begin{equation*}
    P = VI
\end{equation*}

La potencia es el producto de la corriente y el voltaje.