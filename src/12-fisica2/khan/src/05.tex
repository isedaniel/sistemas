\section{Definición del campo eléctrico}

¿Cómo una carga positiva empuja a otra carga positiva sin tocarla?
Hasta ahora podemos calcular la fuerza,
pero no sabemos cómo se aplica.

Pasa algo parecido con la gravedad:
sabemos describir la fuerza,
pero no el mecanismo de transmisión de fuerza a distancia.

Michael Faraday encuentra una explicación.

Una carga crea un \textbf{campo eléctrico} a su alrededor (\(\vec{E}\)),
en todo el espacio que la rodea y todo el tiempo.
Cerca de la carga tendremos un campo más fuerte
y lejos de esta un campo más débil.

Los vectores del campo eléctrico parecen fuerzas,
pero no lo son.
Campo y fuerza eléctrica no son iguales.
No es una fuerza: \textit{puede} ejercer una fuerza eléctrica.
Para que haya fuerza necesitamos que haya por lo menos 2 cargas.
Una carga individual simplemente crea un campo a su alrededor,
pero no ejerce fuerza en ausencia de otra carga.

Faraday explica así cómo un objeto puede ejercer fuerza sobre otro sin que exista contacto.

El campo eléctrico es un concepto útil para conocer fuerzas sin conocer su origen.

Se define como la cantidad de fuerza eléctrica ejercida por cantidad de carga en el espacio,
siendo esta carga lo suficientemente chica como para no ejercer prácticamente ninguna fuerza.

\vspace{.3cm}
\begin{equation}
    \vec{E} = \frac{\vec{F}}{Q_2}
\end{equation}

Siendo \(Q_2\) la carga de prueba. \(\vec{E}\) se mide en \(\frac{N}{C}\).
\(Q_2\) no crea el campo: el campo ejerce la fuerza sobre \(Q_2\).
\(\vec{E}\) puede ser el resultado de una o de muchas cargas,
y ahí radica su utilidad.

Es decir, representa la cantidad de fuerza por carga en ese punto en el espacio.