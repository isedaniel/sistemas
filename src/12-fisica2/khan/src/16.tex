\section{Potencial eléctrico de múltiples cargas}

Puesto que el potencial eléctrico \(v\) es una cantidad escalar,
que refiere a un punto en el espacio,
el potencial eléctrico debido a múltiples cargas es simplemente:

\vspace{.5cm}
\begin{equation*}
    V_t = \sum k\frac{Q_i}{r_i}
\end{equation*}
\vspace{.5cm}

Donde \(k\) es la constante de Coulomb,
\(Q_i\) es cada una de las cargas y 
\(r_i\) es la distancia desde cada una de las cargas hasta el punto.
Importante: tener en cuenta el signo de las cargas.
Como se mencionó con anterioridad:
los signos son importantes para las magnitudes escalares.
Se sacan de las vectoriales solo para que no influyan en el cálculo de la dirección.
