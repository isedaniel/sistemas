\section{Segunda clase}

28 de marzo, 2025

\subsection{Ejercitación Ley de Coulomb}

Repasamos Ley de Coulomb.

\begin{equation*}
    F=k\frac{|q_1q_2|}{r^2}
\end{equation*}

La fuerza eléctrica es constante de Coulomb (k),
por carga 1,
por carga 2,
divido distancia entre las cargas al cuadrado.

Muy similar a la ley gravitación.

La constante de Coulomb,
a la vez, es:

\begin{equation*}
    k = \frac{1}{4\pi\epsilon_0}
\end{equation*}

Según el profe,
variable de ajuste.
Permite considerar diferentes situaciones.

Seguimos repasando fórmulas,
ahora \textbf{campo eléctrico}.

\begin{equation*}
    E=\frac{F}{q}=k\frac{q_2}{r^2}
\end{equation*}

Es decir, campo eléctrico depende de fuerza y de carga.

\begin{equation*}
    \phi=EA 
\end{equation*}

\(\phi\) es \textbf{flujo}.
Esto es la \textbf{Ley de Gauss}.

Con estas tres fórmulas hacemos todo.

Arrancamos.

\subsection{Actividad 1}

\(q_1=3\times10^{-4}C\)

\(q_2=-2\times10^{-6}C\)

\(r = 0.05m\)

¿F?

Simplemente aplicamos ecuación 1: Coulomb.
Nos damos cuenta por los datos que tenemos y la pregunta que nos piden.
La constante viene dada.

\subsection{Actividad 2}

