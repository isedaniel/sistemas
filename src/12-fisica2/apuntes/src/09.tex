\section{Clase 27 de junio}

\subsection{Ejercicio 1}

Revisar foto.

\subsection{Ejercicio 2}

Fuente de \(f = 60 Hz\), resistencia de \(50 mHz\).
Si reactancia es \(X_L = 2\pi f L\), entonces:

\begin{align*}
    X_L = 2\pi \cdot 60 \cdot 50 \times 10^{-3}
    \boxed{X_L = 18.85 \Omega}
\end{align*}

\subsection{Ejercicio 3}

Si \(\epsilon = B\cdot l \cdot v\),
con 

\subsection{Ejercicio 4}

Tenemos un circuito, con pila de \(10V\)

Si velocidad de carga \(V_C = V_0(1-e^{-\frac{t}{RC}})\),
¿Cuál es la tensión de carga en 0s? ¿En \(0.1\)? ¿En \(\inf\)?

En 0, la carga será \(\boxed{0 V}\).

En \(0.1\), la carga será:

\begin{align*}
    
\end{align*}

En \(\inf\) la carga tiende a \(10 V\).

\subsection{Ejercicio 5}

