\section{Clase 16 de mayo}

Después de la parte estática viene la parte magnética.
La corriente produce un campo magnético.
Por un campo eléctrico tengo un campo magnético en torno a la corriente.

\subsection{Transformador}

Tengo una bobina que hace pasar una tensión.
De frente tengo una bobina más chica,
que pesca parte de la tensión.
Así hace bajar el voltaje.
Pero las bobinas no se toca.
Ese traslado es por el campo magnético.

La relación entre espiras y voltaje es:

\begin{equation*}
    \frac{v_1}{v_2} = \frac{n_1}{n_2}
\end{equation*}

Donde \(v\) es voltaje y \(n\) número de espiras.

\subsection{Campo magnético}

Tengo un campo.
Tiro una carga en coulombs.
Campo magnético se mide en tesla (T).

\begin{equation*}
    F = qvB\sen\theta
\end{equation*}

Donde \(F\) es fuerza,
\(q\) es carga,
\(v\) es velocidad,
\(B\) es campo magnético
y \(\theta\) es ángulo con respecto al vector del campo.

\subsection{Campo magnético}

\begin{equation*}
    \Delta B =  \frac{\mu_0I\Delta L \sen\theta}{4\pi r^{2}}
\end{equation*}

Donde \(\mu_0\) es permeabilidad del espacio libre,
que equivale a \(4\pi\times10^{-7}T\frac{m^{2}}{A}\);

