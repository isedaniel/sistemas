\section{Tercera clase}

11 de abril, 2025

\subsection{Corriente eléctrica}

Resistencia \(\alpha\).

\begin{equation*}
    I = \frac{Q}{t}
\end{equation*}

Donde \(I\) es corriente,
\(Q\) es carga
y \(t\) es tiempo.

\begin{equation*}
    A = \frac{C}{\epsilon}
\end{equation*}

Dado conductor x,
tiene una resistencia,
que impide el paso de la corriente:

\begin{equation*}
    V = IR
\end{equation*}

Que es la Ley de Ohm.
Por lo tanto, corriente despejando I:

\begin{equation*}
    I = \frac{V}{R}
\end{equation*}

Resistencia se mide en Ohms,
Potencial en Volts
y corriente en Ampere.

\subsection{Potencial eléctrico}

Es el trabajo por carga que se hace para llevar un carga de A a B.

\begin{equation*}
    V = \frac{d}{c}
\end{equation*}

\begin{equation*}
    R = \rho\frac{L}{A}
\end{equation*}

Es un quilombo esto, tengo que verlo del libro porque no entiendo nada.