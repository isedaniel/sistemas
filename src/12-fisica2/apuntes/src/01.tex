\section{Primera clase}

21 de marzo, 2025.

\subsection{Presentación}

Van a haber cuestionarios.
15 días para resolverlos.
Se le saca foto y se sube.
Entender física es saber resolver ejercicios.
Cuando se graba, se mete presente.
Materia nivel medio.
Hay que tirarse a la promo.

La materia tiene laboratorios prácticos.
\(1^{a}\) quincena de junio.
2 al 13 de junio.

Parciales:
teoría y práctica.
Los cuestionarios son prácticas para el parcial.
20 preguntas teóricas y 5 ejercicios.

La teoría subida se queda.
Cuestionarios y ejercicios van en otro lugar.

\subsection{Ley de Coulomb}

2 cargas puntuales estáticas,
si son contrarias se atraen,
si son iguales se repelen.

Entre las dos hay una fuerza que sigue la \textbf{ley de Coulomb}.

\begin{equation}
    F= k \frac{q_{1}q_{2}}{r^{2}}
\end{equation}

(como la gravedad,
es inversamente proporcional al radio).

K es:

\begin{equation*}
    K= 8.98 \times 10^{9} \frac{Nm^{2}}{c^{2}}
\end{equation*}

c es coulomb,
la unidad de carga.

En algunas fórmulas, K es:

\begin{equation*}
    K = \frac{1}{4\pi\epsilon_{0}}
\end{equation*}

La estática se carga porque perdemos electrones y el equilibrio se tiene que recuperar.

La cosa se complica si cambiamos la posición de las cargas.
Se mete geometría en la cuestión.

\subsection{Campo eléctrico}

Cada vez que hay carga eléctrica, hay campo.
Es el espacio suceptible de sufrir los efectos de la carga.
Si saco una carga y queda la otra,
hay campo.

\begin{equation}
    E = \frac{F}{q}
\end{equation}

Notar semejanza con \(F=mg\).

\subsection{Unidades hasta ahora}

Fuerza: Newton
Radio: metros
Carga: Coulomb
Campo: Newton/Coulomb

