\section{Ejercicios 23 de mayo}

\subsection{Precisiones sobre laboratorio}

Clase que viene vemos la parte práctica.
Con la jefa de trabajos prácticos.

\subsection{Ejercicio 1}

Tengo \(V_1 = 120 V\), \(N_1 = 300\) espiras, \(N_2 = 100\) espiras.
Averiguar voltaje de salida.

Aplicamos igualdad:

\begin{equation*}
    \frac{V_1}{V_2} = \frac{N_1}{N_2} \implies \frac{120 V}{V_2} = \frac{300}{100}
\end{equation*}

Entonces, despejando \(V_2\):

\begin{equation*}
    v_2 = \frac{120 V}{3} = \boxed{40 V}
\end{equation*}

\subsection{Ejercicio 2}

Tenemos un voltaje de \(220V\),
con 800 espiras.
Del otro lado salen 200 espiras, con corriente de \(2 A\).
¿Qué corriente ingresa y que voltaje de salida tenemos?

Primero averiguamos voltaje de salida:

\begin{equation*}
    \frac{V_1}{V_2} = \frac{N_1}{N_2} \implies \frac{220V}{V_2} = \frac{800}{200}
\end{equation*}

Despejando:

\begin{equation*}
    V_2 = \frac{220V}{4} = \boxed{55V}
\end{equation*}

Tenemos \(55V\) de salida. Ahora, por igualdad de potencias:

\begin{equation*}
    I_1V_1 = I_2V_2 \implies I_1 = \frac{2 A\cdot55V}{220V} = \boxed{1/2 A}
\end{equation*}

\subsection{Ejercicio 3}


Tenemos una carga de \(2\mu C\), \(v = 3\times10^{4}\frac{m}{s}\),
con \(B = 0.2T\), \(\theta=90^{\circ}\implies\sen\theta=1\).

Operamos con \(F = qvB\sen\theta\):

\begin{equation*}
    F = 2\times10^{-6} C \cdot 3\times10^{4}\frac{m}{s} \cdot 0.2T = \boxed{0.012N}
\end{equation*}

\subsection{Ejercicio 4}

Tenemos \(q = 5\mu C\), \(v = 2\times10^{5}\frac{m}{s}\),
con \(B=0.1T\) y \(\theta = 45^{\circ}\).

Nuevamente, operamos \(F = qvB\sen\theta\):

\begin{equation*}
    F = 5\times10^{-6}C \cdot 2\times10^{5}\frac{m}{s} \cdot 0.1T \cdot \sen45 = \boxed{0.07 N}
\end{equation*}

\subsection{Ejercicio 5}

Tengo \(I = 10A\), con \(r=5cm\).
Calcular \(B\).

Operamos con \(B = \frac{\mu_0I}{2\pi\cdot r}\),
siendo \(\mu_0 = 4\pi\times10^{-7}Tm/A\).

\begin{equation*}
    B = \frac{4\pi\times10^{-7}Tm/A\cdot10A}{2\pi\cdot 0.05m}=\boxed{4\times10^{-5}T}
\end{equation*}

\subsection{Ejercicio 6}

Tengo corriente \(I = 20A\).
¿A qué distancia tengo \(B = 20 \mu T\)?

\begin{equation*}
    B = \frac{\mu_0I}{2\pi\cdot r} \implies r = \frac{\mu_0I}{2\pi\cdot B}
\end{equation*}

Entonces:

\begin{equation*}
    r = \frac{4\pi\times10^{-7}\cdot20}{2\pi\cdot 20\times10^{-6}} = \boxed{0.2m}
\end{equation*}