\section{Ejercicio 2}

\begin{figure}[H]
    \centering
    \begin{circuitikz}[american voltages]
        % componentes:
        \draw
        (0,0)
        to                    (6,0)
        to [L,l_=$50mH$]      (6,4)
        to                    (0,4)
        to [sV, l_=$60Hz$]    (0,0);
    \end{circuitikz}
\end{figure}

Dado el circuito, calculamos siguiendo la expresión:

\begin{equation*}
    X_L= 2\pi f\cdot L
\end{equation*}

Donde \(X_L\) es reactancia,
que se mide en \textit{ohms} (\(\Omega\));
\(f\) representa la frecuencia,
medida en \textit{hertz} (\(Hz\));
y \(L\) representa la inductancia,
medida en \textit{henrys} (\(H\)).

Dada esa expresión, operamos:

\begin{align*}
    X_L & = 2\pi f\cdot L                             \\
    X_L & = 2\pi \cdot 60 Hz \cdot 50 \times 10^{-3} H \\
    X_L & = \boxed{18.85 \Omega}
\end{align*}