\section{Ejercicio 1}

\begin{equation*}
    L = \mu \cdot \frac{N^{2}A}{l}
\end{equation*}

Donde \(L\) es inductancia,
que se mide en \textit{henry} (\(H\));
\(\mu\) es la permeabilidad magnética del núcleo,
se mide en \textit{henrios por metro} (\(H/m\)) 
o \textit{tesla metros por amperio} (\(T\cdot m/A\));
\(N\) es el número de espiras,
es un valor \textit{adimensional};
\(A\) es el área transversal del solenoide,
se mide -como toda área- en metros cuadrados (\(m^{2}\));
y por último \(l\) es la longitud del solenoide,
naturalmente en metros (\(m\)).

Dado que \(\mu_0 = 4\pi\times10^{-7} H/m\), \(N = 500\), \(l = 0.2 m\),
y diámetro interno es \(d = 2cm\).

Primero determinamos el área:

\begin{align*}
    A & = \pi\cdot r^{2} \\
    A & = \pi\cdot (0.01 m)^{2} \\
    A & = \boxed{\pi \times 10^{-4} m^{2}}
\end{align*}

Por último, calculamos inductancia:

\begin{align*}
    L & = \mu \frac{N^{2}A}{l} \\
    L & = 4\pi\times10^{-7} \frac{H}{m} \cdot \frac{(500)^{2} \cdot \pi \times 10^{-4} m^{2}}{0.2 m} \\
    L & = 4\pi\times10^{-7} \frac{H}{\cancel{m}} \cdot \frac{(500)^{2} \cdot \pi \times 10^{-4} \cancel{m^{2}}}{0.2 \cancel{m}} \\
    L & = \boxed{4.93\times10^{-4} H}
\end{align*}