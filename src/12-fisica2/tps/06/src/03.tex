\section{Ejercicio 3}

Si \(\epsilon = B\cdot l\cdot v\),
donde \(\epsilon\) es la Fuerza Electromotriz (FEM),
medida en \textit{voltios} (\(V\));
\(B\) es el campo magnético,
medido en \textit{teslas} \((T)\);
\(l\) la longitud del conductor,
que se mide naturalmente en \textit{metros} \((m)\);
y \(v\) la velocidad, 
en su unidad habitual de \textit{metros por segundo} \((m/s)\).

Dados \(B=0.4T\), \(v=3m/s\) y \(l=0.5m\), operamos:

\begin{align*}
    \epsilon &= B\cdot l\cdot v \\
    \epsilon &= 0.4T \cdot 0.5m \cdot 3 m/s \\
    \epsilon &= \boxed{0.6V}
\end{align*}