\section{Ejercicio 5}

\begin{figure}[H]
    \centering
    \begin{circuitikz}[american voltages]
        % componentes:
        \draw
        (0,0)
        to [R,l_=$3\Omega$]      (6,0)
        to [R,l_=$1\Omega$]     (6,4)
        to [R,l_=$2\Omega$]     (0,4)
        to [battery1,l_=$12V$,i_=$I$]  (0,0);
    \end{circuitikz}
\end{figure}

Tenemos que calcular corriente \(I\).
Dado que las resistencias están en serie,
la resistencia total \((R_T)\) se obtiene 
de la sumatoria de resistencias,
por lo tanto:

\begin{align*}
    R_T & = (3 + 1 + 2) \Omega \\
    R_T & = 6\Omega 
\end{align*}

Por último, para obtener \(I\) recurrimos a la Ley de Ohm,
cuya expresión es \(V=IR\), si despejamos:

\begin{align*}
    I & = \frac{V}{R} \\
    I & = \frac{12V}{6\Omega} \\
    I & = \boxed{2A}
\end{align*}