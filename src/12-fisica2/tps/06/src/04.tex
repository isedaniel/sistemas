\section{Ejercicio 4}

\begin{figure}[H]
    \centering
    \begin{circuitikz}[american voltages]
        % componentes:
        \draw 
            (0,0) 
            to     (6,0)
            to [C,l_=$100\mu F$]                     (6,4)
            to [R,l_=$1k\Omega$]                     (0,4)
            to [battery1,l_=$10V$] (0,0);
    \end{circuitikz}
\end{figure}

La fórmula de carga del capacitor sigue la expresión
\(V_C=V_0(1-e^{-t/RC})\);
donde \(V_C\) es tensión en tiempo \(t\),
medida el \textit{volts} \((V)\);
\(V_0\) es tensión final o máxima;
\(e\) es el número de Euler;
\(t\) es el tiempo,
en \textit{segundos} \((s)\);
\(R\) es resistencia,
medida en \textit{ohms} \((\Omega)\);
\(C\) es capacitancia,
medida en \textit{faradios} \((F)\);
notar que \(RC\) se conoce como 
constante de tiempo,
que indica la rapidez de carga o descarga del capacitor,
representada por \(\tau\),
y se mide en \textit{segundos} \((s)\).

¿Cuál es la tensión de carga en \(t=0\)?
¿En \(t=0.1s\)?
¿Y en \(t=\infty s\)?

En 0 y en \(\infty\) resolvemos deduciendo:
0 y \(10V\), respectivamente.
Operamos para calcular en \(0.1s\):

\begin{align*}
    RC &= 100\times10^{-6}F \cdot 1\times10^{3} \Omega \\
    RC &= \boxed{0.1}
\end{align*}

Por lo tanto:

\begin{align*}
    V_C = 10V \cdot (1-e^{-0.1/0.1}) \\
    V_C = \boxed{6.32V}
\end{align*}