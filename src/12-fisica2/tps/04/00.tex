\documentclass[11pt]{article}
\usepackage[a4paper, margin=2.54cm]{geometry}

% español
\usepackage[spanish]{babel}

% imágenes
%\usepackage{graphicx}
%\graphicspath{{img}}

% fuentes de conjuntos numéricos
\usepackage{amsfonts}

% símbolos
\usepackage{amsmath, amssymb}

% gráficos
%\usepackage{tikz}

% plots
%\usepackage{pgfplots}
%\pgfplotsset{width=10cm, compat=1.9}

% averiguar
\setlength{\jot}{8pt}
\setlength{\parindent}{0cm}

% espacio entre párrafos
\usepackage[skip=10pt plus1pt, indent=12pt]{parskip}

% cancelar términos
\usepackage{cancel}

% links
%\usepackage[colorlinks=true, 
%    urlcolor=blue]{hyperref}

% shapes
%\usetikzlibrary{shapes.geometric}

% incluir pdfs
%\usepackage{pdfpages}

\title{Física II A\\Ejercicios de Campos Magnéticos\\Prof. Juan Viñas}
\author{Daniel Ise}
\date{Primer Cuatrimestre, 2025}

\begin{document}

\maketitle

\section{Fórmulas}

\subsection{Ejercicio 1}

Tengo \(V_1 = 120 V\), \(N_1 = 300\) espiras, \(N_2 = 100\) espiras.
Averiguar voltaje de salida.

Aplicamos igualdad:

\begin{equation*}
    \frac{V_1}{V_2} = \frac{N_1}{N_2} \implies \frac{120 V}{V_2} = \frac{300}{100}
\end{equation*}

Entonces, despejando \(V_2\):

\begin{equation*}
    v_2 = \frac{120 V}{3} = \boxed{40 V}
\end{equation*}

\subsection{Ejercicio 2}

Tenemos un voltaje de \(220V\),
con 800 espiras.
Del otro lado salen 200 espiras, con corriente de \(2 A\).
¿Qué corriente ingresa y que voltaje de salida tenemos?

Primero averiguamos voltaje de salida:

\begin{equation*}
    \frac{V_1}{V_2} = \frac{N_1}{N_2} \implies \frac{220V}{V_2} = \frac{800}{200}
\end{equation*}

Despejando:

\begin{equation*}
    V_2 = \frac{220V}{4} = \boxed{55V}
\end{equation*}

Tenemos \(55V\) de salida. Ahora, por igualdad de potencias:

\begin{equation*}
    I_1V_1 = I_2V_2 \implies I_1 = \frac{2 A\cdot55V}{220V} = \boxed{1/2 A}
\end{equation*}



\end{document}