\section{Fórmulas}

\subsection{Ejercicio 1}


Tenemos una carga de \(2\mu C\), \(v = 3\times10^{4}\frac{m}{s}\),
con \(B = 0.2T\), \(\theta=90^{\circ}\implies\sen\theta=1\).

Operamos con \(F = qvB\sen\theta\):

\begin{equation*}
    F = 2\times10^{-6} C \cdot 3\times10^{4}\frac{m}{s} \cdot 0.2T = \boxed{0.012N}
\end{equation*}

\subsection{Ejercicio 2}

Tenemos \(q = 5\mu C\), \(v = 2\times10^{5}\frac{m}{s}\),
con \(B=0.1T\) y \(\theta = 45^{\circ}\).

Nuevamente, operamos \(F = qvB\sen\theta\):

\begin{equation*}
    F = 5\times10^{-6}C \cdot 2\times10^{5}\frac{m}{s} \cdot 0.1T \cdot \sen45 = \boxed{0.07 N}
\end{equation*}

\subsection{Ejercicio 3}

Tengo \(I = 10A\), con \(r=5cm\).
Calcular \(B\).

Operamos con \(B = \frac{\mu_0I}{2\pi\cdot r}\),
siendo \(\mu_0 = 4\pi\times10^{-7}Tm/A\).

\begin{equation*}
    B = \frac{4\pi\times10^{-7}Tm/A\cdot10A}{2\pi\cdot 0.05m}=\boxed{4\times10^{-5}T}
\end{equation*}

\subsection{Ejercicio 4}

Tengo corriente \(I = 20A\).
¿A qué distancia tengo \(B = 20 \mu T\)?

\begin{equation*}
    B = \frac{\mu_0I}{2\pi\cdot r} \implies r = \frac{\mu_0I}{2\pi\cdot B}
\end{equation*}

Entonces:

\begin{equation*}
    r = \frac{4\pi\times10^{-7}\cdot20}{2\pi\cdot 20\times10^{-6}} = \boxed{0.2m}
\end{equation*}