\section{Fórmulas}

\subsection{Ejercicio 1}

Tengo \(V_1 = 120 V\), \(N_1 = 300\) espiras, \(N_2 = 100\) espiras.
Averiguar voltaje de salida.

Aplicamos igualdad:

\begin{equation*}
    \frac{V_1}{V_2} = \frac{N_1}{N_2} \implies \frac{120 V}{V_2} = \frac{300}{100}
\end{equation*}

Entonces, despejando \(V_2\):

\begin{equation*}
    v_2 = \frac{120 V}{3} = \boxed{40 V}
\end{equation*}

\subsection{Ejercicio 2}

Tenemos un voltaje de \(220V\),
con 800 espiras.
Del otro lado salen 200 espiras, con corriente de \(2 A\).
¿Qué corriente ingresa y que voltaje de salida tenemos?

Primero averiguamos voltaje de salida:

\begin{equation*}
    \frac{V_1}{V_2} = \frac{N_1}{N_2} \implies \frac{220V}{V_2} = \frac{800}{200}
\end{equation*}

Despejando:

\begin{equation*}
    V_2 = \frac{220V}{4} = \boxed{55V}
\end{equation*}

Tenemos \(55V\) de salida. Ahora, por igualdad de potencias:

\begin{equation*}
    I_1V_1 = I_2V_2 \implies I_1 = \frac{2 A\cdot55V}{220V} = \boxed{1/2 A}
\end{equation*}
