\section*{Fórmulas}

Trabajo que necesito para llevar una carga de un lugar a otro:

\begin{equation}
    W = QV
\end{equation}

Donde \(W\) es trabajo,
\(Q\) es carga,
y \(V\) es diferencia de potencial.

\begin{equation}
    C=k\epsilon\frac{A}{d}
\end{equation}

Donde \(C\) es capacidad,
\(k\) es constante,
\(\epsilon_0\),
y \(V\) es diferencia de potencial.

Con capacitores en serie:

\begin{equation}
    \frac{1}{C_T} = \frac{1}{C_1} + \frac{1}{C_2}
\end{equation}

\section{Ejercicio 1}

Un \(e\), carga \(1.6 \cdot 10^{-19}C\).

Dada pila de \(12 V\),
cuánto trabajo se requiere para llevar un electrón desde el terminal positivo al terminal negativo.

Operamos con fórmula de trabajo:

\begin{align*}
    W = QV \implies W = 1.6 \cdot 10^{-19}C \cdot 12 V \\
    \boxed{W = 1.92 \cdot 10^{-18} J}
\end{align*}

\section{Ejercicio 2}

Dada \(A = 200 cm^2\), con \(d = 0.4 cm\) y \(\epsilon_0 = 8.85\cdot10^{-12}\),
con \(K_{\text{Aire}} = 1\), determinar valor del capacitor (\(C\)).

Operamos con segunda ecuación:

\begin{align*}
    C=k\epsilon\frac{A}{d} \implies C = 1 \cdot 8.85\cdot10^{-12} \frac{0.02 m^2}{0.004 m} \\
    \boxed{4.42 \cdot 10^{-11} F}
\end{align*}

\section{Ejercicio 3}

\(C_1 = 3 pF\) y \(C_2 = 6 pF\). Calcular el capacitor resultante de la serie de \(C_1\) y \(C_2\).

Operamos con la fórmula de capacitancia:

\begin{align*}
    \frac{1}{C_T} = \frac{1}{C_1} + \frac{1}{C_2} \implies \frac{1}{C_T} = \frac{1}{3} + \frac{1}{6} \\
    \frac{1}{C_T} = \frac{3}{6} \\
    \boxed{C_T = 2pF} \\
\end{align*}