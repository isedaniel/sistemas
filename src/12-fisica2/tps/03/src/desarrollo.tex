\section{Fórmulas}

\subsection{Corriente}

\begin{equation}
    I = \frac{Q}{t}
\end{equation}

\subsection{Ley de Ohm}

\begin{equation}
    R = \frac{V}{I}
\end{equation}

\subsection{Resistencia}

\begin{equation}
    R = \rho\frac{L}{A}
\end{equation}

\section{Ejercicio 1}

I = 0.5A

¿Cuánta carga pasa por un alambre en 1 minuto?

Pasamos tiempo a minutos.

\begin{align*}
    t = 1 m = 60 s \\
\end{align*}

Operamos con ecuación de corriente:

\begin{align*}
    0.5A = \frac{Q}{60 s} \\
    \boxed{Q = 30 C}
\end{align*}

\section{Ejercicio 2}

\(R = 240 \omega\), conectada a una fuente de \(120 V\).
¿Qué corriente circula?

Despejamos corriente de Ley de Ohm:

\begin{align*}
    R = \frac{V}{I} \implies I = \frac{V}{R} \\
\end{align*}

Operamos:

\begin{align*}
    I = \frac{120 V}{240 \omega} \\
    \boxed{I = 0.5A}
\end{align*}

\section{Ejercicio 3}

Alambre de 2 metros, diámetro de 8 milímetros,
con \(\rho = 1.76 \cdot 10^{-18} \omega\cdot m\).
Determinar resistencia.

Operamos con fórmula de resistencia:

\begin{align*}
    R = 1.76 \cdot 10^{-18} \omega\cdot m \frac{2 m}{\pi\cdot (0.004 m)^2} \\
    \boxed{R \approx 7\cdot 10^{-14}}
\end{align*}

