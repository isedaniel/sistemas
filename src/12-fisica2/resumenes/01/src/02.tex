\section{Trabajo}

Trabajo necesario para trasladar una carga de una posición a otra:

\vspace{1cm}
\begin{equation}
    W = Q\cdot V
\end{equation}
\vspace{1cm}

Siendo \(W\) el trabajo total,
\(Q\) la carga y \(V\) la diferencia de potencial.

\section{Capacitancia}

\vspace{1cm}
\begin{equation}
    C = k\cdot\epsilon_0 \frac{A}{d}
\end{equation}
\vspace{1cm}

Siendo \(C\) capacitancia,
\(k\) una constante que depende del material,
\(A\) el área y \(d\) la distancia.

\section{Capacitor en serie}

Para sumar capacitores en serie:

\vspace{1cm}
\begin{equation}
    \frac{1}{C_T} = \frac{1}{C_1} + \frac{1}{C_2}  + \cdots + \frac{1}{C_n} 
\end{equation}
\vspace{1cm}