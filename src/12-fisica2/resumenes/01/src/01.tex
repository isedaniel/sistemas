\section{Carga eléctrica}

\subsection{Ley de Coulomb}

La fuerza ejercida entre cargas \(Q_1\) y \(Q_2\).

\vspace{1cm}
\begin{equation}
    F = k\frac{|Q_1Q_2|}{r^{2}}
\end{equation}
\vspace{1cm}

\begin{enumerate}
    \item Siendo F la fuerza entre las cargas (por tercera ley), medida en newton (N)
    \item \(k\) la \textbf{Constante de Coulomb} o \textbf{Constante electrostática},
    con valor aproximado 
    \begin{equation*}
        k_0\approx9\times10^{9} \frac{N\cdot m^{2}}{C^{2}}
    \end{equation*}
    \item \(Q_{1}\) y \(Q_{2}\) las cargas, medidas en coulomb (C)
    \item y \(r\) la distancia entre las mismas, medida en metros (m)
\end{enumerate}

La constante de Coulomb puede expresarse también como:

\vspace{1cm}
\begin{equation*}
    k_0 = \frac{1}{4\pi\epsilon_0}
\end{equation*}
\vspace{1cm}

Siendo \(\epsilon_0\) la \textbf{Constante de permitividad},
cuyo valor aproximado es: 
\begin{equation*}
    \epsilon_0 \approx 8.85 \times 10^{-12} \frac{C^{2}}{N\cdot m^{2}}
\end{equation*}.

\section{Campo eléctrico}

Es la fuerza potencial que una carga puede ejercer sobre una carga de prueba,
ubicada a un distancia \textit{r}.

\vspace{1cm}
\begin{equation}
    E = \frac{F}{Q_0} = k \frac{|Q|}{r^{2}}
\end{equation}
\vspace{1cm}

\begin{itemize}
    \item Siendo \(E\) el campo eléctrico, medido en \(\frac{N}{Q}\)
    \item \(F\) la fuerza que la carga \(Q\) puede potencialmente ejercer sobre la carga de prueba \(Q_0\)
    \item \(k\) la constante de Coulomb 
    \item \(r\) la distancia entre las cargas
\end{itemize}

\section{Flujo Eléctrico - Ley de Gauss}

Refiere al campo eléctrico que atraviesa una superficie de área \(A\).

\vspace{1cm}
\begin{equation*}
    \phi = E \cdot A \cdot \cos \theta
\end{equation*}
\vspace{1cm}

\begin{itemize}
    \item Siendo \(\phi\) el flujo eléctrico 
    \item \(E\) el campo eléctrico 
    \item \(A\) el área considerada
    \item \(\theta\) el ángulo entre el vector normal de la superficie y el vector del campo eléctrico
\end{itemize}

Cuando el vector normal de la superificie y el vector del campo eléctrico son paralelos,
\(\cos\theta = 1\), por lo cual la ecuación se reduce a:

\vspace{1cm}
\begin{equation}
    \phi = E\cdot A
\end{equation}
\vspace{1cm}

La ley de Gauss resulta de reemplazar \(E\) por \(k\frac{|Q|}{r^{2}}\) y,
a su vez,
\(k\) por \(\frac{1}{4\pi\epsilon_0}\).
Por otra parte,
el área considerada es la superficie de una esfera,
por lo cual \(A = 4\pi r^{2}\),
por lo cual llegamos a la expresión:

\vspace{1cm}
\begin{equation*}
    \phi = \frac{1}{4\pi\epsilon} \cdot \frac{|Q|}{r^{2}} \cdot 4\pi r^{2}
\end{equation*}
\vspace{1cm}

Simplificamos:

\vspace{1cm}
\begin{equation*}
    \phi = \frac{1}{\cancel{4\pi}\epsilon} \cdot \frac{|Q|}{\cancel{r^{2}}} \cdot \cancel{4\pi} \cancel{r^{2}}
\end{equation*}
\vspace{1cm}

Finalmente, llegamos a la expresión de la \textbf{Ley de Gauss}:

\vspace{1cm}
\begin{equation}
    \phi = \frac{|Q|}{\epsilon}
\end{equation}
\vspace{1cm}

