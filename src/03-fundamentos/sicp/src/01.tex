\setcounter{section}{-1} % para que prefacio tenga número 0

\section{Prefacios}

Primero está el prólogo.
Diferencia prefacio-prólogo: prólogo generalmente hecho por otra persona.
Prefacio hecho por el propio autor.

\subsection{Prólogo}

Programar es una actividad netamente \textit{intelectual}.
Involucra el pensamiento.
Para aprender a programar hay que escribir -\textit{y leer}- cientos de programas.
Lo que importa no es el problema que resuelven los programas -el qué-,
sino que lo hagan de manera eficiente.
Además, 
deben ser capaces de articularse con otros programas.
Es decir: eficiencia y estructura es lo fundamental.

Todo \textit{programa} es un \textit{modelo} de un proceso real o mental.
En tanto modelo, resulta del \textit{conocimiento humano}, que es a su vez un subproducto de la \textit{experiencia humana}.

Los programas son,
en todo momento,
\textit{parcialmente entendidos}:
nadie los maneja a todos completamente.

A su vez,
a medida que el conocimiento humano sobre el proceso real se incrementa,
-esto es, a medida que crece la experiencia que tenemos con dicho proceso-
el programa evoluciona.
La evolución del programa requiere que seamos capaces de lidiar con la \textit{complejidad}.

Que un programa sirva a su cometido significa que es \textit{verdadero}.
Como en toda activida intelectual humana,
que creamos que algo es verdadedor resulta de la \textit{argumentación} sobre dicha verdad.

Los programas complejos se construyen sobre piezas más simples: 
estas piezas más simples se denominan \textit{idioms}
-a falta de mejor palabra, por ahora, en español-.
Estas estructuras simples,
siguiendo técnicas de estructuración adecuadas,
permiten resolver problemas más complejos manteniendo la capacidad de gestionar la complejidad.

Dentro de las estructuras tenemos también las que,
utilizando eficientemente \textit{procesamiento} y \textit{almacenamiento},
resuelven un problema precisamente definido.
Estas estructuras se denominan \textit{algoritmos}.

El bueno programador sabe mezclar \textit{modismos} y \textit{algoritmos} para resolvemos problemas complejos.

\subsection{Prefacio}

\section{Building Abstractions with Procedures}

Un \textbf{proceso computacional} es una entidad abstacta que,
a medida que \textit{evoluciona},
manipula otra entidad abstacta que denominamos \textbf{datos}.
Las \textit{reglas} que sigue un proceso computacional 
en su \textit{evolución} se conoce como 
\textbf{programa}.