\section{Primera clase}

\subsection{Población}

\begin{itemize}
    \item Ejemplo: habitantes de Argentina
    \item para tomar datos de una población completa: hacemos un \textbf{censo}
    \item es muy caro y complejo
    \item por eso se suele tomar una \textbf{muestra}
    \item \textit{se construye la muestra}, buscando que 
    sea \textbf{representativa}
\end{itemize}

\subsection{Conjunto}
\begin{itemize}
    \item Es un grupo de elementos
    \item Las variables se analizan para un conjunto
    \item Se suele denotar con letras mayúsculas
    \item Sus elementos con letras minúsculas
\end{itemize}

\begin{equation*}
    X = \left\{x_1, x_2, \dots x_n\right\}
\end{equation*}

\subsection{Frecuencia}
\begin{itemize}
    \item Agrupo la cantidad de elementos del conjunto de acuerdo a una 
    característica común
    \item Ejemplo: alumnos de primero, de segundo, etc.
\end{itemize}

\subsection{Frecuencia acumulada}
\begin{itemize}
    \item Va sumando las cantidades de cada una de las categorías
\end{itemize}

\subsection{Frecuencia relativa}
\begin{itemize}
    \item Porcentaje respecto del total que representa la frecuencia
\end{itemize}

\subsection{Frecuencia relativa acumulada}
\begin{itemize}
    \item Sumatoria de frecuencias relativas
\end{itemize}

\subsection{Muestra}
\begin{itemize}
    \item Una muestra bien construida se \textit{debería} aproximar a los valores 
    de la población 
    \item Cuando esto sucede hablamos de que la muestra es \textbf{representativa}
\end{itemize}
