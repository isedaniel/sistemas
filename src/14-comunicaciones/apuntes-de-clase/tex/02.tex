\section{Segunda clase, 18 de agosto}

\subsection{Protocolos}

\begin{itemize}
    \item Conjunto de reglas que gobiernan la comunicación entre entidades
    \item Permite la comunicación
    \item Análogo al idioma
    \item Puntos clave:
    \begin{itemize}
        \item Sintaxis
        \item Semántica
        \item Temporización
    \end{itemize}
\end{itemize}

Pueden ser:
\subsubsection{Directos o indirectos} 
\begin{itemize}
    \item Directo: los dos sistemas comparten línea punto a punto,
    se comunican directamente
    \item Los agentes se comunican sin intervención de \textit{agente activo}
    \item Indirecta: los sistemas se comunican por red conmutada
    \item Si la comunicación entre dos entidades dependen de otra entidad,
    el \textit{agente activo}
\end{itemize}

\subsubsection{Monolíticas o estructuradas} 
\begin{itemize}
    \item Monolítico: contiene toda la lógica necesaria para la comunicación 
    \item Puede ser muy compleja
    \item Protocolo estructurado: en lugar de un único protocolo,
    hay un conjunto, que se ordenan de forma jerárquica o estructurada.
    \item En la estructurada, el conjunto de harware y soft es la \textit{arquitectura del protocolo}
\end{itemize}

\subsubsection{Simétricos o asimétricos} 
\begin{itemize}
    \item Simetría: diseñada para entidades pares, mismo protocolo ambos extremos
    \item Asimétricos: para comunicaciones entre impares,
    resulta de la necesidad de simplificación
\end{itemize}

\subsubsection{Estándares o no estándares}
\begin{itemize}
    \item Sin estándares necesitaríamos muchos protocolos y aun más 
    implementaciones 
    \item Por eso se \textit{estandariza}
    \item Estandarizar reduce complejidad y costos de implementación y operación
    \item Incentiva a distintos proveedores a ser interoperables
\end{itemize}

\subsection{Funciones de un protocolo}

No todos proporcionas todas las funciones, algunas se repiten
en protocolos situados en distintos niveles.

\subsubsection{Encapsulamiento}

Cada protocolo controla datos a transmitir