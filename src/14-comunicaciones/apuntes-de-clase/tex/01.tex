\section{Primera clase, 11 de agosto}

\subsection{Modalidad de cursada}

\begin{itemize}
    \item Domingo: se habilita el contenido de la unidad hasta el viernes
    \item Viernes a la noche: se habilita el cuestionario que hay que resolver,
    hasta el domingo.
    \begin{itemize}
        \item Tenemos que embocar 4/5 para aprobar
        \item Si no aprobamos no podemos acceder al contenido que sigue
        \item Comunicarse con el docente en caso de necesitar extensión o recuperar
    \end{itemize}
    \item Lo ideal: resolver cada sábado
    \item Hay que aprobar todos los parcialitos para acceder al parcial 
\end{itemize}

\subsection{Contenido}

\begin{itemize}
    \item Conceptos generales de comunicaciones 
    \item En la segunda materia (Redes de comunicación) vamos a profundizar 
    en la parte prática
    \item Desarrollamos una unidad por clase 
    \item Se evalua todas las semanas 
    \item Estudiar con bibliografía provista y cuestionarios semanales 
    \item Primer parcial: fines de septiembre/principios de octubre
    \item Semana de parciales no hay tema nuevo
\end{itemize}

\subsection{Unidad I}

\begin{itemize}
    \item Modelos simplificados de comunicaciones 
    \item Input digital \(\rightarrow\) analógico para transmitir \(\rightarrow\) digital para recibir
    \item Se puede ir complejizando incorporando \textbf{wan} y \textbf{lan}
    \item Clasificación por distancia geográfica: 
    dentro de la máquina \(\rightarrow\) lan \(\rightarrow\) wan;
    a medida que crecemos en escala 
\end{itemize}

\subsection{Wan}

En definitiva, historia de la red wan.

\begin{itemize}
    \item Retransmisión \(\rightarrow\) \textit{frame relay}
    \item Modo de transferencia asíncrono
    \begin{itemize}
        \item paquetes de largo variable
        \item velocidad de 10 Mbps a 100Mbps, incluso 1Gbps
        \item múltiples canales 
    \end{itemize}
    \item ISDN: red digital de servicios integrados
    \begin{itemize}
        \item Banda ancha
        \item Cientos de Mbps 
        \item se oriente a la conmutación de paquetes 
    \end{itemize}
\end{itemize}

\subsection{Lan}

\begin{itemize}
    \item Conformada por dispositivos relativamente cercanos 
    \item Generalmente red y dispositivos son propiedad del mismo ente 
    \item Velocidad es mayor a la wan 
    \item Requiere una gran inversión inicial:
    \begin{itemize}
        \item Estudio previo a la instalación 
        \item considerar el mantenimiento 
        \item repartir las responsabilidades de gestión 
        \item distinguir pasillos fríos y calientes 
        \item prever situaciones de crisis, contingencias energéticas,
        posicionamiento de \textit{data centers}, refrigeración 
    \end{itemize}
    \item lan ethernet: basada en conmutación 
\end{itemize}

\subsection{Conceptos}
\begin{itemize}
    \item Comunicación: intercambio de información para cooperar
    \item Red: conjunto de dispositivos 
    \item Networking: interconexión de estaciones de trabajo
    \item Protocolo: descripción formal de normas y reglas de comunicación 
    \item Sistema: conjunto de una o más entidades
\end{itemize}


\subsection{Bibliografía}
\begin{itemize}
    \item William Stallings es la bibliografía básica.
\end{itemize}