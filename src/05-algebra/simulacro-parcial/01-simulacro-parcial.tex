\documentclass[12pt]{article}
\usepackage[a4paper, margin=2.54cm]{geometry}

% español
\usepackage[spanish]{babel}

% imágenes
%\usepackage{graphicx}
%\graphicspath{{img}}

% fuentes de conjuntos numéricos
\usepackage{amsfonts}

% símbolos
\usepackage{amsmath, amssymb}

% gráficos
%\usepackage{tikz}

% plots
%\usepackage{pgfplots}
%\pgfplotsset{width=10cm, compat=1.9}

% averiguar
\setlength{\jot}{8pt}
\setlength{\parindent}{0cm}

% espacio entre párrafos
\usepackage[skip=10pt plus1pt]{parskip}

% cancelar términos
\usepackage{cancel}

% links
%\usepackage[colorlinks=true, 
%    urlcolor=blue]{hyperref}

% shapes
%\usetikzlibrary{shapes.geometric}

% incluir pdfs
%\usepackage{pdfpages}

\title{Álgebra y geometría analítica\\Simulacro de parcial}
\author{Daniel Ise}
\date{17 de septiembre de 2024}

\begin{document}

\maketitle

\tableofcontents

\pagebreak

\section{Parcial 9/19, noche}

\textbf{1. Dadas las rectas \(r_{1}: \lambda (1,4,-2) + (3,9,-4)\)
  y \(r_{2}: \beta(5,7,3) + (-3,-2,1)\),
  los puntos: \(P (1,2,0)\), \(Q (-2,-1,0)\) y \(R (1,0,-3)\),
  y los planos: \(\delta: x + 2z - 6 = 0\), \(\epsilon: x - y + z = 0\)
  y \(\beta: z = 5\).}

\textbf{a. Hallar la intersección entre los planos \(\delta\), \(\epsilon\)
  y \(\beta\).}

Primero, buscamos la intersección entre \(\delta\) y \(\epsilon\).
Para ello, planteamos un sistema de ecuaciones con sus ecuaciones generales:

\begin{align*}
  \begin{cases}
    x + 2z - 6 = 0 \\
    x - y + z = 0  \\
  \end{cases}
\end{align*}

Despejamos \(x\) y sustituimos:

\begin{align*}
  x = 6 - 2z         \\
  6 - 2z - y + z = 0 \\
  y = 6 - z          \\
\end{align*}

Planteamos la ecuación de la recta intersección:

\begin{align*}
  (x,y,z) = (6 - 2z, 6-z, z)              \\
  (x,y,z) = (-2z, -z, z) + (6, 6, 0)      \\
  (x,y,z) = \alpha(-2, -1, 1) + (6, 6, 0) \\
\end{align*}

\(\delta\) y \(\epsilon\) se intersecan en \(\alpha(-2, -1, 1) + (6, 6, 0)\).

Veamos ahora la intersección con \(beta\):

\begin{align*}
   & \begin{cases}
       x = -2\alpha + 6 \\
       y = -\alpha + 6  \\
       5 = \alpha       \\
     \end{cases} \\
   & \boxed{x = -4}   \\
   & \boxed{y = 1}    \\
\end{align*}

Los tres planos se intersecan en el punto \((-4,1,5)\).
Sabemos que solo se intersecan en ese punto puesto que \(\beta\) es paralelo
al plano formado por los ejes \(x\) e \(y\).

\textbf{b. Hallar intersección \(r_{1}\) y \(r_{2}\)}

Planteamos un sistema de ecuaciones:

\begin{align*}
  \begin{cases}
    \lambda (1,4,-2) + (3,9,-4) \\
    \beta(5,7,3) + (-3,-2,1)    \\
  \end{cases}
   & \rightarrow
  \begin{cases}
    \lambda + 3 = 5\beta - 3   \\
    4\lambda + 9 = 7\beta - 2  \\
    -2\lambda - 4 = 3\beta + 1 \\
  \end{cases}
\end{align*}

Tenemos dos incógnitas y tres ecuaciones,
resolvemos tomando dos y comprobamos con la restante:

\begin{align*}
  \lambda           & = 5\beta -6                      \\
  4(5\beta -6) + 9  & = 7\beta - 2                     \\
  20\beta - 24 + 9  & = 7\beta - 2                     \\
  13\beta           & = 13                             \\
  \boxed{\beta = 1} & \rightarrow \boxed{\lambda = -1} \\
  -2 \cdot -1 - 4   & = -2                             \\
  3 \cdot 1 + 1     & = 4                              \\
  -2                & \neq 4                           \\
\end{align*}

Encontramos desigualdad,
por lo tanto las rectas no se intersecan.

\textbf{c. Hallar ecuación de \(\pi\) que contenga a \(r_{2}\) y pase por \(P\)}.

Construimos un plano dada una recta y un punto, para ello construimos
un vector \((1,2,0) - (-3,-2,1) = (4,4,-1)\).

Obtenemos así la ecuación vectorial paramétrica del plano \(\pi\):

\begin{align*}
  \pi: (1,2,0) + \alpha(4,4,-1) + \beta(5,7,-3) \\
\end{align*}

\textbf{d. Hallar intersección entre plano \(\delta\) y recta \(r_{2}\)}

Planteamos un sistema de ecuaciones para obtener los puntos de la intersección:

\begin{align*}
   & \begin{cases}
       x + 2z = 6      \\
       x = 5\beta - 3  \\
       y = 7\beta - 2  \\
       z = -3\beta + 1 \\
     \end{cases}                                     \\
   & 5\beta - 3 + 2(-3\beta + 1) = 6                     \\
   & 5\beta - 3 -6\beta + 2 = 6                          \\
   & - 1 -\beta = 6                                      \\
   & \boxed{\beta = -7} \rightarrow \boxed{(-38,-51,22)} \\
\end{align*}

\textbf{2. Hallar
  \(z \in \mathbb{C}\), tal que \(\left[\frac{1-\sqrt{3i}}{2}\right]^{Z}=-i\).
  Expresar en forma exponencial y binómica.}

Operamos logaritmo natural en cada uno de los componentes de la expresión:
\begin{align*}
  \left[\frac{1-\sqrt{3}i}{2}\right]^{Z}        & =-i      \\
  \ln\left(\frac{1-\sqrt{3}i}{2}\right) \cdot Z & =\ln(-i) \\
\end{align*}

Para operar logaritmo de los dos números complejos,
necesitamos conocer sus ángulos respecto del eje real.
\begin{align*}
   & \alpha = \arctan\left(\frac{-\sqrt{3}}{\cancel{2}} \cdot \frac{\cancel{2}}{1}\right) = -\frac{1}{3}\pi\text{, por estar en 4to cuadrante }\rightarrow \frac{5}{3}\pi \\
   & \beta = \frac{3}{2}\pi\text{, por ser imaginario puro negativo.}                                                                                                     \\
\end{align*}

Expresamos ambos complejos en forma exponencial y operamos logaritmo natural:
\begin{align*}
  \ln\left(\left|\frac{1-\sqrt{3}i}{2}\right|\right)\cdot e^{i\frac{5}{3}\pi} \cdot z = \ln(|-i|) \cdot e^{i\frac{3}{2}\pi} \\
  \ln\left(\sqrt{\frac{1}{4} + \frac{3}{4}}\right)\cdot e^{i\frac{5}{3}\pi} \cdot z = \ln(1) \cdot e^{i\frac{3}{2}\pi}      \\
  \boxed{z = 0}                                                                                                             \\
\end{align*}
\end{document}
