\documentclass{article}
\usepackage[a4paper, margin=2.54cm]{geometry}
\usepackage[spanish]{babel}

% imágenes
%\usepackage{graphicx}               
%\graphicspath{{img}}

% fuentes de conjuntos numéricos
\usepackage{amsfonts}        

% math
\usepackage{amsmath, amssymb}

% gráficos y plots
%\usepackage{tikz}                   
%\usepackage{pgfplots}               
%\pgfplotsset{width=10cm, compat=1.9}

\setlength{\jot}{8pt}
\setlength{\parindent}{0cm}

% espacio entre párrafos
\usepackage{parskip}                

% cancelar términos
\usepackage{cancel}                 

% links
%\usepackage[colorlinks=true, 
%    urlcolor=blue]{hyperref}        

% shapes
%\usetikzlibrary{shapes.geometric}   

% incluir pdfs
%\usepackage{pdfpages}               

\title{Álgebra y geometría analítica\\Apunte teórico}
\author{Daniel Ise}
\date{Septiembre de 2024}

\begin{document}

\maketitle

\section*{Ecuaciones del plano}

\textbf{Ecuación general del plano.}
Si tomamos dos puntos del espacio, 
obtenemos una única \textbf{recta} que pasa por ellos, 
pero infinitos planos pasan por esa recta.

Dado un \textbf{vector dirección}, 
existen infinitos planos \textit{perpendiculares} al mismo.
Podemos determinarlo si tomamos además un \textbf{punto} perteneciente al plano.

Si queremos encontrar la ecuación del plano \(pi\), 
que pasa por \(P_0 (x_0,y_0,z_0)\), 
y es perpendicular al \textbf{vector normal} \(\vec{n} = (a,b,c)\),
preguntamos: ¿qué condición debe cumplir \(P (x, y, z)\) para estar en \(\pi\)?
El vector \(\overline{P_0P}\) debería ser \textit{perpendicular} al 
\textbf{vector normal} del plano:

\begin{align*}
    P (x, y, z) \in \iff \overline{P_0P} \perp \vec{n} \iff \overline{P_0P} \cdot \vec{n} = 0 \\
\end{align*}


\end{document}
