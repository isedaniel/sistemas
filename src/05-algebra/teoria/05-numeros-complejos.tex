\documentclass[12pt]{article}
\usepackage[a4paper, margin=2.54cm]{geometry}
\usepackage[spanish]{babel}

% imágenes
%\usepackage{graphicx}               
%\graphicspath{{img}}

% fuentes de conjuntos numéricos
\usepackage{amsfonts}        

% math
\usepackage{amsmath, amssymb}

% gráficos y plots
\usepackage{tikz}                   
%\usepackage{pgfplots}               
%\pgfplotsset{width=10cm, compat=1.9}
\usetikzlibrary{babel}

\setlength{\jot}{8pt}
\setlength{\parindent}{0cm}

% espacio entre párrafos
\usepackage{parskip}                

% cancelar términos
\usepackage{cancel}                 

% links
%\usepackage[colorlinks=true, 
%    urlcolor=blue]{hyperref}        

% shapes
%\usetikzlibrary{shapes.geometric}   

% incluir pdfs
%\usepackage{pdfpages}               

\title{Álgebra y geometría analítica\\Apunte teórico}
\author{Daniel Ise}
\date{Septiembre de 2024}

\begin{document}

\maketitle

\tableofcontents

\pagebreak

\section{Números complejos}

\subsection{Surgen por necesidad de ampliar números reales.}

Al resolver algunas ecuaciones de segundo grado,
podemos obtener \textbf{raíces negativas},
que \textit{no existen} en el conjunto de los \textbf{reales}.
Pero se pueden resolver recurriendo a \textbf{unidad imaginaria}.

\subsection{Unidad imaginaria.}

Es \(\sqrt{-1}\),
se representa con \(i\).
\textit{Permite operar} \textbf{raíces negativas}.
\begin{align*}
    i = \sqrt{-1}
\end{align*}

\textbf{Ejemplo.}

\begin{align*}
    x^2 - 4x + 13 = 0 \rightarrow x & = \frac{4 \pm \sqrt{4^{2} - 4 \cdot 1 \cdot 13}}{2} \\
    x                               & = \frac{4 \pm \sqrt{16 - 52}}{2}                    \\
    x                               & = \frac{4 \pm \sqrt{-36}}{2}                        \\
    x                               & = \frac{4 \pm \sqrt{36 \cdot (-1)}}{2}              \\
    x                               & = \frac{4 \pm \sqrt{36} \cdot \sqrt{(-1)}}{2}       \\
    x                               & = \frac{4 \pm 6 \cdot i}{2}                         \\
                                    & \boxed{x  = 2 \pm 3i}                               \\
\end{align*}

\subsection{Conjunto de los números complejos}

Un número complejo generalmente se denota con la letra \(z\).
El conjunto se denota \(\mathbb{C}\),
y se define:

\begin{align*}
    \mathbb{C} = \left\{z = a + bi; a,b \in \mathbb{R}, i = \sqrt{-1}\right\} \\
\end{align*}


\section{Formas de representación}

Un número complejo \(z\) puede representarse de diferentes formas.

\subsection{Forma binómica}

Se denomina \textbf{forma binómica} a la representación:

\begin{align*}
    z = a + bi \\
\end{align*}

Donde \(a\) es la \textbf{parte real} del número complejo,
y se denota \(a = Re(z)\).

\(b\) es la \textbf{parte imaginaria} y se denota \(b = Im(z)\).

Si \(a = 0\), es un \textbf{número imaginario puro}.

Si \(b = 0\), se reduce a \textbf{número real}.

Dos números complejos son \textbf{iguales} si tienen mismo
su \textbf{componente real} y \textbf{componente imaginario}.

\subsection{Forma par ordenado}

Un número complejo se puede representar como par ordenado:

\begin{align*}
    z = a + bi = (a,b) \\
\end{align*}

\textbf{Representación gráfica.}

Gráficamente, un par ordenado se puede representar como un \textbf{punto},
de coordenadas \((a,b)\); o como un \textbf{vector},
con \textbf{origen} en \((0,0)\) y \textbf{extremo} en \((a,b)\).

La parte real se representa en el eje de las \textit{abscisas} \(X\),
que se llama \textbf{eje real}.

La parte imaginaria se representa en el eje de las \textit{ordenadas} \(Y\),
que se llama \textbf{eje imaginario}.

\hfil
\begin{center}
    \begin{tikzpicture}[>=stealth,scale=0.6,line cap=round,
            bullet/.style={circle,inner sep=1.5pt,fill}]
        % axis
        \draw[<->] (-9,0) -- (9,0) node[right]{$x$};
        \draw[<->] (0,-7) -- (0,7) node[above]{$y$};
        % marcadores
        \draw (4, 0.1) -- ++ (0, -0.2) node[below]{$a$};
        \draw (0.1, 3) -- ++ (-0.2, 0) node[left]{$b$};
        % vector

        \path (0,0) coordinate (O)
        (4, 0) coordinate (A)
        (0, 3) coordinate (B)
        (4,3) node[bullet,label=right:{$(a,b)$}](P){}
        (O) edge[thick,->] (P)
        (A) edge[dashed] (P)
        (B) edge[dashed] (P);
    \end{tikzpicture}
\end{center}
\hfil

\subsection{Forma trigonométrica}

La \textbf{forma trigonométrica} de un número complejo se expresa como:
\begin{align*}
    z = |z|(\cos \alpha + i \sen \alpha) \\
\end{align*}

Donde \(|z| \in \mathbb{R}\) es el \textbf{módulo} del número complejo,
y se calcula como el módulo de un vector:
\begin{align*}
    z = a+bi \implies |z| = \sqrt{a^2 + b^2} \\
\end{align*}

Y \(\alpha\) es el \textbf{argumento},
un ángulo positivo con respecto al \textbf{eje real},
de medida \(0 \leq \alpha \leq 2\pi\).
Se denota \(arg(z) = \alpha\) y se calcula como:
\begin{align*}
    z = a + bi \implies arg(z) = \alpha = \arctan \left(\frac{b}{a}\right) \\
\end{align*}

\textbf{Ojo \(\arctan\) en calculadora},
prestar atención al \(z\).

\begin{center}
    \begin{tabular}{ c c c }
        \(z\)       & Cuadrante            & \(\arctan\)         \\
        \hline                                                   \\
        \(a + bi\)  & \(1^{er}\) cuadrante & \( \alpha \)        \\
        \(-a + bi\) & \(2^{do}\) cuadrante & \( \alpha + \pi \)  \\
        \(-a - bi\) & \(3^{er}\) cuadrante & \( \alpha + \pi \)  \\
        \(a - bi\)  & \(4^{to}\) cuadrante & \( \alpha + 2\pi \) \\
    \end{tabular}
\end{center}

\subsection{Forma exponencial}

Se deduce de la \textbf{fórmula de Euler}:

\begin{align*}
    e^{i \cdot \alpha} = \cos\alpha + i \sen\alpha \\
\end{align*}

Siendo \(\alpha\) el \textbf{ángulo en radianes}:

\begin{align*}
    180^o = \pi\text{ radianes} \\
\end{align*}

Si \(z\) de forma trigonométrica es:

\begin{align*}
    z = |z| \cdot (\cos\alpha + i\sen\alpha) \\
\end{align*}

Sustituyendo, lo podemos reescribir como:

\begin{align*}
    z = |z| \cdot e^{i \cdot \alpha} \\
\end{align*}

Esta manera de expresar \(z\) permite operar \(\ln\).

\section{Operaciones con números complejos}

\subsection{Potencias de la unidad imaginaria}

\begin{align*}
    i^0 = 1  \\
    i^1 = i  \\
    i^2 = -1 \\
    i^3 = -i \\
\end{align*}

Para resto de potencias, \(i^n = i^r \iff n \mod 4 = r\):
Dividimos \(n\) entre 4 y tomamos el Resto \(r\),
\(i^n\) es lo mismo que \(i^r\).

\textbf{Ejemplos.}

\begin{align*}
    i^{20} \implies 20 \mod 4 = 0 \\
    \boxed{i^{20} = i^0 = 1}      \\
\end{align*}

\begin{align*}
    i^{35} \implies 35 \mod 4 = 3 \\
    \boxed{i^{35} = i^3 = -i}     \\
\end{align*}

\subsection{Suma y resta de números complejos}

\subsubsection{Forma binómica}

Dado un \(z_0,z_1 \in \mathbb{C}\),
expresados de \textbf{forma binómica},
se suman y restan sus partes reales e imaginarias,
respectivamente.

\textbf{Ejemplos.}
\begin{align*}
    (5 - 8i) + (7 + 6i) = (5 + 7) + (6i - 8i) = 12 - 2i                    \\
    (5 + 2i) + (-8 + 3i) - (4-2i) = (5 - 8 - 4) + (2i + 3i + 2i) = -7 + 7i \\
\end{align*}

\subsubsection{Forma trigonométrica}

En \textbf{forma trigonométrica},
\(z \in \mathbb{C}\) no se puede sumar ni restar.

\subsection{Multiplicación de números complejos}

\subsubsection{Forma binómica}

Dados dos números complejos,
expresados de \textbf{forma binómica},
una multiplicación se opera como el producto de un binomio,
recordando que \(i^2 = -1\).

\textbf{Ejemplos.}
\begin{align*}
    (2 + 3i) \cdot (4+5i) & = 8 + 12i + 10i + 15i^2   \\
                          & = 8 + 22i + (15 \cdot -1) \\
                          & = \boxed{-7 + 22i}
\end{align*}

\begin{align*}
    (5-2i)^2 + 2(1-i) \cdot (-3 + 4i) & = 25 - 20i + 4i^2 + (2 - 2i) \cdot (-3 + 4i) \\
                                      & = 21 - 20i - 6 + 6i + 8i - 8i^2              \\
                                      & = \boxed{23 - 6i}                            \\
\end{align*}

\textbf{Producto de complejos conjugados.}

El \textbf{complejo conjugado} de un \(z\)
es otro número complejo de parte real igual y parte imaginaria opuesta.
Se denota \(\overline{z}\):

\begin{align*}
    z = a + bi \implies \overline{z} = a - bi \\
\end{align*}

El producto \(z \cdot \overline{z}\) devuelve un \textbf{número real}:

\begin{align*}
    z \cdot \overline{z} & = (a + bi) cdot (a - bi)                      \\
                         & = a^2 + \cancel{ abi} - \cancel{abi} - b^2i^2 \\
                         & = a^2 - b^2 \cdot -1                          \\
                         & = \boxed{a^2 + b^2}                           \\
\end{align*}

\subsubsection{Forma trigonométrica}

El producto de \(z_{1} \cdot z_{2}\),
en \textbf{forma trigonométrica},
consiste en \textit{multiplicar} los \textbf{módulos} y
\textit{sumar} los \textbf{argumentos}:

\begin{align*}
    \text{Dados: } &                                                                                     \\
                   & z_{1} = |z_{1}|(\cos\alpha + i\sen\alpha) & z_{2} = |z_{2}|(\cos\beta + i\sen\beta) \\
\end{align*}
\begin{align*}
    z_{1} \cdot z_{2} = |z_{1}| \cdot |z_{2}| \cdot [\cos(\alpha+\beta) + i\sen(\alpha+\beta)] \\
\end{align*}

De ello deducimos que la \textbf{potencia natural} \(n\) de \(z\)
consiste en \textit{elevar} \(|z|^n\)
y \textit{multiplicar} \(\alpha \cdot n\):

\begin{align*}
    z_{1}^n = |z_{1}|^n \cdot [\cos(n \cdot \alpha) + i\sen(n \cdot \alpha)] \\
\end{align*}

\subsection{División de números complejos}

\subsubsection{Forma binómica}

Dados \(z_{1}, z_{2}\),
expresados de \textbf{forma binómica},
para dividirlos operamos producto de numerador y denominador por
\textbf{conjugado} del denominador.
De esta manera, sacamos el número complejo del denominador y podemos operar:

\begin{align*}
    \frac{a + bi}{c + di} & = \frac{a + bi}{c + di} \cdot \frac{c - di}{c - di} \\
                          & = \frac{ac + cbi - adi - bdi^2}{c^2 + d^2}          \\
                          & = \frac{(ac + bd) + (cb - ad)i}{c^2 + d^2}          \\
\end{align*}

\textbf{Ejemplo.}
\begin{align*}
    \frac{2+6i}{7-3i} & = \frac{2+6i}{7-3i} \cdot \frac{7+3i}{7+3i} \\
                      & = \frac{14 + 42i + 6i - 18}{7^{2} + 3^{2}}  \\
                      & = \frac{-4 + 48i}{49 + 9}                   \\
                      & = \boxed{\frac{-2}{29} + \frac{24}{29}i}    \\
\end{align*}

\subsubsection{Forma trigonométrica}

El cociente de \(z_{1} \div z_{2}\),
en \textbf{forma trigonométrica},
consiste en \textit{dividir} los \textbf{módulos} y
\textit{restar} los \textbf{argumentos}:

\begin{align*}
    \text{Dados: } &                                                                                     \\
                   & z_{1} = |z_{1}|(\cos\alpha + i\sen\alpha) & z_{2} = |z_{2}|(\cos\beta + i\sen\beta) \\
\end{align*}
\begin{align*}
    z_{1} : z_{2} = |z_{1}| : |z_{2}| \cdot [\cos(\alpha-\beta) + i\sen(\alpha-\beta)] \\
\end{align*}

\subsection{Raíz n-ésima de números complejos en forma trigonométrica}

Todo complejo,
\textbf{excepto el cero},
tiene \textit{exactamente} n raíces n-ésimas diferentes.

Se obtienen siguiendo:

\begin{align*}
    \sqrt[n]{Z} = \sqrt[n]{|W|} \cdot \left(\cos \frac{\alpha + 2k\pi}{n} + i \sen \frac{\alpha + 2k\pi}{n}\right) \text{ con } k=0,1,2 \dots n-1
\end{align*}

\textbf{Representación gráfica.}

\hfil
\begin{center}
    \begin{tikzpicture}[>=stealth,scale=4,line cap=round,
            bullet/.style={circle,inner sep=1.5pt,fill}]
        % axis
        \draw[<->] (-1.1,0) -- (1.1,0) node[right]{$x$};
        \draw[<->] (0,-1.1) -- (0,1.1) node[above]{$y$};

        % vector
        \path (0,0) coordinate (O)
        (1.05, 0.13) coordinate (K0)
        (.4, .97) coordinate (K1)
        (-.64, .84) coordinate (K2)
        (-1.05, -.13) coordinate (K3)
        (-.4, -.97) coordinate (K4)
        (.64, -.84) coordinate (K5)
        (K0) node[bullet]{}
        (K1) node[bullet]{}
        (K2) node[bullet]{}
        (K3) node[bullet]{}
        (K4) node[bullet]{}
        (K5) node[bullet]{}
        (O) edge[dashed] (K0)
        (O) edge[dashed] (K1)
        (O) edge[dashed] (K2)
        (O) edge[dashed] (K3)
        (O) edge[dashed] (K4)
        (O) edge[dashed] (K5)
        [draw=gray] (O) circle (1.06);
    \end{tikzpicture}
\end{center}
\hfil

Generalmente, uniendo los puntos queda un \textbf{polígono regular},
de \(n\) lados, inscripto en una circunferencia de radio \(\sqrt[n]{|z|}\).

\subsection{Logaritmo de número complejo en forma exponencial}

Siendo \(z = |z| \cdot e^{i \cdot \alpha}\):

\begin{align*}
    \ln z & = \ln(|z| \cdot e^{i \cdot \alpha})      \\
          & = \ln(|z|) + \ln(e^{i \cdot \alpha})     \\
          & = \ln(|z|) + i \cdot \alpha \cdot \ln(e) \\
          & = \boxed{\ln(|z|) + \alpha i}            \\
\end{align*}

\subsection{Exponencial compleja}

Dados \(v, u \in \mathbb{C}\), si \(z = v^u\):

\begin{align*}
    z     & = v^u     \\
    \ln z & = \ln v^u \\
    \ln z & = u \ln v \implies v^u = e^{u \ln v} \\
\end{align*}

\end{document}
