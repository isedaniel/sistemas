\documentclass[12pt]{article}
\usepackage[a4paper, margin=2.54cm]{geometry}
\usepackage[spanish]{babel}

% imágenes
%\usepackage{graphicx}               
%\graphicspath{{img}}

% fuentes de conjuntos numéricos
\usepackage{amsfonts}        

% math
\usepackage{amsmath, amssymb}

% gráficos y plots
\usepackage{tikz}                   
%\usepackage{pgfplots}               
%\pgfplotsset{width=10cm, compat=1.9}
\usetikzlibrary{babel}

\setlength{\jot}{8pt}
\setlength{\parindent}{0cm}

% espacio entre párrafos
\usepackage{parskip}                

% cancelar términos
\usepackage{cancel}                 

% links
%\usepackage[colorlinks=true, 
%    urlcolor=blue]{hyperref}        

% shapes
%\usetikzlibrary{shapes.geometric}   

% incluir pdfs
%\usepackage{pdfpages}               

\title{Álgebra y geometría analítica\\Apunte teórico}
\author{Daniel Ise}
\date{Septiembre de 2024}

\begin{document}

\maketitle

\tableofcontents

\pagebreak

\section{Definición de matriz}

\textbf{Matriz.}
Una matriz \(A \text{ de }m \times n\)
es un \textit{ordenamiento rectagular} de \textbf{escalares},
dispuestos en \(m\) filas y \(n\) columnas.
A cada escalar \(a\) le corresponde un \textit{doble subíndice} \(m\cdot n\),
que indica número de fila y columna:

\begin{align*}
  A = \begin{pmatrix}
        a_{11} & a_{12} & \cdots & a_{1n} \\
        a_{21} & a_{22} & \cdots & a_{2n} \\
        \vdots & \vdots & \vdots & \vdots \\
        a_{m1} & a_{m2} & \cdots & a_{mn} \\
      \end{pmatrix}
\end{align*}

\textbf{Notación.}
Se suelen designar con \textit{letras mayúsculas},
indicando conjunto numérico al que pertenecen escalares
y superíndice \(m \times n\):

\begin{align*}
  A \in \mathbb{R}^{m \times n} \\
\end{align*}

\textbf{Elementos.}
Cada uno de los números que forma parte de la matriz.
Un elemento se distingue de otro por la \textbf{posición} que ocupa.

\section{Matrices particulares}

\subsection{Matriz columna}

Matriz de forma \(C \in \mathbb{R}^{m \times 1}\).
Los \textbf{vectores} se pueden pensar como
casos particulares de este tipo de matriz.

\begin{align*}
  C = \begin{pmatrix}
        c_{11} \\
        c_{21} \\
        \vdots \\
        c_{m1} \\
      \end{pmatrix}
\end{align*}

\subsection{Matriz fila}

Matriz de forma \(F \in \mathbb{R}^{1 \times n}\).

\begin{align*}
  F = \begin{pmatrix}
        f_{11} & f_{12} & \cdots & f_{1n} \\
      \end{pmatrix}
\end{align*}

\subsection{Matriz nula}

Matriz cuyos \textit{elementos} son \textit{todos iguales a cero.}
Se simboliza \(\mathbf{0} \in \mathbb{R}^{m \times n}\).

\begin{align*}
  \mathbf{0}_{3 \times 3} = \begin{pmatrix}
                              0 & 0 & 0 \\
                              0 & 0 & 0 \\
                              0 & 0 & 0 \\
                            \end{pmatrix}
\end{align*}

\subsection{Matriz cuadrada}

Matriz con mismo número de filas y columnas: \(m = n\).
Los elementos de forma \(a_{ii}\) forman la \textbf{diagonal principal}.

\begin{align*}
  Q_{4 \times 4} = \begin{pmatrix}
                     1  & 3 & 3 & -5 \\
                     3  & 8 & 5 & 4  \\
                     4  & 7 & 8 & 0  \\
                     -5 & 4 & 0 & 7  \\
                   \end{pmatrix}
\end{align*}

\subsection{Matriz traspuesta}

Dada la matriz \(B\),
llamamos matriz traspuesta \(B^{T}\) a aquella matriz que se obtiene
intercambiando filas por columnas: \(m \times n \rightarrow n \times m\).

\begin{align*}
  B_{2 \times 3} = \begin{pmatrix}
                     1 & 2 & 3 \\
                     4 & 5 & 6 \\
                   \end{pmatrix} & \rightarrow
  B^{T}_{3 \times 2} = \begin{pmatrix}
                         1 & 4 \\
                         2 & 5 \\
                         3 & 6 \\
                       \end{pmatrix}
\end{align*}

\textbf{Propiedades de la trasposición.}

\begin{enumerate}
  \item \((A+B)^{T} = A^{T} + B^{T}\)
  \item \((kA)^{T} = kA^{T}\)
  \item \((A^{T})^{T} = A\)
  \item \((A \cdot B)^{T} = B^{T} \cdot A^{T}\)
\end{enumerate}

\subsection{Matriz simétrica}

Decimos de la matriz \(S \in \mathbb{R}^{m \times m}\) que es simétrica
si y solo si \(S = S^{T}\):

\begin{align*}
  S_{3 \times 3} = \begin{pmatrix}
                     1 & 3 & 4 \\
                     3 & 7 & 9 \\
                     4 & 9 & 5 \\
                   \end{pmatrix} & =
  S^{T}_{3 \times 3} = \begin{pmatrix}
                         1 & 3 & 4 \\
                         3 & 7 & 9 \\
                         4 & 9 & 5 \\
                       \end{pmatrix} \\
\end{align*}

\subsection{Matrices triangulares}

\textbf{Matriz triangular superior.}
\(T \in \mathbb{R}^{n \times n}\) es una matriz triangular superior
cuando los elementos bajo la \textit{diagonal principal} son iguales a 0.

\begin{align*}
  T = \begin{pmatrix}
        1 & 3 & -5 \\
        0 & 8 & 4  \\
        0 & 0 & 7  \\
      \end{pmatrix} & \rightarrow \text{ Si } i > j \implies a_{ij} = 0 \\
\end{align*}

\textbf{Matriz triangular inferior.}
\(U \in \mathbb{R}^{n \times n}\) es una matriz triangular inferior si
todos los elementos sobre la \textit{diagonal principal} son iguales a 0.

\begin{align*}
  U = \begin{pmatrix}
        1  & 0 & 0 \\
        2  & 8 & 0 \\
        -7 & 9 & 7 \\
      \end{pmatrix} & \rightarrow \text{ Si } i < j \implies a_{ij} = 0 \\
\end{align*}

\subsection{Matriz diagonal}
\(D \in \mathbb{R}^{n \times n}\) es una matriz diagonal si es,
a la vez,
triangular superior e inferior:

\begin{align*}
  D = \begin{pmatrix}
        1 & 0 & 0 \\
        0 & 8 & 0 \\
        0 & 0 & 7 \\
      \end{pmatrix} & \implies a_{ij} = 0 \text{, } \forall i \neq j \\
\end{align*}

\subsection{Matriz identidad}
Simbolizada con \(I\),
es una matriz diagonal \(I \in \mathbb{R}^{n \times n}\),
con todos sus elementos en la diagonal principal iguales a \(1\).

\begin{align*}
  I = \begin{pmatrix}
        1 & 0 & 0 \\
        0 & 1 & 0 \\
        0 & 0 & 1 \\
      \end{pmatrix}
\end{align*}


\subsection{Matrices iguales}
Dos matrices \(A\) y \(B\) son iguales,
si y solo si,
todos sus elementos respectivos son iguales,
es decir:

\begin{align*}
  A, B \in \mathbb{R}^{m \times n}, A = B \iff a_{ij} = b_{ij} \text{ } \forall i,j \\
\end{align*}

\section{Operaciones con matrices}

\subsection{Suma de matrices}

Dadas dos matrices \(A, B \in \mathbb{R}^{m \times n}\),
su suma \(C \in \mathbb{R}^{m \times n}\)
es la suma  de sus respectivos elementos:

\begin{align*}
  A + B = C \iff c_{ij} = a_{ij} + b_{ij} \text{ } \forall i,j \\
\end{align*}

\textbf{Propiedades de la suma de matrices.}

\begin{enumerate}
  \item Conmutativa: \(A + B = B + A\)
  \item Asociativa: \((A + B) + C = A + (B + C)\)
  \item Elemento neutro de la suma: \(A + \mathbf{0} = A\)
  \item \(A - A = \mathbf{0}\)
\end{enumerate}

\subsection{Producto de un escalar por matriz}

Dada la matriz \(A \in \mathbb{R}^{m \times n}\),
su multiplicación por \(k \in \mathbb{R}\) es la
multiplicación de cada uno de sus elementos por \(k\).

\begin{align*}
  A \cdot k = P \iff p_{ij} = a_{ij} \cdot k \text{ } \forall i,j \\
\end{align*}

\textbf{Propiedades del producto por un escalar.}

\begin{enumerate}
  \item Distributiva: \(k(A + B) = kA + kB\)
  \item Distributiva 2: \((k + l) \cdot A = kA + lA\)
  \item Asociativa: \(k(lA) = (kl)A\)
  \item Elemento neutro del producto escalar: \(1A = A\)
\end{enumerate}

\subsection{Resta de matrices}

Siendo \(A, B \in \mathbb{R}^{m \times n}\),
definimos la resta de \(A - B\) como la suma de \(A\) y la opuesta de \(B\),
\(B \cdot -1\)

\subsection{Producto de matrices}

\textbf{Definición.}
Es el producto de cada fila de una matriz por las columnas de la otra.

\textbf{Producto de matriz fila por matriz columna.}
Siendo \(F \in \mathbb{R}^{1 \times n}\),
esto es,
una matriz fila,
y \(C \in \mathbb{R}^{n \times 1}\),
una matriz columna,
definimos su producto como:

\begin{align*}
  FC = \begin{pmatrix}
         f_{1} & f_{2} & \cdots & f_{n}
       \end{pmatrix}
  \cdot \begin{pmatrix}
          c_{1}  \\
          c_{2}  \\
          \vdots \\
          c_{n}
        \end{pmatrix}
  = f_{1}c_{1} + f_{2}c_{2} + \cdots + f_{n}c_{n} \\
\end{align*}

Resultando en una matriz \(M \in \mathbb{R}^{1 \times 1}\).

\textbf{Admisión del producto.}
De ello deducimos que,
para que el producto sea posible,
la cantidad de columnas de la primera matriz debe ser igual a la cantidad
de filas de la segunda:

\begin{align*}
  A_{m \times n} \cdot B_{n \times p} = C_{m \times p} \iff c_{ij} = \text{fila }i(A) \cdot \text{columna }j(B) \\
\end{align*}

\textbf{Ejemplo.}

Sean:
\begin{align*}
  A = \begin{pmatrix}
        1 & 2 & 3 \\
        4 & 1 & 0 \\
      \end{pmatrix}
  \text{, }
  B = \begin{pmatrix}
        1 & 1  & 2 \\
        1 & -1 & 0 \\
        2 & 0  & 3 \\
      \end{pmatrix}
\end{align*}

El producto \(A \cdot B\) se puede hacer porque las columnas de \(A\) son
iguales a las filas de \(B\). Entonces:

\begin{align*}
  AB & = \begin{pmatrix}
           [1 \cdot 1 + 2 \cdot 1 + 3 \cdot 2] & [1 \cdot 1 + 2 \cdot (-1) + 3 \cdot 0] & [1 \cdot 2 + 2 \cdot 0 + 3 \cdot 3] \\
           [4 \cdot 1 + 1 \cdot 1 + 0 \cdot 2] & [4 \cdot 1 + 1 \cdot (-1) + 0 \cdot 0] & [4 \cdot 2 + 1 \cdot 0 + 0 \cdot 3] \\
         \end{pmatrix} \\
     & =
  \begin{pmatrix}
    9 & -1 & 11 \\
    5 & 3  & 8  \\
  \end{pmatrix}
\end{align*}

En cambio, \(B \cdot A\) no se puede calcular,
puesto que \(B\) tiene 3 columnas y \(A\) solo 2 filas.

\textbf{Propiedades del producto de matrices.}

\begin{enumerate}
  \item Asociativa: \((AB) \cdot C = A \cdot (BC)\).
  \item Asociativa con escalares: \((kA)B = k(AB) = A(kB)\).
  \item Distributiva a derecha: \((A+B)\cdot C = AC + BC\).
  \item Distributiva a izquierda: \(P(Q+R) = PQ + PR\).
  \item Todo producto por matriz nula es matriz nula:
        \(A\mathbf{0} = \mathbf{0}\).
  \item La \textbf{matriz identidad} es elemento neutro:
        \(AI = IA = A\)
  \item El producto de matrices \underline{no} es conmutativo:
        \(A \cdot B \neq B \cdot A\).
\end{enumerate}

\section{Determinantes}

\textbf{Función determinante.}
Tiene gran importancia en Álgebra,
ya que permite saber si una matriz es \textit{invertible}.
Una matriz tiene inversa si su determinante es \textit{distinto} de 0.
Se define para \textbf{matrices cuadradas}.
Se denota como \(det(A) = |A| = \Delta(A)\).

\subsection{Función determinante en matriz cuadrada \(2 \times 2\).}

Dada una matriz \(B \in \mathbb{R}^{2 \times 2}\):

\begin{align*}
  B = \begin{pmatrix}
        a_{11} & a_{12} \\
        a_{21} & a_{22} \\
      \end{pmatrix} \\
\end{align*}

Definimos su determinante como la \textit{resta} del \textit{producto} de
los \textbf{elementos de las diagonales}:

\begin{align*}
  |B| = \begin{vmatrix}
          a_{11} & a_{12} \\
          a_{21} & a_{22} \\
        \end{vmatrix}
  = a_{11} \cdot a_{22} - a_{21} \cdot a_{12}
\end{align*}

\subsection{Función determinante en matriz cuadrada \(3 \times 3\)}

Dada \(C \in \mathbb{R}^{3 \times 3}\):

\begin{align*}
  C = \begin{pmatrix}
        a_{11} & a_{12} & a_{13} \\
        a_{21} & a_{22} & a_{23} \\
        a_{31} & a_{32} & a_{33} \\
      \end{pmatrix} \\
\end{align*}

Calculamos el determinante mendiante \textbf{regla de Sarrus}.
Primero,
reescribimos matriz adjuntado primera y segunda fila (o columna) nuevamente:

\begin{align*}
  C = \begin{pmatrix}
        a_{11} & a_{12} & a_{13} & a_{11} & a_{12} \\
        a_{21} & a_{22} & a_{23} & a_{21} & a_{22} \\
        a_{31} & a_{32} & a_{33} & a_{31} & a_{32} \\
      \end{pmatrix}
   & \text{, o }
  C = \begin{pmatrix}
        a_{11} & a_{12} & a_{13} \\
        a_{21} & a_{22} & a_{23} \\
        a_{31} & a_{32} & a_{33} \\
        a_{11} & a_{12} & a_{13} \\
        a_{21} & a_{22} & a_{23} \\
      \end{pmatrix} \\
\end{align*}

Y se multiplica en diagonal.

\subsection{Submatriz y adjunto de un elemento}

Podemos obtener el determinante de una matriz cuadrada recurriendo
a los conceptos de \textbf{submatriz} y \textbf{adjunto de un elemento}.

\textbf{Submatriz.}
Dada la matriz cuadrada \(A \in \mathbb{R}^{n \times n}\),
decimos que su submatriz \(A_{ij}\) es la matriz que suprime su elementos
en la fila \(i\) y la columna \(j\).

\textbf{Ejemplo.}

Dada:

\begin{align*}
  A = \begin{pmatrix}
        1 & 2  & 3 \\
        0 & 8  & 1 \\
        0 & -4 & 7 \\
      \end{pmatrix} \\
\end{align*}

La submatriz \(A_{23}\) sería:

\begin{align*}
  A_{23} = \begin{pmatrix}
             1 & 2  \\
             0 & -4 \\
           \end{pmatrix} \\
\end{align*}

\textbf{Adjunto de un elemento \(a_{ij}\).}
Es el determinante de la \textbf{submatriz} \(A_{ij}\).
Positivo si \(i + j\) es par, negativo si es impar.

\begin{align*}
  adj(a_{ij}) = (-1)^{i+j} \cdot |A_{ij}| \\
\end{align*}

\textbf{Determinante de matriz.}
Para calcular determinante de una matriz:
elegimos una fila o columna,
multiplicamos cada elemento por su adjunto,
y sumamos:

\begin{align*}
  A = \begin{pmatrix}
        6 & 0  & 4 \\
        1 & -1 & 5 \\
        4 & 2  & 0 \\
      \end{pmatrix} \\
\end{align*}

Elegimos fila 1:

\begin{align*}
  6 \cdot (-1)^{1 + 1} \cdot \begin{vmatrix}
                               -1 & 5 \\
                               2  & 0 \\
                             \end{vmatrix}
  +
  0 \cdot (-1)^{1 + 2} \cdot \begin{vmatrix}
                               1 & 5 \\
                               4 & 0 \\
                             \end{vmatrix}
  +
  4 \cdot (-1)^{1 + 3} \cdot \begin{vmatrix}
                               1 & -1 \\
                               4 & 2  \\
                             \end{vmatrix}                            \\
  6 \cdot 1 \cdot (-10) + 0 \cdot (-1) \cdot (-20) + 4 \cdot 1 \cdot 6 \\
  -60 + 0 + 24 = -36                                                   \\
\end{align*}

\subsection{Propiedades de los determinantes}

\textbf{Propiedad 1.} 
El determinante del producto de matrices es el producto de sus determinantes:
\begin{align*}
  |A \cdot B| = |A| \cdot |B| \\
\end{align*}

\textbf{Propiedad 2.}
El determinante de una matriz con alguna fila o columna de ceros es 0.

\textbf{Propiedad 3.}
Si multiplicamos una fila o columna por un escalar,
su determinante queda multiplicado por el mismo escalar:
\begin{align*}
  \begin{vmatrix}
    1 & 2\\
    2 & 1\\
  \end{vmatrix} = -1\\
\end{align*}

Multiplicamos segunda columna por 2:
\begin{align*}
  \begin{vmatrix}
    1 & 4\\
    2 & 2\\
  \end{vmatrix} = -2\\
\end{align*}

\textbf{Propiedad 4.}
Si multiplicamos \(n\) filas o columnas por \(k\), 
el determinante se multiplica por \(k^{n}\).

\textbf{Propiedad 5.}
Si cambiamos el orden de una fila o columna,
el determinante cambia de signo.

\textbf{Propiedad 7.}
Si una matriz es \textit{invertible},
el determinante de su inversa es la inversa de su determinante:
\begin{align*}
  |A^{-1}| = \frac{1}{|A|}\\
\end{align*}

\textbf{Propiedad 8.}
El determinante de una matriz es igual al de sus traspuesta:
\begin{align*}
  |A| = |A^{T}| \\
\end{align*}

\textbf{Propiedad 9.}
Si una fila o columna es \textbf{combinación lineal} de otra fila o columna,
su determinante es 0.

\textbf{Combinación lineal.} 
Operar producto escalar, suma o resta entre filas o columnas.

\textbf{Propiedad 10.}
El determinante permanece constante si a una fila o columna
se suman filas o columnas multiplicadas por números distintos de 0.

\textbf{Propiedad 11.}
El determinante de una \textbf{matriz triangular} es el producto
se los elementos de su diagonal.

\section{Inversa de una matriz}

Dada la matriz \(A \in \mathbb{R}^{n \times n}\),
decimos que \(A\) es \textit{invertible} si y solo si
existe una matriz \(A^{-1} \in \mathbb{R}^{n \times n}\)
tal que \(A \cdot A^{-1} = A^{-1} \cdot A = I\).

\textbf{Propiedades de la inversión.}

Siendo \(A, B \in \mathbb{R}^{n \times n}\) invertibles, entonces:
\begin{enumerate}
  \item \(AB\) es invertible: \((AB)^{-1} = B^{-1} \cdot A^{-1}\).
  \item \((kA)^{-1} = k^{-1} \cdot A^{-1}\), con \(k \neq 0\).
  \item La inversa de la trasposición de \(A\)
        es lo mismo que la trasposición de la inversa de \(A\):
        \((A^{T})^{-1} = (A^{-1})^{T}\)
\end{enumerate}

\subsection{Cálculo de la matriz inversa}

Para calcular la matriz inversa primero debemos definir la matriz adjunta.

\textbf{Matriz adjunta.}
Aquella que surge de \textit{reemplazar} cada elemento de una matriz por 
su \textbf{adjunto}.

\end{document}