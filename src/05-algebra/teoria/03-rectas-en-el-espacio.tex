\documentclass{article}
\usepackage[a4paper, margin=2.54cm]{geometry}
\usepackage[spanish]{babel}
%\usepackage{graphicx}               % imágenes
%\graphicspath{{img}}
\usepackage{amsfonts}               % fuentes de conjuntos numéricos
\usepackage{amsmath, amssymb}       % símbolos
%\usepackage{tikz}                   % gráficos
%\usepackage{pgfplots}               % plots
%\pgfplotsset{width=10cm, compat=1.9}
\setlength{\jot}{8pt}
\setlength{\parindent}{0cm}
\usepackage{parskip}                % espacio entre párrafos
\usepackage{cancel}                 % cancelar términos
%\usepackage[colorlinks=true, 
%    urlcolor=blue]{hyperref}        % links
%\usetikzlibrary{shapes.geometric}   % shapes
%\usepackage{pdfpages}               % incluir pdfs

\title{Álgebra y geometría analítica\\Apunte teórico. Rectas en el espacio.}
\author{Daniel Ise}
\date{Agosto de 2024}

\begin{document}

\maketitle

\textbf{Espacio.} Conjunto de \textbf{puntos}. El \textbf{plano} es un
\textit{subconjunto} del \textbf{espacio}. La \textbf{recta} es, a su vez, un
\textit{subconjunto} del \textbf{plano}.

\textbf{Ecuación vectorial de la recta en \(R^2\).}
Una \textbf{recta} es un conjunto de \textbf{puntos} en el \textbf{plano}.
Dados dos puntos, $P$ y $X$,
podemos definir el vector \(\overline{PX}\).
Si este es paralelo a \(\vec{v}\),
entonces \(\overline{PX}\) \textit{resulta} de multiplicar \(\vec{v}\) por un
escalar:

\begin{align*}
    \overline{PX}     & = \lambda \cdot \vec{v}             \\
    (x,y) - (x_1,y_1) & = \lambda \cdot \vec{v}             \\
    (x,y)             & = (x_1,y_1) + \lambda \cdot \vec{v}
\end{align*}

Entonces, dados un punto $P (x_1,y_1)$ y un vector \(\vec{v}\),
siendo \(\lambda\) variable,
el conjunto de puntos \(X (x,y)\) que resuelven la igualdad definen la recta \(r\).

\textbf{Ecuación vectorial de la recta en \(R^3\).}
Dados un vector \(\vec{v} = (v_x,v_y,v_z)\) y un punto \(P = (x_1, y_1, z_1)\),
una recta \(r\) en \(R^3\) se expresa:

\begin{align*}
    (x, y, z) & = (x_1,y_1,z_1) + \lambda \cdot (v_x, v_y, v_z)
\end{align*}

\textbf{Ecuaciones paramétricas de la recta.}
Dada la ecuación vectorial de una recta \(r\),
se puede construir un sistema de ecuaciones:

\begin{align*}
    (x, y, z) & = (x_1,y_1,z_1) + \lambda \cdot (v_x, v_y, v_z)                               \\
    (x, y, z) & = (x_1,y_1,z_1) + (\lambda \cdot v_x, \lambda \cdot v_y,\lambda \cdot v_z)    \\
    (x, y, z) & = (x_1 + \lambda \cdot v_x, y_1 + \lambda \cdot v_y, z_1 + \lambda \cdot v_z)
\end{align*}

Expresado como sistema:

\begin{align*}
    \begin{cases}
        x = x_1 + \lambda \cdot v_x \\
        y = y_1 + \lambda \cdot v_y \\
        z = z_1 + \lambda \cdot v_z
    \end{cases}
\end{align*}

\textbf{Ecuaciones simétricas de la recta.}
Despejando \(\lambda\) de cada una de las ecuaciones paramétricas:

\begin{align*}
    \lambda & = \frac{x - x_1}{v_x} & \lambda & = \frac{y - y_1}{v_y} & \lambda & = \frac{z - z_1}{v_z}
\end{align*}

Igualamos:

\begin{align*}
    \frac{x - x_1}{v_x} = \frac{y - y_1}{v_y} = \frac{z - z_1}{v_z}
\end{align*}

Si y solo si todos los componentes de \(\vec{v} \neq 0\)

Si por ejemplo \(x \in \vec{v} = 0\) entonces:

\begin{align*}
    \frac{y - y_1}{v_y} = \frac{z - z_1}{v_z}; x = x_1
\end{align*}

\textbf{Posiciones relativas de dos rectas en el espacio.}
En el \textbf{plano} las rectas pueden adoptar dos posiciones relativas:
\textbf{intersecarse} en un punto o ser \textbf{paralelas}
(sean o no coincidentes: \(f(x) = g(x)\)).

En el \textbf{espacio} las rectas pueden adoptar tres posiciones:

\begin{enumerate}
    \item \textbf{paralelas}, misma dirección, sean o no coincidentes.
    \item \textbf{incidentes}, se intersecan.
    \item \textbf{alabeadas}, no se cortan ni son paralelas,
          \textit{por ende}, pertenecen a \textbf{planos} diferentes.
\end{enumerate}

\textbf{Punto de intersección de dos rectas en el espacio.}
Se obtiene con las \textbf{ecuaciones paramétricas}.
Se igualan los componentes en $x$, $y$ y $z$.
Por ejemplo, dadas rectas \(r\) y \(s\):

\begin{align*}
    r                        & =
    \begin{cases}
        x = x_1 + \lambda u_x \\
        y = y_1 + \lambda u_y \\
        z = z_1 + \lambda u_z
    \end{cases} &
    s                        & =
    \begin{cases}
        x = x_2 + \mu v_x \\
        y = y_2 + \mu v_y \\
        z = z_2 + \mu v_z
    \end{cases}
\end{align*}

\begin{align*}
    r \cap s & =
    \begin{cases}
        x_1 + \lambda u_x = x_2 + \mu v_x \\
        y_1 + \lambda u_y = y_2 + \mu v_y \\
        z_1 + \lambda u_z = z_2 + \mu v_z
    \end{cases}
\end{align*}

Se opera el sistema de ecuaciones y si no encontramos contradicciones,
se concluye que $s$ y $r$ se intersecan.
Sustituyendo $\lambda$ y $\mu$ se obtiene el \textbf{punto de intersección}.

\textbf{Ejercicios.}

Dadas \(r\) y \(s\) hallar intersección.

\begin{align*}
    r: \frac{x-4}{3} = \frac{y-4}{2} = z - 4 \\
    s: x = \frac{y}{2} = \frac{z}{3}         \\
\end{align*}

\begin{align*}
    r                    & =
    \begin{cases}
        x = 3 \lambda + 4 \\
        y = 2 \lambda + 4 \\
        z = \lambda + 4   \\
    \end{cases} &
    s =
    \begin{cases}
        x = \mu   \\
        y = 2 \mu \\
        z = 3 \mu \\
    \end{cases}             \\
\end{align*}

\begin{align*}
    r \cap s                & =
    \begin{cases}
        \mu = 3 \lambda + 4   \\
        2 \mu = 2 \lambda + 4 \\
        3 \mu = \lambda + 4   \\
    \end{cases}                          \\
    2 \cdot (3 \lambda + 4) & = 2 \lambda + 4      \\
    3 \lambda + 4           & = \lambda + 2        \\
    2 \lambda               & = -2                 \\
                            & \boxed{\lambda = -1} \\
                            & \boxed{\mu = 1}      \\
\end{align*}

Opero en la que me queda para chequear:

\begin{align*}
    3 \cdot 1 = -1 + 4 \\
    \boxed{3 = 3}      \\
\end{align*}

Ta bien. Intersecan en \(P (1,2,3)\)

Para determinar ángulo: \(\angle rs = \angle(\vec{v_r}\vec{v_s})\)

\begin{align*}
    \vec{v_r} = (3,2,1) & \text{  y  } \vec{v_s} = (1,2,3)                                                                                           \\
    \alpha              & = \arccos \left[\frac{(3 \cdot 1)+(2 \cdot 2)+(1 \cdot 3)}{\sqrt{1^{2}+2^{2}+3^{2}} \cdot \sqrt{3^{2}+2^{2}+1^{2}}}\right] \\
    \alpha              & = \arccos \left[\frac{3+4+3}{\sqrt{14} \cdot \sqrt{14}}\right]                                                             \\
    \alpha              & = \arccos \left[\frac{10}{14}\right]                                                                                       \\
    \alpha              & = \arccos \left[\frac{5}{7}\right]                                                                                         \\
                        & \boxed{\alpha \approxeq 44,415^o}
\end{align*}

\subsection*{Ejercicio 3}

Ver si existe y hallar intersección de:

\begin{align*}
    r: (3, 2, -1) + \lambda (0,-1,5) \\
    s: (-1, 2, 4) + \beta (2,-3,1)   \\
\end{align*}

Las rectas están en ecuación vectorial,
primero expresamos ecuaciones paramétricas:

\begin{align*}
    r =                  &
    \begin{cases}
        x = 3             \\
        y = 2 - \lambda   \\
        z = 5 \lambda - 1 \\
    \end{cases} &
    s =
    \begin{cases}
        x = 2 \beta - 1 \\
        y = 2 - 3 \beta \\
        z = \beta + 4   \\
    \end{cases}
\end{align*}

Teniendo las paramétricas, igualamos:

\begin{align*}
    r \cap s & =
    \begin{cases}
        3 = 2 \beta - 1           \\
        2 - \lambda = 2 - 3 \beta \\
        5 \lambda - 1 = \beta + 4 \\
    \end{cases}
\end{align*}

Sistema de 3 ecuaciones, 2 incógnitas, tomamos 2 y operamos:

\begin{align*}
    4 = 2 \beta \rightarrow \beta = 2 \\
    2 - \lambda = 2 - 3 \cdot 2       \\
    2 - \lambda = 2 - 6               \\
    -\lambda = -6                     \\
    \boxed{\lambda = 6}               \\
\end{align*}

Corroboramos con la que queda:

\begin{align*}
    5 \lambda - 1 = \beta + 4 \\
\end{align*}

\textbf{Distancia entre un punto y una recta en el espacio.}
Dado un punto \(P \in R^3\), una recta \(r\)
y su vector director \(\vec{u}\),
la \textbf{distancia} entre \(P\) y \(r\) \(\overline{PQ}\),
es perperdicular a \(r\). Entonces:

\begin{align*}
    \sen A = \frac{\overline{PQ}}{|\overline{AP}|} \\
    \overline{PQ} = |\overline{AP}| \cdot \sen A   \\
\end{align*}

El módulo del producto vectorial de \(\overline{AP}\) y \(\vec{u}\) es:

\begin{align*}
    |\overline{AP} \times \vec{u}| = |\overline{AP}| \cdot |\vec{u}| \cdot \sen A  \\
    \text{si despejo para igualar con la expresión anterior}                      \\
    \sen A \cdot |\overline{AP} = \frac{|\overline{AP} \times \vec{u}|}{|\vec{u}|} \\
\end{align*}

Igualamos:

\begin{align*}
    \overline{PQ} & = \frac{|\overline{AP} \cdot \vec{u}}{\vec{u}}
\end{align*}

\end{document}