\documentclass[12pt]{article}
\usepackage[a4paper, margin=2.54cm]{geometry}

% español
\usepackage[spanish]{babel}

% imágenes
%\usepackage{graphicx}
%\graphicspath{{img}}

% fuentes de conjuntos numéricos
\usepackage{amsfonts}

% símbolos
\usepackage{amsmath, amssymb}

% gráficos
%\usepackage{tikz}

% plots
%\usepackage{pgfplots}
%\pgfplotsset{width=10cm, compat=1.9}

% averiguar
\setlength{\jot}{8pt}
\setlength{\parindent}{0cm}

% espacio entre párrafos
\usepackage[skip=10pt plus1pt]{parskip}

% cancelar términos
\usepackage{cancel}

% links
%\usepackage[colorlinks=true, 
%    urlcolor=blue]{hyperref}

% shapes
%\usetikzlibrary{shapes.geometric}

% incluir pdfs
%\usepackage{pdfpages}

\title{Álgebra y geometría analítica\\Apunte de clase}
\author{Daniel Ise}
\date{2 de octubre de 2024}

\begin{document}

\maketitle

\tableofcontents

\section{Combinación lineal}

Dados dos vectores no paralelos,
podemos generar cualquier vector en el plano
como combinación lineal de los dos primeros.

Y esa combinación es \textit{única}.

Cuando los vectores son coplanares, 
la combinación lineal enfrenta algunas restricciones.

Si son coplanares, puedo generar planos.

Si son paralelos, puedo generar vectores dentro de la recta.

La dimensión del vector que puede resultar de la combinación
es igual al \textbf{rango} de la matriz que puedo construir
a partir del sistema de ecuaciones que se deduce de los 
vectores.

Para saber si un vector está dentro de un sistema,
llego a la solución y opero con los valores del vector,
si es compatible,
debería respetar la igualdad dada por el sistema.

\section{Conjunto de generadores}

Hay conjuntos de vectores que generan un espacio vectorial y otros que no.
Dos vectores no paralelos, generan \(R^{2}\).

Por tanto,
el conjunto de vectores no paralelos/no coplanares es un 
\textbf{conjunto generador} de un espacio vectorial.

\section{Subespacio generado}

Dado el conjunto de vectores \(A\),
del espacio \(V\),
el conjunto \(A\) es generador del subespacio \(W\),
pero no a \(V\).
Está en los ejemplos 5 y 6.

\textbf{Ejemplo.}
(2,1) genera una recta con dirección (2,1).
Ese es un subespacio vectorial.
Es la recta que pasa por el origen con dirección (2,1).

\textbf{Ejemplo II.}
(1,0,2) y (0,0,1). Generan el plano \(y = 0\).

\textbf{Ejemplo III.}

\end{document}

