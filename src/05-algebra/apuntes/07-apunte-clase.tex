\documentclass{article}
\usepackage[a4paper, margin=2.54cm]{geometry}

% español
\usepackage[spanish]{babel}

% imágenes
%\usepackage{graphicx}               
%\graphicspath{{img}}

% fuentes de conjuntos numéricos
\usepackage{amsfonts}               

% símbolos
\usepackage{amsmath, amssymb}       

% gráficos
%\usepackage{tikz}                   

% plots
%\usepackage{pgfplots}               
%\pgfplotsset{width=10cm, compat=1.9}

% averiguar
\setlength{\jot}{8pt}
\setlength{\parindent}{0cm}

% espacio entre párrafos
\usepackage[skip=8pt plus1pt]{parskip}                

% cancelar términos
\usepackage{cancel}                 

% links
%\usepackage[colorlinks=true, 
%    urlcolor=blue]{hyperref}        

% shapes
%\usetikzlibrary{shapes.geometric}   

% incluir pdfs
%\usepackage{pdfpages}               

\title{Álgebra y geometría analítica\\Apunte de clase}
\author{Daniel Ise}
\date{2 de septiembre de 2024}

\begin{document}

\maketitle

\section*{Números complejos}

Hasta ahora trabajamos conjunto \(\mathbb{R}\).
Ahora pasamos al conjunto más amplio: \(\mathbb{C}\).

\textbf{Unidad imaginaria.}
\(i = \sqrt{-1}\).

\textbf{Forma binómica de un número complejo.}
\(z = a + b i\)

\textbf{Igualdad.}
Cuando la parte real es igual y parte imaginaria igual.

\textbf{Conjugado de un complejo.}
Parte real sigue igual, se cambia signo a la parte imaginaria.

\textbf{Como par ordenado.}
\(z = a + b i = (a,b)\)

\textbf{Representación en el plano.}
El número complejo se puede representar como un punto en el plano.
X es la parte real, Y es la parte imaginaria.

\textbf{Potencias de unidad imaginaria.}
\begin{align*}
    i^0 = 1             \\
    i^1 = 1             \\
    i^2 = -1            \\
    i^3 = -i            \\
    i^4 = 1             \\
    i^5 = 1 \cdot i = i \\
\end{align*}
Van cíclicamente \((1, -1 , -i, 1, i, \dots)\)
Hay un \textbf{algoritmo} para resolver.

\section*{Operaciones en forma binómica}

\textbf{Suma y resta de complejos.}
Se opera parte real con parte real,
parte imaginaria con parte imaginaria.

\textbf{Multiplicación.}
Sabiendo que \(i^2 = -1\), se opera como cualquier binomio.
En el fondo, es un alias de \(\sqrt{-1}\).

\textbf{Producto de complejos conjugados.}
\begin{align*}
    (a+bi) \cdot (a-bi) = a^2 \cancel{+ abi} \cancel{- abi} + bi^2 = a^2 + b^2
\end{align*}
Es una diferencia de cuadrados,
pero \(i^2\) cambia el signo de la parte imaginaria.
El resultado es un \textbf{real}.

\textbf{División de complejos.}
Se multiplica numerador y denominador por conjugado del denominador.
\begin{align*}
    \frac{a+bi}{c+di} = \frac{(a+bi) \cdot (c-di)}{(c+di) \cdot (c-di)} =
    \frac{ac + cbi - adi - bdi^2}{c^2 + d^2}
\end{align*}

\section*{Forma polar}

\textbf{Forma polar}.
Tomo el módulo del complejo \(|z| = \sqrt{a^2 + b^2}\),
y el ángulo \(\theta\) que forma en relación al eje \(x\) \((0-360^o)\).

\begin{align*}
    \theta = \arctan \left(\frac{b}{a}\right)
\end{align*}

Ojo con los ángulos.
La función \(\arctan\) no es \textbf{inyectiva}.
Entonces, la calculadora da resultados entre \([-90^o,90^o]\).

\textbf{Grados a radianes.}
Regla de 3 con esta igualdad: \(\pi = 180^o\).

\section*{Pasar de polar a binómica}

\textbf{Polar a trigonométrica.}
\begin{align*}
    \cos \alpha = \frac{a}{|z|} \rightarrow a = |z| \cdot \cos \alpha \\
    \sen \alpha = \frac{b}{|z|} \rightarrow b = |z| \cdot \sen \alpha \\
\end{align*}

\textbf{Forma trigonométrica.} 
\(z = |z| (\cos \alpha + i \sen \alpha)\). Ángulo: \(0 \leq \alpha \leq 2\pi\).

\section*{Operaciones en forma polar o trigonométrica}

En forma \textbf{polar/trigonométrica no} se puede \textbf{sumar ni restar}.
Pero se puede \textbf{multiplicar} y \textbf{dividir}.

\textbf{Multiplicación.} 
Sumamos ángulos y multiplicamos módulos. 
En el fondo, suma de vectores con multiplicación de sus módulos.

\textbf{División.}
Dividimos módulos y restamos ángulos.

\textbf{Potencia.}
Más simple para exponentes grandes. 
Potencia del módulo 
y multiplicamos cada ángulo por el exponente.
\textbf{Fórmula de Le Moivre.}

\textbf{Raíz n-ésima de números complejos en forma trigonométrica.}
Todo complejo, excepto 0, tiene exactamente \(n\) raíces \(n\)-ésimas.

\begin{align*}
    \sqrt[n]{Z} = \sqrt[n]{|W|} \cdot \left(\cos \frac{\alpha + 2k\pi}{n} + i \sen \frac{\alpha + 2k\pi}{n}\right) \text{ con } k=0,1,2 \dots n-1
\end{align*}

\end{document}