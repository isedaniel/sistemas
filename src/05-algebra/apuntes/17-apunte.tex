\documentclass[12pt]{article}
\usepackage[a4paper, margin=2.54cm]{geometry}

% español
\usepackage[spanish]{babel}

% imágenes
%\usepackage{graphicx}
%\graphicspath{{img}}

% fuentes de conjuntos numéricos
\usepackage{amsfonts}

% símbolos
\usepackage{amsmath, amssymb}

% gráficos
%\usepackage{tikz}

% plots
%\usepackage{pgfplots}
%\pgfplotsset{width=10cm, compat=1.9}

% averiguar
\setlength{\jot}{8pt}
\setlength{\parindent}{0cm}

% espacio entre párrafos
\usepackage[skip=10pt plus1pt]{parskip}

% cancelar términos
\usepackage{cancel}

% links
%\usepackage[colorlinks=true, 
%    urlcolor=blue]{hyperref}

% shapes
%\usetikzlibrary{shapes.geometric}

% incluir pdfs
%\usepackage{pdfpages}

\title{Álgebra y geometría analítica\\Profe Sebastián Scalise\\Apunte de clase}
\author{Daniel Ise}
\date{28 de octubre de 2024}

\begin{document}

\maketitle

\tableofcontents

\section{Cónicas}

\section{Circunferencia}

Sección producida por plano perpendicular al eje.

Circunferencia son todos los puntos que equidistan de un punto llamado centro.
Dado un punto y un radio se puede generar una circunferencia.

\textbf{Ecuación canónica de la circunferencia.}
\begin{equation}
    (x-a)^{2} + (y-b)^{2} = r^{2}
\end{equation}

\textbf{Ecuación general de la circunferencia.}
\begin{equation}
    x^{2} + y^{2} Ax + By + C = 0
\end{equation}

Para ir de ecuación general a canónica hay que completar el cuadrado.

\end{document}

