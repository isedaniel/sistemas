\documentclass[12pt]{article}
\usepackage[a4paper, margin=2.54cm]{geometry}

% español
\usepackage[spanish]{babel}

% imágenes
%\usepackage{graphicx}
%\graphicspath{{img}}

% fuentes de conjuntos numéricos
\usepackage{amsfonts}

% símbolos
\usepackage{amsmath, amssymb}

% gráficos
%\usepackage{tikz}

% plots
%\usepackage{pgfplots}
%\pgfplotsset{width=10cm, compat=1.9}

% averiguar
\setlength{\jot}{8pt}
\setlength{\parindent}{0cm}

% espacio entre párrafos
\usepackage[skip=10pt plus1pt]{parskip}

% cancelar términos
\usepackage{cancel}

% links
%\usepackage[colorlinks=true, 
%    urlcolor=blue]{hyperref}

% shapes
%\usetikzlibrary{shapes.geometric}

% incluir pdfs
%\usepackage{pdfpages}

\title{Álgebra y geometría analítica\\Apunte de clase}
\author{Daniel Ise}
\date{7 de octubre de 2024}

\begin{document}

\maketitle

\tableofcontents

\section{Dependencia e independencia lineal}

Hablamos de combinación lineal si \(\vec{v}_{1} = \vec{v}_{2}\cdot \alpha\).
En este caso, hay \textbf{LD}. Sino, hay \textbf{LI}.

\section{Base y dimensión}

B es un conjunto linealmente independiente, en cambio C es linealmente 
dependiente.

\textbf{Base.} Denominamos base al espacio vectorial si y solo si:
\begin{enumerate}
    \item 
\end{enumerate}

Si a una base le agrego elementos, 
los elementos sí o sí son \textbf{LD}.

Puedo agregar cualquier elemento al ser una base: todos los elementos de un 
espacio vectorial se pueden generar a partir de una base. 

Cualquier base \(R^{2}\) requiere de 2 vectores. 
No pueden ser 3 ni 1.

Dados dos \textbf{LI}, tengo una base.

\textbf{Dimensión.} 
Cantidad de elementos que componen a la base.

\end{document}

