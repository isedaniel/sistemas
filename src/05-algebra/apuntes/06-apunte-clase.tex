\documentclass{article}
\usepackage[a4paper, margin=2.54cm]{geometry}

% imágenes
%\usepackage{graphicx}               
%\graphicspath{{img}}

% fuentes de conjuntos numéricos
\usepackage{amsfonts}               

% símbolos
\usepackage{amsmath, amssymb}       

% gráficos
%\usepackage{tikz}                   

% plots
%\usepackage{pgfplots}               
%\pgfplotsset{width=10cm, compat=1.9}

% averiguar
\setlength{\jot}{8pt}
\setlength{\parindent}{0cm}

% espacio entre párrafos
\usepackage{parskip}                

% cancelar términos
\usepackage{cancel}                 

% links
%\usepackage[colorlinks=true, 
%    urlcolor=blue]{hyperref}        

% shapes
%\usetikzlibrary{shapes.geometric}   

% incluir pdfs
%\usepackage{pdfpages}               

\title{Álgebra y geometría analítica\\Apunte de clase}
\author{Daniel Ise}
\date{28 de agosto de 2024}

\begin{document}

\maketitle

\textbf{Ecuación general del plano.} Por una \textbf{recta}
\textit{pasan infinitos} \textbf{planos}. Para determinar un único plano:
necesito un \textbf{punto} que no pertenezca a la recta. Mismo resultado si
tengo \textbf{vector normal} y un \textbf{punto}.

Ejemplo: Dado el \textbf{vector normal} \(\vec{n} = (3,2,1)\) y un
\textbf{punto} \(P (4, 5, 6)\), determinamos la
\textbf{ecuación general del plano}:

\begin{align*}
    3x + 2y + 1z + D                      & = 0            \\
    3 \cdot 4 + 2 \cdot 5 + 1 \cdot 6 + D & = 0            \\
    12 + 10 + 6 + D                       & = 0            \\
    \boxed{D                                      = - 28 } \\
\end{align*}

Entonces:

\begin{center}
    3x + 2y + 1z - 28 = 0
\end{center}

\textbf{Ecuación vectorial del plano.} Igual a la de la recta pero con dos
vectores.

\begin{align*}
    x - 2y - z + 2 & = 0           \\
    x              & = -2 + 2y + z \\
    (x, y, z) & = (-2 + 2y + z, y, z)\\
    (x, y, z) & = (-2, 0, 0) + (2y, y, 0) + (z, 0, z)\\
    (x, y, z) & = (-2, 0, 0) + \lambda (2, 1, 0) + \mu (1, 0, 1)
\end{align*}

\textbf{Teoremas.} Dada \textbf{recta} \(r\) y \textbf{punto} \(P \notin r\), se
determina \textbf{plano}.

Dado \textbf{plano} y \textbf{recta}, si son \textbf{perpendiculares} el 
\textbf{producto escalar} es 0.

Dos \textbf{planos} pueden ser \textbf{paralelos} o \textbf{secantes}. 

Si dos \textbf{rectas} \(r, s \notin \pi\), son \textbf{alabeadas}.

\textbf{Distancia de un punto a un plano.} Consideramos \textbf{punto} 
\(M \in a, a \perp \pi\), la \textbf{distancia} es la menor a desde el 
\textbf{punto} a los \textit{infinitos} \textbf{puntos} del \textbf{plano}.

Ejemplo: Dado \(P (2, 1, 0)\), \(\pi (4x - 2y + 4z - 6 = 0)\).

\begin{align*}
    d(P,\pi) & = \frac{|P \cdot \vec{n} + D|}{|\vec{n}|}\\
    d(P,\pi) & = \frac{2 \cdot 4 + 1 \cdot (-2) + 0 - 6}{\sqrt{16 + 4 + 16}} = 0
\end{align*}

Como \(d(P, \pi) = 0 \rightarrow P \in \pi\).

\textbf{Ejercicios.} 

Hallar plano que contenga a la recta 
\(r: (x,y,z) = (3,2,1) + \alpha(-4,2,1)\) y al punto \(P (5,7,3)\).

\begin{align*}
    \vec{u} & = P - (3,2,1)\\
    \vec{u} & = (2,5,2)
\end{align*}

Entonces:

\begin{align*}
    \pi & = (3,2,1) + \alpha (-4,2,1) + \mu (2,5,2)
\end{align*}

Determinar producto vectorial.

\begin{align*}
    \begin{vmatrix}
        i & j & k \\
        -4 & 2 & 1 \\
        2 & 5 & 2
    \end{vmatrix} = i (4 - 5) - j (-8 - 2) + k (-20 - 4) = -i + 10j - 24k
\end{align*}

\textbf{Ejercicio 2.} 

\end{document}