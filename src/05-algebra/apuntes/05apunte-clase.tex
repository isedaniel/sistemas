\documentclass{article}
\usepackage[margin=2.54cm]{geometry}
%\usepackage{graphicx}               % imágenes
%\graphicspath{{img}}
\usepackage{amsfonts}               % fuentes de conjuntos numéricos
\usepackage{amsmath, amssymb}       % símbolos
%\usepackage{tikz}                   % gráficos
%\usepackage{pgfplots}               % plots
%\pgfplotsset{width=10cm, compat=1.9}
\setlength{\jot}{8pt}
\setlength{\parindent}{0cm}
\usepackage{parskip}                % espacio entre párrafos
\usepackage{cancel}                 % cancelar términos
%\usepackage[colorlinks=true, 
%    urlcolor=blue]{hyperref}        % links
%\usetikzlibrary{shapes.geometric}   % shapes
%\usepackage{pdfpages}               % incluir pdfs

\title{Álgebra y geometría analítica\\Apuntes de clase}
\author{Daniel Ise}
\date{19 de agosto de 2024}

\begin{document}

\maketitle

\textbf{Rectas en el espacio.} 

\textbf{Ecuación vectorial de la recta.}

Una \textbf{recta} es un conjunto de puntos en el plano, alineados con un punto
P $(x_1,y_1)$, con una dirección dada por $\vec{v}$. 

Si P es un punto de la recta r, el vector $\vec{PX}$, que va de P a X, con igual
dirección que $\vec{V}$, por lo tanto es igual a $\vec{v}$ multiplicado por un
escalar.

De ahí se desprende la ecuación vectorial 
$$(x,y) = (x_i, y_i) + \lambda (v_x, v_y)$$

\end{document}