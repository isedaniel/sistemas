\documentclass{article}
\usepackage[margin=2.54cm]{geometry}
%\usepackage{graphicx}               % imágenes
%\graphicspath{{img}}
\usepackage{amsfonts}               % fuentes de conjuntos numéricos
\usepackage{amsmath, amssymb}       % símbolos
%\usepackage{tikz}                   % gráficos
%\usepackage{pgfplots}               % plots
%\pgfplotsset{width=10cm, compat=1.9}
\setlength{\jot}{8pt}
\setlength{\parindent}{0cm}
\usepackage{parskip}                % espacio entre párrafos
\usepackage{cancel}                 % cancelar términos
%\usepackage[colorlinks=true, 
%    urlcolor=blue]{hyperref}        % links
%\usetikzlibrary{shapes.geometric}   % shapes
%\usepackage{pdfpages}               % incluir pdfs

\title{Álgebra - Segunda clase}
\author{Daniel Ise}
\date{7 de agosto de 2024}

\begin{document}

\maketitle

\textbf{Producto escalar de vectores.}

\begin{align*}
	\vec{u} \cdot \vec{v} & = |\vec{u}| \cdot |\vec{v}| \cdot \cos \alpha \\
	\vec{u} \cdot \vec{v} & = u_x \cdot v_x + u_y \cdot v_y
\end{align*}

Además, podemos obtener una expresión para la medida del ángulo $\alpha$.

\begin{align*}
	\alpha & = \arccos \left(\frac{\vec{u} \cdot \vec{v}}{|\vec{u}| \cdot |\vec{v}|}\right)
\end{align*}

Si dos vectores son \textbf{perpendiculares}, el producto escalar es igual a 0.

\begin{align*}
	\vec{u} \perp \vec{v} & \iff \vec{u} \cdot \vec{v} = 0
\end{align*}

\textbf{Propiedades de producto escalar. }

\textbf{Conmutativa. }

\begin{align*}
	\vec{u} \cdot \vec{v} & = \vec{v} \cdot \vec{u}
\end{align*}

\textbf{Asociativa. }

\textbf{Ortonormales. } Cuando dos vectores son a la vez \textbf{versores} y
\textbf{perpendiculares.}

\textbf{Vectores paralelos. } Dados son vectores, se dicen que son paralelos si
comparten dirección. Por lo tanto, tienen los mismos cosenos directores.

\textbf{Normalizar vector. } Normalizarlo es encontrar un vector unitario
palelo al vector normalizado.

\end{document}
