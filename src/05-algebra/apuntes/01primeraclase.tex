\documentclass{article}
\usepackage[margin=2.54cm]{geometry}
%\usepackage{graphicx}               % imágenes
%\graphicspath{{img}}
\usepackage{amsfonts}               % fuentes de conjuntos numéricos
\usepackage{amsmath, amssymb}       % símbolos
%\usepackage{tikz}                   % gráficos
%\usepackage{pgfplots}               % plots
%\pgfplotsset{width=10cm, compat=1.9}
\setlength{\jot}{8pt}
\setlength{\parindent}{0cm}
\usepackage{parskip}                % espacio entre párrafos
\usepackage{cancel}                 % cancelar términos
%\usepackage[colorlinks=true, 
%    urlcolor=blue]{hyperref}        % links
%\usetikzlibrary{shapes.geometric}   % shapes
%\usepackage{pdfpages}               % incluir pdfs

\title{Álgebra - Primera clase}
\author{Daniel Ise}
\date{8 de agosto de 2024}

\begin{document}
\maketitle

\section*{Cursado}

\begin{itemize}
    \item Lunes, online, 18 a 20.
    \item Miércoles, híbrido, 18 a 22.
\end{itemize}

\section*{Evaluación}

Hay trabajo práctico no obligatorio con ejercicios y prácticos obligatorios más cortos.
Cuando se cambia de unidad suelen ser los obligatorios. Hay que llevar al día los
trabajos y al parcial se llega cómodo.

Hay cuestionarios sincrónicos para hacer en clase los miércoles. Ninguno es obligatorio.
Pero sirven para practicar.

Se hacen parciales de 4 o 5 preguntas. Si no nos conocen la cara nos van a hacer
alguna instancia oral.

Hay que entregar los trabajos de la manera más profesional posible.

El workflow es: no obligatorio, sincrónico y obligatorio, para llegar bien al parcial.

\section*{Clase en sí}

\textbf{Hay un apunte de conceptos preliminares para revisar.} Trigo es importante.
Es importante respetar el workflow porque está todo enganchado.

\section*{Vectores en el plano}

\textbf{Estudiar esto con el material de los profes}. Porque hay diferencias en el
tratamiento que le dan en otros países o regiones.

La teoría está para leerla.

Los \textbf{vectores} se pueden representar en $R^2$ t $R^3$. Un vector es un segmento
orientado, que va de A (origen) a B (extremo).

$AB = v$

\begin{itemize}
    \item La dirección es la recta a la que pertenece el vector.
    \item El sentido es la orientación de la flecha.
    \item El módulo es la longitud.
\end{itemize}

Si un vector tiene módulo 0 es el vector nulo.

Si su módulo es 1, es el vector unitario o versor. Versor es
vector de módulo 1.

Vector equipolente: cuando tienen mismo módulo dirección y sentido. Misma dirección
alcanza con que sea recta paralela.

\textbf{Opuesto de un vector: } Mismo módulo y dirección, pero distinto sentido.

\textbf{Vectores concurrentes: } coinciden en origen.

\textbf{Vector de posición: } Origen en origen de los ejes y extremo en P.

\section*{Operaciones con vectores}

\textbf{Suma: } Método 1. Regla de la poligonal.
Tomo como origen de un vector el extremo del otro.
Método 2. Regla del paralelogramo. Hacemos un paralelogramo con los dos.

\textbf{Resta: } Dados los vectores u y v, u - v es igual a la suma de u más el
vector opuesto de v.

\textbf{Ángulo entre dos vectores. } El menor: 0. CUando tienen misma dirección y
sentido. El mayor: 180. Cuando tienen misma dirección y sentido opuesto. Por definición
se toma el menor ángulo entre los vectores u y v.

\begin{align*}
    0 &\leq \alpha \leq 180
\end{align*}

\textbf{Coordenadas de un vector: } Un vector se puede definir por sus coordenadas.
Dado el vector V, puede proyectar sobre el eje x e y. $v_y v_x$.
Si asocio un versor para cada eje, $i$ el versor sobre x, $j$ el versor sobre y,
Entonces, dado el vector V, tenemos que 
\begin{align}
    v &= v_xi + v_yj
\end{align}

De manera abreviada, un vector se puede notar como $v=(v_x,v_y)$. Es la forma
abreviada de la suma de los vectores que hemos visto en el punto anterior.

\end{document}
