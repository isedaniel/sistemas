\documentclass[12pt]{article}
\usepackage[a4paper, margin=2.54cm]{geometry}

% español
\usepackage[spanish]{babel}

% imágenes
%\usepackage{graphicx}
%\graphicspath{{img}}

% fuentes de conjuntos numéricos
\usepackage{amsfonts}

% símbolos
\usepackage{amsmath, amssymb}

% gráficos
%\usepackage{tikz}

% plots
%\usepackage{pgfplots}
%\pgfplotsset{width=10cm, compat=1.9}

% averiguar
\setlength{\jot}{8pt}
\setlength{\parindent}{0cm}

% espacio entre párrafos
\usepackage[skip=10pt plus1pt]{parskip}

% cancelar términos
\usepackage{cancel}

% links
%\usepackage[colorlinks=true, 
%    urlcolor=blue]{hyperref}

% shapes
%\usetikzlibrary{shapes.geometric}

% incluir pdfs
%\usepackage{pdfpages}

\title{Álgebra y geometría analítica\\Apunte de clase
\\Profes: Silvia Ranieri\\Sebastian Scalise}
\author{Daniel Ise}
\date{14 de octubre de 2024}

\begin{document}

\maketitle

\tableofcontents

\section{Transformaciones lineales}

Es una función.
Por lo tanto,
tiene \textbf{dominio} y \textbf{codominio}.
Estas funciones tienen como dominio espacios vectoriales,
como codominio espacios vectoriales.
\begin{equation}
    T: V \rightarrow W
\end{equation}

\(T\) es una transformación lineal si y solo si:
\begin{enumerate}
    \item \(T(0,0) = (0,0)\)
    \item \(T(u + v) = T(u) + T(v)\)
    \item \(T(\alpha v) = \alpha T(v)\)
\end{enumerate}

De esto se deduce que:
\begin{enumerate}
    \item Toda componente debe ser grado 1
    \item No debe tener coeficientes independientes
\end{enumerate}

\section{Teorema fundamental de las transformaciones lineales}

Qué datos necesitaría para obtener una única transformación lineal.
También conocido como teorema de existencia y unicidad de 
una transformación lineal.
Sean los espacios vectoriales \(V = R^{n}\) y W, sea \(B = \{v_{1}, ... v_{n}\}\)
base de \(V\) y \(w_{1} \dots w_{n}\) vectores de \(W\), entonces existe 
una única transformación lineal que verifica:
\begin{align}
    \begin{cases}
        T(v_{1}) = w_{1}  \\
        \hfil \vdots      \\
        t(v_{n}) = w_{n}  \\
    \end{cases}
\end{align}

\section{Nucleo e imagen de una transformación lineal}

Sean \(V\) y \(W\) dos espacios vectoriales,
sea \(T: V \rightarrow W\) una transformación lineal,
entonces el núcleo de \(T\),
denotado \(Nu(T)\),
está dado por:
\begin{equation}
    Nu(T) = \left\{v \in V:T(v) = 0w\right\}
\end{equation}

Asimismo,
se llama imagen de \(T\),
denotada \(Im(T)\) al conjunto de vectores \(w \in W\)
tales que existe \(v \in V\) con \(T(v) = w\).
Similar al conjunto imagen en funciones.
Es un subespacio vectorial del codominio:
\begin{equation}
    Im(T) \subseteq W
\end{equation}

\end{document}
