\documentclass{article}
\usepackage[margin=2.54cm]{geometry}
%\usepackage{graphicx}               % imágenes
%\graphicspath{{img}}
\usepackage{amsfonts}               % fuentes de conjuntos numéricos
\usepackage{amsmath, amssymb}       % símbolos
%\usepackage{tikz}                   % gráficos
%\usepackage{pgfplots}               % plots
%\pgfplotsset{width=10cm, compat=1.9}
\setlength{\jot}{8pt}
\setlength{\parindent}{0cm}
\usepackage{parskip}                % espacio entre párrafos
\usepackage{cancel}                 % cancelar términos
%\usepackage[colorlinks=true, 
%    urlcolor=blue]{hyperref}        % links
%\usetikzlibrary{shapes.geometric}   % shapes
%\usepackage{pdfpages}               % incluir pdfs

\title{Álgebra - Tercera clase}
\author{Daniel Ise}
\date{12 de agosto de 2024}

\begin{document}

\maketitle

Queda pendiente hacer \textbf{todas las guías}. Sino me voy a quedar y me voy a
la B.

\textbf{$R^3$}. El origen es $(0,0,0)$. El nuevo eje es Z. Para simbolizar un
vector $\vec{a}$ se proyecta sobre los tres ejes. Para expresarlo utilizamos los
versores $\breve{i}$, $\breve{j}$ y $\breve{k}$.

\textbf{Distancia entre dos puntos.} Hacemos resta de extremo menos origen.
Igual que en el plano.

Todo es más o menos igual que en plano, solo que hay que considerar el tercer
componente.

\textbf{Lo nuevo. }El producto vectorial. No tiene sentido en el plano. Se hace
solo en el espacio. Fórmula:

\begin{align*}
    |\vec{u}x\vec{v}| & = |\vec{u}| \cdot |\vec{v}| \cdot \sin\alpha
\end{align*}

\textbf{Determinante.} Se hace con una matriz.

Cuando da todo 0 el producto vectorial es porque son paralelos.

Para determinar el área de un paralelogramo que forman dos vectores podemos
hacer el producto vectorial. Se puede usar en el plano tomando el tercer 
componente como 0.

\textbf{Producto mixto.} Vectorial por escalar. Se puede hacer derecho en
matriz de 3x3.

\end{document}