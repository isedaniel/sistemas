\documentclass{article}
\usepackage[a4paper, margin=2.54cm]{geometry}

% español
\usepackage[spanish]{babel}

% imágenes
%\usepackage{graphicx}
%\graphicspath{{img}}

% fuentes de conjuntos numéricos
\usepackage{amsfonts}

% símbolos
\usepackage{amsmath, amssymb}

% gráficos
%\usepackage{tikz}

% plots
%\usepackage{pgfplots}
%\pgfplotsset{width=10cm, compat=1.9}

% averiguar
\setlength{\jot}{8pt}
\setlength{\parindent}{0cm}

% espacio entre párrafos
\usepackage[skip=8pt plus1pt]{parskip}

% cancelar términos
\usepackage{cancel}

% links
%\usepackage[colorlinks=true, 
%    urlcolor=blue]{hyperref}

% shapes
%\usetikzlibrary{shapes.geometric}

% incluir pdfs
%\usepackage{pdfpages}

\title{Álgebra y geometría analítica\\Apunte de clase}
\author{Daniel Ise}
\date{4 de septiembre de 2024}

\begin{document}

\maketitle

\tableofcontents

\section{Prácticos y parciales}

Se valora \textbf{explicar paso por paso} lo que estamos haciendo.
Es lo que permite saber que estamos entendiendo.
No solo aplicando fórmulas.

\section{Raíz enésima de número complejo}

\begin{align*}
    \sqrt[n]{W} = \sqrt[n]{|W|} \left(\cos \frac{\alpha + 2\pi k}{n} + \sen \frac{\alpha + 2\pi k}{n} \right)
\end{align*}

\section{Forma exponencial del número complejo}

\textbf{Fórmula de Euler}

Se demuestra con \textbf{serie de Taylor}. 
Permite demostrar la \textbf{identidad de Euler}.
Permite representar complejos con notación más corta.

\begin{align*}
    -2+2i = \sqrt{8}(\cos 135^o + i \sen 135^o) = \sqrt{8} \cdot 2^{\frac{@pi}{2}}
\end{align*}

\section{Logaritmo natural de un número complejo}

Se puede resolver a partir de la \textbf{notación exponencial}.

\begin{align*}
    \ln z = \ln (|z|\cdot e^{i\cdot\alpha})
\end{align*}

\section{Exponencial compleja}

Siendo \(V, U \in \mathbb{C}\).

\begin{align*}
    V^{U} = e^{u \cdot\ln v}
\end{align*}

\end{document}
