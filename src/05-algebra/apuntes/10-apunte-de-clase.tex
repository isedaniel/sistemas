\documentclass[12pt]{article}
\usepackage[a4paper, margin=2.54cm]{geometry}

% español
\usepackage[spanish]{babel}

% imágenes
%\usepackage{graphicx}
%\graphicspath{{img}}

% fuentes de conjuntos numéricos
\usepackage{amsfonts}

% símbolos
\usepackage{amsmath, amssymb}

% gráficos
%\usepackage{tikz}

% plots
%\usepackage{pgfplots}
%\pgfplotsset{width=10cm, compat=1.9}

% averiguar
\setlength{\jot}{8pt}
\setlength{\parindent}{0cm}

% espacio entre párrafos
\usepackage[skip=10pt plus1pt]{parskip}

% cancelar términos
\usepackage{cancel}

% links
%\usepackage[colorlinks=true, 
%    urlcolor=blue]{hyperref}

% shapes
%\usetikzlibrary{shapes.geometric}

% incluir pdfs
%\usepackage{pdfpages}

\title{Álgebra y geometría analítica\\Apunte de clase}
\author{Daniel Ise}
\date{11 de septiembre de 2024}

\begin{document}

\maketitle

\tableofcontents

\section{Propiedades de los determinantes}

\subsection{Propiedad 10}
Determinante no cambia si a una fila (o columna)
se le suma otra fila (o columna),
multiplicadas por números distintos de 0.

\textbf{Ejemplo.}

\subsection{Propiedad 11}

\textbf{Submatriz \(A_{ij}\).}
Matriz en la que se suprimen fila \(i\) y columna \(j\).

\textbf{Adjunto de elemento \(a_{ij}\).}
Determinante de submatriz \(A_{ij}\),
con signo positivo o negativo de acuerdo a suma de subíndices par o impar.

\textbf{Matriz adjunta.}
Se calcula reemplazando a cada elemento por su adjunto.

\textbf{Nota}: ojo que algunos llaman adjunta a la traspuesta de la adjunta.

\section{Matriz inversa}

Dada matriz \(A\),
denominamos \textbf{matriz inversa} \(A^{-1}\) a aquella que,
multiplicando \(A \cdot A^{-1}\),
devuelve la \textbf{matriz identidad}.

\textbf{Admisión de inversa.}
Tiene dos requerimientos,
1. matriz cuadrada y
2. determinante \textit{distinto de 0}.

\textbf{Matriz singular.} 
Aquella matriz que no admite inversa.

\textbf{Metodos.}
Hay dos métodos para calcular.
\begin{itemize}
    \item Método de 
    \item Método de Gauss
\end{itemize}



\end{document}
