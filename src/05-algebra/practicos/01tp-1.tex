\documentclass{article}
\usepackage[margin=2.54cm]{geometry}
%\usepackage{graphicx}               % imágenes
%\graphicspath{{img}}
\usepackage{amsfonts}               % fuentes de conjuntos numéricos
\usepackage{amsmath, amssymb}       % símbolos
%\usepackage{tikz}                   % gráficos
%\usepackage{pgfplots}               % plots
%\pgfplotsset{width=10cm, compat=1.9}
\setlength{\jot}{8pt}
\setlength{\parindent}{0cm}
\usepackage{parskip}                % espacio entre párrafos
\usepackage{cancel}                 % cancelar términos
%\usepackage[colorlinks=true, 
%    urlcolor=blue]{hyperref}        % links
%\usetikzlibrary{shapes.geometric}   % shapes
%\usepackage{pdfpages}               % incluir pdfs

\title{Álgebra y Geometría Analítica\\ Trabajo Práctico $N^o1$}
\author{Daniel Ise}
\date{Agosto de 2024}

\begin{document}

\maketitle

1. Obtener un vector de módulo 1 (versor) ortogonal a los vectores
$\vec{u} = (1, -2, 3)$ y $\vec{v} = (0, -1, 5)$.
\begin{align*}
    \vec{u}\times\vec{v} & = \begin{vmatrix}
                                 \breve{i} & \breve{j} & \breve{k} \\
                                 1         & -2        & 3         \\
                                 0         & -1        & 5
                             \end{vmatrix}                                         \\
    \vec{u}\times\vec{v} & = \breve{i}\begin{vmatrix}
                                          -2 & 3 \\
                                          -1 & 5
                                      \end{vmatrix} - \breve{j}\begin{vmatrix}
                                                                   1 & 3 \\
                                                                   0 & 5
                                                               \end{vmatrix} + \breve{k}\begin{vmatrix}
                                                                                            1 & -2 \\
                                                                                            0 & -1
                                                                                        \end{vmatrix} \\
    \vec{u}\times\vec{v} & = \breve{i}(-7) - \breve{j}(5) + \breve{k}(-1)                              \\
    \vec{u}\times\vec{v} & = (-7, -5, -1)
\end{align*}

El vector $(-7, -5, -1)$ es ortogonal a los vectores $\vec{u}$ y $\vec{v}$. Para
convertirlo en un versor, operamos el cociente del mismo por su módulo:

\begin{align*}
    \frac{(-7, -5, -1)}{\sqrt{(-7)^2 + (-5)^2 + (-1)^2}} & =
    \boxed{\left(\frac{-7}{\sqrt{75}},\frac{-5}{\sqrt{75}},\frac{-1}{\sqrt{75}}\right)}
\end{align*}

2. Los vértices de un triángulo son los puntos $A = (4,1,-1)$,
$B = (1,-4,-2)$ y $C = (0,-1,3)$. Hallar:
\begin{itemize}
    \item[a.] Los vectores posición de los vértices.
    \item[b.] El perímetro del triángulo.
    \item[c.] Los ángulos del triángulo.
    \item[d.] El área del triángulo.
\end{itemize}

a. Vectores posición:

\begin{align*}
    \overline{AB} = B - A & = \boxed{(-3, -5, -1)} \\
    \overline{BC} = C - B & = \boxed{(-1, 3, 5)}   \\
    \overline{CA} = A - C & = \boxed{(4, 2, -4)}
\end{align*}

b. Perímetro del triangulo:

\begin{align*}
    |\overline{AB}| + |\overline{BC}| + |\overline{CA}| =
    \sqrt{35} + \sqrt{35} + 6 & = \boxed{2\sqrt{35} + 6}
\end{align*}

c. Ángulos:

Al ser un triángulo isósceles, tomamos primero el ángulo del vértice
$\angle \overline{BA}\overline{BC}$, que al encontrarse frente al funto $B$
denominaremos $\beta$.

\begin{align*}
    \beta & = \arccos \left(\frac{\overline{BA} \cdot \overline{BC}}
    {|\overline{BA}| \cdot |\overline{BC}|}\right)                      \\
    \beta & = \arccos \left(\frac{3 \cdot (-1) + 5 \cdot 3 + 1 \cdot 5}
    {\sqrt{35} \cdot \sqrt{35}}\right)                                  \\
    \beta & \approx \boxed{60.94}
\end{align*}

Dado que los ángulos de la base deben ser iguales entre sí por tratarse de un
triángulo isósceles, así como la suma de los tres debe ser igual a $180^o$,
por lo tanto:

\begin{align*}
    180^o - \beta    & = 119.06                          \\
    \frac{119.06}{2} & = \alpha = \gamma = \boxed{59.53}
\end{align*}

3. Sean los vectores:
\begin{align*}
    \vec{u} & = (-1,2,0) & \vec{v} & = (-2, 1, 2) & \vec{w} & = (0, 4, -3)
\end{align*}

Obtener en cada caso $\vec{s}$ si:

\begin{itemize}
    \item[i.]  $2\vec{u} + 3 \vec{s} - \vec{w} = \vec{s} + \vec{v}$
    \item[ii.] $3(\vec{u}+\vec{v}) - 2 \vec{s} = -(\vec{s} + \vec{w})$
\end{itemize}

i.
\begin{align*}
    2\vec{u} + 3\vec{s} - \vec{w} & = \vec{s} + \vec{v}                                 \\
    2\vec{s}                      & = \vec{v} - 2\vec{u} + \vec{w}                      \\
    2\vec{s}                      & = (-2, 1, 2) - (-2, 4, 0) + (0, 4, -3)              \\
    2\vec{s}                      & = (0, 1, -1)                                        \\
    \vec{s}                       & = \boxed{\left(0, \frac{1}{2}, \frac{-1}{2}\right)}
\end{align*}

ii.
\begin{align*}
    3\vec{u} + 3\vec{v} - 2\vec{s}       & = -\vec{s} - \vec{w}  \\
    3\vec{u} + 3\vec{v} + \vec{w}        & = \vec{s}             \\
    (-3, 6, 0) + (-6, 3, 6) + (0, 4, -3) & = \vec{s}             \\
    \vec{s}                              & = \boxed{(-9, 13, 3)}
\end{align*}

4. Hallar la distancia entre los puntos R y S.
\begin{align*}
    R & = (-3,2,0) & S & = (0, 2, 4)
\end{align*}

Para hallar la distancia tomamos $|\overline{RS}|$:

\begin{align*}
    |\overline{RS}| & = |R - S|                      \\
    |\overline{RS}| & = |(-3,2,0) - (0,2,4)|         \\
    |\overline{RS}| & = |(-3,0,-4)|                  \\
    |\overline{RS}| & = \sqrt{(-3)^2 + 0^2 + (-4)^2} \\
    |\overline{RS}| & = \boxed{5}
\end{align*}

5. Calcule el volumen del paralelepípedo que tiene por aristas PQ, PR y PS si
P, Q, R y S son: $(1, -3, 4);\\(3,-5,3); (2,-1,4); (2,2,5)$.

\begin{align*}
    \overline{PQ} = Q - P & = (2, -2, -1) \\
    \overline{PR} = R - P & = (1, 2, 0)   \\
    \overline{PS} = S - P & = (1, 5, 1)
\end{align*}

\begin{align*}
    \left(\overline{PQ} \times \overline{PR}\right) \cdot \overline{PS} & = \begin{vmatrix}
                                                                                2 & -2 & -1 \\
                                                                                1 & 2  & 0  \\
                                                                                1 & 5  & 1
                                                                            \end{vmatrix} \\
    \left(\overline{PQ} \times \overline{PR}\right) \cdot \overline{PS} & =
    2 \begin{vmatrix}
          2 & 0 \\
          5 & 1
      \end{vmatrix} - (-2) \begin{vmatrix}
                               1 & 0 \\
                               1 & 1
                           \end{vmatrix} + (-1) \begin{vmatrix}
                                                    1 & 2 \\
                                                    1 & 5
                                                \end{vmatrix}                             \\
    \left(\overline{PQ} \times \overline{PR}\right) \cdot \overline{PS} & =
    2 \cdot (2 - 0) - (-2) \cdot (1 - 0) + (-1) \cdot (5 - 2)                              \\
    \left(\overline{PQ} \times \overline{PR}\right) \cdot \overline{PS} & = 4 + 2 - 3      \\
    \left(\overline{PQ} \times \overline{PR}\right) \cdot \overline{PS} & = \boxed{3}
\end{align*}


6. Los puntos $A (1,2,-1)$, $B(2,0,2)$ y $C(-4,1,-3)$ son vértices consecutivos
de un paralelogramo. Hallar el vértice D, centro M, el perímetro y el área del
paralelogramo.

i. Vértice D:

\begin{align*}
    \overline{AB} & = \overline{DC}                        \\
    B - A         & = C - D                                \\
    D             & = C - B + A                            \\
    D             & = (-4, 1, -3) - (2, 0, 2) + (1, 2, -1) \\
    D             & = (-5, 3, -6)
\end{align*}

ii. Centro M:

\begin{align*}
    M & = \frac{\overline{AB} + \overline{AD}}{2}
\end{align*}

Determinamos $\overline{AB}$ y $\overline{AD}$:

\begin{align*}
    \overline{AB} = B - A & = (1, -2, 3)  \\
    \overline{AD} = D - A & = (-6, 1, -5)
\end{align*}

Entonces:

\begin{align*}
    M & = \frac{(1, -2, 3) + (-6, 1, -5)}{2}                  \\
    M & = \boxed{\left(\frac{-5}{2}, \frac{-1}{2}, -1\right)}
\end{align*}

iii. Perímetro:

Podemos determinar el perímetro, por tratarse de un paralelogramo, con la
expresión $2\overline{AB} + 2\overline{AD}$:

\begin{align*}
    2\overline{AB} + 2\overline{AD} & = 2 \cdot (1, -2, 3) + 2 \cdot (-6, 1, -5) \\
    2\overline{AB} + 2\overline{AD} & = \boxed{2 \sqrt{14} + 2 \sqrt{62}}
\end{align*}

iv. Área:

Por último, para determinar el área podemos recurrir al módulo del producto
factorial de $\overline{AB}$ y $\overline{AD}$. Primero operamos el producto:

\begin{align*}
    \overline{AB} \times \overline{AD} & = \begin{vmatrix}
        \breve{i} & \breve{j} & \breve{k} \\
        1 & -2 & 3 \\
        -6 & 1 & -5
    \end{vmatrix}\\
    \overline{AB} \times \overline{AD} & = \breve{i}  \begin{vmatrix}
        -2 & 3\\
        1 & -5
    \end{vmatrix} - \breve{j} \begin{vmatrix}
        1 & 3\\
        -6 & -5
    \end{vmatrix} + \breve{k}  \begin{vmatrix}
        1 & -2\\
        -6 & 1
    \end{vmatrix}\\
    \overline{AB} \times \overline{AD} & = \breve{i}(10 - 3) -
    \breve{j}(-5 +18) + \breve{k}(1-12)\\
    \overline{AB} \times \overline{AD} & = (7, -13, -11)
\end{align*}

Por último, operamos módulo de $(7, -13, -11)$:


\begin{align*}
    |(7, -13, -11)| & = \sqrt{7^2 + (-13)^2 + (-11)^2}\\
    (7, -13, -11) &= \boxed{339}
\end{align*}

\end{document}
