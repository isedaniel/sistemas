\documentclass{article}
\usepackage[margin=2.54cm]{geometry}
%\usepackage{graphicx}               % imágenes
%\graphicspath{{img}}
\usepackage{amsfonts}               % fuentes de conjuntos numéricos
\usepackage{amsmath, amssymb}       % símbolos
%\usepackage{tikz}                   % gráficos
%\usepackage{pgfplots}               % plots
%\pgfplotsset{width=10cm, compat=1.9}
\setlength{\jot}{8pt}
\setlength{\parindent}{0cm}
\usepackage{parskip}                % espacio entre párrafos
\usepackage{cancel}                 % cancelar términos
%\usepackage[colorlinks=true, 
%    urlcolor=blue]{hyperref}        % links
%\usetikzlibrary{shapes.geometric}   % shapes
%\usepackage{pdfpages}               % incluir pdfs

\title{Álgebra y Geometría Analítica\\ Trabajo Práctico N$^o$1}
\author{Daniel Ise}
\date{Agosto de 2024}

\begin{document}

\maketitle

1. Obtener un vector de módulo 1 (versor) ortogonal a los vectores 
$\vec{u} = (1, -2, 3)$ y $\vec{v} = (0, -1, 5)$.
\begin{align*}
    \vec{w} &= (2, 3, 4)
\end{align*}

2. Los vértices de un triángulo son los puntos $A = (4,1,-1)$,
$B = (1,-4,-2)$ y $C = (0,-1,3)$. Hallar: 
\begin{itemize}
    \item[a.] Los vectores posición de los vértices.
    \item[b.] El perímetro del triángulo.
    \item[c.] Los ángulos del triángulo.
    \item[d.] El área del triángulo.
\end{itemize}

3. Sean los vectores:
\begin{align*}
    \vec{u} &= (-1,2,0) & \vec{v} &= (-2, 1, 2) & \vec{w} &= (0, 4, -3)
\end{align*}

Obtener en cada caso $\vec{s}$ si:

\begin{itemize}
    \item[i.]  $2\vec{u} + 3 \vec{s} - \vec{w} = \vec{s} + \vec{v}$
    \item[ii.] $3(\vec{u}+\vec{v}) - 2 \vec{s} = -(\vec{s} + \vec{w})$
\end{itemize}

4. Hallar la distancia entre los puntos R y S.
\begin{align*}
    R &= (-3,2,0) & S &= (0, 2, 4)
\end{align*}

5. Calcule el volumen del paralelepípedo que tiene por aristas PQ, PR y PS si
P, Q, R y S son: $(1, -3, 4);\\(3,-5,3); (2,-1,4); (2,2,5)$.

6. Los puntos $A (1,2,-1)$, $B(2,0,2)$ y $C(-4,1,-3)$ son vértices consecutivos
de un paralelogramo. Hallar el vértice D, centro M, el perímetro y el área del
paralelogramo.

\end{document}
