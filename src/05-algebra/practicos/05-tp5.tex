\documentclass[12pt]{article}
\usepackage[a4paper, margin=2.54cm]{geometry}
\usepackage[spanish]{babel}

% imágenes
%\usepackage{graphicx}
%\graphicspath{{img}}

% fuentes de conjuntos numéricos
\usepackage{amsfonts}

% math
\usepackage{amsmath, amssymb}

% gráficos y plots
\usepackage{tikz}
%\usepackage{pgfplots}
%\pgfplotsset{width=10cm, compat=1.9}
\usetikzlibrary{babel}

\setlength{\jot}{8pt}
\setlength{\parindent}{0cm}

% espacio entre párrafos
\usepackage{parskip}

% cancelar términos
\usepackage{cancel}

% links
%\usepackage[colorlinks=true,
%    urlcolor=blue]{hyperref}

% shapes
%\usetikzlibrary{shapes.geometric}

% incluir pdfs
%\usepackage{pdfpages}

\begin{document}

\thispagestyle{empty}

\begin{center}
  \vspace*{.5cm}
  \includegraphics[scale=0.4]{../../img/udemm-logo.png}\\
  \vspace{.2cm}
  \Large
  \textbf{Facultad de Ingeniería}\\
  \textbf{Ingeniería en Sistemas}\\
  \vspace{2cm}

  \Huge
  Álgebra y Geometría Analítica\\
  Trabajo Práctico N\(^{\circ} 5\) \\
  Espacios Vectoriales
  \vfill

  \raggedright
  \Large
  Docentes:
  \begin{itemize}
    \item[] Mg. Silvia Ranieri\\
    \item[] Prof. Sebastian Scalise\\
  \end{itemize}
  Alumno:
  \begin{itemize}
    \item[] Daniel Ise
  \end{itemize}
  \vspace{1cm}

  \centering
  2024
\end{center}

\pagebreak

\textbf{1.}
Siendo \(A = \left\{(x,y,z) \in R^{3} / x = y = z\right\}\) y
\(B = \left\{(x,y,z) \in R^{3} / x-y-z=0\right\}\),
encontrar:

\hspace{6mm}\textbf{a.}
\(A \cap B\), una base y su dimensión.

Comprobamos que \(A \not \subset B\):
\begin{align*}
  \begin{cases}
    y = x \\
    z = x
  \end{cases} \implies (x,y,z) = \alpha (1,1,1)
\end{align*}

Sustituimos en \(B\):
\begin{align*}
  1 - 1 - 1 = -1 \neq 0 \implies A \not \subset B
\end{align*}

Por lo tanto:
\begin{align*}
  B_{A \cap B} = \left\{(0,0,0)\right\} \implies dim(A \cap B) = 0
\end{align*}

\hspace{6mm}\textbf{b.}
\(A + B\), una base y su dimensión. ¿Es directa?

Sabemos que \(B_{A} = \left\{(1,1,1)\right\}\).
Determinamos \(B_{B}\):
\begin{align*}
  x = y + z & \rightarrow (x,y,z) = (y+z,y,z) = y(1,1,0) + z(1,0,1) \\
            & \boxed{B_{B} = \left\{(1,1,0),(1,0,1)\right\}}
\end{align*}

Por lo tanto, \(A + B = \left\{(1,1,0),(1,0,1),(1,1,1)\right\}\).
Como \(A + B\) está formado por 3 vectores, resta saber si es linealmente
independiente:
\begin{align*}
  \begin{vmatrix}
    1 & 1 & 0 \\
    1 & 0 & 1 \\
    1 & 1 & 1 \\
  \end{vmatrix} = 1 \cdot (-1) - 1 \cdot 0 = \boxed{-1 \neq 0}
\end{align*}

Como el determinante de \(A + B\) es distinto de 0,
concluimos que \(A \oplus B = R^{3}\), así como \(dim(A\oplus B) = 3\).

Además, en el punto anterior mostramos que la intersección entre ambos
es el vector nulo, por lo cual también concluimos que la suma es directa.

\textbf{2.}
Dados los vectores:
\(U_{1} = (1,0,2)\),
\(U_{2} = (0,2,0)\) y
\(U_{3} = (0,1,-1)\).

\hspace{6mm}\textbf{a.}
Expresar, de ser posible, \(U_{1}\) como combinación lineal de \(U_{2}\) y
\(U_{3}\).

\(U_{1}\) no se puede expresar como combinación lineal de \(U_{2}\) y
\(U_{3}\), puesto que el componente en \(x\) de ambos vectores es igual a 0,
y \(\alpha\cdot 0 + \beta\cdot 0 = 1\) es un absurdo.

\hspace{6mm}\textbf{b.}
Decidir si el conjunto \(A = \left\{U_{1}, U_{2}, U_{3}\right\}\) es una base
de \(R^{3}\).

Dado que el conjunto \(A\) está compuesto por tres vectores,
restaría determinar que estos son linealmente independientes para saber
si consituyen base de \(R_{3}\), por lo tanto:
\begin{align*}
  \begin{vmatrix}
    1 & 0 & 2  \\
    0 & 2 & 0  \\
    0 & 1 & -1 \\
  \end{vmatrix} = 1 \cdot (-2) - 0 + 2 \cdot 0 = \boxed{-2 \neq 0}
\end{align*}

Como el determinante de la matriz es distinto de 0,
concluimos que el conjunto \(A\) es linealmente independiente y,
por lo tanto,
es base del espacio vectorial \(R^{3}\).

\hspace{6mm}\textbf{c.}
Presentar un subespacio de \(R^{3}\) de dimensión 1 que contenga a \(U_{1}\).
Dar una base del mismo.

Tomando a \(U_{1}\) como base, obtenemos el conjunto \(\left\{(1,0,2)\right\}\),
de dimensión 1, expresado como \(\left\{(x,y,z) \in R^{3} / y = 0 \land z=2x \right\}\).

\hspace{6mm}\textbf{d.}
Presentar un subespacio de \(R^{3}\) de dimensión 2 que contenga a \(U_{2}\).
Dar una base del mismo.

Podemos utilizar a \(U_{2}\) y \(U_{3}\) como base, 
\(\left\{(0,1,-1), (0,2,0)\right\}\). 
Determinamos la ecuación del plano mediante el producto factorial 
\((0,1,-1) \times (0,2,0) = (2,0,0)\), por lo tanto,
\(\left\{(x,y,z) \in R^{3} / 2x = 0 \right\}\).

\hspace{6mm}\textbf{e.}
Presentar un subespacio de \(R^{3}\) de dimensión 0.

Por convención, el único subespacio de \(R^{3}\) con dimensión igual a 0 es
el que tiene como base al vector nulo:
\(\left\{(x,y,z) \in R^{3} / x = y = z = 0 \right\}\).

\hspace{6mm}\textbf{f.}
Presentar un subespacio de \(R^{3}\) de dimensión 3.
Dar una base del mismo.

Dado que tenemos un conjunto de 3 vectores que, 
por lo que vimos en el punto \textbf{b.},
constituyen un conjunto linealmente independiente,
podemos utilizarlos como base:
\(B = \left\{(1,0,2), (0,2,0), (0,1,-1)\right\}\).

Al ser un conjunto LI de dimensión 3, 
sabemos que permiten expresar al subespacio trivial \(R^{3} \subseteq R^{3}\).

\end{document}
