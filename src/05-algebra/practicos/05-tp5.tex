\documentclass[12pt]{article}
\usepackage[a4paper, margin=2.54cm]{geometry}
\usepackage[spanish]{babel}

% imágenes
%\usepackage{graphicx}
%\graphicspath{{img}}

% fuentes de conjuntos numéricos
\usepackage{amsfonts}

% math
\usepackage{amsmath, amssymb}

% gráficos y plots
\usepackage{tikz}
%\usepackage{pgfplots}
%\pgfplotsset{width=10cm, compat=1.9}
\usetikzlibrary{babel}

\setlength{\jot}{8pt}
\setlength{\parindent}{0cm}

% espacio entre párrafos
\usepackage{parskip}

% cancelar términos
\usepackage{cancel}

% links
%\usepackage[colorlinks=true,
%    urlcolor=blue]{hyperref}

% shapes
%\usetikzlibrary{shapes.geometric}

% incluir pdfs
%\usepackage{pdfpages}

\begin{document}

\thispagestyle{empty}

\begin{center}
  \vspace*{.5cm}
  \includegraphics[scale=0.4]{../../img/udemm-logo.png}\\
  \vspace{.2cm}
  \Large
  \textbf{Facultad de Ingeniería}\\
  \textbf{Ingeniería en Sistemas}\\
  \vspace{2cm}

  \Huge
  Álgebra y Geometría Analítica\\
  Trabajo Práctico N\(^{\circ} 5\) \\
  Espacios Vectoriales
  \vfill

  \raggedright
  \Large
  Docentes:
  \begin{itemize}
    \item[] Mg. Silvia Ranieri\\
    \item[] Prof. Sebastian Scalise\\
  \end{itemize}
  Alumno:
  \begin{itemize}
    \item[] Daniel Ise
  \end{itemize}
  \vspace{1cm}

  \centering
  2024
\end{center}

\pagebreak

\textbf{1.}
Siendo \(A = \left\{(x,y,z) \in R^{3} / x = y = z\right\}\) y
\(B = \left\{(x,y,z) \in R^{3} / x-y-z=0\right\}\),
encontrar:

\hspace{6mm}\textbf{a.}
\(A \cap B\), una base y su dimensión.

Comprobamos que \(A \not \subset B\):

\begin{align*}
  \begin{cases}
    y = x \\
    z = x
  \end{cases} \implies (x,y,z) = \alpha (1,1,1)
\end{align*}

Sustituimos en \(B\):

\begin{align*}
  1 - 1 - 1 = -1 \neq 0 \implies A \not \subset B
\end{align*}

Por lo tanto:

\begin{align*}
  B_{A \cap B} = \left\{(0,0,0)\right\} \implies dim(A \cap B) = 0
\end{align*}

\hspace{6mm}\textbf{b.}
\(A + B\), una base y su dimensión. ¿Es directa?

Sabemos que \(B_{A} = \left\{(1,1,1)\right\}\).
Determinamos \(B_{B}\):

\begin{align*}
  x = y + z & \rightarrow (x,y,z) = (y+z,y,z) = y(1,1,0) + z(1,0,1) \\
            & \boxed{B_{B} = \left\{(1,1,0),(1,0,1)\right\}}
\end{align*}

Por lo tanto, \(A + B = \left\{(1,1,0),(1,0,1),(1,1,1)\right\}\).
Como \(A + B\) está formado por 3 vectores, resta saber si es linealmente
independiente:

\begin{align*}
  \begin{vmatrix}
    1 & 1 & 0 \\
    1 & 0 & 1 \\
    1 & 1 & 1 \\
  \end{vmatrix} = 1 \cdot (-1) - 1 \cdot 0 = \boxed{-1 \neq 0}
\end{align*}

Como el determinante de \(A + B\) es distinto de 0,
concluimos que \(A \oplus B = R^{3}\), así como \(dim(A\oplus B) = 3\).

Además, en el punto anterior mostramos que la intersección entre ambos
es el vector nulo, por lo cual también concluimos que la suma es directa.

\textbf{2.}
Dados los vectores:
\(U_{1} = (1,0,2)\),
\(U_{2} = (0,2,0)\) y
\(U_{3} = (0,1,-1)\).

\hspace{6mm}\textbf{a.}
Expresar, de ser posible, \(U_{1}\) como combinación lineal de \(U_{2}\) y
\(U_{3}\).

\(U_{1}\) no se puede expresar como combinación lineal de \(U_{2}\) y
\(U_{3}\), puesto que el componente en \(x\) de ambos vectores es igual a 0,
y \(\alpha\cdot 0 + \beta\cdot 0 = 1\) es un absurdo.

\hspace{6mm}\textbf{b.}
Decidir si el conjunto \(A = \left\{U_{1}, U_{2}, U_{3}\right\}\) es una base
de \(R^{3}\).

Dado que el conjunto \(A\) está compuesto por tres vectores,
restaría determinar que estos son linealmente independientes para saber
si consituyen base de \(R_{3}\), por lo tanto:

\begin{align*}
  \begin{vmatrix}
    1 & 0 & 2  \\
    0 & 2 & 0  \\
    0 & 1 & -1 \\
  \end{vmatrix} = 1 \cdot (-2) - 0 + 2 \cdot 0 = \boxed{-2 \neq 0}
\end{align*}

Como el determinante de la matriz es distinto de 0,
concluimos que el conjunto \(A\) es linealmente independiente y,
por lo tanto,
es base del espacio vectorial \(R^{3}\).

\hspace{6mm}\textbf{c.}
Presentar un subespacio de \(R^{3}\) de dimensión 1 que contenga a \(U_{1}\).
Dar una base del mismo.

Tomando a \(U_{1}\) como base, obtenemos el conjunto \(\left\{(1,0,2)\right\}\),
de dimensión 1, expresado como \(\left\{(x,y,z) \in R^{3} / y = 0 \land z=2x \right\}\).

\hspace{6mm}\textbf{d.}
Presentar un subespacio de \(R^{3}\) de dimensión 2 que contenga a \(U_{2}\).
Dar una base del mismo.

Podemos utilizar a \(U_{2}\) y \(U_{3}\) como base,
\(\left\{(0,1,-1), (0,2,0)\right\}\).
Determinamos la ecuación del plano mediante el producto factorial
\((0,1,-1) \times (0,2,0) = (2,0,0)\), por lo tanto,
\(\left\{(x,y,z) \in R^{3} / 2x = 0 \right\}\).

\hspace{6mm}\textbf{e.}
Presentar un subespacio de \(R^{3}\) de dimensión 0.

Por convención, el único subespacio de \(R^{3}\) con dimensión igual a 0 es
el que tiene como base al vector nulo:
\(\left\{(x,y,z) \in R^{3} / x = y = z = 0 \right\}\).

\hspace{6mm}\textbf{f.}
Presentar un subespacio de \(R^{3}\) de dimensión 3.
Dar una base del mismo.

Dado que tenemos un conjunto de 3 vectores que,
por lo que vimos en el punto \textbf{b.},
constituyen un conjunto linealmente independiente,
podemos utilizarlos como base:
\(B = \left\{(1,0,2), (0,2,0), (0,1,-1)\right\}\).

Al ser un conjunto LI de dimensión 3,
sabemos que permiten expresar al subespacio trivial \(R^{3} \subseteq R^{3}\).

\textbf{3.}
Dado el conjunto de vectores: \(S = \left\{(2,0,2), (0,3,3),(4,3,7)\right\}\)

\hspace{6mm}\textbf{a.}
Determinar si \(U = (1,2,4)\) es combinación lineal del conjunto de vectores de
\(S\).

Dado que \(S\) tiene 3 vectores,
de ser un conjunto LI podría generar el espacio vectorial \(R^{3}\) y,
por lo tanto,
a \(U\).
Calculamos su determinante:

\begin{align*}
  \begin{vmatrix}
    2 & 0 & 2 \\
    0 & 3 & 3 \\
    4 & 3 & 7 \\
  \end{vmatrix} & = 2 \cdot (21 - 9) - 0 + 2 \cdot (0 - 12) \\
                  & = 24 - 24 = \boxed{0}
\end{align*}

Como el determinante de la matriz es igual a 0,
el conjunto \(S\) no es LI y por lo tanto no genera \(R^{3}\).
Planteamos la ecuación del plano que generan los dos primeros vectores:

\begin{align*}
  (x,y,z) = \alpha(2,0,2) + \beta(0,3,3) \implies
  \begin{vmatrix}
    i & j & k \\
    2 & 0 & 2 \\
    0 & 3 & 3 \\
  \end{vmatrix} = i \cdot (-6) - j \cdot (6) + k \cdot 6
\end{align*}

Tomando los dos primeros vectores del conjunto obtenemos un plano,
que sigue la expresión \(-6x-6y+6z=0\).
Con esta expresión,
observamos que el tecer vector es coplanar a los dos primeros,
por lo cual restaría determinar si \(U\) también lo es o no.

\begin{align*}
  -6 \cdot 1 - 6 \cdot 2 + 6 \cdot 4 = \boxed{6 \neq 0}
\end{align*}

Como se observa,
\(U\) no es coplanar a los vectores de \(S\),
por lo cual concluimos que no es combinación lineal de los mismos.

\hspace{6mm}\textbf{b.}
¿Cuánto debe valer \(k \in R\) para que \(v = (-3,k,3)\) sea combinación
lineal de S?

Para poder expresar \(v\) como combinación lineal de \(S\) necesitamos que
aquel sea coplanar. Para ello, recurrimos a la expresión del plano que hemos
obtenido en el punto anterior:

\begin{align*}
  -6x-6y+6z=0 & \rightarrow -6\cdot (-3) - 6\cdot k + 6 \cdot 3 = 0 \\
              & 18 - 6k + 18 = 0                                    \\
              & -6k = -36                                           \\
              & \boxed{k = 6}
\end{align*}

Con \(k = 6\) \(v\) es coplanar al plano que genera \(S\) y, por lo tanto,
es combinación lineal de este.

\hspace{6mm}\textbf{c.}
¿\(S\) es una base de \(R^{3}\)?

Como notamos en el punto \textbf{a},
\(S\) cumple con el criterio de la cantidad de vectores para ser base de
\(R^{3}\).
Sin embargo,
no cumple con el segundo requisito,
esto es,
no es linealmente independiente.
Por ello mismo no genera todo \(R^{3}\),
sino un subespacio vectorial del mismo: un plano.

\hspace{6mm}\textbf{d.}
¿Qué subespacio genera \(S\)? Llámelo \(W\). Dar una dimensión y una base.

El subespacio generado por \(S\) es el plano \(-6x-6y+6z=0\).
Para dar con una base debemos de quedarnos con dos de los vectores que lo
integran, de manera tal que el conjunto sea linealmente independiente.
Este subconjunto es \(W = \left\{(2,0,2),(0,3,3)\right\}\) y se obtiene
simplemente quedándonos con los dos primeros vectores que mantienen la
independencia lineal.

Al delimitar un plano, sabemos que su \(dim(W) = 2\).

\hspace{6mm}\textbf{e.}
Siendo \(H = \left\{(x,y,z) \in R^{3} / 2x = y = -2z\right\}\),
dar una base de \(H\).
Determinar \(W + H\), una base y su dimensión.
Decir si la suma es o no directa.

Primero determinamos una expresión para \(H\):

\begin{align*}
          &
  \begin{cases}
    y = 2x \\
    z = \frac{2x}{-2} = -x
  \end{cases} \\
  (x,y,z) & = (x, 2x, -x) \\
  & = x(1,2,-1) \\
  &\boxed{\lambda(1,2,-1)}
\end{align*}

Con esta expresión de \(H\) podemos afirmar que es una recta que,
al no ser coplanar a \(W\),
podemos concluir que constituya una suma directa:

\begin{align*}
  W \oplus H & = R^{3}
\end{align*}

\textbf{4.}
En \(R^{2\times2}\) se consideran las matrices:
\(A = \begin{pmatrix}
  2&0\\0&-1
\end{pmatrix}\),
\(B = \begin{pmatrix}
  1&3\\-1&0
\end{pmatrix}\) y
\(C = \begin{pmatrix}
  0&0\\2&-1
\end{pmatrix}\)

\hspace{6mm}\textbf{a.}
Decidir conjunto generado por dichas matrices.

El conjunto generado por las matrices sigue la expresión:

\begin{align*}
  \begin{pmatrix}
    a&b\\c&d
  \end{pmatrix} & =
  \alpha
  \begin{pmatrix}
    2&0\\0&-1
  \end{pmatrix} +
  \beta
  \begin{pmatrix}
    1&3\\-1&0
  \end{pmatrix} +
  \gamma
  \begin{pmatrix}
    0&0\\2&-1
  \end{pmatrix}
\end{align*}

Que se puede expresar como un sistema de ecuaciones:

\begin{align*}
  \begin{cases}
    a = 2\alpha + \beta\\
    b = 3\beta \\
    c = -\beta + 2\gamma \\
    d = -\alpha - \gamma
  \end{cases}
\end{align*}

\hspace{6mm}\textbf{b.}
¿Es un conjunto linealmente independiente?

Para saber si las matrices son linealmente independientes podemos ver si 
la única solución para la combinación lineal  \(\alpha
\begin{pmatrix}
  2&0\\0&-1
\end{pmatrix} +
\beta
\begin{pmatrix}
  1&3\\-1&0
\end{pmatrix} +
\gamma
\begin{pmatrix}
  0&0\\2&-1
\end{pmatrix} = 0
\) es la solución trivial \(\alpha = \beta = \gamma = 0\):

\begin{align*}
  \begin{cases}
    0 = 2\alpha + \beta\\
    0 = 3\beta \\
    0 = -\beta + 2\gamma \\
    0 = -\alpha - \gamma
  \end{cases} & \rightarrow 3\beta = 0 \implies \beta = 0 \\
  & 2\alpha = 0 \implies \alpha = 0 \\
  & -\beta + 2\gamma = 0 \implies \gamma = 0
\end{align*}

Como la única solución es la solución trivial, 
concluimos que el conjunto es linealmente independiente.

\hspace{6mm}\textbf{c.}
¿Son base de \(R^{2\times2}\)? Justificar.

Aunque el conjunto de matrices sea linealmente independiente,
para constitur base de \(R^{2\times2}\) necesitamos 4 matrices,
por lo cual concluimos que el conjunto es generador de un subespacio vectorial
\(\subset R^{2\times2}\).

\end{document}
