\documentclass[11pt]{article}
\usepackage[a4paper, margin=2.54cm]{geometry}
\usepackage[spanish]{babel}

% imágenes
%\usepackage{graphicx}               
%\graphicspath{{img}}

% fuentes de conjuntos numéricos
\usepackage{amsfonts}        

% math
\usepackage{amsmath, amssymb}

% gráficos y plots
\usepackage{tikz}                   
%\usepackage{pgfplots}               
%\pgfplotsset{width=10cm, compat=1.9}
\usetikzlibrary{babel}

\setlength{\jot}{8pt}
\setlength{\parindent}{0cm}

% espacio entre párrafos
\usepackage{parskip}                

% cancelar términos
\usepackage{cancel}                 

% links
%\usepackage[colorlinks=true, 
%    urlcolor=blue]{hyperref}        

% shapes
%\usetikzlibrary{shapes.geometric}   

% incluir pdfs
%\usepackage{pdfpages}  

\title{Álgebra y Geometría Analítica\\ Trabajo Práctico $N^{o}2$\\Números complejos}
\author{Profesores:\\Mg. Silvia Ranieri\\Prof. Sebastian Scalise\\Alumno:\\Daniel Ise}
\date{Septiembre de 2024}

\begin{document}

\thispagestyle{empty}

\begin{center}
  \vspace*{.5cm}
  \includegraphics[scale=0.4]{../../img/udemm-logo.png}\\
  \vspace{.2cm}
  \Large
  \textbf{Facultad de Ingeniería}\\
  \textbf{Ingeniería en Sistemas}\\
  \vspace{2cm}

  \Huge
  Álgebra y Geometría Analítica\\
  Trabajo Práctico \(N^{o} 3\)
  \vfill

  \raggedright
  \Large
  Docentes:
  \begin{itemize}
    \item[] Mg. Silvia Ranieri\\
    \item[] Prof. Sebastian Scalise\\
  \end{itemize}
  Alumno:
  \begin{itemize}
    \item[] Daniel Ise
  \end{itemize}
  \vspace{1cm}

  \centering
  2024
\end{center}

\pagebreak

\textbf{1. Dadas las siguientes matrices, efectuar cuando sea posible:}

\textbf{a. \(C \cdot A\)}

Siendo \(C \in \mathbb{R}^{3 \times 3}\)
y \(A \in \mathbb{R}^{3 \times 1}\),
es posible operar el producto \(C \cdot A\),
ya que \(C\) tiene 3 columnas y \(A\) 3 filas.
El resultado del producto sería una matriz \(\mathbb{R}^{3 \times 1}\):

\begin{align*}
  \begin{pmatrix}
    4  & 1 & -1 \\
    -1 & 0 & 1  \\
    -3 & 0 & 0  \\
  \end{pmatrix}
  \cdot
  \begin{pmatrix}
    1  \\
    -3 \\
    0  \\
  \end{pmatrix}
   & =
  \boxed{
    \begin{pmatrix}
      1  \\
      -1 \\
      -3 \\
    \end{pmatrix}
  }
\end{align*}

\textbf{b. \(C \cdot D\)}

El producto se puede operar porque \(C\) tiene 3 columnas,
en tanto \(D\) tiene 3 filas.
El resultado sería una matriz \(\mathbb{R}^{3 \times 2}\).

\begin{align*}
  \begin{pmatrix}
    4  & 1 & -1 \\
    -1 & 0 & 1  \\
    -3 & 0 & 0  \\
  \end{pmatrix}
  \cdot
  \begin{pmatrix}
    2  & -1 \\
    -4 & 0  \\
    0  & 1  \\
  \end{pmatrix}
   & =
  \boxed{
    \begin{pmatrix}
      4  & -5 \\
      -2 & 2  \\
      -6 & 3  \\
    \end{pmatrix}
  }
\end{align*}

\textbf{c. \(A \cdot B\)}

El producto se puede operar,
puesto que \(A\) cuenta con 1 columna y \(B\) con 1 fila.
El resultado sería una matriz de tamaño \(\mathbb{R}^{3 \times 3}\).

\begin{align*}
  \begin{pmatrix}
    1  \\
    -3 \\
    0  \\
  \end{pmatrix}
  \cdot
  \begin{pmatrix}
    3 & -1 & 2 \\
  \end{pmatrix}
   & =
  \boxed{
    \begin{pmatrix}
      3  & -1 & 2  \\
      -9 & 3  & -6 \\
      0  & 0  & 0  \\
    \end{pmatrix}
  }
\end{align*}

\textbf{d. \(B \cdot A\)}

Este producto también puede operarse,
en la medida que \(B \in \mathbb{R}^{1 \times 3}\)
y \(A \in \mathbb{R}^{3 \times 1}\).
El resultado sería una matriz de un único elemento.

\begin{align*}
  \begin{pmatrix}
    3 & -1 & 2 \\
  \end{pmatrix}
  \cdot
  \begin{pmatrix}
    1  \\
    -3 \\
    0  \\
  \end{pmatrix}
   & =
  \boxed{
    \begin{pmatrix}
      6 \\
    \end{pmatrix}
  }
\end{align*}

\textbf{e. \(B \cdot D - 2B \cdot E\)}

\(B\) es una matriz de forma \(\mathbb{R}^{1 \times 3}\),
por lo tanto, existe el producto tanto con \(D\), como con \(E\),
de tamaño \(\mathbb{R}^{3 \times 2}\).
El resultado debería ser una matriz de tipo \(\mathbb{R}^{1 \times 2}\).

\begin{align*}
  \begin{pmatrix}
    3 & -1 & 2 \\
  \end{pmatrix}
  \cdot
  \begin{pmatrix}
    2  & -1 \\
    -4 & 0  \\
    0  & 1  \\
  \end{pmatrix}
   & +
  \begin{pmatrix}
    6 & -2 & 4 \\
  \end{pmatrix}
  \cdot
  \begin{pmatrix}
    4 & 1  \\
    0 & -6 \\
    2 & 0  \\
  \end{pmatrix}
  =    \\
  \begin{pmatrix}
    10 & -1 \\
  \end{pmatrix}
   & +
  \begin{pmatrix}
    32 & 18 \\
  \end{pmatrix}
  =
  \boxed{
    \begin{pmatrix}
      42 & 17 \\
    \end{pmatrix}
  }
\end{align*}

\textbf{f. \(det(C)\)}

Podemos calcular el determinante de \(C\) tomando su fila 3
y calculando el producto de cada uno de sus elementos por su adjunto:

\begin{align*}
  \begin{vmatrix}
    4  & 1 & -1 \\
    -1 & 0 & 1  \\
    -3 & 0 & 0  \\
  \end{vmatrix}
  =
  -3
  \cdot
  (-1)^{3+1}
  \cdot
  \begin{vmatrix}
    1 & -1 \\
    0 & 1  \\
  \end{vmatrix}
  = \boxed{-3}
\end{align*}

\textbf{g. \(det(B\cdot A)\)}

Sabemos que \(B\cdot A = (6)\),
por lo tanto,
\(det(B\cdot A) = 6\).

\textbf{h. \(C^{-1}\)}.

Sabemos que \(C \in \mathbb{R}^{3 \times 3}\) es regular,
puesto que \(-3 \neq 0\).
Determinamos \(C^{-1}\) por método de la matriz adjunta.
Para ello, determinamos primero \(adj(C)\).

\begin{align*}
  adj(C) = \begin{pmatrix}
             0 & -3 & 0  \\
             0 & -3 & -3 \\
             1 & -3 & 1  \\
           \end{pmatrix}
\end{align*}

Multiplicamos transpuesta por \(\frac{1}{|C|}\):

\begin{align*}
  \frac{adj(C)^{T}}{|C|}
  =
  \begin{pmatrix}
    0  & 0  & 1  \\
    -3 & -3 & -3 \\
    0  & -3 & 1  \\
  \end{pmatrix}
  \cdot
  \frac{1}{-3}
  =
  \boxed{
    \begin{pmatrix}
      0 & 0 & -1/3 \\
      1 & 1 & 1    \\
      0 & 1 & -1/3 \\
    \end{pmatrix}
  }
\end{align*}

Comprobamos:
\begin{align*}
  \begin{pmatrix}
    0 & 0 & -1/3 \\
    1 & 1 & 1    \\
    0 & 1 & -1/3 \\
  \end{pmatrix}
  \cdot
  \begin{pmatrix}
    4  & 1 & -1 \\
    -1 & 0 & 1  \\
    -3 & 0 & 0  \\
  \end{pmatrix}
  =
  \begin{pmatrix}
    1 & 0 & 0 \\
    0 & 1 & 0 \\
    0 & 0 & 1 \\
  \end{pmatrix}
\end{align*}

\textbf{i. \((B\cdot A)^{-1}\)}

\(B\cdot A = (6)\),
por lo tanto,
\((B\cdot A)^{-1}=\boxed{(1/6)}\)

\textbf{j. \((E\cdot D^{T})^{-1}\)}

Operamos primero producto de \(E \cdot D^{T}\):

\begin{align*}
  \begin{pmatrix}
    4 & 1  \\
    0 & -6 \\
    2 & 0  \\
  \end{pmatrix}
  \cdot
  \begin{pmatrix}
    2  & -4 & 0 \\
    -1 & 0  & 1 \\
  \end{pmatrix}
  =
  \begin{pmatrix}
    7 & -16 & 1  \\
    6 & 0   & -6 \\
    4 & -8  & 0  \\
  \end{pmatrix}
\end{align*}

Comprobamos que sea matriz regular,
para ello calculamos determinante tomando tercera columna:

\begin{align*}
  1 \cdot (-48) - (-6) \cdot 8 = \boxed{0}
\end{align*}

Como el determinante es igual a 0,
concluimos que es matriz singular, 
por lo tanto, no tiene inversa.

\end{document}
