\documentclass[12pt]{article}
\usepackage[a4paper, margin=2.54cm]{geometry}
\usepackage[spanish]{babel}

% imágenes
%\usepackage{graphicx}
%\graphicspath{{img}}

% fuentes de conjuntos numéricos
\usepackage{amsfonts}

% math
\usepackage{amsmath, amssymb}

% gráficos y plots
\usepackage{tikz}
%\usepackage{pgfplots}
%\pgfplotsset{width=10cm, compat=1.9}
\usetikzlibrary{babel}

\setlength{\jot}{8pt}
\setlength{\parindent}{0cm}

% espacio entre párrafos
\usepackage[skip=10pt plus1pt, indent=12pt]{parskip}

% cancelar términos
\usepackage{cancel}

% links
%\usepackage[colorlinks=true,
%    urlcolor=blue]{hyperref}

% shapes
%\usetikzlibrary{shapes.geometric}

% incluir pdfs
%\usepackage{pdfpages}

\begin{document}

\thispagestyle{empty}

\begin{center}
	\vspace*{.5cm}
	\includegraphics[scale=.6]{~/Pictures/udemm-logo.png}\\
	\vspace{.2cm}
	\Large
	\textbf{Facultad de Ingeniería}\\
	\textbf{Ingeniería en Sistemas}\\
	\vspace{2cm}

	\Huge
	Álgebra y Geometría Analítica\\
	Trabajo Práctico N\(^\circ\) 7\\
	\vfill

	\raggedright
	\Large
	Docentes:
	\begin{itemize}
		\item[] Mg. Silvia Ranieri 		\\
		\item[] Prof. Sebastián Scalise \\
	\end{itemize}
	Alumno:
	\begin{itemize}
		\item[] Daniel Ise
	\end{itemize}
	Legajo:
	\begin{itemize}
		\item[] 28547
	\end{itemize}
	Fecha:
	\begin{itemize}
		\item[] Octubre, 2024
	\end{itemize}
\end{center}

\pagebreak

1. Hallar todos los valores de \(k \in \mathbb{R}\) para los cuales la matriz
\(\begin{pmatrix}
	0 & 0 & -3 \\1&0&-1\\k&1&-1
\end{pmatrix}\) tiene a 1 como autovalor.

Si 1 es autovalor de \(A\), entonces debe cumplirse que:

\begin{align*}
	(A-1I)\vec{v}=\vec{0}
\end{align*}

Operamos la resta de matrices:

\begin{align*}
	\begin{pmatrix}
		(0-1) & 0     & -3     \\
		1     & (0-1) & -1     \\
		k     & 1     & (-1-1) \\
	\end{pmatrix}
	\begin{pmatrix}
		x \\
		y \\
		z \\
	\end{pmatrix} =
	\begin{pmatrix}
		0 \\
		0 \\
		0 \\
	\end{pmatrix}
\end{align*}

Construimos el sistema de ecuaciones:

\begin{align*}
	\begin{cases}
		-x-3z=0 \implies x=-3z          \\
		x-y-z=0 \implies y=-4z          \\
		kx+y-2z=0 \implies -3zk-4z-2z=0 \\
	\end{cases} \\
	-3zk-4z-2z=0                    \\
	(k+2)(-3z)=0
\end{align*}

En este punto, deducimos que \(z\) debe ser distinto que 0,
ya que \(z = 0\) implicaría \(x=0\) e \(y=0\) y,
como sabemos,
el sistema de ecuaciones debe tener una solución que no sea
la trivial. Por lo tanto, el término que hace 0 a la expresión es
\(k+2=0 \implies \boxed{k=-2}\).

\pagebreak

2. Dada la matriz \(A=\begin{pmatrix}
	-3 & -2 & 5 \\2&1&-1\\0&0&-2
\end{pmatrix}\), hallar sus autovalores,
subespacio asociado a cada uno de ellos y los autovectores.
Decir si A es diagonalizable. Justificar.

Primero buscamos autovalores:

\begin{align*}
	\begin{vmatrix}
		(-3-\lambda) & -2          & 5            \\
		2            & (1-\lambda) & -1           \\
		0            & 0           & (-2-\lambda)
	\end{vmatrix} = 0       \\
	(-2-\lambda)[(-3-\lambda)(1-\lambda)+4]=0       \\
	(-2-\lambda)[-3+3\lambda-\lambda+\lambda^{2}+4] \\
	(-2-\lambda)(\lambda^{2}+2\lambda+1)            \\
	(-2-\lambda)(x+1)^{2}
\end{align*}

Entonces: \(\lambda_{1}= -1\) y \(\lambda_{2}=-2\).
Calculamos autovectores.

\begin{align*}
	E(\lambda_{1}) =
	\begin{pmatrix}
		-2 & -2 & 5  \\
		2  & 2  & -1 \\
		0  & 0  & -1 \\
	\end{pmatrix}
	\begin{pmatrix}
		x \\y\\z
	\end{pmatrix}
	=
	\begin{pmatrix}
		0 \\0\\0
	\end{pmatrix}
\end{align*}

Deducimos que \(z=0 \implies x=-y\). Entonces:

\begin{align*}
	E(\lambda_{1})=\begin{Bmatrix}
		               \begin{pmatrix}
			-y \\y\\0
		\end{pmatrix}
	               \end{Bmatrix}
	\implies
	\vec{v_{1}}=
	\begin{pmatrix}
		-1 \\1\\0
	\end{pmatrix}
\end{align*}

Calculamos \(E(\lambda_{2})\):

\begin{align*}
	E(\lambda_{2}) =
	\begin{pmatrix}
		-1 & -2 & 5  \\
		2  & 3  & -1 \\
		0  & 0  & 0  \\
	\end{pmatrix}
	\begin{pmatrix}
		x \\y\\z
	\end{pmatrix}
	=
	\begin{pmatrix}
		0 \\0\\0
	\end{pmatrix}                                                \\
	\begin{cases}
		-x-2y+5z = 0 \implies x=5z-2y \\
		2(5z-2y) + 3y -z =0 \implies \boxed{y=9z} \land \boxed{x=-13z}
	\end{cases} \\
	E(\lambda_{2})
	=
	\begin{Bmatrix}
		\begin{pmatrix}
			-13z \\9z\\z
		\end{pmatrix}
	\end{Bmatrix}
	\implies
	v_{2}
	=
	\begin{pmatrix}
		-13 \\9\\1
	\end{pmatrix}
\end{align*}

Por último,
concluimos que \(A\) no es diagonalizable,
puesto que es una matriz de orden 3,
pero tiene solos 2 autovectores.

\pagebreak

3. Dada la matriz \(A=\begin{pmatrix}
	-3 & 0 & 0 \\0&2&0\\-1&4&2
\end{pmatrix}\), hallar sus autovalores,
subespacio asociado a cada uno de ellos y los autovectores.
Decir si A es diagonalizable. Justificar.

Dado que \(A\) es una matriz triangular,
esto es,
una matriz con todos sus elementos por encima de la diagonal principal iguales a 0,
podemos deducir que sus autovalores son \(\lambda_{1} = -3\)
y \(\lambda_{2} = 2\).

Calculamos el subespacio para \(\lambda_{1} = -3\):

\begin{align*}
	E(\lambda_{1}) =
	\begin{pmatrix}
		0 & 0 & 0 \\0&5&0\\-1&4&5
	\end{pmatrix}
	\begin{pmatrix}
		x \\y\\z
	\end{pmatrix}
	=
	\begin{pmatrix}
		0 \\0\\0
	\end{pmatrix}                                   \\
	\boxed{y=0} \implies -x+5z \implies \boxed{x=5z} \\
	E(\lambda_{1}) =
	\begin{Bmatrix}
		\begin{pmatrix}
			5z \\0\\z
		\end{pmatrix}
	\end{Bmatrix}
	\implies
	v_{1} =
	\begin{pmatrix}
		5 \\0\\1
	\end{pmatrix}
\end{align*}

Ahora el subespacio para \(\lambda_{2} = 2\):

\begin{align*}
	E(\lambda_{2}) =
	\begin{pmatrix}
		-5 & 0 & 0 \\
		0  & 0 & 0 \\
		-1 & 4 & 0
	\end{pmatrix}
	\begin{pmatrix}
		x \\y\\z
	\end{pmatrix}
	=
	\begin{pmatrix}
		0 \\0\\0
	\end{pmatrix}                \\
	\boxed{x=0} \land \boxed{y=0} \\
	E(\lambda_{2}) =
	\begin{Bmatrix}
		\begin{pmatrix}
			0 \\0\\z
		\end{pmatrix}
	\end{Bmatrix}
	\implies
	v_{2} =
	\begin{pmatrix}
		0 \\0\\1
	\end{pmatrix}
\end{align*}

En este caso,
\(A\) tampoco sería diagonalizable,
puesto que es una matriz de \(3x3\),
pero solo cuenta con 2 autovectores.

\pagebreak

4. Determinar si
\(A = \begin{pmatrix}
	0 & 0 & -2 \\
	1 & 2 & 1  \\
	1 & 0 & 3
\end{pmatrix}\)
es diagonalizable.
Justificar.

Para saber si \(A\) es diagonalizable,
primero determinamos sus autovalores:

\begin{align*}
	\begin{vmatrix}
		-\lambda & 0           & -2          \\
		1        & (2-\lambda) & 1           \\
		1        & 0           & (3-\lambda)
	\end{vmatrix}
	= 0                                                \\
	2(2-\lambda) + (3-\lambda)(-\lambda)(2-\lambda) =0 \\
	(2-\lambda)[2+(3-\lambda)(-\lambda)] =0            \\
	(2-\lambda)(\lambda^{2}-3\lambda+2) = 0            \\
	\boxed{\lambda_{1} = 1, m = 1} \land \boxed{\lambda_{2} = 2, m=2}
\end{align*}

Determinamos autovector de \(\lambda_{1}\):

\begin{align*}
	\begin{pmatrix}
		-1 & 0 & -2 \\1&1&1\\1&0&2
	\end{pmatrix} \implies \boxed{x=-z} \land \boxed{y=z} \\
	E(\lambda_{1})=
	\begin{Bmatrix}
		\begin{pmatrix}
			-2z \\z\\z
		\end{pmatrix}
	\end{Bmatrix}
	\implies
	v_{1} =
	\begin{pmatrix}
		-2 \\1\\1
	\end{pmatrix}
\end{align*}

Por último, buscamos autovectores de \(\lambda_{2}\):

\begin{align*}
	\begin{pmatrix}
		-2 & 0 & -2 \\1&0&1\\1&0&1\\
	\end{pmatrix} \\
	E(\lambda_{2})=
	\begin{Bmatrix}
		\begin{pmatrix}
			-z \\y\\z
		\end{pmatrix}
	\end{Bmatrix}
	\implies
	v_{2}=
	\begin{pmatrix}
		-1 \\0\\1
	\end{pmatrix}
	\land
	v_{3}
	\begin{pmatrix}
		0 \\1\\0
	\end{pmatrix}
\end{align*}

Como \(A\) es una matriz de orden 3 y tiene 3 autovectores,
concluimos que \(A\) es diagonalizable.

\pagebreak

5. Dada la matriz
\(A = \begin{pmatrix}
	-1 & 0 & 0 \\\alpha&-1&0\\1&1&2
\end{pmatrix}\)
estudiar,
en cada caso,
para qué valores de \(\alpha\) la matriz es diagonalizable.
Elegir uno de esos valores y encontrar la matriz P que la diagonaliza.

Al ser \(A\) una matriz triangular inferior,
deducimos que sus autovalores son \(\lambda_{1}=-1\),
con multiplicidad 2,
y \(\lambda=2\),
con multiplicidad 1.

Por lo tanto, con el autovalor -1 deberiamos tener 2 autovectores.
Planteamos la búsqueda del subespacio:

\begin{align*}
	\begin{pmatrix}
		0 & 0 & 0 \\\alpha&0&0\\1&1&3
	\end{pmatrix}
\end{align*}

Notamos que \(\alpha \neq 0 \implies x=0\) y,
por lo tanto,
podríamos expresar el autovector en función de una sola de las variables.
Ello implicaría que \(A\) no sería diagonalizable.
Por ello concluimos que \(\alpha = 0\) para que \(A\) sea diagonalizable.

\begin{align*}
	\begin{pmatrix}
		0 & 0 & 0 \\0&0&0\\1&1&3
	\end{pmatrix}
	\implies
	x=-y-3z \\
	E(\lambda_{1}) =
	\begin{Bmatrix}
		\begin{pmatrix}
			-y-3z \\y\\z
		\end{pmatrix}
	\end{Bmatrix}
	\implies
	v_{1} =
	\begin{pmatrix}
		-1 \\1\\0
	\end{pmatrix}
	\land
	v_{2} =
	\begin{pmatrix}
		-3 \\0\\1
	\end{pmatrix}
\end{align*}

Y para \(\lambda_{2}\):

\begin{align*}
	\begin{pmatrix}
		-3 & 0 & 0 \\0&-3&0\\1&1&0
	\end{pmatrix} \\
	E(\lambda_{2}) =
	\begin{Bmatrix}
		\begin{pmatrix}
			0 \\0\\z
		\end{pmatrix}
	\end{Bmatrix}
	\implies
	v_{3} =
	\begin{pmatrix}
		0 \\0\\1
	\end{pmatrix}
\end{align*}

Con ello, obtenemos \(P\):

\begin{align*}
	P =
	\begin{pmatrix}
		-1 & -3 & 0 \\
		1  & 0  & 0 \\
		0  & 1  & 1
	\end{pmatrix}
	 & \land
	D =
	\begin{pmatrix}
		-1 & 0  & 0 \\
		0  & -1 & 0 \\
		0  & 0  & 2
	\end{pmatrix}
\end{align*}

\end{document}
