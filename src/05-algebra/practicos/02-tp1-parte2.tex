\documentclass{article}
\usepackage[margin=2.54cm]{geometry}
%\usepackage{graphicx}               % imágenes
%\graphicspath{{img}}
\usepackage{amsfonts}               % fuentes de conjuntos numéricos
\usepackage{amsmath, amssymb}       % símbolos
%\usepackage{tikz}                   % gráficos
%\usepackage{pgfplots}               % plots
%\pgfplotsset{width=10cm, compat=1.9}
\setlength{\jot}{8pt}
\setlength{\parindent}{0cm}
\usepackage{parskip}                % espacio entre párrafos
\usepackage{cancel}                 % cancelar términos
%\usepackage[colorlinks=true, 
%    urlcolor=blue]{hyperref}        % links
%\usetikzlibrary{shapes.geometric}   % shapes
%\usepackage{pdfpages}               % incluir pdfs

\title{Álgebra y Geometría Analítica\\ Trabajo Práctico $N^o1$\\Parte 2: Rectas y planos}
\author{Profesores:\\Mg. Silvia Ranieri\\Prof. Sebastian Scalise\\Alumno:\\Daniel Ise}
\date{Septiembre de 2024}

\begin{document}

\maketitle

\textbf{1. Encontrar ecuación paramétrica de:}

\textbf{a. Recta que pasa por (1, 3, -1) y (5,2,3)}

\begin{align*}
    (5,2,3) - (1,3,-1) = (4, -1, 4) \\
    r = (5,2,3) + \lambda (4,-1,4)  \\
    r =
    \begin{cases}
        x = 4 \lambda + 5 \\
        y = - \lambda + 2 \\
        z = 4 \lambda + 3 \\
    \end{cases}
\end{align*}

\textbf{b. Recta que pasa por origen, paralela a \(\frac{x+3}{5} = \frac{y+2}{-2} = \frac{z-5}{-3}\)}

\begin{align*}
    \begin{cases}
        x = 5 \lambda  \\
        y = -2 \lambda \\
        z = -3 \lambda \\
    \end{cases}
\end{align*}

\textbf{2. Dadas \(r_1: \lambda (1,4,-2) + (4,-3,0)\),
    \(r_2: \beta (5,7,-3) + (-5,-13,4)\) y
    los puntos \(P (1,-2,0)\), \(Q (-2,-1,0)\) y \(R(1,0,4)\):}

\textbf{a. Intersección \(r_1\) y \(r_2\)}

\begin{align*}
    r_1 =
    \begin{cases}
        x = \lambda + 4  \\
        y = 4 \lambda -3 \\
        z = -2 \lambda   \\
    \end{cases}   &
    r_2 =
    \begin{cases}
        x = 5 \beta - 5  \\
        y = 7 \beta - 13 \\
        z = -3 \beta + 4 \\
    \end{cases}                                                       \\
    r_1 \cap r_2          & =
    \begin{cases}
        \lambda + 4 = 5 \beta - 5   \\
        4 \lambda -3 = 7 \beta - 13 \\
        -2 \lambda = -3 \beta + 4   \\
    \end{cases}                                            \\
    \lambda = 5 \beta - 9 & \rightarrow 4 (5 \beta - 9) - 3 = 7 \beta - 13 \\
    20 \beta - 36 - 3     & = 7 \beta - 13                                 \\
    13 \beta - 39         & = - 13                                         \\
    \beta                 & = \frac{26}{13}                                \\
    \boxed{\beta = 2}     & \rightarrow \boxed{\lambda = 1}                \\
    \text{Comprobamos:}                                                    \\
    -2 \cdot 1 = -3 \cdot 2 + 4                                            \\
    \boxed{-2 = -2}                                                        \\
    \text{Entonces, intersección:}                                         \\
    \boxed{(5, 1, -2)}                                                     \\
\end{align*}

\textbf{b. Paralela a \(r_2\) que pase por \((1,-2,0)\)}

\begin{align*}
    \boxed{\gamma (5,7,-3) + (1,-2,0)}
\end{align*}

\textbf{c. Distancia de \(R\) a \(r_2\)}

\begin{align*}
    d(r_2, R) & = \frac{|[(1,0,4) - (-5,-13,4)] \times (5,7,-3)|}{|5,7,-3|}\\
    d(r_2, R) & = \frac{|(6,13,0) \times (5,7,-3)|}{\sqrt{5^2 + 7^2 + (-3)^2}}\\
    d(r_2, R) & = \frac{|(-39,18,-23)|}{\sqrt{25 + 49 + 9}}\\
    d(r_2, R) & = \frac{|\sqrt{(-39)^2 + 18^2 + (-23)^2}}{\sqrt{83}}\\
    d(r_2, R) & = \frac{\sqrt{2374}}{\sqrt{83}}\\
    & \boxed{d(r_2, R) \approxeq 5.348}\\
\end{align*}

\textbf{4. Hallar k para que las rectas se intercepten e intersección.}

\begin{align*}
    r_1 =
    \begin{cases}
        x - y = 3\\
        y - 2z = 1\\
    \end{cases}\\
    y = x - 3 \rightarrow 
\end{align*}

\end{document}
