\documentclass[aspectratio=1610,t, 10pt]{beamer}
\usetheme{singapore}

\title{Practica Álgebra y Geometría Analítica\\Unidad I}
\author{Daniel Ise}
\institute{Universidad de la Marina Mercante}
\date{2025}

% sacamos nav
\setbeamertemplate{navigation symbols}{}

\begin{document}

\frame{\titlepage}

\begin{frame}
1. Dados: \(\vec{u} = (5,2)\), \(\vec{v} = (-1, -1)\) y \(\vec{w} = (1, -2)\),
obtener \(\vec{s}\):
\begin{itemize}
    \item[a.]\(2u + 3s - w = s + v\)
    \item[b.]\(3(u+v) - 2s = -(s+w)\)
\end{itemize} 
Calcular: \(2u - 5w - 3v\)
\end{frame}

\begin{frame}
2. Obtener \(\alpha\) y \(\beta\) tales que: 
\(\vec{v} = \alpha\vec{u} + \beta\vec{w}\),
siendo \(\vec{v} = (1,4)\), \(\vec{u} = (1,2)\), \(\vec{w} = (0,2)\).
\end{frame}

\begin{frame} 
3. Obtener distancia entre R y S:
\begin{itemize}
    \item[a.] \(R=(-3,2)\) \(S=(0,2)\)
    \item[b.] \(R=(2,3)\) \(S=(5,3)\)
\end{itemize}
\end{frame}

\begin{frame} 
4. Hallar producto escalar de:
\begin{itemize}
    \item[a.] \(\vec{v}=(2,1)\) \(\vec{u}=(0,1)\)
    \item[b.] \(\vec{v}=(1,0)\) \(\vec{u}=(2,0)\)
\end{itemize}
\end{frame}

\begin{frame} 
5. Hallar ángulo entre vectores:  \(\vec{v}=(2,-3)\) y \(\vec{u}=(3,-1)\).
\begin{itemize}
    \item[a.] Obtener \(\vec{w}\) perpendicular a \(\vec{v}\) y \(\vec{u}\)
    \item[b.] Obtener \(\vec{s}\) paralelo a \(\vec{v}\) y \(\vec{u}\)
\end{itemize}
\end{frame}

\begin{frame} 
6. Sean:  \(\vec{u}= 4i + 3j\) y \(\vec{v}= \alpha i - 2j\),
obtener \(\alpha\in\mathbb{R}\) tal que:
\begin{itemize}
    \item[a.] \(\vec{u}\) y \(\vec{v}\) sean ortogonales o perpendiculares
    \item[b.] \(\vec{u}\) y \(\vec{v}\) sean paralelos
    \item[c.] \(\vec{u}\) y \(\vec{v}\) tengan un ángulo de \(2\pi/3\)
\end{itemize}
\end{frame}

\begin{frame} 
7. Dados los puntos \(A = (2,3)\),\(B = (1,-5)\) y \(C = (0,-2)\),
calcular perímetro del triángulo y ángulos interiores.
\end{frame}

\begin{frame} 
8. Dados: \(\vec{u}=(5,2)\), \(\vec{v}=(-3,7)\) y \(\vec{w}=(4,3)\),
hallar:
\begin{itemize}
    \item[a.] \(\vec{u}+\vec{v}\)
    \item[b.] \(\vec{u}-\vec{w}\)
    \item[c.] \(3\vec{u}-2\vec{v}+5{w}\)
    \item[d.] \(|\vec{u}|\)
    \item[e.] \(|\vec{v}|\)
    \item[f.] \(|\vec{w}|\)
    \item[g.] Obtener \(\vec{s}\): \(3u-2s+4v=2(w-u)\)
\end{itemize}
\end{frame}

\begin{frame} 
9. Dados los puntos \(A = (5,4)\),\(B = (1,-3)\) y \(C = (-2,0)\),
calcular perímetro, área y ángulos interiores del triángulo.
\end{frame}

\begin{frame} 
10. Dados \(\vec{u}=(-5,2)\), \(\vec{v}=(-2,6)\) y \(\vec{w}=(1,1)\):
\begin{itemize}
    \item[a.] Obtener \(\vec{t}\): \(3(u+v) - 2t = -(t - 2w)\)
    \item[b.] Obtener \(\alpha\) y \(\beta\), tales que \(v = \alpha u + \beta w\)
\end{itemize}
\end{frame}

\begin{frame} 
11. Dados \(P = (-1,2)\), \(Q = (0,5)\) y \(R = (1,1)\), hallar:
\begin{itemize}
    \item[a.] Cuarto vértice \(S\) y centro \(M\) del paralelogramo.
    \item[b.] Perímetro
    \item[c.] Ángulos
    \item[d.] Área
\end{itemize}
\end{frame}


\end{document}