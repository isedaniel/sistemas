\section{Cónicas}

\subsection{Recta}

\subsubsection{Ecuación}

\begin{align*}
    y = mx + b
\end{align*}

Siendo $m$ la pendiente, $b$ la ordenada.

Dos rectas paralelas tienen igual pendiente.

Dos rectas son perpediculares si su pendiente es la \textbf{inversa multiplicativa cambiada de signo}.

\begin{align*}
    m = -\frac{1}{m_2}
\end{align*}

\subsubsection{Ecuación dados dos puntos}

\begin{align*}
    m = \frac{y_2 - y_1}{x_2 - x_1}
\end{align*}

\subsubsection{Punto medio entre dos puntos}

\begin{align*}
    M = \left( \frac{x_1 + x_2}{2}, \frac{y_1 + y_2}{2} \right)
\end{align*}

\subsubsection{Distancia}

Pitágoras.

\subsubsection{Distancia de la recta a un punto}

Escribimos la recta de forma implícita:

\begin{align*}
    Ax + By + C = 0
\end{align*}

Y dado el punto $P(p_1,p_2)$:

\begin{align*}
    d(P,r) = \frac{|A \cdot p_1 + B \cdot p_2 + C|}{\sqrt{A^2 + B^2}}
\end{align*}

\subsubsection{Distancia de una recta al origen}

Se deduce de la expresión anterior:

\begin{align*}
    d(O, r) = \frac{|C|}{\sqrt{A^2 + B^2}}
\end{align*}


\subsection{Cónicas}

Las cónicas resultan de la intersección de un cono de expresión $z^2 = y^2 + x^2$,
con un plano.

En función de la relación entre el ángulo de conicidad $\alpha$ y la inclinación
del plano respecto del eje del cono $\beta$, se obtienen las distintas cónicas.

\begin{itemize}
    \item Circunferencia, la intersencción de la cónica y un plano perpendicular al eje principal del cono.
    \item Elipse, plano inclinado que corta una de las hojas de la superficie
    \item Parábola, corta una de las hojas y es paralelo a la generatriz
    \item Hipérbola, corta ambas hojas del cono
\end{itemize}

\subsection{Circunferencia}

El plano que corta a la cónica tiene un ángulo $\beta = 90^o$ respecto del 
eje, es decir, es perpendicular al eje.

\textbf{Definición.}

Es el lugar geométrico donde los puntos que la forman equidistan de un punto 
fijo que se llama centro:

\begin{align*}
    Circ(C, r) = P: d(C,P) = r
\end{align*}

\subsubsection{Ecuación de la circunferencia centro (0,0)}

\begin{align*}
    x^2 + y^2 = r^2
\end{align*}

\subsubsection{Ecuación canónica, de centro (a,b)}

\begin{align*}
    (x-a)^2 + (y-b)^2 = r^2
\end{align*}

\subsubsection{Ecuación general}

Resulta de desarrollar la canónica:

\begin{align*}
    x^2 + y^2 + Ax + By + C = 0
\end{align*}

Para volver a la canónica completamos el cuadrado.

Si tenemos puntos para construir la circunferencia usamos esta.

\textbf{Condiciones para que sea circunferencia.}
\begin{enumerate}
    \item Los coeficientes de $x^2$ e $y^2$ deben ser iguales
    \item No deben tener término $xy$
    \item El radio al cuadrado debe ser mayor a 0
\end{enumerate}

\subsubsection{Ecuación de la tangente y normal a la circunferencia}

Con circunferencia de centro $C$ y un punto $P$ perteneciente a la misma.

La \textbf{recta normal} es la que contiene a $C$ y $P$.

\textbf{recta tangente} es la perpendicular a la normal.

Simplemente calculo la pendiente de la recta normal y con la perpendicular 
tengo la tangente.

\subsection{Elipse}

Es la sección producida en una superficie cónica por un plano oblícuo al eje,
que no sea paralelo a la generatriz, y que su ángulo sea mayor al de la 
generatriz y distinto de $90^o$.

\begin{align*}
    \alpha < \beta \neq 90^o
\end{align*}

Es una curva cerrada.

Se define como el lugar geométrico de los puntos cuya suma de distancias a 
dos puntos fijos -llamados \textit{focos}- es constante.

\subsubsection{Elementos de la elipse}

Focos: son los puntos fijos $F$ y $F'$.

Eje mayor o focal: es el segmento que pasa por los focos y llega a la elipse. Mide $2a$

Eje menor: es el eje que divide por la mitad a la distancia focal, de medida $2b$.

Centro: es donde se interscan los ejes.

Distancia focal: Distancia entre los dos focos, $2c$

Vértices: puntos de intersección de la elipse con los ejes mayor y menor

Ejes de simetría: rectas que contienen al eje mayor y menor 

Centro de simetría: coincide con el centro de la elipse 

La distancia de cualquier punto $P$ de la elipse a los focos permanece constante:

\begin{align*}
    PF + PF' = 2a
\end{align*}

\subsubsection{Excentricidad de la elipse}

Determina que tan alargada o redonda es. Es el cociente entre el 
semieje focal y el semieje mayor:

\begin{align*}
    e = \frac{c}{a}
\end{align*}

Puede variar entre 0 y 1.

Cuando la excentricidad es 0, estamos ante una circunferencia.

\subsubsection{Ecuación de la elipse centrada en origen}

\begin{align*}
    \frac{x^2}{a^2} + \frac{y^2}{b^2} = 1
\end{align*}

Recordar que:

\begin{align*}
    c^2 = a^2 + b^2
\end{align*}

Cuando los focos están sobre la vertical:

\begin{align*}
        \frac{x^2}{b^2} + \frac{y^2}{a^2} = 1
\end{align*}

Es decir, $a > b$ siempre.

\subsubsection{Ecuación de la elipse centro $(x_0, y_0)$}

\begin{align*}
    \frac{(x - x_0)^2}{a^2} + \frac{(y^2 - y_0)^2}{b^2} = 1
\end{align*}

Si está centrada en el eje vertical, intercambian denominador.
Solo si el denominador de y es mayor al denominador de x.

\subsubsection{Ecuación general de la elipse}

\begin{align*}
    Ax^2 + By^2 + Cx + Dy + E = 0
\end{align*}

Siendo $A$ distinto de  $B$ y mayores a 0.

Para pasar de esta a la otra, igual que en la circunferencia, 
completamos cuadrado e igualamos a uno el segundo componente.

\subsubsection{Ecuación de recta tangente a la elipse en un punto}

\begin{align*}
        \frac{x*x_0}{a^2} + \frac{y*y_0}{b^2} = 1
\end{align*}


\subsection{Hipérbola}

La hipérbola es el lugar geométrico que se mueven de tal manera que el 
valor absoluto de la diferencia de sus distancias a dos puntos fijos 
-llamados focos- es constante. Es decir:

\begin{align*}
    |PF-PF'| = 2a
\end{align*}

\subsubsection{Elementos}

Vértices: $(\pm a, 0)$

Focos: $(\pm c, 0)$

Extremos del eje conjugado: $(0, \pm b)$

Eje real: $2a$

Eje imaginario o conjugado: $2b$

Eje focal: $2c$

Centro: centro de simetría.

Condición es:

\begin{align*}
    c^2 = a^2 + b^2 \\
    c > b \\
    c > a \\
\end{align*}

\subsubsection{Excentricidad}

\begin{align*}
    e = \frac{c}{a} \, (e > 1)
\end{align*}

\subsubsection{Ecuación canónica de centro en origen}

Con eje focal sobre el eje x:

\begin{align*}
    \frac{x^2}{a^2} - \frac{y^2}{b^2} = 1
\end{align*}

Con eje focal sobre y:

\begin{align*}
    - \frac{x^2}{b^2} + \frac{y^2}{a^2} = 1
\end{align*}

(Se centra en el que queda positivo)

\subsubsection{Ecuación canónica con centro $(x_0, y_0)$}

\begin{align*}
    \frac{(x-x_0)^2}{a^2} - \frac{(y-y_0)^2}{b^2} = 1
\end{align*}

Recordar que si se centra en eje vertical cambian los signos y los denominadores.

Se trabaja igual que la otra,
dejamos sobre el segundo término el independiente, 
buscamos igualarlo a 0 y completar los cuadrados.

\subsubsection{Hipérbola equilátera}

\begin{align*}
    xy = k
\end{align*}

\subsubsection{Ecuación de la recta tangete a una hipérbola centrada en origen}

\begin{align*}
        \frac{x*x_0}{a^2} - \frac{y*y_0}{b^2} = 1
\end{align*}


\subsection{Parábola}

Es el lugar geométrico de los puntos del plano que equidistan de un punto fijo,
que se llama foco, y de una recta fija, que llamamos directriz.

\subsubsection{Elementos}

Foco: punto fijo F 

Directriz: recta fija d 

Parámetro p: distancia del foco a la directriz, se designa por $p$

Eje: Recta perpendicular a la directriz, pasa por el foco 

Vértice: intersección parábola con su eje, punto medio entre directriz y foco 

\subsubsection{Ecuación de la parábola centrada en origen}

Hay dos casos, Parábola de $V(0,0)$, eje horizontal:

\begin{align*}
    y^2 = 2px
\end{align*}

Parábola de $V(0,0)$, eje vertical:

\begin{align*}
    x^2 = 2py
\end{align*}


\subsubsection{Ecuación de la parábola centrada en $V(a,b)$}

Centrada en eje horizontal:

\begin{align*}
    (y-b)^2 = 2p(x-a)
\end{align*}

Centrada en vertical: 

\begin{align*}
    (x-a)^2 = 2p(y-b)
\end{align*}

\subsubsection{Pasaje general a canónica}

Ponemos la cuadrática para un lado, la lineal para el otro con el coeficiente 
independiente.

Completamos el cuadrado y tomamos factor comun para el vértice $V(a,b)$.