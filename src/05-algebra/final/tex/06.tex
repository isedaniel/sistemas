\section{Transformaciones lineales}

Una transformación lineal es una función, que tiene dominio y codominio,
con la particualidad que estos son \textbf{espacios vectoriales}.

Es una regla que transforma vectores del espacio vectorial $V$ a vectores
en $W$.

No toda función que transforme $V\to W$ es una transformación lineal.

Tiene que cumplir tres condiciones:

\begin{align*}
    T(u+v) = T(u) + T(v)      \\
    T(k\cdot v) = k\cdot T(v) \\
    T(0_v) = 0_w
\end{align*}

En la práctica, esto se logra con expresiones que:

\begin{enumerate}
    \item sean lineales
    \item no multipliquen argumentos, o sea, no $xy$ y del estilo.
    \item no haya términos independientes
\end{enumerate}

\subsection{Teorema fundamental de las tranformaciones lineales}

Dados los espacios vectoriales $V$ y $W$,
siendo $B = v_n$ base de $V$,
y $w_n$ vectores de $w$,
existe \textbf{una única} transformación lineal que verifica:

\begin{align*}
    T(v_n) = w_n
\end{align*}

Para buscar la transformación tomamos un vector genérico $(x,y)$,
lo escribimos como resultado de transformación lineal de la base,
$(x,y) = \alpha(1,0) + \beta(1,1)$.

Las letras griegas son las coordenadas del vector de la base.
Planteamos el sistema de ecuaciones y las depejamos.

\begin{align*}
    \begin{cases}
        x = \alpha + \beta
        y = \beta \implies \alpha = x - y
    \end{cases}
\end{align*}

Aplicamos $T$ a ambos lados: $T(x,y) = \alpha T(1,0) + \beta T(1,1)$

Sabemos que $T(1,0) = (0,1,0)$ y que $T(1,1) = (0,1,0)$, por lo tanto:

\begin{align*}
    T(x,y) = \alpha (0,1,0) + \beta (0,1,0) \\
    T(x,y) = (x - y) (0,1,0) + y (0,1,0) \\
    T(x,y) = (0, x - y + y, 0)
\end{align*}

Así obtenemos la expresión de la transformación: $T(x,y) = (0,x,0)$


\subsection{Núcleo de una TL}

El núcleo de la TL se denota $Nu(T)$, y es la base que resulta de igualar 
la TL a $0_v$.

Igualamos la TL a 0, despejamos y obtenemos una expresión, que la separamos en 
vectores, esa es la base del núcleo.

\begin{align*}
    (x-2y, 0, 2x-4y) = (0,0,0) \implies x = 2y
\end{align*}

Como $z$ no aparece, \textbf{está libre}, entonces:

\begin{align*}
    (2y,y,z) = y(2,1,0) + z(0,0,1)
\end{align*}

La base del núcleo es $Nu(T) = (2,1,0),(0,0,1)$. Es de dimensión 2.

\subsection{Imagen de una TL}

Resulta de dividir en vectores la expresión que define la TL,
por ejemplo:

\begin{align*}
    (x-2y, 0, 2x-4y) = x(1,0,2) + y(-2,0,-4)
\end{align*}

Como son LD, tomamos uno solo para la base: $Im(T) = (1,0,2)$. Es de dimensión 1.

\subsection{Teorema de las dimensiones}

Establece la relación entre dimensión de $V$, $Nu(V)$ e $Im(V)$:

\begin{align*}
    dim(V) = dim(Nu(T)) + dim(Im(T))
\end{align*}


\subsection{Clasificación de las transformaciones lineales}

Son funciones y como tales pueden ser inyectivas, sobreyectivas y biyectivas.

Sobreyectivas quería decir que podemos llegar a todos los elementos del 
codominio. En otras palabras, si va a $R^2$, nuestra función abarca todo $R^2$
con almenos un valor del dominio.
Para todo valor en $Y$ hay un $x$ con el cual llegar.

Inyectiva quiere decir que, para todo $Y$, hay solo un $x$ con el que podemos 
llegar.

Biyectivas, llegamos a todos y además solo a través de un $x$.

Una transformación lineal es:

\begin{itemize}
    \item Monomorfismo si es inyectiva 
    \item Epimorfismo si es sobreyectiva 
    \item Isomorfismo si es biyectiva 
    \item Endomorfismo si $V = W$
\end{itemize}

Si el núcleo de una TL es el vector nulo $0_v$,
la TL es un monomorfismo. 

Si la imagen es todo $W$, es epimorfismo.

Si el núcleo es $0_v$ y además genera todo $W$, es isomorfismo.

No es necesario analizar inyectividad y sobreyectividad por separado,
una implica a la otra y solo ocurre cuando dominio y codominio tienen la 
misma dimensión.

\subsection{Matriz asociada a una TL}

Es la matriz de coeficientes de la transformación lineal.

\subsection{Inversa de la TL}

Si tengo una transformación lineal biyectiva,
es decir, que es un isomorfismo, 
existe una transformación lineal inversa.

Para determinarla construimos la asociada:

\begin{align*}
    M = \begin{pmatrix}
        x \\ y \\ z
    \end{pmatrix}
\end{align*}

Buscamos su inversa, y con ella encontramos la TL inversa.