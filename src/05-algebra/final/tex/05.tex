\section{Subespacios vectoriales}

\subsection{Introducción}

Un \textbf{espacio vectorial} es un conjunto no vacío $V$,
cuyos elementos se denominan \textbf{vectores},
y tiene definidas dos operaciones:
suma como ley de composición interna (sigue dentro de $V$)
y producto de escalar por vector composición externa ($k$ es externo, pero 
el resultado $v$ sigue dentro de $V$).

De acuerdo a la definición, $R^3$ es espacio vectorial. 
Todos los espacios $R^n$, con $n \geq 1$ son espacios vectoriales.

\subsection{Subespacio vectorial}

$W$ es subespacio de $V$, si cumple con los criterios de ser \textit{espacio 
vectorial}.

\subsection{Condiciones necesarias y suficientes para caracterizar un subespacio}

\begin{enumerate}
    \item $\vec{0}$ está en $W$
    \item Si $\vec{u}$ y $\vec{v}$ están, $\vec{u}+\vec{v}$ está 
    \item Si $\vec{v}$ está, $k\cdot\vec{v}$ está.
\end{enumerate}

Entonces:

\begin{enumerate}
    \item Vemos que esté el vector nulo
    \item Vemos dos vectores y los sumamos
    \item Multiplicamos un vector por -1 para buscar su opuesto
\end{enumerate}

\subsection{Subespacios triviales}

\begin{enumerate}
    \item V es subespacio trivial de si mismo
    \item El subespacio que contiene solo al vector nulo
\end{enumerate}


\subsection{Combinación lineal}

Una combinación lineal de dos o más vectores es un vector,
que se obtiene de sumar esos vectores multiplicados por algun escalar.

Cualquier vector en el plano se puede representar como una combinación lineal 
de dos vectores, \textit{siempre que estos no sean paralelos},
es decir, que tengan distinta dirección.

La combinación lineal para obtener cada vector es \textit{única}.

\begin{equation*}
    \vec{m} = x\vec{u} + y\vec{v} + z\vec{w}
\end{equation*}

Si resolviendo este sistema llegamos a un SCI, se puede expresar como combinación 
lineal de infinitas maneras.

En $R^2$ podemos generar cualquier vector con dos vectores no paralelos.
Si tenemos 3 vectores no paralelos, podemos generar todo de infinitas formas.

En $R^3$ podemos generar cualquier vector con tres vectores 
\textit{no coplanares}, es decir, que ninguno sea combinación lineal de los otros
(se chequea con Teorema de Cramer).
Si tenemos menos de 3, no generamos todo $R^3$ y hay que usar 
método de Gauss para ver sobre qué plano estamos.

Si tenemos más de 3, generamos todo vector de infinitas formas.

\subsection{Conjuntos generadores}

(Un poco se anticipa en el apartado anterior).

Un conjunto generador de $V$ es un conjunto de vectores $A$ que,
con combinaciones lineales, pueden generar a $V$, por ejemplo:

\begin{equation*}
    A = \left\{ (1,0),(0,1) \right\} \,\text{generan } R^3
\end{equation*}

\subsection{Subespacio generado}

Dado un conjunto generador $A$, el \textbf{subespacio generado} $W$ 
es el conjunto de combinaciones lineales posibles usando $A$,
que generan $W$ pero no a todo el espacio $V$.

Para ver el subespacio generado por $(1,0,2)$ y $(0,0,1)$:

\begin{equation*}
    \alpha(1,0,2) + \beta(0,0,1) = (\alpha, 0, 2\alpha + \beta)
\end{equation*}

Es decir, genera todos los vectores del plano $y=0$.

Si tengo 3 vectores $(1,0,0), (0,1,0), (0,2,1)$, calculo el determinante,
da distinto de 0, por lo tanto, para construir el vector que quiera voy a 
tener valores de los parámetros, este conjunto \textbf{genera} $R^3$.

Si tengo 3 vectores $(1,0,0),(0,1,1),(1,1,1)$, 
calculo el determinante tengo 0.
Esto quiere decir que no voy a poder generar el vector que quiera.

Para conocer el plano uso Método de Gauss con matriz ampliada y 
busco que una de sus filas sea 0.

\begin{equation*}
\left(
\begin{array}{ccc|c}
1&0&1&x \\
0&1&1&y \\
0&1&1&z \\
\end{array}
\right)
\end{equation*}

A la tercera fila le resto la segunda y obtengo:

\begin{equation*}
\left(
\begin{array}{ccc|c}
1&0&1&x \\
0&1&1&y \\
0&0&0&z-y \\
\end{array}
\right)
\end{equation*}

Hecho que la última fila sea 0, concluyo que es compatible para el plano $z-y=0$.
Estos vectores generan el \textbf{subespacio} $z-y=0$.

En el caso de matrices,
para generar $R^{2\times3}$ necestaría $2\times3$ matrices, 
que no sean combinación lineal.

\subsection{Dependencia lineal}

Si tengo un conjunto de vectores $V$,
lo igualo al vector $0_v$,
es decir,
planteo un \textit{sistema homogéneo},
decimos que el conjunto es linealmente independiente si la solución trivial 
es la \textbf{única} solución.

Si además de la solución trivial tengo infinitas soluciones,
el sistema es sistema compatible indeterminado,
por lo tanto, es linealmente dependiente.

Alternativamente,
si uno de los vectores es combinación lineal de los otros,
el conjunto es \textbf{dependiente}.

Planteo el sistema homogéneo,
si la única solución es 0, LI, sino LD.

\textbf{Propiedades.}

\textbf{1.}

El conjunto \(\left\{ 0_v \right\}\) es el único conjunto de un solo 
vector que es LD. Todos los demás son LI.

\textbf{2.}

Si un conjunto de vectores contiene $0_v$, es LD.

\subsection{Intepretación geométrica}

Son LD solo si son paralelos.
En $R^3$ son paralelos si son coplanares,
es decir, el determinante de la matriz de coeficientes tiene que dar 0.

\subsection{Base}

Decimos que el conjunto de vectores $B$ es base de $V$ si:
1. es linealmente independiente y 2. genera $V$.

Ejemplo: $B=(1,0)(1,1)$ es base de $R^2$. 
$C=(1,0),(1,1),(2,0)$ no es base de $R^2$, porque es linealmente dependiente.

La base canónica de $R^2$ es $B=(1,0),(0,1)$.

Toda base de $R^n$ tiene $n$ vectores.

\subsection{Dimensión}

La dimensión de un espacio vectorial es la cantidad de vectores que componen su 
base, es decir, la base de $R^n$ es $n$.

Al espacio compuesto por el vector nulo se le asigna base 0:

\begin{equation*}
    dim(\left\{ 0_v \right\}) = 0
\end{equation*}

\textbf{Propiedades.}

1. Todo conjunto de $n$ vectores LI es una base.
2. Todo conjunto de más de $n$ vectores es LD, no es base.

\subsection{Base y dimensión de subespacio vectorial}

Si $S$ es subespacio de $V$, entonces $dim(S) \leq dim(V)$.
La dimensión del subespacio tiene que ver con la cantidad de variables libres 
de las que disponemos para combinar linealmente y generar vectores.

\subsection{Operaciones con subespacios}

\textbf{Intersección}

Si tenemos los subespacios $S$ y $T$, tales que:

\begin{align*}
    S = {P: x - 3z = 0} \\
    T = {Q: x + y - z = 0}
\end{align*}

La intersección $T\cap S$ está formado por los vectores que pertenecen 
simultáneamente a $S$ y a $T$.
Es decir, aquellos que \textit{satisfacen ambas ecuaciones}.

Llegamos a $(3,-2,1)$, que es base de $S\cap T$.

\textbf{Suma de subespacios.}

Es el conjunto de los generadores de los dos subespacios, $S$ y $T$.

La base de $S$ es $(3z,y,z) = z(3,0,1) + y(0,1,0)$, es decir,
$B_S = \left\{ (3,0,1),(0,1,0) \right\}$.

Base de $T$: $x = z - y \implies (z-y,y,z) = z(1,0,1) + y(-1,1,0)$,
entonces la base es $(1,0,1) + (-1,1,0)$.

Por definición, como vamos a tener 4 vectores,
el conjunto sería LD, para armar la base nos quedamos con el primero no coplanar,
en este caso probamos si $(1,0,1)$ verifica la ecuación de $S$:
$1 - 3 \neq 0$, por lo tanto no es coplanar, lo usamos.

La base de $S+T$ es $(3,0,1),(0,1,0),(1,0,1)$, genera $R^3$.

