\section{Matrices}

\subsection{Introducción}

Una matriz es un ordenamiento rectangular de escalares,
en $m$ filas y $n$ columnas,
Se designa: $A \in \mathbb{R}^{m \times n}$.
Cada uno de los números que la compone es un \textbf{elemento},
que se \textit{distingue} por su posición.

$m\times n$ es el tamaño, \textbf{orden}, tipo o \textbf{forma} de la matriz.

\subsection{Matrices particulares}

\textbf{Matriz columna.}

Matriz de forma:

\begin{equation*}
    A \in R^{m\times1}
\end{equation*}

\textbf{Matriz fila.}

Matriz de forma:

\begin{equation*}
    A \in R^{1\times n}
\end{equation*}

\textbf{Matriz nula.}

Matriz con todos elementos 0.

\textbf{Matriz cuadrada.}

Matriz donde $m = n$, es decir, $A \in R^{n\times n}$

\textbf{Matriz traspuesta.}

Se denota $A^T$, se obtiene cambiado cada fila en columna.

Propiedades:
\begin{itemize}
    \item $(A+B)^T = A^T + B^T$
    \item $(kA)^T = k\cdot A^T$
    \item $(A^T)^T = A$
    \item $(A\cdot B)^T = B^T \cdot A^T$ (da vuelta el producto)
\end{itemize}

\textbf{Matriz simétrica.}

Cuando $A=A^T$.

\textbf{Matriz triangular superior.}

Cuando todos los elementos bajo la diagonal principal son 0.

\textbf{Matriz triangular inferior.}

Cuando todos los elementos bajo la diagonal principal son 0.

\textbf{Matriz diagonal.}

Todos los elementos, salvo la diagonal, son 0.

\textbf{Matriz identidad.}

Matriz diagonal donde todos los elementos de la diagonal son 1.

\subsection{Igualdad de matrices}

Cuando las matrices son de igual forma, y además sus elementos respectivos 
son iguales.

\subsection{Operaciones con matrices}

\textbf{Suma y resta.}

Cuando son iguales: se suma o resta elemento por elemento.

\textbf{Producto de un escalar por matriz.}

Se multiplica cada elemento por el escalar.

\textbf{Producto de matrices.}

Dada $A \in R^{m\times n}$ y $B \in R^{n\times p}$, devuelve una matriz 
$C \in R^{m\times p}$.

Columnas de la primera DEBEN ser iguales a filas de la segunda.

El producto de matrices NO ES CONMUTATIVO.

\subsection{Determinante}

Permite saber si una matriz tiene inversa
y si un sistema de ecuaciones lineales tiene solución,
cuando es \textbf{distinto de 0}.

\textbf{Propiedades.}

Propiedad 1.

El determinante del producto es igual al producto de sus determinantes.

\begin{equation*}
    |A\cdot B| = |A| \cdot |B|
\end{equation*}

Propiedad 2.

Si hay fila o columna 0, el determinante es 0.

Propiedad 3.

Si multiplico fila o columna por escalar $k$, 
el determinante es multiplicado por $k$.

Propiedad 4.

Se puede extraer $k$ a $n$ filas o columnas,
dividiendo el determinante por el factor común $k^n$.

Esto es consecuencia de la propiedad anterior.

Propiedad 5.

Si se cambia el orden de una fila o columna,
el determinante cambia de signo.

Propiedad 7.

Si una matriz es invertible, el determinante de la inversa es el inverso 
del determinante:

\begin{equation*}
    |A^{-1}| = 1/|A|
\end{equation*}

Propiedad 8.

El determinante de $A$ es igual al determinante de traspuesta de $A$:

\begin{equation*}
    |A| = |A^T|
\end{equation*}

Propiedad 9.

Si una fila o columna es combinación lineal de otras, 
el determinante es 0.

Propiedad 10.

El determinante permanece constante si a una fila le sumamos otra 
multiplicada por un escalar.

Propiedad 11.

El determinante de una matriz triangular es el producto de su diagonal.

\textbf{Submatriz.}

Es la matriz que obtenemos cuando suprimimos una fila $i$ y una columna $j$.


\subsection{Matriz adjunta}

\textbf{Adjunto de un elemento.}

Es el determinante de la submatriz, con signo positivo o negativo de acuerdo 
a si $i+j$ es par o impar.

Básicamente con esto calculamos el determinante.

La matriz adjunta es la que se obtiene remplazando cada elemento por su adjunto.


\subsection{Matriz inversa}

Si $A$ es matriz cuadrada, con determinante $|A| \neq 0$,
existe $A^{-1}$, tal que:

\begin{equation*}
    A^{-1}\cdot A = A\cdot A^{-1} = I 
\end{equation*}

Una matriz que admite inversa es \textbf{regular}.
Una matriz que no admite inversa es \textbf{singular}.

\textbf{Propiedades de la inversión.}

\textbf{1.} Si $AB$ es invertible, entonces: $(AB)^{-1} = B^{-1}\cdot A^{-1}$.

\textbf{2.} $(kA)^{-1} = k^{-1}\cdot A^{-1}$

\textbf{3.} $(A^{-1})^T = (A^T)^{-1}$

\textbf{Cálculo.}

Dos formas:
\begin{enumerate}
    \item Método de la adjunta 
    \item Método de Gauss
\end{enumerate}

\textbf{Método de la adjunta.}

\begin{align*}
    A^{-1} = \frac{\left[ adj(A) \right]^T}{|A|}
\end{align*}

Adjunta era sustituir cada elemento por el determinante de su submatriz.

\textbf{Método de Gauss.}

Definimos la matriz por bloques $G=(A|I)$, siendo $I$ la matriz identidad.
Operamos por fila hasta que del lado izquierdo quede la identidad,
quedando la inversa del lado derecho.

\textbf{Aplicación.}

Sirve para despejar productos de matrices en ecuaciones matriciales:

Obtener X siendo $A\cdot X = B \implies A^{-1}\cdot A\cdot X = I\cdot X = A^{-1}\cdot B$
