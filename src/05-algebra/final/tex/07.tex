\section{Autovalores, autovectores, autoespacios}

Tengo $A$ matriz $n \times n$,
$\lambda$ es autovalor de $A$ si hay un vector columna $v$ no nulo,
que cumpla:

\begin{align*}
    A \cdot v = \lambda \cdot v
\end{align*}

$v$ es el autovector asociado al autovalor $\lambda$.

Para encontrarlos operamos:

\begin{align*}
    |A-\lambda I| = 0_v
\end{align*}

Esta es la \textbf{ecuación característica de A}.
Las raíces de la ecuación característica son los \textbf{autovalores} de A.

Para encontrar los autovectores resolvemos, con cada autovalor:

\begin{align*}
    (A-\lambda I)
    \begin{pmatrix}
        x \\ y
    \end{pmatrix} =
    \begin{pmatrix}
        0 \\ 0
    \end{pmatrix}
\end{align*}

La solución de este sistema siempre es un \textbf{subespacio},
denominado \textbf{autoespacio}.
Devolvemos la \textbf{base}.

Los autovectores siempre son LI.

Para verificar que esté bien usamos la primera expresión:

\begin{align*}
    A \cdot v = \lambda \cdot v
\end{align*}

\textbf{Propiedades.}

\begin{enumerate}
    \item En una matriz triangulada, los autovalores son los elementos de la 
    diagonal principal
    \item Si por lo menos uno de los autovalores es 0, 
    la matriz es singular (determinante 0).
    \item La suma de autovalores es la suma de elementos de la diagonal principal.
\end{enumerate}

\subsection{Matrices similares}

$A$ y $B$ son matrices semejantes si:

\begin{align*}
    B=P^{-1} \cdot A \cdot P
\end{align*}

\textbf{Propiedades de las semejantes.}

Tienen los mismos autovalores.

\subsection{Matriz diagonalizable}

$A$ es diagonalizable si es similar a una matriz diagonal $D$:

\begin{align*}
    P^{-1} \cdot A \cdot P = D
\end{align*}

\textbf{Condiciones que tiene que cumplir.}

\begin{enumerate}
    \item $A$ tiene autovectores LI
\end{enumerate}

Construimos $P$ con los autovectores como columnas.
Si $|P| \neq 0$, esto es, es \textit{regular} y \textit{no singular},
es invertible.

Si calculando autovectores, encontramos un autovalor con multiplicidad algebraica
mayor a la dimensión del autoespacio asociado, entonces nos va a faltar un 
autovector para que $P$ sea regular, por lo tanto esa matriz no es diagonalizable.

\textbf{El orden de los autovalores en D es igual al de los autovectores en P.}