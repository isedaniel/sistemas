\section{Sistemas de ecuaciones}

\subsection{Introducción}

Un sistema de dos ecuaciones y dos incógnitas tiene una solución cuando hay un par
$(x_1, y_1)$ que satisface ambas ecuaciones a la vez.

Para comprobar se reemplaza.

\subsection{Clasificación}

Esta clasificación abarca todo sistema lineal $m$ ecuaciones con $n$ incógnitas 
$m \times n$:

\subsubsection{Sistema compatible determinado}

Tiene una sola solución $(x_1, y_1)$.
En sistemas lineales en $R^2$ equivale a dos rectas que se intersecan en un punto.

\subsubsection{Sistema compatible indeterminado}

Tiene infinitas soluciones, en $R^2$ son dos rectas coincidentes,
cualquier punto de la recta funciona como solución.

En $R^3$ puede ser la intersección de dos planos, que también forma una recta,
de infinitos puntos.

\subsubsection{Sistema incompatible}

No tiene solución. Gráficamente en $R^2$ son dos rectas paralelas.

En $R^3$ existe el caso de las \textit{rectas alabeadas}, es decir,
que no son coplanares y por lo tanto no se intersecan.

\subsection{Dos métodos}

Los dos métodos clásicos de resolución son \textbf{igualación} y \textbf{sustitución}.

Si es compatible \textit{determinado}, devuelven un número.
Si es \textit{indeterminado}, devuelve una igualdad verdadera de dos números.

Si es \textit{incompatible}, devuelve un absurdo.

\subsection{Sistema lineal de tres incógnitas}

Son \textit{lineales}, porque son polinomios de grado uno, y no hay 
productos de las variables entre si, tipo $xy$.

\subsection{Método de Gauss}

Escribimos la matriz de coeficientes ampliada, que incluye la matriz de 
coeficientes $A$ y la matriz de términos independientes $K$:

\begin{equation*}
    G = (A|K)
\end{equation*}

Hacemos combinaciones lineales hasta que tenemos un sistema escalonado,
y de ahí resolvemos con ecuaciones.

Si llego a una \textbf{contradicción}, el sistema es \textbf{incompatible}.

Si llego a una fila $(0,0,0|0)$, es \textbf{compatible indeterminado},
con lo que queda puedo establecer la recta intersección.

\subsection{Regla de Cramer}

Aplica para lineales que cumplan:

\begin{enumerate}
    \item Número de ecuaciones igual a número de incógnitas
    \item Determinante de matriz de coeficientes distinto de 0
\end{enumerate}

Si cumplen, es compatible determinado y se denomina \textbf{sistema de Cramer}.

\begin{enumerate}
    \item Calculo el determinante de la matriz de coeficientes.
    \item Para cada variable, sustituyo su columna por la matriz de términos independientes.
    \item El valor de cada incógnita resulta de dividir el determinante de cada 
    incógnita por el determinate de la matriz de coeficientes
\end{enumerate}

Ejemplo: 

\begin{equation*}
    \frac{\Delta_X}{\Delta}
\end{equation*}

\subsection{Teorema de Cramer}

Permite analizar las soluciones de sistemas de ecuaciones lineales.

El sistema tiene que ser cuadrado (\textbf{ecuaciones = incógnitas}).

Si el determinante de la matriz de coeficientes $A$ es $|A|\neq 0$,
es compatible determinado.

Si $|A| = 0$, puede ser indeterminado o incompatible.

\subsection{Sistema homogéneo de ecuaciones lineales}

Un sistema de ecuaciones lineales es homogéneo si todos los \textbf{términos 
independientes} son \textit{iguales a 0}, es decir, la matriz columna $k$ 
de términos independientes tiene todos elementos 0.

Todo sistema de ecuaciones lineales homogéneo tiene, como mínimo,
la solución trivial, pero puede tener además infinidad de soluciones
si es \textbf{SCI}.

Aplicando regla de Crame, si el determinante es 0, un sistema de ecuaciones 
homogéneo tiene infinitas soluciones.


\subsection{Rango de una matriz}

Es la mayor submatriz cuadrada que podemos encontrar cuyo determinante sea 
distinto de 0. 

Se puede entender como el número de filas o columnas linealmente independientes 
que la matriz contiene.

Una línea es linealmente dependiente cuando resulta de la combinación 
lineal de las otras líneas de la matriz.

Se simboliza $rang(A)$.


\subsection{Teorema Rouché-Frobenius}

Si el rango de la matriz de coeficientes $A$ es distinto del rango de la ampliada,
el sistema es \textbf{incompatible}.

\begin{equation*}
    rang(A) \neq rang(A*) \implies\,\text{Sistema incompatible}
\end{equation*}

Si coinciden, es compatible. Ahora, para ser determinado,
el rango debe ser igual al número de incógnitas,
es decir,
debe haber tantas líneas linealmente independientes como incógnitas.

Si hay menos líneas independientes que incógnitas,
es compatible indeterminado.

Cuando el rango es igual, pero menor a $n$, consecuentemente el sistema 
es compatible pero indeterminado, podemos calcular el 
\textbf{grado de libertad} del sistema, de acuerdo a la diferencia entre 
el rango y $n$. Por ejemplo, un sistema de rango 2, con 3 variables,
tendrá una variable libre, su grado de libertad será 1.