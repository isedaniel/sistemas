\section{Vectores, rectas y planos}

\subsection{Vector}

Elemento del espacio vectorial.
Se pueden representar en el plano $R^2$ o en el espacio $R^3$.

Ejemplos de magnitudes físicas vectoriales: velocidad,
fuerza. Puesto que su efecto no depende solo de su magnitud,
sino de la dirección en la que actúan.

Un vector $AB$ es un segmento orientado que va de un \textbf{origen}
a un \textbf{extremo}.

\subsubsection{Elementos}

Dirección: dirección de la recta que contiene al vector.

Sentido: va del origen $A$ al extremo $B$.

Módulo: longitud del segmento, se representa $|AB|$.
Es siempre positivo o 0.
Si es 1: es un vector unitario o \textbf{versor}.

\subsubsection{Vectores equipolentes}

Decimos que dos vectores son \textbf{equipolentes} cuando tienen igual
módulo, dirección y sentido.

\subsubsection{Vectores libres}

El conjunto de todos los vectores equipolentes se llama \textbf{vector libre}.
Cada \textbf{vector fijo} es un representante del conjunto vector libre.

\subsubsection{Vector fijo}

Es un representante del vector libre, tiene módulo, dirección, sentido y 
origen.

\subsubsection{Vector opuesto}

Mismo módulo, dirección, sentido opuesto.

\begin{align*}
    AB = -BA
\end{align*}

\subsubsection{Vectores concurrentes}

Vectores concurrentes son vectores que comparten origen.

\subsubsection{Vector de posición}

El vector de posición $P$ es el que une el origen de coordenadas con el punto 
$P$.

\subsection{Operaciones con vectores}

Suma: con regla de la poligonal o regla del paralelogramo.

Regla de la poligonal es hacer coincidir el origen de un vector con el extremo 
del otro, la suma es el vector que va del origen del primero al extremo del 
segundo.

Regla del paralelogramo es poner el origen de los dos en común,
trazar rectas paralelas y obteniendo un paralelogramo,
la diagonal coincide con la suma.

Resta: es la suma del opuesto del vector.

\begin{align*}
    \vec{w} = \vec{v} + -\vec{u}
\end{align*}

\textbf{Producto de un escalar por un vector.}

Mantiene dirección, cambia \textbf{módulo}. 
El \textbf{sentido} cambia si el escalar $k$ es negativo.

\textbf{Ángulo entre dos vectores.}

Es el menor ángulo que forman dos vectores,
por lo que puede variar desde $0^o$ a $180^o$.

\textbf{Proyección de un vector sobre otro}

Se traza, desde el extremo de un vector, una recta perpendicular al otro.
Desde el origen común, se traza el \textbf{vector proyección} de $u$ sobre $v$.
Si el ángulo fuese obtuso, se traza desde el extremo de $u$ hasta la proyección 
de $v$ (y viceversa).

Si forman un ángulo de $90^o$ no hay proyección, por lo que el vector proyección 
sería el vector nulo.

\textbf{Calcular el módulo del vector proyección.}

Como trazamos una perpendicular desde el extremo hacia el otro vector 
-o hacia su proyección, en caso de que el ángulo sea obtuso-
el vector $u$ sería hipotenusa de un triángulo rectángulo, 
por lo que el módulo del vector proyección sería:

\begin{align*}
    |P| = \cos\alpha \cdot |u|
\end{align*}

Aqui se ve que si $\alpha$ mide $90^o$, $|P|$ = 0.

\subsubsection{Vectores opuestos}

Mismo módulo, misma dirección, \textbf{sentidos opuestos}.

\subsubsection{Coordenada del vector}

Extremo menos origen:

\begin{align*}
    \vec{AB} = B - A
\end{align*}

\subsection{Módulo del vector}

Pitágoras con las componentes del vector:

\begin{align*}
    |\vec{v}| = \sqrt{v_x^2 + v_y^2}
\end{align*}

\subsubsection{Operaciones analíticas}

Suma y resta es suma y resta de componentes.

\textbf{Producto escalar.}

Sumatoria del producto de las componentes.

\begin{align*}
    u \cdot v = u_x \cdot v_x + u_y \cdot v_y
\end{align*}

También resulta de multiplicar los módulos por el ángulo que forman:

\begin{align*}
    u \cdot v = |u| \cdot |v| \cdot \cos\alpha
\end{align*}

De aquí deducimos que, \textbf{para saber el ángulo que forman} dos vectores:

\begin{align*}
    \cos\alpha = \frac{u \cdot v}{|u| \cdot |v|}
\end{align*}

\textbf{Escalar por vector.}

Vector que resulta de multiplicar el escalar por cada componente.

\subsubsection{Condición de ortogonalidad}

Dos vectores son ortogonales -perpendiculares- si su producto escalar es 0:

\begin{align*}
    u \cdot v = 0
\end{align*}

\subsubsection{Ortonormales}

Cuando son perpendiculares y \textbf{además son unitarios}.

\subsubsection{Cosenos directores}

Coseno de los ángulos que forma el vector respecto de los ejes coordenados:

\begin{align*}
    \cos\alpha = \frac{v_x}{|v|} &\hspace{24pt} \cos\beta = \frac{v_y}{|v|}
\end{align*}

Con esto puedo saber si dos vectores son paralelos, porque tendrían 
iguales cosenos directores:

\begin{align*}
    \frac{v_x}{u_x} = \frac{v_y}{u_y} = \frac{|v|}{|u|}
\end{align*}

\subsubsection{Normalización}

Para convertir un vector en vector unitario -con módulo = 1-:

\begin{align*}
    \vec{n} = \frac{v}{|v|}
\end{align*}


\subsection{Vectores en el espacio}

Es la generalización de lo descrito para el vector en el plano,
pero a una \textbf{tercera coordenada $z$}.

\subsubsection{Proyección de $u$ sobre $v$}

Igual que en el plano:

\begin{align*}
    P_{u,v} = \frac{u \cdot v}{|v|}
\end{align*}

Resulta en un número real, obviamente.

\subsubsection{Producto vectorial}

Esta es una operación propia de $R^3$.
Resulta en otro vector \textit{perpendicular} tanto a $u$ como a $v$,
con sentido siguiendo la regla del pulgar.

\begin{align*}
    u \times v = 
    \begin{vmatrix}
    i & j & k \\
    u_x & u_y & u_z \\
    v_x & v_y & v_z \\    
    \end{vmatrix}
\end{align*}

Cuando los vectores son \textbf{paralelos},
el producto vectorial devuelve el \textbf{vector nulo} $0_v$

El módulo del producto vectorial es:

\begin{align*}
    |u \times v| = |u||v| \cdot \sen\alpha
\end{align*}

El módulo del producto vectorial coincide con el área del paralelogramo 
que forman $u$ y $v$.

El producto vectorial no es conmutativo.

\subsubsection{Producto mixto}

Es combinar producto vectorial y producto escalar. Devuelve un \textbf{real}.

\begin{align*}
    (u \times v) \cdot w = u \cdot (v \times w)
\end{align*}

Se calcula con un determinante de los tres vectores:

\begin{align*}
    \begin{vmatrix}
        u_x & u_y & u_z \\
        v_x & v_y & v_z \\
        w_x & w_y & w_z \\
    \end{vmatrix}
\end{align*}

Geométricamente, el producto mixto es igual al volumen del paralelepípedo 
que forman $u$, $v$ y $w$.

\textbf{Propiedad.}
Tres vectores son \textbf{coplanares} si el producto mixto entre ellos es igual a 0.

\subsubsection{Condición de paralelismo}

Igual que con los cosenos directores, pero ahora incluyendo la tercera dimensión:

\begin{align*}
    \frac{a_x}{b_x} = \frac{a_y}{b_y} = \frac{a_z}{b_z} 
\end{align*}

\subsubsection{Vector unitario}

Se calcula igual:

\begin{align*}
    v_n = \frac{v}{|v|}
\end{align*}


\subsection{Rectas en el espacio}

Tenemos el conjunto \textbf{espacio} $R^3$, sus elementos son \textbf{puntos},
destacamos el subcojunto \textbf{planos}, dentro de cada plano hay a su vez 
otro subconjunto que son las \textbf{rectas}.

\subsubsection{Ecuación vectorial de la recta}

La recta es un conjunto de puntos de un plano, alineados con un punto $P$,
con una dirección dada por un \textbf{vector director} $v$, siendo su expresión:

\begin{align*}
    (x,y,z) = \lambda v + P
\end{align*}

\subsubsection{Ecuación paramétrica de la recta}

Si operamos el producto por un escalar y la suma en la expresión anterior:

\begin{align*}
    (x,y,z) = (\lambda x_v + x_p,\cdots)
\end{align*}

Si con eso planteamos un sistema tenemos:

\begin{align*}
    \begin{cases}
        x = x_p + \lambda \cdot x_v \\
        y = y_p + \lambda \cdot y_v \\
        z = z_p + \lambda \cdot z_v
    \end{cases}
\end{align*}

Estas son las ecuaciones paramétricas, expresamos cada componente del punto 
en función de un parámetro $\lambda$.

\subsubsection{Ecuación general o ecuaciones simétricas de la recta}

Si despejamos el parámetro e igualamos:

\begin{align*}
    \frac{x - x_p}{x_v} = \frac{y - y_p}{y_v} = \frac{z - z_p}{z_v}
\end{align*}

\subsubsection{Posiciones relativas de dos rectas en el espacio}

En el \textbf{plano} las rectas pueden:
\begin{enumerate}
    \item Cortarse
    \item Ser paralelas (siendo coincidentes cuando son la misma)
\end{enumerate}

En el \textbf{espacio} las rectas pueden tener tres posiciones:
\begin{enumerate}
    \item Paralelas, tienen la misma dirección, pudiendo ser coincidentes
    \item Incidentes o secantes, se tocan en un punto $P$
    \item \textbf{Alabeadas}, no se cortan ni son paralelas: 
    están en planos distintos
\end{enumerate}

\subsubsection{Punto de intersección de dos rectas}

Se obtiene igualando las paramétricas de la recta.
Si el sistema es compatible determinado, vamos a encontrar el punto que iguala 
los parámetros.

Si es \textbf{incompatible} las rectas son \textbf{alabeadas}.

\subsubsection{Distancia entre un punto y la recta en el espacio}

La distancia entre un punto $M$ y la recta se deduce de la relación entre 
el vector director $v$ y el punto $P$ de la recta:

\begin{align*}
    d(M, r) = \frac{|\vec{PM} \times \vec{v}}{|v|}
\end{align*}

\subsection{Ecuaciones del plano}

\subsubsection{Ecuación general del plano}

Dados dos puntos, tenemos una única recta que pasa por esos dos $\bar{PQ}$.
Por $PQ$ pasan infinitos planos.

Si tengo un vector $v$, existen infinitos planos perpendiculares a él.

Pero si tenemos un \textbf{vector normal} $n$ y un \textbf{punto} $P$, 
un \textbf{plano} queda determinado.

La ecuación general del plano es:

\begin{align*}
    Ax + By + Cz + d = 0
\end{align*}

Siendo $n = (A,B,C)$ componentes del vector director,
y $d$ se obtiene evaluando la ecuación general en $P$.

Para graficarlo, naturalmente hacemos $y=0,z=0$ y así sucesivamente,
para ver donde intersecan con los ejes.

De aquí se deduce también que: \textbf{tres puntos no alineados definen un plano}.

\subsubsection{Plano a partir de tres puntos}

Con los tres puntos $P$, $Q$, y $M$, hacemos dos vectores,
por ejemplo, $\bar{PQ}$ y $\bar{PM}$.

Después operamos producto vectorial para obtener el \textbf{vector normal}
al plano.

Con el vector normal más uno de los puntos, determinamos la ecuación general.

\subsubsection{Ecuación vectorial paramétrica del plano}

Asi mismo, dados dos vectores no paralelos y un punto,
podemos definir el plano.

Con la ecuación vectorial paramétrica, dados $u$, $v$ y $P$:

\begin{align*}
    (x,y,z) =  P + \alpha u + \beta v
\end{align*}

Para pasar de la general a la paramétrica despejamos una de las variables,
que queda dependiendo de las otras dos,
con eso construimos un vector y reescribimos.

Para volver calculamos al revés, unificamos en una sola cosa,
procurando que quede en función de dos variables y llegamos a la ecuación general.

\subsubsection{Propiedades de los planos}

\begin{enumerate}
    \item Dado un plano, hay puntos que pertenecen y puntos que no
    \item Para saber si pertenecen, evaluamos con la ecuación general
    \item Si dos planos tienen un punto en común, se cortan en una recta,
    que se obtiene igualando las ecuaciones generales 
    \item Si dos rectas tienen un punto en común, se puede trazar un plano 
    que las contiene 
    \item Por tres puntos que no pertenecen a una recta se puede trazar un solo plano 
\end{enumerate}

\subsubsection{Posiciones relativas de dos rectas}

Si $a$ y $b$ son dos rectas, estas pueden ser:
\begin{enumerate}
    \item copalanares: secantes, paralelas, o la misma 
    \item Alabeadas: no coplanares
\end{enumerate}

\subsubsection{Posición relativa entre recta y plano}

Un plano y una recta que no está en él, ni se tocan, son \textbf{paralelos}.

Se sabe porque el \textbf{vector director} de la recta es \textbf{perpendicular} 
al \textbf{vector normal} del plano, es decir:

\begin{align*}
    u \cdot v = 0
\end{align*}

Si existe un punto donde se toquen: son \textbf{incidentes}.

Si una recta es \textbf{perpendicular al plano}, es perpendicular a todas las 
rectas pertenecientes al plano.

\subsubsection{Posiciones relativas de dos planos}

Puede ocurrir que:
\begin{enumerate}
    \item Son paralelos: no intersecan o son el mismo 
    \item Son secantes: existe una recta que es su intersección
\end{enumerate}

Si dos planos son perpendiculares a una misma recta,
quiere decir que son paralelos. 

Y viceversa: dos rectas perpendiculares a un plano son paralelas.

\subsubsection{Distancia punto a un plano}

Es la menor distancia del punto a los infinitos puntos del plano:

\begin{align*}
    d(P,\pi) = \frac{|P \cdot n + D|}{|n|}
\end{align*}

\subsubsection{Intersección entre dos planos}

Es una recta $r$ si los planos no son paralelos.

Se obtiene igualando sus ecuaciones generales.
Podemos poner dos variables en función de una,
juntar y encontrar una expresión para el vector director de la recta y el punto.

Verificamos el punto reemplazando en los planos.

\subsubsection{Intersección tres planos}

Puede dar:
\begin{enumerate}
    \item Un punto
    \item Una recta 
    \item Vacío (no solo si son paralelos, pueden ser alabeados entre si)
\end{enumerate}

\subsubsection{Intersección recta y plano}

Tomamos las paramétricas de la recta y reemplazamos en la general del plano.
La intersección es el valor del parámetro.