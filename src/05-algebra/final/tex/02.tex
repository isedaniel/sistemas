\section{Números complejos}

\subsection{Introducción}

Surgen por la necesidad de resolver ecuaciones de segundo grado con raíces de
números negativos.
Para ello, se amplía el conjunto de los números reales.

\textbf{La unidad imaginaria.}

\begin{equation*}
    i = \sqrt{-1}
\end{equation*}

Por ejemplo, $\sqrt{-4} = 2 \cdot \sqrt{-1} = 2i$.

\subsection{Forma binómica}

\begin{equation*}
    z = a + bi
\end{equation*}

\subsection{Conjugado de complejo}

\begin{equation*}
    \bar{z} = a - bi
\end{equation*}

\subsection{Forma par ordenado}

\begin{equation*}
    (a,b)
\end{equation*}

Se representa parte real en el eje real, parte imaginaria en el eje imaginario Y.
El gráfico es un \textbf{punto} o un \textbf{vector}.

\subsection{Potencias de i}

Se toma el resto de la división $n/4$, y:

\begin{align*}
    i^{0} = 1  \\
    i^{1} = i  \\
    i^{2} = -1 \\
    i^{3} = -i
\end{align*}

\subsection{Operaciones en forma binómica}

\textbf{Suma}

Se suma lo real con lo real, lo imaginario con lo imaginario.

\textbf{Multiplicación}

Se multiplican como binomios:

\begin{align*}
    (2 + 3i) \cdot (4 + 5i) = 8 + 12i + 10i + 15i^2 = 8 + 22i - 15 = -7 + 22i
\end{align*}

\textbf{Producto de conjugados}

Es $a^2 + b^2$ (sin $i$):

\begin{equation*}
    (3 - 5i) \cdot (3 + 5i) = 9 + 15i - 15i - 25i^2 = 9 + 25 = 34
\end{equation*}

\textbf{División de complejos}

Se deduce multiplicando el numerador y denominador por el conjugado del
denominador:

\begin{align*}
    \frac{a + bi}{c + di} & = \frac{(a + bi) * (c - di)}{(c + di) * (c - di)} \\
                          & = \frac{(ac + bd) + (bc - ad)i}{c^2 + d^2}                                                \\
\end{align*}

\subsection{Forma polar}

Formado por el \textbf{módulo} y su \textbf{argumento} (ángulo respecto del 
semieje positivo x).

\textbf{Módulo.}

\begin{equation*}
    |z| = \sqrt{a^2 + b^2}
\end{equation*}

\textbf{Argumento.}

\begin{align*}
    arg(z) = \arctan\left( \frac{b}{a} \right)
\end{align*}

Según cuadrante:
\begin{itemize}
    \item $1^o$: Queda como está
    \item $2^o$: Le sumo $180^o$ ($\pi$)
    \item $3^o$: Le sumo $180^o$ ($\pi$)
    \item $4^o$: Le sumo $360^o$ ($2\pi$)
\end{itemize}

\textbf{Forma trigonométrica y polar}

\begin{equation*}
    z = a + bi = |z|(\cos\alpha + i\sen\alpha) = (|z||\alpha)
\end{equation*}

\textbf{Conjugado.}

Geométricamente, 
es como si midiera el ángulo $\alpha$ en sentido de las agujas del reloj,
respecto del semieje positivo $x$.
El vector se representa simétrico respecto del eje horizontal,
equivalente a $\bar{z}$.

Una forma de llegar a $\bar{\alpha} = 2\pi - \alpha$

\textbf{Opuesto.}

Ángulo más $\pi$. Es equivalente a $-z$.

\subsection{Operaciones forma polar}

Suma y resta no se pueden hacer.

\textbf{Multiplicación.}

Multiplicamos módulos y sumamos argumentos.

\begin{align*}
    z_1 \cdot z_2 = 
    |z_1|\cdot |z_2|\cdot \left[ \cos\alpha+\beta + i\sen\alpha+\beta \right]
\end{align*}

\textbf{División.}

Divido módulos y resto argumentos:

\begin{align*}
    z_1/z_2 = |z_1|/|z_2|\cdot \left[ \sen(\alpha-\beta) + i\cos(\alpha-\beta) \right]
\end{align*}

\textbf{Potencia.}

Elevo el módulo a $n$ y multiplico el argumento por $n$:

\begin{align*}
    z_1^n = |z_1|^n\cdot \left[ \sen(\alpha\cdot n) + i\cos(\alpha\cdot n) \right]
\end{align*}

Esto es el \textbf{Teorema de De Moivre}.

\textbf{Raíz n-ésima.}

Todo complejo tiene $n$ raíces $n$-ésimas.
Hacemos raíz $n$-ésima del módulo y dividimos el argumento por $n$,
pero como son $n$ raíces, hay que ir de $k=0...n-1$:

\begin{align*}
    \sqrt[n]{z} = \sqrt[n]{|z|} \left[ \cos\left(\frac{\alpha+2k\pi}{n}\right) + i\sen\left(\frac{\alpha + 2k\pi}{n}\right) \right] \, \text{, con } \, k = 0,1...n-1
\end{align*}

Queda un polígono regular, en una circunferencia de radio $\sqrt[n]{|z|}$.

\subsection{Complejos en forma exponencial}

Napier descubre $e$, y Euler trabaja y desarrolla la constante.

La relación $e$ con $\sen$ y $\cos$  se conoce como \textbf{fórmula de Euler:}

\begin{align*}
    e^{i\cdot\alpha} = \cos\alpha + i\sen\alpha
\end{align*}

Permite representar complejos con una notación más corta:

\begin{align*}
    z = |z|\cdot e^{i\cdot\alpha}
\end{align*}

Notar que es \textbf{necesario trabajar con radianes}.

\textbf{Logaritmo de complejos.}

\begin{align*}
    \ln z = \ln |z| + \alpha\cdot i
\end{align*}

\textbf{Exponencial compleja.}

$v^u$, ambos complejos:

\begin{align*}
    v^u = e^{u\cdot\ln v}
\end{align*}