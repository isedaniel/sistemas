\documentclass{article}
\usepackage[a4paper, margin=2.54cm]{geometry}
%\usepackage{graphicx}               % imágenes
%\graphicspath{{img}}
\usepackage{amsfonts}               % fuentes de conjuntos numéricos
\usepackage{amsmath, amssymb}       % símbolos
%\usepackage{tikz}                   % gráficos
%\usepackage{pgfplots}               % plots
%\pgfplotsset{width=10cm, compat=1.9}
\setlength{\jot}{8pt}
\setlength{\parindent}{0cm}
\usepackage{parskip}                % espacio entre párrafos
\usepackage{cancel}                 % cancelar términos
%\usepackage[colorlinks=true, 
%    urlcolor=blue]{hyperref}        % links
%\usetikzlibrary{shapes.geometric}   % shapes
%\usepackage{pdfpages}               % incluir pdfs

\title{Álgebra y geometría analítica\\Guía de ejercicios}
\author{Daniel Ise}
\date{Agosto de 2024}

\begin{document}

\maketitle

Primero, un repaso conceptual.

\textbf{Espacio.} Es un conjunto de \textbf{puntos}. Un \textbf{plano} es un
\textit{subconjunto} del \textbf{espacio}. La \textbf{recta} es, a su vez, un
\textit{subconjunto} del \textbf{plano}.

\textbf{Ecuación vectorial de la recta en \(R^2\).} Una \textbf{recta} es un
conjunto de \textbf{puntos} en el \textbf{plano}.
Dados dos puntos, $P$ y $X$, pertenecientes a la recta $r$, podemos definir el
vector \(\overline{PX}\). Si este vector tiene la misma dirección que
\(\vec{v}\), entonces \(\overline{PX}\) resulta de multiplicar \(\vec{v}\)
por un escalar:

\begin{align*}
    \overline{PX}     & = \lambda \cdot \vec{v}             \\
    (x,y) - (x_1,y_1) & = \lambda \cdot \vec{v}             \\
    (x,y)             & = (x_1,y_1) + \lambda \cdot \vec{v}
\end{align*}

Concluimos que, dados un punto $P (x_1,y_1)$ y un vector \(\vec{v}\), podemos
definir $r$. Esta definición se denomina
\textbf{ecuación vectorial de la recta.}

\textbf{Ecuación vectorial de la recta en \(R^3\).} Dados un vector
\(\vec{v} = (v_x,v_y,v_z)\) y un punto \(P = (x_1, y_1, z_1)\), podemos definir
una recta en \(R^3\) como:

\begin{align*}
    (x, y, z) & = (x_1,y_1,z_1) + \lambda \cdot (v_x, v_y, v_z)
\end{align*}

Esta es la \textbf{ecuación vectorial de la recta} en el \textbf{espacio}
$(R^3)$.

\textbf{Ecuaciones paramétricas de la recta.} Dada una ecuación vectorial de una
recta \(r\), es posible construir un sistema de ecuaciones:

\begin{align*}
    (x, y, z) & = (x_1,y_1,z_1) + \lambda \cdot (v_x, v_y, v_z)                               \\
    (x, y, z) & = (x_1,y_1,z_1) + (\lambda \cdot v_x, \lambda \cdot v_y,\lambda \cdot v_z)    \\
    (x, y, z) & = (x_1 + \lambda \cdot v_x, y_1 + \lambda \cdot v_y, z_1 + \lambda \cdot v_z)
\end{align*}

Expresado como un sistema:

\begin{align*}
    \begin{cases}
        x = x_1 + \lambda \cdot v_x \\
        y = y_1 + \lambda \cdot v_y \\
        z = z_1 + \lambda \cdot v_z
    \end{cases}
\end{align*}

Estas son las \textbf{ecuaciones paramétricas de la recta}.

\textbf{Ecuaciones simétricas de la recta.} Si despejo \(\lambda\) de cada una
de las ecuaciones paramétricas obtengo:

\begin{align*}
    \lambda & = \frac{x - x_1}{v_x} & \lambda & = \frac{y - y_1}{v_y} & \lambda & = \frac{z - z_1}{v_z}
\end{align*}

Igualamos:

\begin{align*}
    \frac{x - x_1}{v_x} = \frac{y - y_1}{v_y} = \frac{z - z_1}{v_z}
\end{align*}

Con la \textbf{condición} que todos los componentes de \(\vec{v} \neq 0\)

Estas son las \textbf{ecuaciones simétricas de la recta}.

\textbf{Posiciones relativas de dos rectas en el espacio.} En el \textbf{plano}
las rectas pueden adoptar dos posiciones: \textbf{intersecarse} en un punto o
ser \textbf{paralelas} (sean o no coincidentes: \(f(x) = g(x)\)).

En el \textbf{espacio} las rectas pueden adoptar tres posiciones:

\begin{enumerate}
    \item \textbf{paralelas}, tienen la misma dirección, pueden ser
          coincidentes.
    \item \textbf{incidentes}, se intersecan.
    \item \textbf{alabeadas}, no se cortan ni son paralelas, \textit{por ende}
          pertenecen a \textbf{planos} diferentes.
\end{enumerate}

\textbf{Punto de intersección de dos rectas.} Se obtiene a partir de las 
\textbf{ecuaciones paramétricas}.  Se igualan los componentes en $x$, $y$ y $z$.

\begin{align*}
    r & =
    \begin{cases}
       x = x_1 + \lambda u_x\\
       y = y_1 + \lambda u_y\\
       z = z_1 + \lambda u_z 
    \end{cases} &
    s & =
    \begin{cases}
        x = x_2 + \mu v_x\\
        y = y_2 + \mu v_y\\
        z = z_2 + \mu v_z 
     \end{cases}
\end{align*}

\begin{align*}
    r \cap s & =
    \begin{cases}
       x_1 + \lambda u_x = x_2 + \mu v_x\\
       y_1 + \lambda u_y = y_2 + \mu v_y\\
       z_1 + \lambda u_z = z_2 + \mu v_z 
    \end{cases}
\end{align*}

Se opera el sistema de ecuaciones y si no encontramos contradicciones, se 
concluye que las rectas $s$ y $r$ se intersecan. Sustituyendo $\lambda$ y $\mu$
se obtiene el \textbf{punto de intersección}.

\end{document}