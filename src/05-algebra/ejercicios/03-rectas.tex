\documentclass{article}
\usepackage[a4paper, margin=2.54cm]{geometry}
%\usepackage{graphicx}               % imágenes
%\graphicspath{{img}}
\usepackage{amsfonts}               % fuentes de conjuntos numéricos
\usepackage{amsmath, amssymb}       % símbolos
%\usepackage{tikz}                   % gráficos
%\usepackage{pgfplots}               % plots
%\pgfplotsset{width=10cm, compat=1.9}
\setlength{\jot}{8pt}
\setlength{\parindent}{0cm}
\usepackage{parskip}                % espacio entre párrafos
\usepackage{cancel}                 % cancelar términos
%\usepackage[colorlinks=true, 
%    urlcolor=blue]{hyperref}        % links
%\usetikzlibrary{shapes.geometric}   % shapes
%\usepackage{pdfpages}               % incluir pdfs

\title{Álgebra y geometría analítica\\Guía de ejercicios}
\author{Daniel Ise}
\date{Agosto de 2024}

\begin{document}

\maketitle

Primero, un repaso conceptual.

\textbf{Espacio.} Es un conjunto, cuyos elementos son \textbf{puntos}. Dentro 
del espacio se destaca el subconjunto que denominamos \textbf{plano}. Cada plano
está formado, a su vez, por un subconjunto de \textbf{rectas}.

\textbf{Ecuación vectorial de la recta en \(R^2\).} Una \textbf{recta} $R$ es un
conjunto de \textbf{puntos} en el plano, alineados con un punto $P (x_1, y_1)$,
con \textbf{dirección} dada por $\vec{v}$. Siendo $P$ un punto de la recta $r$,
podemos afirmar que el vector $\vec{PX}$, que va desde $P$ a un punto cualquiera
$X$, tiene la misma \textbf{dirección} que $\vec{v}$. Por lo tanto, \(\vec{PX}\)
es el resultado de multiplicar \(\vec{v}\) por un escalar, que representamos 
como la letra griega gamma:

\begin{equation*}
    \vec{PX} = _\lambda \cdot \vec{v}
\end{equation*}

De esta ecuación podemos deducir la ecuación de la recta $r$, que requeriría del
vector \(\vec{v}\) y de un punto $P$:

\begin{align*}
    \vec{PX} & = \lambda \cdot \vec{v}\\
    (x,y) - (x_1,y_1) & = _\lambda \cdot \vec{v}\\
    (x,y) & = (x_1,y_1) + _\lambda \cdot \vec{v}
\end{align*}

Esta forma de definir $r$ se conoce como 
\textbf{ecuación vectorial de la recta.}

\textbf{Ecuación vectorial de la recta en \(R^3\).} Dados un vector 
\(\vec{v} = (v_x,v_y,v_z)\) y un punto \(P = (x_1, y_1, z_1)\), podemos definir
una recta en \(R^3\) como:

\begin{align*}
    (x, y, z) & = (x_1,y_1,z_1) + _\lambda (v_x, v_y, v_z)
\end{align*}

Esta es la \textbf{ecuación vectorial de la recta} en el \textbf{espacio}.

\textbf{Ecuaciones paramétricas de la recta.} Dada una ecuación vectorial de una 
recta \(r\), es posible construir un sistema de ecuaciones:

\begin{align*}
    (x, y, z) & = (x_1,y_1,z_1) + _\lambda (v_x, v_y, v_z)\\
    (x, y, z) & = (x_1,y_1,z_1) + (_\lambda \cdot v_x, _\lambda \cdot v_y,_\lambda \cdot v_z)\\
    (x, y, z) & = (x_1 + _\lambda \cdot v_x, y_1 + _\lambda \cdot v_y, z_1 + _\lambda \cdot v_z)
\end{align*}

Expresado como un sistema:

\begin{align*}
    \begin{cases}
        x = x_1 + _\lambda \cdot v_x\\
        y = y_1 + _\lambda \cdot v_y\\
        z = z_1 + _\lambda \cdot v_z
    \end{cases}
\end{align*}

Estas son las \textbf{ecuaciones paramétricas de la recta}.

\textbf{Ecuaciones simétricas de la recta.} Si despejo \(\lambda\) de cada una
de las ecuaciones paramétricas obtengo:

\begin{align*}
    \lambda & = \frac{x - x_1}{v_x} & \lambda & = \frac{y - y_1}{v_y} & \lambda & = \frac{z - z_1}{v_z}
\end{align*}

Igualando estas expresiones:

\begin{align*}
    \frac{x - x_1}{v_x} = \frac{y - y_1}{v_y} = \frac{z - z_1}{v_z}
\end{align*}

Siempre con la \textbf{condición} que todos los componentes de \(\vec{v} \neq 0\)

Estas son las \textbf{ecuaciones simétricas de la recta}.

\end{document}
